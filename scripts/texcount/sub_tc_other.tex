%% Other TC instructions

\begin{description}

\option[ignore]
Indicates start of a region to be ignored. End region with the TC-instruction \code{endignore}.

\option[break \parm{title}]
Break point which initiates a new subcount. The title is used to identify the following region in the summary output.

\option[incbib]
Sets bibliography inclusion, same as running \TeXcount{} with the option \code{-incbib}.

\option[subst \parm{macro} \parm{text}]
This substitutes a macro with any text. The verbose output will show the substituted text: e.g. \code{\%TC:subst \bs{test} TEST} will cause a following \code{\bs{newcommand}\bs{test}\{TEST\}} to be changed into \code{\bs{newcommand} TEST\{TEST\}}, which \TeXcount{} will interpret differently. Use with care!

\option[newcounter \parm{name} \opt{description}]
Define a new counter with the given name and description (optional). A corresponding parsing rule will also be added with the same name.

\end{description}



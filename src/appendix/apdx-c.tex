\lstloadlanguages{C}
\lstset{language=C}
\begin{lstlisting}
/* Includes */
#include <float.h> /* for DBL_MAX */
#include <string.h> /* for memset, memcpy */
#include "macros.h"
#include "top_n_outlier_pruning_block.h"

/* Check compatibility of defined macros. */
#if (defined(UNSORTED_INSERT) && defined(SORTED_INSERT)) || (!defined(UNSORTED_INSERT) && !defined(SORTED_INSERT))
	#error "Exactly one of UNSORTED_INSERT and SORTED_INSERT should be defined."
#endif /* #if defined(UNSORTED_INSERT) && defined(SORTED_INSERT) */

#if !defined(DEBUG) && defined(STATS)
	#error "STATS should only be defined in DEBUG mode."
#endif /* #if !defined(DEBUG) && defined(STATS) */

/* Forward declarations */
static inline double_t distance_squared(
	const double_t * const vector1, 
	const double_t * const vector2, 
	const size_t vector_dims
	);
static inline double_t insert(
	index_t * const outliers,
	double_t * const outlier_scores,
	const size_t k,
	uint_t * const found,
	const index_t new_outlier, 
	const double_t new_outlier_index
	);
static inline void best_outliers(
	index_t * const outliers,
	double_t * const outlier_scores,
	size_t * outliers_size,
	const size_t N,
	index_t * const current_block,
	double_t * const scores,
	const size_t block_size
	);
static inline void sort_vectors_descending(
	index_t *  const current_block,
	double_t * const scores,
	const size_t block_size
	);
static inline void merge(
	index_t * const global_outliers,
	double_t * const global_outlier_scores,
	const size_t global_outliers_size,
	const size_t N,
	index_t * const local_outliers,
	double_t * const local_outlier_scores,
	const size_t block_size,
	index_t * const new_outliers,
	double_t * const new_outlier_scores,
	size_t * new_outliers_size
	);

#ifdef STATS
static lint_t calls_counter = 0;
static uint_t num_pruned = 0;

void get_stats(lint_t * const counter, uint_t * const prune_count) {
	ASSERT_NOT_NULL(counter);
	ASSERT_NOT_NULL(prune_count);
	
	*counter = calls_counter;
	*prune_count = num_pruned;
}
#endif /* #ifdef STATS */

static inline double_t distance_squared(const double_t * const vector1, const double_t * const vector2, const size_t vector_dims) {
	ASSERT_NOT_NULL(vector1);
	ASSERT_NOT_NULL(vector2);
	ASSERT(vector_dims > 0);
	
#ifdef STATS
	const UNUSED lint_t old_calls_counter = calls_counter;
	calls_counter++;
	ASSERT(calls_counter > old_calls_counter);
#endif /* #ifdef STATS */
	
	double_t sum_of_squares = 0;
	
	uint_t dim;
	for (dim = 0; dim < vector_dims; dim++) {
		const double_t val = vector1[dim] - vector2[dim];
		const double_t val_squared = val * val;
		sum_of_squares += val_squared;
	}
	
	return sum_of_squares;
}

static inline double_t insert(index_t * const outliers,
							  double_t * const outlier_scores,
							  const size_t k,
							  uint_t * const found,
							  const index_t new_outlier, 
							  const double_t new_outlier_score) {
	/* Error checking. */
	ASSERT_NOT_NULL(outliers);
	ASSERT_NOT_NULL(outlier_scores);
	ASSERT(k > 0);
	ASSERT_NOT_NULL(found);
	ASSERT(*found <= k);
	ASSERT(new_outlier >= start_index);
	
	int_t	insert_index  = -1; /* the index at which the new outlier will be inserted */
	double_t removed_value = -1; /* the value that was removed from the outlier_scores array */
	
#if defined(SORTED_INSERT)
	/*
	 * Shuffle array elements from front to back. Elements greater than the new
	 * value will be right-shifted by one index in the array.
	 *
	 * Note that uninitialised values in the array will appear on the left. That
	 * is, if the array is incomplete (has a size n < N) then the data in the
	 * array is stored in the rightmost n indexes.
	 */
	
	if (*found < k) {
		/* Special handling required if the array is incomplete. */
		
		uint_t i;
		for (i = k - *found - 1; i < k; i++) {
			if (new_outlier_score > outlier_scores[i] || i == (k - *found - 1)) {
				/* Shuffle values down the array. */
				if (i != 0) {
					outliers[i-1] = outliers[i];
					outlier_scores[i-1] = outlier_scores[i];
				}
				insert_index  = i;
				removed_value = 0;
			} else {
				/* We have found the insertion point. */
				break;
			}
		}
	} else {
		int_t i;
		for (i = k-1; i >= 0; i--) {
			if (new_outlier_score < outlier_scores[i]) {
				if ((unsigned) i == k-1)
					/*
					 * The removed value is the value of the last element in the
					 * array.
					 */
					removed_value = outlier_scores[i];

				/* Shuffle values down the array. */
				if (i != 0) {
					outliers[i] = outliers[i-1];
					outlier_scores[i] = outlier_scores[i-1];
				}
				insert_index = i;
			} else {
				/* We have found the insertion point. */
				break;
			}
		}
	}
#elif defined(UNSORTED_INDEX)
	if (*found < k) {
		insert_index = *found + 1;
		removed_value = 0;
	} else {
		int_t max_index = -1;
		double_t max_value = DBL_MAX;
	
		int_t i;
		for (i = k-1; i >= 0; i--) {
			if (max_index <= 0 || outlier_scores[i] > max_value) {
				max_index = i;
				max_value = outlier_scores[i];
			}
		}
		
		if (new_outlier_score < max_value) {
			insert_index  = max_index;
			removed_value = max_value;
		}
	}
#endif /* #if defined(SORTED_INSERT) */
	
	/*
	 * Insert the new pair and increment the current_size of the array (if
	 * necessary).
	 */
	if (insert_index >= 0) {
		outliers[insert_index] = new_outlier;
		outlier_scores[insert_index] = new_outlier_score;
		
		if (*found < k)
			(*found)++;
	}
	
	return removed_value;
}

static inline void best_outliers(index_t * const outliers,
								 double_t * const outlier_scores,
								 size_t * outliers_size,
								 const size_t N,
								 index_t * const current_block,
								 double_t * const scores,
								 const size_t block_size) {
	/* Error checking. */
	ASSERT_NOT_NULL(outliers);
	ASSERT_NOT_NULL(outlier_scores);
	ASSERT_NOT_NULL(outliers_size);
	ASSERT(*outliers_size <= N);
	ASSERT(N > 0);
	ASSERT_NOT_NULL(current_block);
	ASSERT_NOT_NULL(scores);
	ASSERT(block_size > 0);
	
	/* Sort the (current_block, scores) vectors in descending order of value. */
	sort_vectors_descending(current_block, scores, block_size);
	
	/* Create two temporary vectors for the output of the "merge" function. */
	index_t  new_outliers[N];
	double_t new_outlier_scores[N];
	size_t   new_outliers_size = 0;
	
	memset(new_outliers, null_index, N * sizeof(index_t));
	memset(new_outlier_scores, 0, N * sizeof(double_t));
	
	/* Merge the two vectors. */
	merge(outliers, outlier_scores, *outliers_size, N, current_block, scores, block_size, new_outliers, new_outlier_scores, &new_outliers_size);
	
	/* Copy values from temporary vectors to real vectors. */
	memcpy(outliers, new_outliers, N * sizeof(index_t));
	memcpy(outlier_scores, new_outlier_scores, N * sizeof(double_t));
	*outliers_size = new_outliers_size;
}

static inline void sort_vectors_descending(index_t *  const current_block,
										   double_t * const scores,
										   const size_t block_size) {
	/* Error checking. */
	ASSERT_NOT_NULL(current_block);
	ASSERT_NOT_NULL(scores);
	ASSERT(block_size > 0);
	
	uint_t i;
	for (i = 0; i < block_size; i++) {
		int_t j;
		index_t  ind = current_block[i];
		double_t val = scores	   [i];
		for (j = i-1; j >= 0; j--) {
			if (scores[j] >= val)
				break;
			current_block[j+1] = current_block[j];
			scores	   [j+1] = scores	   [j];
		}
		current_block[j+1] = ind;
		scores	   [j+1] = val;
	}
}

static inline void merge(index_t * const global_outliers, double_t * const global_outlier_scores, const size_t global_outliers_size, const size_t N,
						 index_t * const local_outliers,  double_t * const local_outlier_scores,  const size_t block_size,
						 index_t * const new_outliers, double_t * const new_outlier_scores, size_t * new_outliers_size) {
	/* Error checking. */
	ASSERT_NOT_NULL(global_outliers);
	ASSERT_NOT_NULL(global_outlier_scores);
	ASSERT(global_outliers_size <= N);
	ASSERT(N > 0);
	ASSERT_NOT_NULL(local_outliers);
	ASSERT_NOT_NULL(local_outlier_scores);
	ASSERT(block_size > 0);
	ASSERT_NOT_NULL(new_outliers);
	ASSERT_NOT_NULL(new_outlier_scores);
	ASSERT_NOT_NULL(new_outliers_size);
	
	*new_outliers_size  = 0;
	uint_t iter = 0; /* iterator through output array */
	uint_t global = 0; /* iterator through global array */
	uint_t local = 0; /* iterator through local array */
	while (iter < N && (global < global_outliers_size || local < block_size)) {
		if (global >= global_outliers_size && local < block_size) {
			/* There are no remaining elements in the global arrays. */
			new_outliers[iter] = local_outliers[local];
			new_outlier_scores[iter] = local_outlier_scores[local];
			local ++;
			global++;
		} else if (global < global_outliers_size && local >= block_size) {
			/* There are no remaining elements in the local arrays. */
			new_outliers[iter] = global_outliers[global];
			new_outlier_scores[iter] = global_outlier_scores[global];
			local ++;
			global++;
		} else if (global >= global_outliers_size && local >= block_size) {
			/*
			 * There are no remaining elements in either the global or local 
			 * arrays.
			 */
			break;
		} else if (global_outlier_scores[global] >= local_outlier_scores[local]) {
			new_outliers[iter] = global_outliers[global];
			new_outlier_scores[iter] = global_outlier_scores[global];
			global++;
		} else if (global_outlier_scores[global] <= local_outlier_scores[local]) {
			new_outliers[iter] = local_outliers[local];
			new_outlier_scores[iter] = local_outlier_scores[local];
			local++;
		}
		
		iter++;
		(*new_outliers_size)++;
	}
}

void top_n_outlier_pruning_block(const double_t * const data,
								 const size_t num_vectors, const size_t vector_dims,
								 const size_t k, const size_t N, const UNUSED size_t default_block_size,
								 index_t * outliers, double_t * outlier_scores) {
	/* Error checking. */
	ASSERT_NOT_NULL(data);
	ASSERT(vector_dims > 0);
	ASSERT(k > 0);
	ASSERT(N > 0);
	ASSERT(default_block_size > 0);
	ASSERT_NOT_NULL(outliers);
	ASSERT_NOT_NULL(outlier_scores);
	
	/* Set output to zero. */
	memset(outliers, null_index, N * sizeof(index_t));
	memset(outlier_scores, 0, N * sizeof(double_t));
	
	double_t cutoff = 0; /* vectors with a score less than the cutoff will be removed from the block */
	size_t   outliers_found = 0; /* the number of initialised elements in the outliers array */
	
#ifndef NO_BLOCKING
	index_t  block_begin; /* the index of the first vector in the block currently being processed */
	size_t   block_size; /* block_size may be smaller than devfault_block_size if "num_vectors mod default_block_size != 0" */
	
	for (block_begin = 0; block_begin < num_vectors; block_begin += block_size) { /* while there are still blocks to process */
		block_size = MIN(block_begin + default_block_size, num_vectors) - block_begin; /* the number of vectors in the current block */
		ASSERT(block_size <= default_block_size);
		
		index_t current_block[block_size]; /* the indexes of the vectors in the current block */
		index_t neighbours[block_size][k]; /* the "k" nearest neighbours for each vector in the current block */
		double neighbours_dist[block_size][k]; /* the distance of the "k" nearest neighbours for each vector in the current block */
		double score[block_size]; /* the average distance to the "k" neighbours */
		uint_t found[block_size]; /* how many nearest neighbours we have found, for each vector in the block */
		
		/* Reset array contents */
		uint_t i;
		for (i = 0; i < block_size; i++) {
			if (i < block_size)
				current_block[i] = (index_t)((block_begin + i) + start_index);
			else
				current_block[i] = null_index;
		}
		memset(&neighbours, null_index, block_size * k * sizeof(index_t));
		memset(&neighbours_dist, 0, block_size * k * sizeof(double));
		memset(&score, 0, block_size * sizeof(double));
		memset(&found, 0, block_size * sizeof(uint_t));
		
		index_t vector1;
		for (vector1 = start_index; vector1 < num_vectors + start_index; vector1++) {
			uint_t block_index;
			for (block_index = 0; block_index < block_size; block_index++) {
				const index_t vector2 = current_block[block_index];
				
				if (vector1 != vector2 && vector2 >= start_index) {
					/*
					 * Calculate the square of the distance between the two
					 * vectors (indexed by "vector1" and "vector2")
					 */
					const double_t dist_squared = distance_squared(&data[(vector1-start_index) * vector_dims], &data[(vector2-start_index) * vector_dims], vector_dims);
					
					/*
					 * Insert the new (index, distance) pair into the neighbours
					 * array for the current vector.
					 */
					
					const double_t removed_distance = insert(neighbours[block_index], neighbours_dist[block_index], k, &found[block_index], vector1, dist_squared);
					
					/*
					 * Update the score (if the neighbours array was changed).
					 */
					if (removed_distance >= 0)
						score[block_index] = (double_t) ((score[block_index] * k - removed_distance + dist_squared) / k);
					
					/*
					 * If the score for this vector is less than the cutoff,
					 * then prune this vector from the block.
					 */
					if (found[block_index] >= k && score[block_index] < cutoff) {
						current_block[block_index] = null_index;
						score[block_index] = 0;
#ifdef STATS
						const UNUSED uint_t old_num_pruned = num_pruned;
						num_pruned++;
						ASSERT(num_pruned > old_num_pruned);
#endif /* #ifdef STATS */
					}
				}
			}
		}
		
		/* Keep track of the top "N" outliers. */
		best_outliers(outliers, outlier_scores, &outliers_found, N, current_block, score, block_size);
		
		/*
		 * Set "cutoff" to the score of the weakest outlier. There is no need to
		 * store an outlier in future iterations if its score is better than the
		 * cutoff.
		 */
		cutoff = outlier_scores[N-1];
	}
#else
	index_t vector1;
	for (vector1 = start_index; vector1 < num_vectors + start_index; vector1++) {
		index_t neighbours[k]; /* the "k" nearest neighbours for the current vector */
		double_t neighbours_dist[k]; /* the distance of the "k" nearest neighbours for the current vector */
		double_t score = 0; /* the average distance to the "k" neighbours */
		uint_t found = 0; /* how many nearest neighbours we have found */
		boolean removed = false; /* true if vector1 has been pruned */
		
		memset(neighbours, null_index, k * sizeof(index_t));
		memset(neighbours_dist, 0, k * sizeof(double_t));
		
		index_t vector2;
		for (vector2 = start_index; vector2 < num_vectors + start_index && !removed; vector2++) {
			if (vector1 != vector2) {
				/*
				 * Calculate the square of the distance between the two
				 * vectors (indexed by "vector1" and "vector2")
				 */
				const double_t dist_squared = distance_squared(&data[(vector1-start_index) * vector_dims], &data[(vector2-start_index) * vector_dims], vector_dims);
				
				/*
				 * Insert the new (index, distance) pair into the neighbours
				 * array for the current vector.
				 */
				const double_t removed_distance = insert(neighbours, neighbours_dist, k, &found, vector2, dist_squared);
				
				/* Update the score (if the neighbours array was changed). */
				if (removed_distance >= 0)
					score = (double_t) ((score * k - removed_distance + dist_squared) / k);
				
				/*
				 * If the score for this vector is less than the cutoff,
				 * then prune this vector from the block.
				 */
				if (found >= k && score < cutoff) {
					removed = true;
#ifdef STATS
					const UNUSED uint_t old_num_pruned = num_pruned;
					num_pruned++;
					ASSERT(num_pruned > old_num_pruned);
#endif /* #ifdef STATS */
					break;
				}
			}
		}
		
		if (!removed) {
			/* Keep track of the top "N" outliers. */
			best_outliers(outliers, outlier_scores, &outliers_found, N, &vector1, &score, 1);
			
			/*
			 * Set "cutoff" to the score of the weakest outlier. There is no need to
			 * store an outlier in future iterations if its score is better than the
			 * cutoff.
			 */
			cutoff = outlier_scores[N-1];
		}
	}
#endif /* #ifndef NO_BLOCKING */
}
\end{lstlisting}

%%%%%%%%%%%%%%%%%%%%%%%%%%%%%%%%%%%%%%%%%%%%%%%%%%%%%%%%%%%%%%%%%%%%%%%%%%%%%%%%
% Introduction
%%%%%%%%%%%%%%%%%%%%%%%%%%%%%%%%%%%%%%%%%%%%%%%%%%%%%%%%%%%%%%%%%%%%%%%%%%%%%%%%
\subsection{Introduction}
\label{profiling:blockSize:introduction}
To further investigate the effect of blocking within the
\command{TopN_Outlier_Pruning_Block} algorithm, it was decided to map a series
of key metrics against the block size, and to vary this block size over a wide
range for all of our test data sets. The key metrics of interest to these tests
were:
\begin{itemize}
    \item Total execution time of the \algm{anomaly detection using commute
        distance} algorithm.
    \item Total execution time of the \command{TopN_Outlier_Pruning_Block}
        function itself.
    \item The number of vectors that were pruned from the input data set
        during the execution of the algorithm.
    \item Number of calls that were made to the \command{distance_squared}
        function throughout the algorithm's execution.
\end{itemize}

%%%%%%%%%%%%%%%%%%%%%%%%%%%%%%%%%%%%%%%%%%%%%%%%%%%%%%%%%%%%%%%%%%%%%%%%%%%%%%%%
% Method
%%%%%%%%%%%%%%%%%%%%%%%%%%%%%%%%%%%%%%%%%%%%%%%%%%%%%%%%%%%%%%%%%%%%%%%%%%%%%%%%
\subsection{Method}
\label{profiling:blockSize:method}
In order to investigate the effect of blocking on the performance of the
\command{TopN_Outlier_Pruning_Block} algorithm, a suite of test scripts were
constructed to iterate through all of the test data sets, running the
\algm{anomaly detection using commute distance} algorithm over each data set
and for each block size to be tested.

%%%%%%%%%%%%%%%%%%%%%%%%%%%%%%%%%%%%%%%%%%%%%%%%%%%%%%%%%%%%%%%%%%%%%%%%%%%%%%%%
% Results
%%%%%%%%%%%%%%%%%%%%%%%%%%%%%%%%%%%%%%%%%%%%%%%%%%%%%%%%%%%%%%%%%%%%%%%%%%%%%%%%
\subsection{Results}
\label{profiling:blockSize:results}
This section details the results of profiling the initial \progLang{C}
implementation of the \command{TopN_Outlier_Pruning_Block} function across
numerous data sets, and with a varying block size. In addition, the execution
time of the algorithm was compared with a modified implementation which does not
use blocking in any way.

The data for the plots was obtained using the \software{MATLAB}
\command{profile} command. Descriptions of the data sets can be found in
\autoref{dataSets}.

Please note that the legend in \autoref{profiling:blockSize:legend:datasets} is
common to \autoref{profiling:blockSize:totalExecutionTime},
\autoref{profiling:blockSize:functionExecutionTime},
\autoref{profiling:blockSize:distanceCalls} and
\autoref{profiling:blockSize:vectorsPruned}. Additionally, the legend in
\autoref{profiling:blockSize:legend:block_sizes} is common to
\autoref{profiling:blockSize:totalRunTimeComplexity:linear},
\autoref{profiling:blockSize:totalRunTimeComplexity:logarithmic},
\autoref{profiling:blockSize:functionRunTimeComplexity:linear} and
\autoref{profiling:blockSize:functionRunTimeComplexity:logarithmic}.

\begin{figure}
    \centering
    \begin{tikzpicture}[gnuplot]
\gpsetlinetype{gp lt plot 2}
\gpsetlinewidth{1.00}
\node[gp node right] at (7.623,19.666) {ball1};
\gpcolor{gp lt color axes}
\gpsetlinetype{gp lt plot 0}
\draw[gp path] (7.807,19.666)--(8.723,19.666);
\node[gp node right] at (7.623,19.358) {ball1 (no blocking)};
\gpcolor{gp lt color axes}
\gpsetlinetype{gp lt axes}
\draw[gp path] (7.807,19.358)--(8.723,19.358);
\node[gp node right] at (7.623,19.050) {connect4};
\gpcolor{gp lt color 0}
\gpsetlinetype{gp lt plot 0}
\draw[gp path] (7.807,19.050)--(8.723,19.050);
\node[gp node right] at (7.623,18.742) {connect4 (no blocking)};
\gpcolor{gp lt color 0}
\gpsetlinetype{gp lt axes}
\draw[gp path] (7.807,18.742)--(8.723,18.742);
\node[gp node right] at (7.623,18.434) {letter-recognition};
\gpcolor{gp lt color 1}
\gpsetlinetype{gp lt plot 0}
\draw[gp path] (7.807,18.434)--(8.723,18.434);
\node[gp node right] at (7.623,18.126) {letter-recognition (no blocking)};
\gpcolor{gp lt color 1}
\gpsetlinetype{gp lt axes}
\draw[gp path] (7.807,18.126)--(8.723,18.126);
\node[gp node right] at (7.623,17.818) {magicgamma};
\gpcolor{gp lt color 2}
\gpsetlinetype{gp lt plot 0}
\draw[gp path] (7.807,17.818)--(8.723,17.818);
\node[gp node right] at (7.623,17.510) {magicgamma (no blocking)};
\gpcolor{gp lt color 2}
\gpsetlinetype{gp lt axes}
\draw[gp path] (7.807,17.510)--(8.723,17.510);
\node[gp node right] at (7.623,17.202) {mesh\_network};
\gpcolor{gp lt color 3}
\gpsetlinetype{gp lt plot 0}
\draw[gp path] (7.807,17.202)--(8.723,17.202);
\node[gp node right] at (7.623,16.894) {mesh\_network (no blocking)};
\gpcolor{gp lt color 3}
\gpsetlinetype{gp lt axes}
\draw[gp path] (7.807,16.894)--(8.723,16.894);
\node[gp node right] at (7.623,16.586) {musk};
\gpcolor{gp lt color 4}
\gpsetlinetype{gp lt plot 0}
\draw[gp path] (7.807,16.586)--(8.723,16.586);
\node[gp node right] at (7.623,16.278) {musk (no blocking)};
\gpcolor{gp lt color 4}
\gpsetlinetype{gp lt axes}
\draw[gp path] (7.807,16.278)--(8.723,16.278);
\node[gp node right] at (7.623,15.970) {pendigits};
\gpcolor{gp lt color 5}
\gpsetlinetype{gp lt plot 0}
\draw[gp path] (7.807,15.970)--(8.723,15.970);
\node[gp node right] at (7.623,15.662) {pendigits (no blocking)};
\gpcolor{gp lt color 5}
\gpsetlinetype{gp lt axes}
\draw[gp path] (7.807,15.662)--(8.723,15.662);
\node[gp node right] at (7.623,15.354) {runningex10k};
\gpcolor{gp lt color 6}
\gpsetlinetype{gp lt plot 0}
\draw[gp path] (7.807,15.354)--(8.723,15.354);
\node[gp node right] at (7.623,15.046) {runningex10k (no blocking)};
\gpcolor{gp lt color 6}
\gpsetlinetype{gp lt axes}
\draw[gp path] (7.807,15.046)--(8.723,15.046);
\node[gp node right] at (7.623,14.738) {runningex1k};
\gpcolor{gp lt color 7}
\gpsetlinetype{gp lt plot 0}
\draw[gp path] (7.807,14.738)--(8.723,14.738);
\node[gp node right] at (7.623,14.430) {runningex1k (no blocking)};
\gpcolor{gp lt color 7}
\gpsetlinetype{gp lt axes}
\draw[gp path] (7.807,14.430)--(8.723,14.430);
\node[gp node right] at (7.623,14.122) {runningex20k};
\gpcolor{gp lt color 0}
\gpsetlinetype{gp lt plot 0}
\draw[gp path] (7.807,14.122)--(8.723,14.122);
\node[gp node right] at (7.623,13.814) {runningex20k (no blocking)};
\gpcolor{gp lt color 0}
\gpsetlinetype{gp lt axes}
\draw[gp path] (7.807,13.814)--(8.723,13.814);
\node[gp node right] at (7.623,13.506) {runningex30k};
\gpcolor{gp lt color 1}
\gpsetlinetype{gp lt plot 0}
\draw[gp path] (7.807,13.506)--(8.723,13.506);
\node[gp node right] at (7.623,13.198) {runningex30k (no blocking)};
\gpcolor{gp lt color 1}
\gpsetlinetype{gp lt axes}
\draw[gp path] (7.807,13.198)--(8.723,13.198);
\node[gp node right] at (7.623,12.890) {runningex40k};
\gpcolor{gp lt color 2}
\gpsetlinetype{gp lt plot 0}
\draw[gp path] (7.807,12.890)--(8.723,12.890);
\node[gp node right] at (7.623,12.582) {runningex40k (no blocking)};
\gpcolor{gp lt color 2}
\gpsetlinetype{gp lt axes}
\draw[gp path] (7.807,12.582)--(8.723,12.582);
\node[gp node right] at (7.623,12.274) {runningex50k};
\gpcolor{gp lt color 3}
\gpsetlinetype{gp lt plot 0}
\draw[gp path] (7.807,12.274)--(8.723,12.274);
\node[gp node right] at (7.623,11.966) {runningex50k (no blocking)};
\gpcolor{gp lt color 3}
\gpsetlinetype{gp lt axes}
\draw[gp path] (7.807,11.966)--(8.723,11.966);
\node[gp node right] at (7.623,11.658) {segmentation};
\gpcolor{gp lt color 4}
\gpsetlinetype{gp lt plot 0}
\draw[gp path] (7.807,11.658)--(8.723,11.658);
\node[gp node right] at (7.623,11.350) {segmentation (no blocking)};
\gpcolor{gp lt color 4}
\gpsetlinetype{gp lt axes}
\draw[gp path] (7.807,11.350)--(8.723,11.350);
\node[gp node right] at (7.623,11.042) {spam};
\gpcolor{gp lt color 5}
\gpsetlinetype{gp lt plot 0}
\draw[gp path] (7.807,11.042)--(8.723,11.042);
\node[gp node right] at (7.623,10.734) {spam (no blocking)};
\gpcolor{gp lt color 5}
\gpsetlinetype{gp lt axes}
\draw[gp path] (7.807,10.734)--(8.723,10.734);
\node[gp node right] at (7.623,10.426) {spam\_train};
\gpcolor{gp lt color 6}
\gpsetlinetype{gp lt plot 0}
\draw[gp path] (7.807,10.426)--(8.723,10.426);
\node[gp node right] at (7.623,10.118) {spam\_train (no blocking)};
\gpcolor{gp lt color 6}
\gpsetlinetype{gp lt axes}
\draw[gp path] (7.807,10.118)--(8.723,10.118);
\node[gp node right] at (7.623,9.810) {testCD};
\gpcolor{gp lt color 7}
\gpsetlinetype{gp lt plot 0}
\draw[gp path] (7.807,9.810)--(8.723,9.810);
\node[gp node right] at (7.623,9.502) {testCD (no blocking)};
\gpcolor{gp lt color 7}
\gpsetlinetype{gp lt axes}
\draw[gp path] (7.807,9.502)--(8.723,9.502);
\node[gp node right] at (7.623,9.194) {testCDST};
\gpcolor{gp lt color 0}
\gpsetlinetype{gp lt plot 0}
\draw[gp path] (7.807,9.194)--(8.723,9.194);
\node[gp node right] at (7.623,8.886) {testCDST (no blocking)};
\gpcolor{gp lt color 0}
\gpsetlinetype{gp lt axes}
\draw[gp path] (7.807,8.886)--(8.723,8.886);
\node[gp node right] at (7.623,8.578) {testCDST2};
\gpcolor{gp lt color 1}
\gpsetlinetype{gp lt plot 0}
\draw[gp path] (7.807,8.578)--(8.723,8.578);
\node[gp node right] at (7.623,8.270) {testCDST2 (no blocking)};
\gpcolor{gp lt color 1}
\gpsetlinetype{gp lt axes}
\draw[gp path] (7.807,8.270)--(8.723,8.270);
\node[gp node right] at (7.623,7.962) {testCDST3};
\gpcolor{gp lt color 2}
\gpsetlinetype{gp lt plot 0}
\draw[gp path] (7.807,7.962)--(8.723,7.962);
\node[gp node right] at (7.623,7.654) {testCDST3 (no blocking)};
\gpcolor{gp lt color 2}
\gpsetlinetype{gp lt axes}
\draw[gp path] (7.807,7.654)--(8.723,7.654);
\node[gp node right] at (7.623,7.346) {testoutrank};
\gpcolor{gp lt color 3}
\gpsetlinetype{gp lt plot 0}
\draw[gp path] (7.807,7.346)--(8.723,7.346);
\node[gp node right] at (7.623,7.038) {testoutrank (no blocking)};
\gpcolor{gp lt color 3}
\gpsetlinetype{gp lt axes}
\draw[gp path] (7.807,7.038)--(8.723,7.038);
\end{tikzpicture}
    \caption[Block size profiling legend]{The data sets legend for all figures
        in \autoref{profiling:blockSize}}
    \label{profiling:blockSize:legend:datasets}
\end{figure}

\begin{figure}
    \centering
    \begin{tikzpicture}[gnuplot]
%% generated with GNUPLOT 4.4p3 (Lua 5.1.4; terminal rev. 97, script rev. 96a)
%% Mon 22 Oct 2012 22:30:13 EST
\gpcolor{\gprgb{502}{502}{502}}
\gpsetlinetype{gp lt plot 0}
\gpsetlinewidth{1.00}
\draw[gp path] (1.688,0.985)--(1.868,0.985);
\draw[gp path] (9.447,0.985)--(9.267,0.985);
\node[gp node right] at (1.504,0.985) { 0.75};
\draw[gp path] (1.688,2.714)--(1.868,2.714);
\draw[gp path] (9.447,2.714)--(9.267,2.714);
\node[gp node right] at (1.504,2.714) { 0.8};
\draw[gp path] (1.688,4.443)--(1.868,4.443);
\draw[gp path] (9.447,4.443)--(9.267,4.443);
\node[gp node right] at (1.504,4.443) { 0.85};
\draw[gp path] (1.688,6.173)--(1.868,6.173);
\draw[gp path] (9.447,6.173)--(9.267,6.173);
\node[gp node right] at (1.504,6.173) { 0.9};
\draw[gp path] (1.688,7.902)--(1.868,7.902);
\draw[gp path] (9.447,7.902)--(9.267,7.902);
\node[gp node right] at (1.504,7.902) { 0.95};
\draw[gp path] (1.688,9.631)--(1.868,9.631);
\draw[gp path] (9.447,9.631)--(9.267,9.631);
\node[gp node right] at (1.504,9.631) { 1};
\draw[gp path] (1.688,0.985)--(1.688,1.165);
\draw[gp path] (1.688,9.631)--(1.688,9.451);
\node[gp node center] at (1.688,0.677) {$10^{0}$};
\draw[gp path] (2.077,0.985)--(2.077,1.075);
\draw[gp path] (2.077,9.631)--(2.077,9.541);
\draw[gp path] (2.592,0.985)--(2.592,1.075);
\draw[gp path] (2.592,9.631)--(2.592,9.541);
\draw[gp path] (2.856,0.985)--(2.856,1.075);
\draw[gp path] (2.856,9.631)--(2.856,9.541);
\draw[gp path] (2.981,0.985)--(2.981,1.165);
\draw[gp path] (2.981,9.631)--(2.981,9.451);
\node[gp node center] at (2.981,0.677) {$10^{1}$};
\draw[gp path] (3.370,0.985)--(3.370,1.075);
\draw[gp path] (3.370,9.631)--(3.370,9.541);
\draw[gp path] (3.885,0.985)--(3.885,1.075);
\draw[gp path] (3.885,9.631)--(3.885,9.541);
\draw[gp path] (4.149,0.985)--(4.149,1.075);
\draw[gp path] (4.149,9.631)--(4.149,9.541);
\draw[gp path] (4.274,0.985)--(4.274,1.165);
\draw[gp path] (4.274,9.631)--(4.274,9.451);
\node[gp node center] at (4.274,0.677) {$10^{2}$};
\draw[gp path] (4.664,0.985)--(4.664,1.075);
\draw[gp path] (4.664,9.631)--(4.664,9.541);
\draw[gp path] (5.178,0.985)--(5.178,1.075);
\draw[gp path] (5.178,9.631)--(5.178,9.541);
\draw[gp path] (5.442,0.985)--(5.442,1.075);
\draw[gp path] (5.442,9.631)--(5.442,9.541);
\draw[gp path] (5.568,0.985)--(5.568,1.165);
\draw[gp path] (5.568,9.631)--(5.568,9.451);
\node[gp node center] at (5.568,0.677) {$10^{3}$};
\draw[gp path] (5.957,0.985)--(5.957,1.075);
\draw[gp path] (5.957,9.631)--(5.957,9.541);
\draw[gp path] (6.471,0.985)--(6.471,1.075);
\draw[gp path] (6.471,9.631)--(6.471,9.541);
\draw[gp path] (6.735,0.985)--(6.735,1.075);
\draw[gp path] (6.735,9.631)--(6.735,9.541);
\draw[gp path] (6.861,0.985)--(6.861,1.165);
\draw[gp path] (6.861,9.631)--(6.861,9.451);
\node[gp node center] at (6.861,0.677) {$10^{4}$};
\draw[gp path] (7.250,0.985)--(7.250,1.075);
\draw[gp path] (7.250,9.631)--(7.250,9.541);
\draw[gp path] (7.765,0.985)--(7.765,1.075);
\draw[gp path] (7.765,9.631)--(7.765,9.541);
\draw[gp path] (8.029,0.985)--(8.029,1.075);
\draw[gp path] (8.029,9.631)--(8.029,9.541);
\draw[gp path] (8.154,0.985)--(8.154,1.165);
\draw[gp path] (8.154,9.631)--(8.154,9.451);
\node[gp node center] at (8.154,0.677) {$10^{5}$};
\draw[gp path] (8.543,0.985)--(8.543,1.075);
\draw[gp path] (8.543,9.631)--(8.543,9.541);
\draw[gp path] (9.058,0.985)--(9.058,1.075);
\draw[gp path] (9.058,9.631)--(9.058,9.541);
\draw[gp path] (9.322,0.985)--(9.322,1.075);
\draw[gp path] (9.322,9.631)--(9.322,9.541);
\draw[gp path] (9.447,0.985)--(9.447,1.165);
\draw[gp path] (9.447,9.631)--(9.447,9.451);
\node[gp node center] at (9.447,0.677) {$10^{6}$};
\draw[gp path] (1.688,9.631)--(1.688,0.985)--(9.447,0.985);
\gpcolor{gp lt color border}
\node[gp node center,rotate=-270] at (0.246,5.308) {Total execution time (normalised)};
\node[gp node center] at (5.567,0.215) {Block size};
\gpcolor{gp lt color axes}
\draw[gp path] (1.688,9.125)--(2.981,9.079)--(3.209,9.202)--(3.598,9.213)--(3.760,9.251)%
  --(3.826,9.227)--(3.987,9.276)--(4.215,9.336)--(4.274,9.346)--(5.568,9.511)--(6.861,9.519)%
  --(8.154,9.620)--(9.447,9.631);
\gpsetpointsize{4.00}
\gppoint{gp mark 1}{(1.688,9.125)}
\gppoint{gp mark 1}{(2.981,9.079)}
\gppoint{gp mark 1}{(3.209,9.202)}
\gppoint{gp mark 1}{(3.598,9.213)}
\gppoint{gp mark 1}{(3.760,9.251)}
\gppoint{gp mark 1}{(3.826,9.227)}
\gppoint{gp mark 1}{(3.987,9.276)}
\gppoint{gp mark 1}{(4.215,9.336)}
\gppoint{gp mark 1}{(4.274,9.346)}
\gppoint{gp mark 1}{(5.568,9.511)}
\gppoint{gp mark 1}{(6.861,9.519)}
\gppoint{gp mark 1}{(8.154,9.620)}
\gppoint{gp mark 1}{(9.447,9.631)}
\gpsetlinetype{gp lt axes}
\draw[gp path] (1.688,9.132)--(1.766,9.132)--(1.845,9.132)--(1.923,9.132)--(2.001,9.132)%
  --(2.080,9.132)--(2.158,9.132)--(2.237,9.132)--(2.315,9.132)--(2.393,9.132)--(2.472,9.132)%
  --(2.550,9.132)--(2.628,9.132)--(2.707,9.132)--(2.785,9.132)--(2.864,9.132)--(2.942,9.132)%
  --(3.020,9.132)--(3.099,9.132)--(3.177,9.132)--(3.255,9.132)--(3.334,9.132)--(3.412,9.132)%
  --(3.491,9.132)--(3.569,9.132)--(3.647,9.132)--(3.726,9.132)--(3.804,9.132)--(3.882,9.132)%
  --(3.961,9.132)--(4.039,9.132)--(4.118,9.132)--(4.196,9.132)--(4.274,9.132)--(4.353,9.132)%
  --(4.431,9.132)--(4.509,9.132)--(4.588,9.132)--(4.666,9.132)--(4.745,9.132)--(4.823,9.132)%
  --(4.901,9.132)--(4.980,9.132)--(5.058,9.132)--(5.136,9.132)--(5.215,9.132)--(5.293,9.132)%
  --(5.372,9.132)--(5.450,9.132)--(5.528,9.132)--(5.607,9.132)--(5.685,9.132)--(5.763,9.132)%
  --(5.842,9.132)--(5.920,9.132)--(5.999,9.132)--(6.077,9.132)--(6.155,9.132)--(6.234,9.132)%
  --(6.312,9.132)--(6.390,9.132)--(6.469,9.132)--(6.547,9.132)--(6.626,9.132)--(6.704,9.132)%
  --(6.782,9.132)--(6.861,9.132)--(6.939,9.132)--(7.017,9.132)--(7.096,9.132)--(7.174,9.132)%
  --(7.253,9.132)--(7.331,9.132)--(7.409,9.132)--(7.488,9.132)--(7.566,9.132)--(7.644,9.132)%
  --(7.723,9.132)--(7.801,9.132)--(7.880,9.132)--(7.958,9.132)--(8.036,9.132)--(8.115,9.132)%
  --(8.193,9.132)--(8.271,9.132)--(8.350,9.132)--(8.428,9.132)--(8.507,9.132)--(8.585,9.132)%
  --(8.663,9.132)--(8.742,9.132)--(8.820,9.132)--(8.898,9.132)--(8.977,9.132)--(9.055,9.132)%
  --(9.134,9.132)--(9.212,9.132)--(9.290,9.132)--(9.369,9.132)--(9.447,9.132);
\gpcolor{gp lt color 0}
\gpsetlinetype{gp lt plot 0}
\draw[gp path] (1.688,5.494)--(2.981,4.573)--(3.209,4.394)--(3.598,5.297)--(3.760,5.328)%
  --(3.826,5.450)--(3.987,7.756)--(4.215,4.483)--(4.274,7.747)--(5.568,4.928)--(6.861,9.631);
\gppoint{gp mark 2}{(1.688,5.494)}
\gppoint{gp mark 2}{(2.981,4.573)}
\gppoint{gp mark 2}{(3.209,4.394)}
\gppoint{gp mark 2}{(3.598,5.297)}
\gppoint{gp mark 2}{(3.760,5.328)}
\gppoint{gp mark 2}{(3.826,5.450)}
\gppoint{gp mark 2}{(3.987,7.756)}
\gppoint{gp mark 2}{(4.215,4.483)}
\gppoint{gp mark 2}{(4.274,7.747)}
\gppoint{gp mark 2}{(5.568,4.928)}
\gppoint{gp mark 2}{(6.861,9.631)}
\gpsetlinetype{gp lt axes}
\draw[gp path] (1.688,6.238)--(1.766,6.238)--(1.845,6.238)--(1.923,6.238)--(2.001,6.238)%
  --(2.080,6.238)--(2.158,6.238)--(2.237,6.238)--(2.315,6.238)--(2.393,6.238)--(2.472,6.238)%
  --(2.550,6.238)--(2.628,6.238)--(2.707,6.238)--(2.785,6.238)--(2.864,6.238)--(2.942,6.238)%
  --(3.020,6.238)--(3.099,6.238)--(3.177,6.238)--(3.255,6.238)--(3.334,6.238)--(3.412,6.238)%
  --(3.491,6.238)--(3.569,6.238)--(3.647,6.238)--(3.726,6.238)--(3.804,6.238)--(3.882,6.238)%
  --(3.961,6.238)--(4.039,6.238)--(4.118,6.238)--(4.196,6.238)--(4.274,6.238)--(4.353,6.238)%
  --(4.431,6.238)--(4.509,6.238)--(4.588,6.238)--(4.666,6.238)--(4.745,6.238)--(4.823,6.238)%
  --(4.901,6.238)--(4.980,6.238)--(5.058,6.238)--(5.136,6.238)--(5.215,6.238)--(5.293,6.238)%
  --(5.372,6.238)--(5.450,6.238)--(5.528,6.238)--(5.607,6.238)--(5.685,6.238)--(5.763,6.238)%
  --(5.842,6.238)--(5.920,6.238)--(5.999,6.238)--(6.077,6.238)--(6.155,6.238)--(6.234,6.238)%
  --(6.312,6.238)--(6.390,6.238)--(6.469,6.238)--(6.547,6.238)--(6.626,6.238)--(6.704,6.238)%
  --(6.782,6.238)--(6.861,6.238)--(6.939,6.238)--(7.017,6.238)--(7.096,6.238)--(7.174,6.238)%
  --(7.253,6.238)--(7.331,6.238)--(7.409,6.238)--(7.488,6.238)--(7.566,6.238)--(7.644,6.238)%
  --(7.723,6.238)--(7.801,6.238)--(7.880,6.238)--(7.958,6.238)--(8.036,6.238)--(8.115,6.238)%
  --(8.193,6.238)--(8.271,6.238)--(8.350,6.238)--(8.428,6.238)--(8.507,6.238)--(8.585,6.238)%
  --(8.663,6.238)--(8.742,6.238)--(8.820,6.238)--(8.898,6.238)--(8.977,6.238)--(9.055,6.238)%
  --(9.134,6.238)--(9.212,6.238)--(9.290,6.238)--(9.369,6.238)--(9.447,6.238);
\gpcolor{gp lt color 1}
\gpsetlinetype{gp lt plot 0}
\draw[gp path] (1.688,1.948)--(2.981,2.059)--(3.209,2.933)--(3.598,3.015)--(3.760,2.134)%
  --(3.826,2.206)--(3.987,2.465)--(4.215,2.646)--(4.274,2.422)--(5.568,2.695)--(6.861,5.563)%
  --(8.154,9.205)--(9.447,9.631);
\gppoint{gp mark 3}{(1.688,1.948)}
\gppoint{gp mark 3}{(2.981,2.059)}
\gppoint{gp mark 3}{(3.209,2.933)}
\gppoint{gp mark 3}{(3.598,3.015)}
\gppoint{gp mark 3}{(3.760,2.134)}
\gppoint{gp mark 3}{(3.826,2.206)}
\gppoint{gp mark 3}{(3.987,2.465)}
\gppoint{gp mark 3}{(4.215,2.646)}
\gppoint{gp mark 3}{(4.274,2.422)}
\gppoint{gp mark 3}{(5.568,2.695)}
\gppoint{gp mark 3}{(6.861,5.563)}
\gppoint{gp mark 3}{(8.154,9.205)}
\gppoint{gp mark 3}{(9.447,9.631)}
\gpsetlinetype{gp lt axes}
\draw[gp path] (1.688,1.915)--(1.766,1.915)--(1.845,1.915)--(1.923,1.915)--(2.001,1.915)%
  --(2.080,1.915)--(2.158,1.915)--(2.237,1.915)--(2.315,1.915)--(2.393,1.915)--(2.472,1.915)%
  --(2.550,1.915)--(2.628,1.915)--(2.707,1.915)--(2.785,1.915)--(2.864,1.915)--(2.942,1.915)%
  --(3.020,1.915)--(3.099,1.915)--(3.177,1.915)--(3.255,1.915)--(3.334,1.915)--(3.412,1.915)%
  --(3.491,1.915)--(3.569,1.915)--(3.647,1.915)--(3.726,1.915)--(3.804,1.915)--(3.882,1.915)%
  --(3.961,1.915)--(4.039,1.915)--(4.118,1.915)--(4.196,1.915)--(4.274,1.915)--(4.353,1.915)%
  --(4.431,1.915)--(4.509,1.915)--(4.588,1.915)--(4.666,1.915)--(4.745,1.915)--(4.823,1.915)%
  --(4.901,1.915)--(4.980,1.915)--(5.058,1.915)--(5.136,1.915)--(5.215,1.915)--(5.293,1.915)%
  --(5.372,1.915)--(5.450,1.915)--(5.528,1.915)--(5.607,1.915)--(5.685,1.915)--(5.763,1.915)%
  --(5.842,1.915)--(5.920,1.915)--(5.999,1.915)--(6.077,1.915)--(6.155,1.915)--(6.234,1.915)%
  --(6.312,1.915)--(6.390,1.915)--(6.469,1.915)--(6.547,1.915)--(6.626,1.915)--(6.704,1.915)%
  --(6.782,1.915)--(6.861,1.915)--(6.939,1.915)--(7.017,1.915)--(7.096,1.915)--(7.174,1.915)%
  --(7.253,1.915)--(7.331,1.915)--(7.409,1.915)--(7.488,1.915)--(7.566,1.915)--(7.644,1.915)%
  --(7.723,1.915)--(7.801,1.915)--(7.880,1.915)--(7.958,1.915)--(8.036,1.915)--(8.115,1.915)%
  --(8.193,1.915)--(8.271,1.915)--(8.350,1.915)--(8.428,1.915)--(8.507,1.915)--(8.585,1.915)%
  --(8.663,1.915)--(8.742,1.915)--(8.820,1.915)--(8.898,1.915)--(8.977,1.915)--(9.055,1.915)%
  --(9.134,1.915)--(9.212,1.915)--(9.290,1.915)--(9.369,1.915)--(9.447,1.915);
\gpcolor{gp lt color 2}
\gpsetlinetype{gp lt plot 0}
\draw[gp path] (1.688,3.202)--(2.981,3.253)--(3.209,2.560)--(3.598,2.979)--(3.760,3.192)%
  --(3.826,3.504)--(3.987,3.108)--(4.215,3.024)--(4.274,2.810)--(5.568,3.264)--(6.861,6.432)%
  --(8.154,9.609)--(9.447,9.631);
\gppoint{gp mark 4}{(1.688,3.202)}
\gppoint{gp mark 4}{(2.981,3.253)}
\gppoint{gp mark 4}{(3.209,2.560)}
\gppoint{gp mark 4}{(3.598,2.979)}
\gppoint{gp mark 4}{(3.760,3.192)}
\gppoint{gp mark 4}{(3.826,3.504)}
\gppoint{gp mark 4}{(3.987,3.108)}
\gppoint{gp mark 4}{(4.215,3.024)}
\gppoint{gp mark 4}{(4.274,2.810)}
\gppoint{gp mark 4}{(5.568,3.264)}
\gppoint{gp mark 4}{(6.861,6.432)}
\gppoint{gp mark 4}{(8.154,9.609)}
\gppoint{gp mark 4}{(9.447,9.631)}
\gpsetlinetype{gp lt axes}
\draw[gp path] (1.688,2.321)--(1.766,2.321)--(1.845,2.321)--(1.923,2.321)--(2.001,2.321)%
  --(2.080,2.321)--(2.158,2.321)--(2.237,2.321)--(2.315,2.321)--(2.393,2.321)--(2.472,2.321)%
  --(2.550,2.321)--(2.628,2.321)--(2.707,2.321)--(2.785,2.321)--(2.864,2.321)--(2.942,2.321)%
  --(3.020,2.321)--(3.099,2.321)--(3.177,2.321)--(3.255,2.321)--(3.334,2.321)--(3.412,2.321)%
  --(3.491,2.321)--(3.569,2.321)--(3.647,2.321)--(3.726,2.321)--(3.804,2.321)--(3.882,2.321)%
  --(3.961,2.321)--(4.039,2.321)--(4.118,2.321)--(4.196,2.321)--(4.274,2.321)--(4.353,2.321)%
  --(4.431,2.321)--(4.509,2.321)--(4.588,2.321)--(4.666,2.321)--(4.745,2.321)--(4.823,2.321)%
  --(4.901,2.321)--(4.980,2.321)--(5.058,2.321)--(5.136,2.321)--(5.215,2.321)--(5.293,2.321)%
  --(5.372,2.321)--(5.450,2.321)--(5.528,2.321)--(5.607,2.321)--(5.685,2.321)--(5.763,2.321)%
  --(5.842,2.321)--(5.920,2.321)--(5.999,2.321)--(6.077,2.321)--(6.155,2.321)--(6.234,2.321)%
  --(6.312,2.321)--(6.390,2.321)--(6.469,2.321)--(6.547,2.321)--(6.626,2.321)--(6.704,2.321)%
  --(6.782,2.321)--(6.861,2.321)--(6.939,2.321)--(7.017,2.321)--(7.096,2.321)--(7.174,2.321)%
  --(7.253,2.321)--(7.331,2.321)--(7.409,2.321)--(7.488,2.321)--(7.566,2.321)--(7.644,2.321)%
  --(7.723,2.321)--(7.801,2.321)--(7.880,2.321)--(7.958,2.321)--(8.036,2.321)--(8.115,2.321)%
  --(8.193,2.321)--(8.271,2.321)--(8.350,2.321)--(8.428,2.321)--(8.507,2.321)--(8.585,2.321)%
  --(8.663,2.321)--(8.742,2.321)--(8.820,2.321)--(8.898,2.321)--(8.977,2.321)--(9.055,2.321)%
  --(9.134,2.321)--(9.212,2.321)--(9.290,2.321)--(9.369,2.321)--(9.447,2.321);
\gpcolor{gp lt color 3}
\gpsetlinetype{gp lt plot 0}
\draw[gp path] (1.688,9.212)--(2.981,9.296)--(3.209,9.455)--(3.598,9.302)--(3.760,9.293)%
  --(3.826,9.402)--(3.987,9.437)--(4.215,9.297)--(4.274,9.517)--(5.568,9.631)--(6.861,9.240)%
  --(8.154,9.330)--(9.447,9.603);
\gppoint{gp mark 5}{(1.688,9.212)}
\gppoint{gp mark 5}{(2.981,9.296)}
\gppoint{gp mark 5}{(3.209,9.455)}
\gppoint{gp mark 5}{(3.598,9.302)}
\gppoint{gp mark 5}{(3.760,9.293)}
\gppoint{gp mark 5}{(3.826,9.402)}
\gppoint{gp mark 5}{(3.987,9.437)}
\gppoint{gp mark 5}{(4.215,9.297)}
\gppoint{gp mark 5}{(4.274,9.517)}
\gppoint{gp mark 5}{(5.568,9.631)}
\gppoint{gp mark 5}{(6.861,9.240)}
\gppoint{gp mark 5}{(8.154,9.330)}
\gppoint{gp mark 5}{(9.447,9.603)}
\gpsetlinetype{gp lt axes}
\draw[gp path] (1.688,9.055)--(1.766,9.055)--(1.845,9.055)--(1.923,9.055)--(2.001,9.055)%
  --(2.080,9.055)--(2.158,9.055)--(2.237,9.055)--(2.315,9.055)--(2.393,9.055)--(2.472,9.055)%
  --(2.550,9.055)--(2.628,9.055)--(2.707,9.055)--(2.785,9.055)--(2.864,9.055)--(2.942,9.055)%
  --(3.020,9.055)--(3.099,9.055)--(3.177,9.055)--(3.255,9.055)--(3.334,9.055)--(3.412,9.055)%
  --(3.491,9.055)--(3.569,9.055)--(3.647,9.055)--(3.726,9.055)--(3.804,9.055)--(3.882,9.055)%
  --(3.961,9.055)--(4.039,9.055)--(4.118,9.055)--(4.196,9.055)--(4.274,9.055)--(4.353,9.055)%
  --(4.431,9.055)--(4.509,9.055)--(4.588,9.055)--(4.666,9.055)--(4.745,9.055)--(4.823,9.055)%
  --(4.901,9.055)--(4.980,9.055)--(5.058,9.055)--(5.136,9.055)--(5.215,9.055)--(5.293,9.055)%
  --(5.372,9.055)--(5.450,9.055)--(5.528,9.055)--(5.607,9.055)--(5.685,9.055)--(5.763,9.055)%
  --(5.842,9.055)--(5.920,9.055)--(5.999,9.055)--(6.077,9.055)--(6.155,9.055)--(6.234,9.055)%
  --(6.312,9.055)--(6.390,9.055)--(6.469,9.055)--(6.547,9.055)--(6.626,9.055)--(6.704,9.055)%
  --(6.782,9.055)--(6.861,9.055)--(6.939,9.055)--(7.017,9.055)--(7.096,9.055)--(7.174,9.055)%
  --(7.253,9.055)--(7.331,9.055)--(7.409,9.055)--(7.488,9.055)--(7.566,9.055)--(7.644,9.055)%
  --(7.723,9.055)--(7.801,9.055)--(7.880,9.055)--(7.958,9.055)--(8.036,9.055)--(8.115,9.055)%
  --(8.193,9.055)--(8.271,9.055)--(8.350,9.055)--(8.428,9.055)--(8.507,9.055)--(8.585,9.055)%
  --(8.663,9.055)--(8.742,9.055)--(8.820,9.055)--(8.898,9.055)--(8.977,9.055)--(9.055,9.055)%
  --(9.134,9.055)--(9.212,9.055)--(9.290,9.055)--(9.369,9.055)--(9.447,9.055);
\gpcolor{gp lt color 4}
\gpsetlinetype{gp lt plot 0}
\draw[gp path] (1.688,8.025)--(2.981,7.219)--(3.209,7.146)--(3.598,7.024)--(3.760,7.533)%
  --(3.826,7.654)--(3.987,7.347)--(4.215,7.503)--(4.274,7.590)--(5.568,8.916)--(6.861,9.628)%
  --(8.154,9.631)--(9.447,9.522);
\gppoint{gp mark 6}{(1.688,8.025)}
\gppoint{gp mark 6}{(2.981,7.219)}
\gppoint{gp mark 6}{(3.209,7.146)}
\gppoint{gp mark 6}{(3.598,7.024)}
\gppoint{gp mark 6}{(3.760,7.533)}
\gppoint{gp mark 6}{(3.826,7.654)}
\gppoint{gp mark 6}{(3.987,7.347)}
\gppoint{gp mark 6}{(4.215,7.503)}
\gppoint{gp mark 6}{(4.274,7.590)}
\gppoint{gp mark 6}{(5.568,8.916)}
\gppoint{gp mark 6}{(6.861,9.628)}
\gppoint{gp mark 6}{(8.154,9.631)}
\gppoint{gp mark 6}{(9.447,9.522)}
\gpsetlinetype{gp lt axes}
\draw[gp path] (1.688,7.178)--(1.766,7.178)--(1.845,7.178)--(1.923,7.178)--(2.001,7.178)%
  --(2.080,7.178)--(2.158,7.178)--(2.237,7.178)--(2.315,7.178)--(2.393,7.178)--(2.472,7.178)%
  --(2.550,7.178)--(2.628,7.178)--(2.707,7.178)--(2.785,7.178)--(2.864,7.178)--(2.942,7.178)%
  --(3.020,7.178)--(3.099,7.178)--(3.177,7.178)--(3.255,7.178)--(3.334,7.178)--(3.412,7.178)%
  --(3.491,7.178)--(3.569,7.178)--(3.647,7.178)--(3.726,7.178)--(3.804,7.178)--(3.882,7.178)%
  --(3.961,7.178)--(4.039,7.178)--(4.118,7.178)--(4.196,7.178)--(4.274,7.178)--(4.353,7.178)%
  --(4.431,7.178)--(4.509,7.178)--(4.588,7.178)--(4.666,7.178)--(4.745,7.178)--(4.823,7.178)%
  --(4.901,7.178)--(4.980,7.178)--(5.058,7.178)--(5.136,7.178)--(5.215,7.178)--(5.293,7.178)%
  --(5.372,7.178)--(5.450,7.178)--(5.528,7.178)--(5.607,7.178)--(5.685,7.178)--(5.763,7.178)%
  --(5.842,7.178)--(5.920,7.178)--(5.999,7.178)--(6.077,7.178)--(6.155,7.178)--(6.234,7.178)%
  --(6.312,7.178)--(6.390,7.178)--(6.469,7.178)--(6.547,7.178)--(6.626,7.178)--(6.704,7.178)%
  --(6.782,7.178)--(6.861,7.178)--(6.939,7.178)--(7.017,7.178)--(7.096,7.178)--(7.174,7.178)%
  --(7.253,7.178)--(7.331,7.178)--(7.409,7.178)--(7.488,7.178)--(7.566,7.178)--(7.644,7.178)%
  --(7.723,7.178)--(7.801,7.178)--(7.880,7.178)--(7.958,7.178)--(8.036,7.178)--(8.115,7.178)%
  --(8.193,7.178)--(8.271,7.178)--(8.350,7.178)--(8.428,7.178)--(8.507,7.178)--(8.585,7.178)%
  --(8.663,7.178)--(8.742,7.178)--(8.820,7.178)--(8.898,7.178)--(8.977,7.178)--(9.055,7.178)%
  --(9.134,7.178)--(9.212,7.178)--(9.290,7.178)--(9.369,7.178)--(9.447,7.178);
\gpcolor{gp lt color 5}
\gpsetlinetype{gp lt plot 0}
\draw[gp path] (1.688,3.932)--(2.981,4.600)--(3.209,4.339)--(3.598,4.389)--(3.760,4.292)%
  --(3.826,4.869)--(3.987,4.761)--(4.215,4.842)--(4.274,4.241)--(5.568,4.513)--(6.861,8.720)%
  --(8.154,9.094)--(9.447,9.631);
\gppoint{gp mark 7}{(1.688,3.932)}
\gppoint{gp mark 7}{(2.981,4.600)}
\gppoint{gp mark 7}{(3.209,4.339)}
\gppoint{gp mark 7}{(3.598,4.389)}
\gppoint{gp mark 7}{(3.760,4.292)}
\gppoint{gp mark 7}{(3.826,4.869)}
\gppoint{gp mark 7}{(3.987,4.761)}
\gppoint{gp mark 7}{(4.215,4.842)}
\gppoint{gp mark 7}{(4.274,4.241)}
\gppoint{gp mark 7}{(5.568,4.513)}
\gppoint{gp mark 7}{(6.861,8.720)}
\gppoint{gp mark 7}{(8.154,9.094)}
\gppoint{gp mark 7}{(9.447,9.631)}
\gpsetlinetype{gp lt axes}
\draw[gp path] (1.688,3.806)--(1.766,3.806)--(1.845,3.806)--(1.923,3.806)--(2.001,3.806)%
  --(2.080,3.806)--(2.158,3.806)--(2.237,3.806)--(2.315,3.806)--(2.393,3.806)--(2.472,3.806)%
  --(2.550,3.806)--(2.628,3.806)--(2.707,3.806)--(2.785,3.806)--(2.864,3.806)--(2.942,3.806)%
  --(3.020,3.806)--(3.099,3.806)--(3.177,3.806)--(3.255,3.806)--(3.334,3.806)--(3.412,3.806)%
  --(3.491,3.806)--(3.569,3.806)--(3.647,3.806)--(3.726,3.806)--(3.804,3.806)--(3.882,3.806)%
  --(3.961,3.806)--(4.039,3.806)--(4.118,3.806)--(4.196,3.806)--(4.274,3.806)--(4.353,3.806)%
  --(4.431,3.806)--(4.509,3.806)--(4.588,3.806)--(4.666,3.806)--(4.745,3.806)--(4.823,3.806)%
  --(4.901,3.806)--(4.980,3.806)--(5.058,3.806)--(5.136,3.806)--(5.215,3.806)--(5.293,3.806)%
  --(5.372,3.806)--(5.450,3.806)--(5.528,3.806)--(5.607,3.806)--(5.685,3.806)--(5.763,3.806)%
  --(5.842,3.806)--(5.920,3.806)--(5.999,3.806)--(6.077,3.806)--(6.155,3.806)--(6.234,3.806)%
  --(6.312,3.806)--(6.390,3.806)--(6.469,3.806)--(6.547,3.806)--(6.626,3.806)--(6.704,3.806)%
  --(6.782,3.806)--(6.861,3.806)--(6.939,3.806)--(7.017,3.806)--(7.096,3.806)--(7.174,3.806)%
  --(7.253,3.806)--(7.331,3.806)--(7.409,3.806)--(7.488,3.806)--(7.566,3.806)--(7.644,3.806)%
  --(7.723,3.806)--(7.801,3.806)--(7.880,3.806)--(7.958,3.806)--(8.036,3.806)--(8.115,3.806)%
  --(8.193,3.806)--(8.271,3.806)--(8.350,3.806)--(8.428,3.806)--(8.507,3.806)--(8.585,3.806)%
  --(8.663,3.806)--(8.742,3.806)--(8.820,3.806)--(8.898,3.806)--(8.977,3.806)--(9.055,3.806)%
  --(9.134,3.806)--(9.212,3.806)--(9.290,3.806)--(9.369,3.806)--(9.447,3.806);
\gpcolor{gp lt color 6}
\gpsetlinetype{gp lt plot 0}
\draw[gp path] (1.688,6.507)--(2.981,6.280)--(3.209,6.291)--(3.598,6.268)--(3.760,6.275)%
  --(3.826,6.373)--(3.987,6.384)--(4.215,6.430)--(4.274,6.449)--(5.568,6.867)--(6.861,9.551)%
  --(8.154,9.532)--(9.447,9.631);
\gppoint{gp mark 8}{(1.688,6.507)}
\gppoint{gp mark 8}{(2.981,6.280)}
\gppoint{gp mark 8}{(3.209,6.291)}
\gppoint{gp mark 8}{(3.598,6.268)}
\gppoint{gp mark 8}{(3.760,6.275)}
\gppoint{gp mark 8}{(3.826,6.373)}
\gppoint{gp mark 8}{(3.987,6.384)}
\gppoint{gp mark 8}{(4.215,6.430)}
\gppoint{gp mark 8}{(4.274,6.449)}
\gppoint{gp mark 8}{(5.568,6.867)}
\gppoint{gp mark 8}{(6.861,9.551)}
\gppoint{gp mark 8}{(8.154,9.532)}
\gppoint{gp mark 8}{(9.447,9.631)}
\gpsetlinetype{gp lt axes}
\draw[gp path] (1.688,6.256)--(1.766,6.256)--(1.845,6.256)--(1.923,6.256)--(2.001,6.256)%
  --(2.080,6.256)--(2.158,6.256)--(2.237,6.256)--(2.315,6.256)--(2.393,6.256)--(2.472,6.256)%
  --(2.550,6.256)--(2.628,6.256)--(2.707,6.256)--(2.785,6.256)--(2.864,6.256)--(2.942,6.256)%
  --(3.020,6.256)--(3.099,6.256)--(3.177,6.256)--(3.255,6.256)--(3.334,6.256)--(3.412,6.256)%
  --(3.491,6.256)--(3.569,6.256)--(3.647,6.256)--(3.726,6.256)--(3.804,6.256)--(3.882,6.256)%
  --(3.961,6.256)--(4.039,6.256)--(4.118,6.256)--(4.196,6.256)--(4.274,6.256)--(4.353,6.256)%
  --(4.431,6.256)--(4.509,6.256)--(4.588,6.256)--(4.666,6.256)--(4.745,6.256)--(4.823,6.256)%
  --(4.901,6.256)--(4.980,6.256)--(5.058,6.256)--(5.136,6.256)--(5.215,6.256)--(5.293,6.256)%
  --(5.372,6.256)--(5.450,6.256)--(5.528,6.256)--(5.607,6.256)--(5.685,6.256)--(5.763,6.256)%
  --(5.842,6.256)--(5.920,6.256)--(5.999,6.256)--(6.077,6.256)--(6.155,6.256)--(6.234,6.256)%
  --(6.312,6.256)--(6.390,6.256)--(6.469,6.256)--(6.547,6.256)--(6.626,6.256)--(6.704,6.256)%
  --(6.782,6.256)--(6.861,6.256)--(6.939,6.256)--(7.017,6.256)--(7.096,6.256)--(7.174,6.256)%
  --(7.253,6.256)--(7.331,6.256)--(7.409,6.256)--(7.488,6.256)--(7.566,6.256)--(7.644,6.256)%
  --(7.723,6.256)--(7.801,6.256)--(7.880,6.256)--(7.958,6.256)--(8.036,6.256)--(8.115,6.256)%
  --(8.193,6.256)--(8.271,6.256)--(8.350,6.256)--(8.428,6.256)--(8.507,6.256)--(8.585,6.256)%
  --(8.663,6.256)--(8.742,6.256)--(8.820,6.256)--(8.898,6.256)--(8.977,6.256)--(9.055,6.256)%
  --(9.134,6.256)--(9.212,6.256)--(9.290,6.256)--(9.369,6.256)--(9.447,6.256);
\gpcolor{gp lt color 7}
\gpsetlinetype{gp lt plot 0}
\draw[gp path] (1.688,8.913)--(2.981,8.930)--(3.209,8.943)--(3.598,8.902)--(3.760,9.074)%
  --(3.826,9.144)--(3.987,9.215)--(4.215,9.172)--(4.274,9.177)--(5.568,9.513)--(6.861,9.558)%
  --(8.154,9.622)--(9.447,9.631);
\gppoint{gp mark 9}{(1.688,8.913)}
\gppoint{gp mark 9}{(2.981,8.930)}
\gppoint{gp mark 9}{(3.209,8.943)}
\gppoint{gp mark 9}{(3.598,8.902)}
\gppoint{gp mark 9}{(3.760,9.074)}
\gppoint{gp mark 9}{(3.826,9.144)}
\gppoint{gp mark 9}{(3.987,9.215)}
\gppoint{gp mark 9}{(4.215,9.172)}
\gppoint{gp mark 9}{(4.274,9.177)}
\gppoint{gp mark 9}{(5.568,9.513)}
\gppoint{gp mark 9}{(6.861,9.558)}
\gppoint{gp mark 9}{(8.154,9.622)}
\gppoint{gp mark 9}{(9.447,9.631)}
\gpsetlinetype{gp lt axes}
\draw[gp path] (1.688,8.883)--(1.766,8.883)--(1.845,8.883)--(1.923,8.883)--(2.001,8.883)%
  --(2.080,8.883)--(2.158,8.883)--(2.237,8.883)--(2.315,8.883)--(2.393,8.883)--(2.472,8.883)%
  --(2.550,8.883)--(2.628,8.883)--(2.707,8.883)--(2.785,8.883)--(2.864,8.883)--(2.942,8.883)%
  --(3.020,8.883)--(3.099,8.883)--(3.177,8.883)--(3.255,8.883)--(3.334,8.883)--(3.412,8.883)%
  --(3.491,8.883)--(3.569,8.883)--(3.647,8.883)--(3.726,8.883)--(3.804,8.883)--(3.882,8.883)%
  --(3.961,8.883)--(4.039,8.883)--(4.118,8.883)--(4.196,8.883)--(4.274,8.883)--(4.353,8.883)%
  --(4.431,8.883)--(4.509,8.883)--(4.588,8.883)--(4.666,8.883)--(4.745,8.883)--(4.823,8.883)%
  --(4.901,8.883)--(4.980,8.883)--(5.058,8.883)--(5.136,8.883)--(5.215,8.883)--(5.293,8.883)%
  --(5.372,8.883)--(5.450,8.883)--(5.528,8.883)--(5.607,8.883)--(5.685,8.883)--(5.763,8.883)%
  --(5.842,8.883)--(5.920,8.883)--(5.999,8.883)--(6.077,8.883)--(6.155,8.883)--(6.234,8.883)%
  --(6.312,8.883)--(6.390,8.883)--(6.469,8.883)--(6.547,8.883)--(6.626,8.883)--(6.704,8.883)%
  --(6.782,8.883)--(6.861,8.883)--(6.939,8.883)--(7.017,8.883)--(7.096,8.883)--(7.174,8.883)%
  --(7.253,8.883)--(7.331,8.883)--(7.409,8.883)--(7.488,8.883)--(7.566,8.883)--(7.644,8.883)%
  --(7.723,8.883)--(7.801,8.883)--(7.880,8.883)--(7.958,8.883)--(8.036,8.883)--(8.115,8.883)%
  --(8.193,8.883)--(8.271,8.883)--(8.350,8.883)--(8.428,8.883)--(8.507,8.883)--(8.585,8.883)%
  --(8.663,8.883)--(8.742,8.883)--(8.820,8.883)--(8.898,8.883)--(8.977,8.883)--(9.055,8.883)%
  --(9.134,8.883)--(9.212,8.883)--(9.290,8.883)--(9.369,8.883)--(9.447,8.883);
\gpcolor{gp lt color 0}
\gpsetlinetype{gp lt plot 0}
\draw[gp path] (1.688,5.129)--(2.981,4.319)--(3.209,4.602)--(3.598,4.605)--(3.760,4.828)%
  --(3.826,4.216)--(3.987,4.649)--(4.215,4.862)--(4.274,4.668)--(5.568,5.116)--(6.861,7.539)%
  --(8.154,9.136)--(9.447,9.631);
\gppoint{gp mark 10}{(1.688,5.129)}
\gppoint{gp mark 10}{(2.981,4.319)}
\gppoint{gp mark 10}{(3.209,4.602)}
\gppoint{gp mark 10}{(3.598,4.605)}
\gppoint{gp mark 10}{(3.760,4.828)}
\gppoint{gp mark 10}{(3.826,4.216)}
\gppoint{gp mark 10}{(3.987,4.649)}
\gppoint{gp mark 10}{(4.215,4.862)}
\gppoint{gp mark 10}{(4.274,4.668)}
\gppoint{gp mark 10}{(5.568,5.116)}
\gppoint{gp mark 10}{(6.861,7.539)}
\gppoint{gp mark 10}{(8.154,9.136)}
\gppoint{gp mark 10}{(9.447,9.631)}
\gpsetlinetype{gp lt axes}
\draw[gp path] (1.688,5.040)--(1.766,5.040)--(1.845,5.040)--(1.923,5.040)--(2.001,5.040)%
  --(2.080,5.040)--(2.158,5.040)--(2.237,5.040)--(2.315,5.040)--(2.393,5.040)--(2.472,5.040)%
  --(2.550,5.040)--(2.628,5.040)--(2.707,5.040)--(2.785,5.040)--(2.864,5.040)--(2.942,5.040)%
  --(3.020,5.040)--(3.099,5.040)--(3.177,5.040)--(3.255,5.040)--(3.334,5.040)--(3.412,5.040)%
  --(3.491,5.040)--(3.569,5.040)--(3.647,5.040)--(3.726,5.040)--(3.804,5.040)--(3.882,5.040)%
  --(3.961,5.040)--(4.039,5.040)--(4.118,5.040)--(4.196,5.040)--(4.274,5.040)--(4.353,5.040)%
  --(4.431,5.040)--(4.509,5.040)--(4.588,5.040)--(4.666,5.040)--(4.745,5.040)--(4.823,5.040)%
  --(4.901,5.040)--(4.980,5.040)--(5.058,5.040)--(5.136,5.040)--(5.215,5.040)--(5.293,5.040)%
  --(5.372,5.040)--(5.450,5.040)--(5.528,5.040)--(5.607,5.040)--(5.685,5.040)--(5.763,5.040)%
  --(5.842,5.040)--(5.920,5.040)--(5.999,5.040)--(6.077,5.040)--(6.155,5.040)--(6.234,5.040)%
  --(6.312,5.040)--(6.390,5.040)--(6.469,5.040)--(6.547,5.040)--(6.626,5.040)--(6.704,5.040)%
  --(6.782,5.040)--(6.861,5.040)--(6.939,5.040)--(7.017,5.040)--(7.096,5.040)--(7.174,5.040)%
  --(7.253,5.040)--(7.331,5.040)--(7.409,5.040)--(7.488,5.040)--(7.566,5.040)--(7.644,5.040)%
  --(7.723,5.040)--(7.801,5.040)--(7.880,5.040)--(7.958,5.040)--(8.036,5.040)--(8.115,5.040)%
  --(8.193,5.040)--(8.271,5.040)--(8.350,5.040)--(8.428,5.040)--(8.507,5.040)--(8.585,5.040)%
  --(8.663,5.040)--(8.742,5.040)--(8.820,5.040)--(8.898,5.040)--(8.977,5.040)--(9.055,5.040)%
  --(9.134,5.040)--(9.212,5.040)--(9.290,5.040)--(9.369,5.040)--(9.447,5.040);
\gpcolor{gp lt color 1}
\gpsetlinetype{gp lt plot 0}
\draw[gp path] (1.688,5.245)--(2.981,4.352)--(3.209,4.164)--(3.598,4.209)--(3.760,4.034)%
  --(3.826,4.313)--(3.987,4.163)--(4.215,4.246)--(4.274,4.304)--(5.568,4.551)--(6.861,7.345)%
  --(8.154,9.631)--(9.447,9.597);
\gppoint{gp mark 11}{(1.688,5.245)}
\gppoint{gp mark 11}{(2.981,4.352)}
\gppoint{gp mark 11}{(3.209,4.164)}
\gppoint{gp mark 11}{(3.598,4.209)}
\gppoint{gp mark 11}{(3.760,4.034)}
\gppoint{gp mark 11}{(3.826,4.313)}
\gppoint{gp mark 11}{(3.987,4.163)}
\gppoint{gp mark 11}{(4.215,4.246)}
\gppoint{gp mark 11}{(4.274,4.304)}
\gppoint{gp mark 11}{(5.568,4.551)}
\gppoint{gp mark 11}{(6.861,7.345)}
\gppoint{gp mark 11}{(8.154,9.631)}
\gppoint{gp mark 11}{(9.447,9.597)}
\gpsetlinetype{gp lt axes}
\draw[gp path] (1.688,3.408)--(1.766,3.408)--(1.845,3.408)--(1.923,3.408)--(2.001,3.408)%
  --(2.080,3.408)--(2.158,3.408)--(2.237,3.408)--(2.315,3.408)--(2.393,3.408)--(2.472,3.408)%
  --(2.550,3.408)--(2.628,3.408)--(2.707,3.408)--(2.785,3.408)--(2.864,3.408)--(2.942,3.408)%
  --(3.020,3.408)--(3.099,3.408)--(3.177,3.408)--(3.255,3.408)--(3.334,3.408)--(3.412,3.408)%
  --(3.491,3.408)--(3.569,3.408)--(3.647,3.408)--(3.726,3.408)--(3.804,3.408)--(3.882,3.408)%
  --(3.961,3.408)--(4.039,3.408)--(4.118,3.408)--(4.196,3.408)--(4.274,3.408)--(4.353,3.408)%
  --(4.431,3.408)--(4.509,3.408)--(4.588,3.408)--(4.666,3.408)--(4.745,3.408)--(4.823,3.408)%
  --(4.901,3.408)--(4.980,3.408)--(5.058,3.408)--(5.136,3.408)--(5.215,3.408)--(5.293,3.408)%
  --(5.372,3.408)--(5.450,3.408)--(5.528,3.408)--(5.607,3.408)--(5.685,3.408)--(5.763,3.408)%
  --(5.842,3.408)--(5.920,3.408)--(5.999,3.408)--(6.077,3.408)--(6.155,3.408)--(6.234,3.408)%
  --(6.312,3.408)--(6.390,3.408)--(6.469,3.408)--(6.547,3.408)--(6.626,3.408)--(6.704,3.408)%
  --(6.782,3.408)--(6.861,3.408)--(6.939,3.408)--(7.017,3.408)--(7.096,3.408)--(7.174,3.408)%
  --(7.253,3.408)--(7.331,3.408)--(7.409,3.408)--(7.488,3.408)--(7.566,3.408)--(7.644,3.408)%
  --(7.723,3.408)--(7.801,3.408)--(7.880,3.408)--(7.958,3.408)--(8.036,3.408)--(8.115,3.408)%
  --(8.193,3.408)--(8.271,3.408)--(8.350,3.408)--(8.428,3.408)--(8.507,3.408)--(8.585,3.408)%
  --(8.663,3.408)--(8.742,3.408)--(8.820,3.408)--(8.898,3.408)--(8.977,3.408)--(9.055,3.408)%
  --(9.134,3.408)--(9.212,3.408)--(9.290,3.408)--(9.369,3.408)--(9.447,3.408);
\gpcolor{gp lt color 2}
\gpsetlinetype{gp lt plot 0}
\draw[gp path] (1.688,6.437)--(2.981,6.243)--(3.209,6.362)--(3.598,6.383)--(3.760,6.543)%
  --(3.826,6.310)--(3.987,6.338)--(4.215,6.283)--(4.274,6.358)--(5.568,6.558)--(6.861,9.631);
\gppoint{gp mark 12}{(1.688,6.437)}
\gppoint{gp mark 12}{(2.981,6.243)}
\gppoint{gp mark 12}{(3.209,6.362)}
\gppoint{gp mark 12}{(3.598,6.383)}
\gppoint{gp mark 12}{(3.760,6.543)}
\gppoint{gp mark 12}{(3.826,6.310)}
\gppoint{gp mark 12}{(3.987,6.338)}
\gppoint{gp mark 12}{(4.215,6.283)}
\gppoint{gp mark 12}{(4.274,6.358)}
\gppoint{gp mark 12}{(5.568,6.558)}
\gppoint{gp mark 12}{(6.861,9.631)}
\gpsetlinetype{gp lt axes}
\draw[gp path] (1.688,5.182)--(1.766,5.182)--(1.845,5.182)--(1.923,5.182)--(2.001,5.182)%
  --(2.080,5.182)--(2.158,5.182)--(2.237,5.182)--(2.315,5.182)--(2.393,5.182)--(2.472,5.182)%
  --(2.550,5.182)--(2.628,5.182)--(2.707,5.182)--(2.785,5.182)--(2.864,5.182)--(2.942,5.182)%
  --(3.020,5.182)--(3.099,5.182)--(3.177,5.182)--(3.255,5.182)--(3.334,5.182)--(3.412,5.182)%
  --(3.491,5.182)--(3.569,5.182)--(3.647,5.182)--(3.726,5.182)--(3.804,5.182)--(3.882,5.182)%
  --(3.961,5.182)--(4.039,5.182)--(4.118,5.182)--(4.196,5.182)--(4.274,5.182)--(4.353,5.182)%
  --(4.431,5.182)--(4.509,5.182)--(4.588,5.182)--(4.666,5.182)--(4.745,5.182)--(4.823,5.182)%
  --(4.901,5.182)--(4.980,5.182)--(5.058,5.182)--(5.136,5.182)--(5.215,5.182)--(5.293,5.182)%
  --(5.372,5.182)--(5.450,5.182)--(5.528,5.182)--(5.607,5.182)--(5.685,5.182)--(5.763,5.182)%
  --(5.842,5.182)--(5.920,5.182)--(5.999,5.182)--(6.077,5.182)--(6.155,5.182)--(6.234,5.182)%
  --(6.312,5.182)--(6.390,5.182)--(6.469,5.182)--(6.547,5.182)--(6.626,5.182)--(6.704,5.182)%
  --(6.782,5.182)--(6.861,5.182)--(6.939,5.182)--(7.017,5.182)--(7.096,5.182)--(7.174,5.182)%
  --(7.253,5.182)--(7.331,5.182)--(7.409,5.182)--(7.488,5.182)--(7.566,5.182)--(7.644,5.182)%
  --(7.723,5.182)--(7.801,5.182)--(7.880,5.182)--(7.958,5.182)--(8.036,5.182)--(8.115,5.182)%
  --(8.193,5.182)--(8.271,5.182)--(8.350,5.182)--(8.428,5.182)--(8.507,5.182)--(8.585,5.182)%
  --(8.663,5.182)--(8.742,5.182)--(8.820,5.182)--(8.898,5.182)--(8.977,5.182)--(9.055,5.182)%
  --(9.134,5.182)--(9.212,5.182)--(9.290,5.182)--(9.369,5.182)--(9.447,5.182);
\gpcolor{gp lt color 3}
\gpsetlinetype{gp lt plot 0}
\draw[gp path] (1.688,6.659)--(2.981,6.014)--(3.209,6.030)--(3.598,6.110)--(3.760,6.232)%
  --(3.826,6.432)--(3.987,6.261)--(4.215,6.443)--(4.274,6.073)--(5.568,6.393)--(6.861,9.631);
\gppoint{gp mark 13}{(1.688,6.659)}
\gppoint{gp mark 13}{(2.981,6.014)}
\gppoint{gp mark 13}{(3.209,6.030)}
\gppoint{gp mark 13}{(3.598,6.110)}
\gppoint{gp mark 13}{(3.760,6.232)}
\gppoint{gp mark 13}{(3.826,6.432)}
\gppoint{gp mark 13}{(3.987,6.261)}
\gppoint{gp mark 13}{(4.215,6.443)}
\gppoint{gp mark 13}{(4.274,6.073)}
\gppoint{gp mark 13}{(5.568,6.393)}
\gppoint{gp mark 13}{(6.861,9.631)}
\gpsetlinetype{gp lt axes}
\draw[gp path] (1.688,6.059)--(1.766,6.059)--(1.845,6.059)--(1.923,6.059)--(2.001,6.059)%
  --(2.080,6.059)--(2.158,6.059)--(2.237,6.059)--(2.315,6.059)--(2.393,6.059)--(2.472,6.059)%
  --(2.550,6.059)--(2.628,6.059)--(2.707,6.059)--(2.785,6.059)--(2.864,6.059)--(2.942,6.059)%
  --(3.020,6.059)--(3.099,6.059)--(3.177,6.059)--(3.255,6.059)--(3.334,6.059)--(3.412,6.059)%
  --(3.491,6.059)--(3.569,6.059)--(3.647,6.059)--(3.726,6.059)--(3.804,6.059)--(3.882,6.059)%
  --(3.961,6.059)--(4.039,6.059)--(4.118,6.059)--(4.196,6.059)--(4.274,6.059)--(4.353,6.059)%
  --(4.431,6.059)--(4.509,6.059)--(4.588,6.059)--(4.666,6.059)--(4.745,6.059)--(4.823,6.059)%
  --(4.901,6.059)--(4.980,6.059)--(5.058,6.059)--(5.136,6.059)--(5.215,6.059)--(5.293,6.059)%
  --(5.372,6.059)--(5.450,6.059)--(5.528,6.059)--(5.607,6.059)--(5.685,6.059)--(5.763,6.059)%
  --(5.842,6.059)--(5.920,6.059)--(5.999,6.059)--(6.077,6.059)--(6.155,6.059)--(6.234,6.059)%
  --(6.312,6.059)--(6.390,6.059)--(6.469,6.059)--(6.547,6.059)--(6.626,6.059)--(6.704,6.059)%
  --(6.782,6.059)--(6.861,6.059)--(6.939,6.059)--(7.017,6.059)--(7.096,6.059)--(7.174,6.059)%
  --(7.253,6.059)--(7.331,6.059)--(7.409,6.059)--(7.488,6.059)--(7.566,6.059)--(7.644,6.059)%
  --(7.723,6.059)--(7.801,6.059)--(7.880,6.059)--(7.958,6.059)--(8.036,6.059)--(8.115,6.059)%
  --(8.193,6.059)--(8.271,6.059)--(8.350,6.059)--(8.428,6.059)--(8.507,6.059)--(8.585,6.059)%
  --(8.663,6.059)--(8.742,6.059)--(8.820,6.059)--(8.898,6.059)--(8.977,6.059)--(9.055,6.059)%
  --(9.134,6.059)--(9.212,6.059)--(9.290,6.059)--(9.369,6.059)--(9.447,6.059);
\gpcolor{gp lt color 4}
\gpsetlinetype{gp lt plot 0}
\draw[gp path] (1.688,8.527)--(2.981,8.347)--(3.209,8.506)--(3.598,8.444)--(3.760,8.546)%
  --(3.826,8.573)--(3.987,8.504)--(4.215,8.451)--(4.274,8.701)--(5.568,9.072)--(6.861,9.470)%
  --(8.154,9.631)--(9.447,9.569);
\gppoint{gp mark 14}{(1.688,8.527)}
\gppoint{gp mark 14}{(2.981,8.347)}
\gppoint{gp mark 14}{(3.209,8.506)}
\gppoint{gp mark 14}{(3.598,8.444)}
\gppoint{gp mark 14}{(3.760,8.546)}
\gppoint{gp mark 14}{(3.826,8.573)}
\gppoint{gp mark 14}{(3.987,8.504)}
\gppoint{gp mark 14}{(4.215,8.451)}
\gppoint{gp mark 14}{(4.274,8.701)}
\gppoint{gp mark 14}{(5.568,9.072)}
\gppoint{gp mark 14}{(6.861,9.470)}
\gppoint{gp mark 14}{(8.154,9.631)}
\gppoint{gp mark 14}{(9.447,9.569)}
\gpsetlinetype{gp lt axes}
\draw[gp path] (1.688,8.787)--(1.766,8.787)--(1.845,8.787)--(1.923,8.787)--(2.001,8.787)%
  --(2.080,8.787)--(2.158,8.787)--(2.237,8.787)--(2.315,8.787)--(2.393,8.787)--(2.472,8.787)%
  --(2.550,8.787)--(2.628,8.787)--(2.707,8.787)--(2.785,8.787)--(2.864,8.787)--(2.942,8.787)%
  --(3.020,8.787)--(3.099,8.787)--(3.177,8.787)--(3.255,8.787)--(3.334,8.787)--(3.412,8.787)%
  --(3.491,8.787)--(3.569,8.787)--(3.647,8.787)--(3.726,8.787)--(3.804,8.787)--(3.882,8.787)%
  --(3.961,8.787)--(4.039,8.787)--(4.118,8.787)--(4.196,8.787)--(4.274,8.787)--(4.353,8.787)%
  --(4.431,8.787)--(4.509,8.787)--(4.588,8.787)--(4.666,8.787)--(4.745,8.787)--(4.823,8.787)%
  --(4.901,8.787)--(4.980,8.787)--(5.058,8.787)--(5.136,8.787)--(5.215,8.787)--(5.293,8.787)%
  --(5.372,8.787)--(5.450,8.787)--(5.528,8.787)--(5.607,8.787)--(5.685,8.787)--(5.763,8.787)%
  --(5.842,8.787)--(5.920,8.787)--(5.999,8.787)--(6.077,8.787)--(6.155,8.787)--(6.234,8.787)%
  --(6.312,8.787)--(6.390,8.787)--(6.469,8.787)--(6.547,8.787)--(6.626,8.787)--(6.704,8.787)%
  --(6.782,8.787)--(6.861,8.787)--(6.939,8.787)--(7.017,8.787)--(7.096,8.787)--(7.174,8.787)%
  --(7.253,8.787)--(7.331,8.787)--(7.409,8.787)--(7.488,8.787)--(7.566,8.787)--(7.644,8.787)%
  --(7.723,8.787)--(7.801,8.787)--(7.880,8.787)--(7.958,8.787)--(8.036,8.787)--(8.115,8.787)%
  --(8.193,8.787)--(8.271,8.787)--(8.350,8.787)--(8.428,8.787)--(8.507,8.787)--(8.585,8.787)%
  --(8.663,8.787)--(8.742,8.787)--(8.820,8.787)--(8.898,8.787)--(8.977,8.787)--(9.055,8.787)%
  --(9.134,8.787)--(9.212,8.787)--(9.290,8.787)--(9.369,8.787)--(9.447,8.787);
\gpcolor{gp lt color 5}
\gpsetlinetype{gp lt plot 0}
\draw[gp path] (1.688,6.866)--(2.981,6.813)--(3.209,6.856)--(3.598,7.379)--(3.760,9.154)%
  --(3.826,6.790)--(3.987,7.273)--(4.215,6.854)--(4.274,6.959)--(5.568,7.172)--(6.861,9.631)%
  --(8.154,9.037)--(9.447,8.966);
\gppoint{gp mark 15}{(1.688,6.866)}
\gppoint{gp mark 15}{(2.981,6.813)}
\gppoint{gp mark 15}{(3.209,6.856)}
\gppoint{gp mark 15}{(3.598,7.379)}
\gppoint{gp mark 15}{(3.760,9.154)}
\gppoint{gp mark 15}{(3.826,6.790)}
\gppoint{gp mark 15}{(3.987,7.273)}
\gppoint{gp mark 15}{(4.215,6.854)}
\gppoint{gp mark 15}{(4.274,6.959)}
\gppoint{gp mark 15}{(5.568,7.172)}
\gppoint{gp mark 15}{(6.861,9.631)}
\gppoint{gp mark 15}{(8.154,9.037)}
\gppoint{gp mark 15}{(9.447,8.966)}
\gpsetlinetype{gp lt axes}
\draw[gp path] (1.688,6.993)--(1.766,6.993)--(1.845,6.993)--(1.923,6.993)--(2.001,6.993)%
  --(2.080,6.993)--(2.158,6.993)--(2.237,6.993)--(2.315,6.993)--(2.393,6.993)--(2.472,6.993)%
  --(2.550,6.993)--(2.628,6.993)--(2.707,6.993)--(2.785,6.993)--(2.864,6.993)--(2.942,6.993)%
  --(3.020,6.993)--(3.099,6.993)--(3.177,6.993)--(3.255,6.993)--(3.334,6.993)--(3.412,6.993)%
  --(3.491,6.993)--(3.569,6.993)--(3.647,6.993)--(3.726,6.993)--(3.804,6.993)--(3.882,6.993)%
  --(3.961,6.993)--(4.039,6.993)--(4.118,6.993)--(4.196,6.993)--(4.274,6.993)--(4.353,6.993)%
  --(4.431,6.993)--(4.509,6.993)--(4.588,6.993)--(4.666,6.993)--(4.745,6.993)--(4.823,6.993)%
  --(4.901,6.993)--(4.980,6.993)--(5.058,6.993)--(5.136,6.993)--(5.215,6.993)--(5.293,6.993)%
  --(5.372,6.993)--(5.450,6.993)--(5.528,6.993)--(5.607,6.993)--(5.685,6.993)--(5.763,6.993)%
  --(5.842,6.993)--(5.920,6.993)--(5.999,6.993)--(6.077,6.993)--(6.155,6.993)--(6.234,6.993)%
  --(6.312,6.993)--(6.390,6.993)--(6.469,6.993)--(6.547,6.993)--(6.626,6.993)--(6.704,6.993)%
  --(6.782,6.993)--(6.861,6.993)--(6.939,6.993)--(7.017,6.993)--(7.096,6.993)--(7.174,6.993)%
  --(7.253,6.993)--(7.331,6.993)--(7.409,6.993)--(7.488,6.993)--(7.566,6.993)--(7.644,6.993)%
  --(7.723,6.993)--(7.801,6.993)--(7.880,6.993)--(7.958,6.993)--(8.036,6.993)--(8.115,6.993)%
  --(8.193,6.993)--(8.271,6.993)--(8.350,6.993)--(8.428,6.993)--(8.507,6.993)--(8.585,6.993)%
  --(8.663,6.993)--(8.742,6.993)--(8.820,6.993)--(8.898,6.993)--(8.977,6.993)--(9.055,6.993)%
  --(9.134,6.993)--(9.212,6.993)--(9.290,6.993)--(9.369,6.993)--(9.447,6.993);
\gpcolor{gp lt color 6}
\gpsetlinetype{gp lt plot 0}
\draw[gp path] (1.688,7.055)--(2.981,7.268)--(3.209,7.627)--(3.598,7.483)--(3.760,7.767)%
  --(3.826,7.924)--(3.987,7.626)--(4.215,8.198)--(4.274,8.108)--(5.568,8.284)--(6.861,9.631)%
  --(8.154,9.264)--(9.447,9.235);
\gppoint{gp mark 1}{(1.688,7.055)}
\gppoint{gp mark 1}{(2.981,7.268)}
\gppoint{gp mark 1}{(3.209,7.627)}
\gppoint{gp mark 1}{(3.598,7.483)}
\gppoint{gp mark 1}{(3.760,7.767)}
\gppoint{gp mark 1}{(3.826,7.924)}
\gppoint{gp mark 1}{(3.987,7.626)}
\gppoint{gp mark 1}{(4.215,8.198)}
\gppoint{gp mark 1}{(4.274,8.108)}
\gppoint{gp mark 1}{(5.568,8.284)}
\gppoint{gp mark 1}{(6.861,9.631)}
\gppoint{gp mark 1}{(8.154,9.264)}
\gppoint{gp mark 1}{(9.447,9.235)}
\gpsetlinetype{gp lt axes}
\draw[gp path] (1.688,8.407)--(1.766,8.407)--(1.845,8.407)--(1.923,8.407)--(2.001,8.407)%
  --(2.080,8.407)--(2.158,8.407)--(2.237,8.407)--(2.315,8.407)--(2.393,8.407)--(2.472,8.407)%
  --(2.550,8.407)--(2.628,8.407)--(2.707,8.407)--(2.785,8.407)--(2.864,8.407)--(2.942,8.407)%
  --(3.020,8.407)--(3.099,8.407)--(3.177,8.407)--(3.255,8.407)--(3.334,8.407)--(3.412,8.407)%
  --(3.491,8.407)--(3.569,8.407)--(3.647,8.407)--(3.726,8.407)--(3.804,8.407)--(3.882,8.407)%
  --(3.961,8.407)--(4.039,8.407)--(4.118,8.407)--(4.196,8.407)--(4.274,8.407)--(4.353,8.407)%
  --(4.431,8.407)--(4.509,8.407)--(4.588,8.407)--(4.666,8.407)--(4.745,8.407)--(4.823,8.407)%
  --(4.901,8.407)--(4.980,8.407)--(5.058,8.407)--(5.136,8.407)--(5.215,8.407)--(5.293,8.407)%
  --(5.372,8.407)--(5.450,8.407)--(5.528,8.407)--(5.607,8.407)--(5.685,8.407)--(5.763,8.407)%
  --(5.842,8.407)--(5.920,8.407)--(5.999,8.407)--(6.077,8.407)--(6.155,8.407)--(6.234,8.407)%
  --(6.312,8.407)--(6.390,8.407)--(6.469,8.407)--(6.547,8.407)--(6.626,8.407)--(6.704,8.407)%
  --(6.782,8.407)--(6.861,8.407)--(6.939,8.407)--(7.017,8.407)--(7.096,8.407)--(7.174,8.407)%
  --(7.253,8.407)--(7.331,8.407)--(7.409,8.407)--(7.488,8.407)--(7.566,8.407)--(7.644,8.407)%
  --(7.723,8.407)--(7.801,8.407)--(7.880,8.407)--(7.958,8.407)--(8.036,8.407)--(8.115,8.407)%
  --(8.193,8.407)--(8.271,8.407)--(8.350,8.407)--(8.428,8.407)--(8.507,8.407)--(8.585,8.407)%
  --(8.663,8.407)--(8.742,8.407)--(8.820,8.407)--(8.898,8.407)--(8.977,8.407)--(9.055,8.407)%
  --(9.134,8.407)--(9.212,8.407)--(9.290,8.407)--(9.369,8.407)--(9.447,8.407);
\gpcolor{gp lt color 7}
\gpsetlinetype{gp lt plot 0}
\draw[gp path] (1.688,9.073)--(2.981,9.067)--(3.209,9.164)--(3.598,9.202)--(3.760,9.195)%
  --(3.826,9.267)--(3.987,9.299)--(4.215,9.352)--(4.274,9.380)--(5.568,9.565)--(6.861,9.534)%
  --(8.154,9.584)--(9.447,9.631);
\gppoint{gp mark 2}{(1.688,9.073)}
\gppoint{gp mark 2}{(2.981,9.067)}
\gppoint{gp mark 2}{(3.209,9.164)}
\gppoint{gp mark 2}{(3.598,9.202)}
\gppoint{gp mark 2}{(3.760,9.195)}
\gppoint{gp mark 2}{(3.826,9.267)}
\gppoint{gp mark 2}{(3.987,9.299)}
\gppoint{gp mark 2}{(4.215,9.352)}
\gppoint{gp mark 2}{(4.274,9.380)}
\gppoint{gp mark 2}{(5.568,9.565)}
\gppoint{gp mark 2}{(6.861,9.534)}
\gppoint{gp mark 2}{(8.154,9.584)}
\gppoint{gp mark 2}{(9.447,9.631)}
\gpsetlinetype{gp lt axes}
\draw[gp path] (1.688,9.070)--(1.766,9.070)--(1.845,9.070)--(1.923,9.070)--(2.001,9.070)%
  --(2.080,9.070)--(2.158,9.070)--(2.237,9.070)--(2.315,9.070)--(2.393,9.070)--(2.472,9.070)%
  --(2.550,9.070)--(2.628,9.070)--(2.707,9.070)--(2.785,9.070)--(2.864,9.070)--(2.942,9.070)%
  --(3.020,9.070)--(3.099,9.070)--(3.177,9.070)--(3.255,9.070)--(3.334,9.070)--(3.412,9.070)%
  --(3.491,9.070)--(3.569,9.070)--(3.647,9.070)--(3.726,9.070)--(3.804,9.070)--(3.882,9.070)%
  --(3.961,9.070)--(4.039,9.070)--(4.118,9.070)--(4.196,9.070)--(4.274,9.070)--(4.353,9.070)%
  --(4.431,9.070)--(4.509,9.070)--(4.588,9.070)--(4.666,9.070)--(4.745,9.070)--(4.823,9.070)%
  --(4.901,9.070)--(4.980,9.070)--(5.058,9.070)--(5.136,9.070)--(5.215,9.070)--(5.293,9.070)%
  --(5.372,9.070)--(5.450,9.070)--(5.528,9.070)--(5.607,9.070)--(5.685,9.070)--(5.763,9.070)%
  --(5.842,9.070)--(5.920,9.070)--(5.999,9.070)--(6.077,9.070)--(6.155,9.070)--(6.234,9.070)%
  --(6.312,9.070)--(6.390,9.070)--(6.469,9.070)--(6.547,9.070)--(6.626,9.070)--(6.704,9.070)%
  --(6.782,9.070)--(6.861,9.070)--(6.939,9.070)--(7.017,9.070)--(7.096,9.070)--(7.174,9.070)%
  --(7.253,9.070)--(7.331,9.070)--(7.409,9.070)--(7.488,9.070)--(7.566,9.070)--(7.644,9.070)%
  --(7.723,9.070)--(7.801,9.070)--(7.880,9.070)--(7.958,9.070)--(8.036,9.070)--(8.115,9.070)%
  --(8.193,9.070)--(8.271,9.070)--(8.350,9.070)--(8.428,9.070)--(8.507,9.070)--(8.585,9.070)%
  --(8.663,9.070)--(8.742,9.070)--(8.820,9.070)--(8.898,9.070)--(8.977,9.070)--(9.055,9.070)%
  --(9.134,9.070)--(9.212,9.070)--(9.290,9.070)--(9.369,9.070)--(9.447,9.070);
\gpcolor{gp lt color 0}
\gpsetlinetype{gp lt plot 0}
\draw[gp path] (1.688,8.312)--(2.981,8.405)--(3.209,8.371)--(3.598,8.474)--(3.760,8.782)%
  --(3.826,8.639)--(3.987,8.350)--(4.215,8.425)--(4.274,8.438)--(5.568,8.903)--(6.861,9.631)%
  --(8.154,9.432)--(9.447,9.371);
\gppoint{gp mark 3}{(1.688,8.312)}
\gppoint{gp mark 3}{(2.981,8.405)}
\gppoint{gp mark 3}{(3.209,8.371)}
\gppoint{gp mark 3}{(3.598,8.474)}
\gppoint{gp mark 3}{(3.760,8.782)}
\gppoint{gp mark 3}{(3.826,8.639)}
\gppoint{gp mark 3}{(3.987,8.350)}
\gppoint{gp mark 3}{(4.215,8.425)}
\gppoint{gp mark 3}{(4.274,8.438)}
\gppoint{gp mark 3}{(5.568,8.903)}
\gppoint{gp mark 3}{(6.861,9.631)}
\gppoint{gp mark 3}{(8.154,9.432)}
\gppoint{gp mark 3}{(9.447,9.371)}
\gpsetlinetype{gp lt axes}
\draw[gp path] (1.688,8.321)--(1.766,8.321)--(1.845,8.321)--(1.923,8.321)--(2.001,8.321)%
  --(2.080,8.321)--(2.158,8.321)--(2.237,8.321)--(2.315,8.321)--(2.393,8.321)--(2.472,8.321)%
  --(2.550,8.321)--(2.628,8.321)--(2.707,8.321)--(2.785,8.321)--(2.864,8.321)--(2.942,8.321)%
  --(3.020,8.321)--(3.099,8.321)--(3.177,8.321)--(3.255,8.321)--(3.334,8.321)--(3.412,8.321)%
  --(3.491,8.321)--(3.569,8.321)--(3.647,8.321)--(3.726,8.321)--(3.804,8.321)--(3.882,8.321)%
  --(3.961,8.321)--(4.039,8.321)--(4.118,8.321)--(4.196,8.321)--(4.274,8.321)--(4.353,8.321)%
  --(4.431,8.321)--(4.509,8.321)--(4.588,8.321)--(4.666,8.321)--(4.745,8.321)--(4.823,8.321)%
  --(4.901,8.321)--(4.980,8.321)--(5.058,8.321)--(5.136,8.321)--(5.215,8.321)--(5.293,8.321)%
  --(5.372,8.321)--(5.450,8.321)--(5.528,8.321)--(5.607,8.321)--(5.685,8.321)--(5.763,8.321)%
  --(5.842,8.321)--(5.920,8.321)--(5.999,8.321)--(6.077,8.321)--(6.155,8.321)--(6.234,8.321)%
  --(6.312,8.321)--(6.390,8.321)--(6.469,8.321)--(6.547,8.321)--(6.626,8.321)--(6.704,8.321)%
  --(6.782,8.321)--(6.861,8.321)--(6.939,8.321)--(7.017,8.321)--(7.096,8.321)--(7.174,8.321)%
  --(7.253,8.321)--(7.331,8.321)--(7.409,8.321)--(7.488,8.321)--(7.566,8.321)--(7.644,8.321)%
  --(7.723,8.321)--(7.801,8.321)--(7.880,8.321)--(7.958,8.321)--(8.036,8.321)--(8.115,8.321)%
  --(8.193,8.321)--(8.271,8.321)--(8.350,8.321)--(8.428,8.321)--(8.507,8.321)--(8.585,8.321)%
  --(8.663,8.321)--(8.742,8.321)--(8.820,8.321)--(8.898,8.321)--(8.977,8.321)--(9.055,8.321)%
  --(9.134,8.321)--(9.212,8.321)--(9.290,8.321)--(9.369,8.321)--(9.447,8.321);
\gpcolor{gp lt color 1}
\gpsetlinetype{gp lt plot 0}
\draw[gp path] (1.688,8.463)--(2.981,8.468)--(3.209,8.502)--(3.598,8.592)--(3.760,8.643)%
  --(3.826,8.681)--(3.987,8.703)--(4.215,8.722)--(4.274,8.803)--(5.568,9.123)--(6.861,9.546)%
  --(8.154,9.631)--(9.447,9.617);
\gppoint{gp mark 4}{(1.688,8.463)}
\gppoint{gp mark 4}{(2.981,8.468)}
\gppoint{gp mark 4}{(3.209,8.502)}
\gppoint{gp mark 4}{(3.598,8.592)}
\gppoint{gp mark 4}{(3.760,8.643)}
\gppoint{gp mark 4}{(3.826,8.681)}
\gppoint{gp mark 4}{(3.987,8.703)}
\gppoint{gp mark 4}{(4.215,8.722)}
\gppoint{gp mark 4}{(4.274,8.803)}
\gppoint{gp mark 4}{(5.568,9.123)}
\gppoint{gp mark 4}{(6.861,9.546)}
\gppoint{gp mark 4}{(8.154,9.631)}
\gppoint{gp mark 4}{(9.447,9.617)}
\gpsetlinetype{gp lt axes}
\draw[gp path] (1.688,8.424)--(1.766,8.424)--(1.845,8.424)--(1.923,8.424)--(2.001,8.424)%
  --(2.080,8.424)--(2.158,8.424)--(2.237,8.424)--(2.315,8.424)--(2.393,8.424)--(2.472,8.424)%
  --(2.550,8.424)--(2.628,8.424)--(2.707,8.424)--(2.785,8.424)--(2.864,8.424)--(2.942,8.424)%
  --(3.020,8.424)--(3.099,8.424)--(3.177,8.424)--(3.255,8.424)--(3.334,8.424)--(3.412,8.424)%
  --(3.491,8.424)--(3.569,8.424)--(3.647,8.424)--(3.726,8.424)--(3.804,8.424)--(3.882,8.424)%
  --(3.961,8.424)--(4.039,8.424)--(4.118,8.424)--(4.196,8.424)--(4.274,8.424)--(4.353,8.424)%
  --(4.431,8.424)--(4.509,8.424)--(4.588,8.424)--(4.666,8.424)--(4.745,8.424)--(4.823,8.424)%
  --(4.901,8.424)--(4.980,8.424)--(5.058,8.424)--(5.136,8.424)--(5.215,8.424)--(5.293,8.424)%
  --(5.372,8.424)--(5.450,8.424)--(5.528,8.424)--(5.607,8.424)--(5.685,8.424)--(5.763,8.424)%
  --(5.842,8.424)--(5.920,8.424)--(5.999,8.424)--(6.077,8.424)--(6.155,8.424)--(6.234,8.424)%
  --(6.312,8.424)--(6.390,8.424)--(6.469,8.424)--(6.547,8.424)--(6.626,8.424)--(6.704,8.424)%
  --(6.782,8.424)--(6.861,8.424)--(6.939,8.424)--(7.017,8.424)--(7.096,8.424)--(7.174,8.424)%
  --(7.253,8.424)--(7.331,8.424)--(7.409,8.424)--(7.488,8.424)--(7.566,8.424)--(7.644,8.424)%
  --(7.723,8.424)--(7.801,8.424)--(7.880,8.424)--(7.958,8.424)--(8.036,8.424)--(8.115,8.424)%
  --(8.193,8.424)--(8.271,8.424)--(8.350,8.424)--(8.428,8.424)--(8.507,8.424)--(8.585,8.424)%
  --(8.663,8.424)--(8.742,8.424)--(8.820,8.424)--(8.898,8.424)--(8.977,8.424)--(9.055,8.424)%
  --(9.134,8.424)--(9.212,8.424)--(9.290,8.424)--(9.369,8.424)--(9.447,8.424);
\gpcolor{gp lt color 2}
\gpsetlinetype{gp lt plot 0}
\draw[gp path] (1.688,8.370)--(2.981,8.426)--(3.209,8.493)--(3.598,8.482)--(3.760,8.526)%
  --(3.826,8.760)--(3.987,8.763)--(4.215,8.777)--(4.274,8.804)--(5.568,9.172)--(6.861,9.631)%
  --(8.154,9.465)--(9.447,9.493);
\gppoint{gp mark 5}{(1.688,8.370)}
\gppoint{gp mark 5}{(2.981,8.426)}
\gppoint{gp mark 5}{(3.209,8.493)}
\gppoint{gp mark 5}{(3.598,8.482)}
\gppoint{gp mark 5}{(3.760,8.526)}
\gppoint{gp mark 5}{(3.826,8.760)}
\gppoint{gp mark 5}{(3.987,8.763)}
\gppoint{gp mark 5}{(4.215,8.777)}
\gppoint{gp mark 5}{(4.274,8.804)}
\gppoint{gp mark 5}{(5.568,9.172)}
\gppoint{gp mark 5}{(6.861,9.631)}
\gppoint{gp mark 5}{(8.154,9.465)}
\gppoint{gp mark 5}{(9.447,9.493)}
\gpsetlinetype{gp lt axes}
\draw[gp path] (1.688,8.394)--(1.766,8.394)--(1.845,8.394)--(1.923,8.394)--(2.001,8.394)%
  --(2.080,8.394)--(2.158,8.394)--(2.237,8.394)--(2.315,8.394)--(2.393,8.394)--(2.472,8.394)%
  --(2.550,8.394)--(2.628,8.394)--(2.707,8.394)--(2.785,8.394)--(2.864,8.394)--(2.942,8.394)%
  --(3.020,8.394)--(3.099,8.394)--(3.177,8.394)--(3.255,8.394)--(3.334,8.394)--(3.412,8.394)%
  --(3.491,8.394)--(3.569,8.394)--(3.647,8.394)--(3.726,8.394)--(3.804,8.394)--(3.882,8.394)%
  --(3.961,8.394)--(4.039,8.394)--(4.118,8.394)--(4.196,8.394)--(4.274,8.394)--(4.353,8.394)%
  --(4.431,8.394)--(4.509,8.394)--(4.588,8.394)--(4.666,8.394)--(4.745,8.394)--(4.823,8.394)%
  --(4.901,8.394)--(4.980,8.394)--(5.058,8.394)--(5.136,8.394)--(5.215,8.394)--(5.293,8.394)%
  --(5.372,8.394)--(5.450,8.394)--(5.528,8.394)--(5.607,8.394)--(5.685,8.394)--(5.763,8.394)%
  --(5.842,8.394)--(5.920,8.394)--(5.999,8.394)--(6.077,8.394)--(6.155,8.394)--(6.234,8.394)%
  --(6.312,8.394)--(6.390,8.394)--(6.469,8.394)--(6.547,8.394)--(6.626,8.394)--(6.704,8.394)%
  --(6.782,8.394)--(6.861,8.394)--(6.939,8.394)--(7.017,8.394)--(7.096,8.394)--(7.174,8.394)%
  --(7.253,8.394)--(7.331,8.394)--(7.409,8.394)--(7.488,8.394)--(7.566,8.394)--(7.644,8.394)%
  --(7.723,8.394)--(7.801,8.394)--(7.880,8.394)--(7.958,8.394)--(8.036,8.394)--(8.115,8.394)%
  --(8.193,8.394)--(8.271,8.394)--(8.350,8.394)--(8.428,8.394)--(8.507,8.394)--(8.585,8.394)%
  --(8.663,8.394)--(8.742,8.394)--(8.820,8.394)--(8.898,8.394)--(8.977,8.394)--(9.055,8.394)%
  --(9.134,8.394)--(9.212,8.394)--(9.290,8.394)--(9.369,8.394)--(9.447,8.394);
\gpcolor{gp lt color 3}
\gpsetlinetype{gp lt plot 0}
\draw[gp path] (1.688,8.366)--(2.981,7.756)--(3.209,7.621)--(3.598,7.758)--(3.760,7.589)%
  --(3.826,7.710)--(3.987,7.688)--(4.215,7.724)--(4.274,7.738)--(5.568,7.843)--(6.861,7.988)%
  --(8.154,7.915)--(9.447,7.865);
\gppoint{gp mark 6}{(1.688,8.366)}
\gppoint{gp mark 6}{(2.981,7.756)}
\gppoint{gp mark 6}{(3.209,7.621)}
\gppoint{gp mark 6}{(3.598,7.758)}
\gppoint{gp mark 6}{(3.760,7.589)}
\gppoint{gp mark 6}{(3.826,7.710)}
\gppoint{gp mark 6}{(3.987,7.688)}
\gppoint{gp mark 6}{(4.215,7.724)}
\gppoint{gp mark 6}{(4.274,7.738)}
\gppoint{gp mark 6}{(5.568,7.843)}
\gppoint{gp mark 6}{(6.861,7.988)}
\gppoint{gp mark 6}{(8.154,7.915)}
\gppoint{gp mark 6}{(9.447,7.865)}
\gpsetlinetype{gp lt axes}
\draw[gp path] (1.688,9.631)--(1.766,9.631)--(1.845,9.631)--(1.923,9.631)--(2.001,9.631)%
  --(2.080,9.631)--(2.158,9.631)--(2.237,9.631)--(2.315,9.631)--(2.393,9.631)--(2.472,9.631)%
  --(2.550,9.631)--(2.628,9.631)--(2.707,9.631)--(2.785,9.631)--(2.864,9.631)--(2.942,9.631)%
  --(3.020,9.631)--(3.099,9.631)--(3.177,9.631)--(3.255,9.631)--(3.334,9.631)--(3.412,9.631)%
  --(3.491,9.631)--(3.569,9.631)--(3.647,9.631)--(3.726,9.631)--(3.804,9.631)--(3.882,9.631)%
  --(3.961,9.631)--(4.039,9.631)--(4.118,9.631)--(4.196,9.631)--(4.274,9.631)--(4.353,9.631)%
  --(4.431,9.631)--(4.509,9.631)--(4.588,9.631)--(4.666,9.631)--(4.745,9.631)--(4.823,9.631)%
  --(4.901,9.631)--(4.980,9.631)--(5.058,9.631)--(5.136,9.631)--(5.215,9.631)--(5.293,9.631)%
  --(5.372,9.631)--(5.450,9.631)--(5.528,9.631)--(5.607,9.631)--(5.685,9.631)--(5.763,9.631)%
  --(5.842,9.631)--(5.920,9.631)--(5.999,9.631)--(6.077,9.631)--(6.155,9.631)--(6.234,9.631)%
  --(6.312,9.631)--(6.390,9.631)--(6.469,9.631)--(6.547,9.631)--(6.626,9.631)--(6.704,9.631)%
  --(6.782,9.631)--(6.861,9.631)--(6.939,9.631)--(7.017,9.631)--(7.096,9.631)--(7.174,9.631)%
  --(7.253,9.631)--(7.331,9.631)--(7.409,9.631)--(7.488,9.631)--(7.566,9.631)--(7.644,9.631)%
  --(7.723,9.631)--(7.801,9.631)--(7.880,9.631)--(7.958,9.631)--(8.036,9.631)--(8.115,9.631)%
  --(8.193,9.631)--(8.271,9.631)--(8.350,9.631)--(8.428,9.631)--(8.507,9.631)--(8.585,9.631)%
  --(8.663,9.631)--(8.742,9.631)--(8.820,9.631)--(8.898,9.631)--(8.977,9.631)--(9.055,9.631)%
  --(9.134,9.631)--(9.212,9.631)--(9.290,9.631)--(9.369,9.631)--(9.447,9.631);
%% coordinates of the plot area
\gpdefrectangularnode{gp plot 1}{\pgfpoint{1.688cm}{0.985cm}}{\pgfpoint{9.447cm}{9.631cm}}
\end{tikzpicture}
%% gnuplot variables

    \caption{Total algorithm execution time with various block size
        implementations}
    \label{profiling:blockSize:totalExecutionTime}
\end{figure}

\begin{figure}
    \centering
    \begin{tikzpicture}[gnuplot]
%% generated with GNUPLOT 4.4p3 (Lua 5.1.4; terminal rev. 97, script rev. 96a)
%% Wed 24 Oct 2012 15:01:58 EST
\gpcolor{gp lt color border}
\gpsetlinetype{gp lt border}
\gpsetlinewidth{1.00}
\draw[gp path] (1.504,0.985)--(1.684,0.985);
\draw[gp path] (8.447,0.985)--(8.267,0.985);
\node[gp node right] at (1.320,0.985) { 0};
\draw[gp path] (1.504,1.750)--(1.684,1.750);
\draw[gp path] (8.447,1.750)--(8.267,1.750);
\node[gp node right] at (1.320,1.750) { 0.1};
\draw[gp path] (1.504,2.514)--(1.684,2.514);
\draw[gp path] (8.447,2.514)--(8.267,2.514);
\node[gp node right] at (1.320,2.514) { 0.2};
\draw[gp path] (1.504,3.279)--(1.684,3.279);
\draw[gp path] (8.447,3.279)--(8.267,3.279);
\node[gp node right] at (1.320,3.279) { 0.3};
\draw[gp path] (1.504,4.043)--(1.684,4.043);
\draw[gp path] (8.447,4.043)--(8.267,4.043);
\node[gp node right] at (1.320,4.043) { 0.4};
\draw[gp path] (1.504,4.808)--(1.684,4.808);
\draw[gp path] (8.447,4.808)--(8.267,4.808);
\node[gp node right] at (1.320,4.808) { 0.5};
\draw[gp path] (1.504,5.573)--(1.684,5.573);
\draw[gp path] (8.447,5.573)--(8.267,5.573);
\node[gp node right] at (1.320,5.573) { 0.6};
\draw[gp path] (1.504,6.337)--(1.684,6.337);
\draw[gp path] (8.447,6.337)--(8.267,6.337);
\node[gp node right] at (1.320,6.337) { 0.7};
\draw[gp path] (1.504,7.102)--(1.684,7.102);
\draw[gp path] (8.447,7.102)--(8.267,7.102);
\node[gp node right] at (1.320,7.102) { 0.8};
\draw[gp path] (1.504,7.866)--(1.684,7.866);
\draw[gp path] (8.447,7.866)--(8.267,7.866);
\node[gp node right] at (1.320,7.866) { 0.9};
\draw[gp path] (1.504,8.631)--(1.684,8.631);
\draw[gp path] (8.447,8.631)--(8.267,8.631);
\node[gp node right] at (1.320,8.631) { 1};
\draw[gp path] (1.504,0.985)--(1.504,1.165);
\draw[gp path] (1.504,8.631)--(1.504,8.451);
\node[gp node center] at (1.504,0.677) {$10^{0}$};
\draw[gp path] (1.852,0.985)--(1.852,1.075);
\draw[gp path] (1.852,8.631)--(1.852,8.541);
\draw[gp path] (2.313,0.985)--(2.313,1.075);
\draw[gp path] (2.313,8.631)--(2.313,8.541);
\draw[gp path] (2.549,0.985)--(2.549,1.075);
\draw[gp path] (2.549,8.631)--(2.549,8.541);
\draw[gp path] (2.661,0.985)--(2.661,1.165);
\draw[gp path] (2.661,8.631)--(2.661,8.451);
\node[gp node center] at (2.661,0.677) {$10^{1}$};
\draw[gp path] (3.010,0.985)--(3.010,1.075);
\draw[gp path] (3.010,8.631)--(3.010,8.541);
\draw[gp path] (3.470,0.985)--(3.470,1.075);
\draw[gp path] (3.470,8.631)--(3.470,8.541);
\draw[gp path] (3.706,0.985)--(3.706,1.075);
\draw[gp path] (3.706,8.631)--(3.706,8.541);
\draw[gp path] (3.818,0.985)--(3.818,1.165);
\draw[gp path] (3.818,8.631)--(3.818,8.451);
\node[gp node center] at (3.818,0.677) {$10^{2}$};
\draw[gp path] (4.167,0.985)--(4.167,1.075);
\draw[gp path] (4.167,8.631)--(4.167,8.541);
\draw[gp path] (4.627,0.985)--(4.627,1.075);
\draw[gp path] (4.627,8.631)--(4.627,8.541);
\draw[gp path] (4.863,0.985)--(4.863,1.075);
\draw[gp path] (4.863,8.631)--(4.863,8.541);
\draw[gp path] (4.976,0.985)--(4.976,1.165);
\draw[gp path] (4.976,8.631)--(4.976,8.451);
\node[gp node center] at (4.976,0.677) {$10^{3}$};
\draw[gp path] (5.324,0.985)--(5.324,1.075);
\draw[gp path] (5.324,8.631)--(5.324,8.541);
\draw[gp path] (5.784,0.985)--(5.784,1.075);
\draw[gp path] (5.784,8.631)--(5.784,8.541);
\draw[gp path] (6.021,0.985)--(6.021,1.075);
\draw[gp path] (6.021,8.631)--(6.021,8.541);
\draw[gp path] (6.133,0.985)--(6.133,1.165);
\draw[gp path] (6.133,8.631)--(6.133,8.451);
\node[gp node center] at (6.133,0.677) {$10^{4}$};
\draw[gp path] (6.481,0.985)--(6.481,1.075);
\draw[gp path] (6.481,8.631)--(6.481,8.541);
\draw[gp path] (6.941,0.985)--(6.941,1.075);
\draw[gp path] (6.941,8.631)--(6.941,8.541);
\draw[gp path] (7.178,0.985)--(7.178,1.075);
\draw[gp path] (7.178,8.631)--(7.178,8.541);
\draw[gp path] (7.290,0.985)--(7.290,1.165);
\draw[gp path] (7.290,8.631)--(7.290,8.451);
\node[gp node center] at (7.290,0.677) {$10^{5}$};
\draw[gp path] (7.638,0.985)--(7.638,1.075);
\draw[gp path] (7.638,8.631)--(7.638,8.541);
\draw[gp path] (8.099,0.985)--(8.099,1.075);
\draw[gp path] (8.099,8.631)--(8.099,8.541);
\draw[gp path] (8.335,0.985)--(8.335,1.075);
\draw[gp path] (8.335,8.631)--(8.335,8.541);
\draw[gp path] (8.447,0.985)--(8.447,1.165);
\draw[gp path] (8.447,8.631)--(8.447,8.451);
\node[gp node center] at (8.447,0.677) {$10^{6}$};
\draw[gp path] (1.504,8.631)--(1.504,0.985)--(8.447,0.985)--(8.447,8.631)--cycle;
\node[gp node center,rotate=-270] at (0.246,4.808) {\textbf{Function execution time (normalised)}};
\node[gp node center] at (4.975,0.215) {\textbf{Block size}};
\gpcolor{gp lt color axes}
\gpsetlinetype{gp lt plot 0}
\draw[gp path] (1.504,5.573)--(2.661,5.573)--(2.865,5.573)--(3.213,5.573)--(3.358,5.573)%
  --(3.417,5.573)--(3.562,6.337)--(3.765,6.337)--(3.818,7.102)--(4.976,7.866)--(6.133,8.631)%
  --(7.290,8.631)--(8.447,8.631);
\gpsetpointsize{4.00}
\gppoint{gp mark 1}{(1.504,5.573)}
\gppoint{gp mark 1}{(2.661,5.573)}
\gppoint{gp mark 1}{(2.865,5.573)}
\gppoint{gp mark 1}{(3.213,5.573)}
\gppoint{gp mark 1}{(3.358,5.573)}
\gppoint{gp mark 1}{(3.417,5.573)}
\gppoint{gp mark 1}{(3.562,6.337)}
\gppoint{gp mark 1}{(3.765,6.337)}
\gppoint{gp mark 1}{(3.818,7.102)}
\gppoint{gp mark 1}{(4.976,7.866)}
\gppoint{gp mark 1}{(6.133,8.631)}
\gppoint{gp mark 1}{(7.290,8.631)}
\gppoint{gp mark 1}{(8.447,8.631)}
\gpsetlinetype{gp lt axes}
\draw[gp path] (1.504,5.573)--(1.574,5.573)--(1.644,5.573)--(1.714,5.573)--(1.785,5.573)%
  --(1.855,5.573)--(1.925,5.573)--(1.995,5.573)--(2.065,5.573)--(2.135,5.573)--(2.205,5.573)%
  --(2.275,5.573)--(2.346,5.573)--(2.416,5.573)--(2.486,5.573)--(2.556,5.573)--(2.626,5.573)%
  --(2.696,5.573)--(2.766,5.573)--(2.836,5.573)--(2.907,5.573)--(2.977,5.573)--(3.047,5.573)%
  --(3.117,5.573)--(3.187,5.573)--(3.257,5.573)--(3.327,5.573)--(3.398,5.573)--(3.468,5.573)%
  --(3.538,5.573)--(3.608,5.573)--(3.678,5.573)--(3.748,5.573)--(3.818,5.573)--(3.888,5.573)%
  --(3.959,5.573)--(4.029,5.573)--(4.099,5.573)--(4.169,5.573)--(4.239,5.573)--(4.309,5.573)%
  --(4.379,5.573)--(4.450,5.573)--(4.520,5.573)--(4.590,5.573)--(4.660,5.573)--(4.730,5.573)%
  --(4.800,5.573)--(4.870,5.573)--(4.940,5.573)--(5.011,5.573)--(5.081,5.573)--(5.151,5.573)%
  --(5.221,5.573)--(5.291,5.573)--(5.361,5.573)--(5.431,5.573)--(5.501,5.573)--(5.572,5.573)%
  --(5.642,5.573)--(5.712,5.573)--(5.782,5.573)--(5.852,5.573)--(5.922,5.573)--(5.992,5.573)%
  --(6.063,5.573)--(6.133,5.573)--(6.203,5.573)--(6.273,5.573)--(6.343,5.573)--(6.413,5.573)%
  --(6.483,5.573)--(6.553,5.573)--(6.624,5.573)--(6.694,5.573)--(6.764,5.573)--(6.834,5.573)%
  --(6.904,5.573)--(6.974,5.573)--(7.044,5.573)--(7.115,5.573)--(7.185,5.573)--(7.255,5.573)%
  --(7.325,5.573)--(7.395,5.573)--(7.465,5.573)--(7.535,5.573)--(7.605,5.573)--(7.676,5.573)%
  --(7.746,5.573)--(7.816,5.573)--(7.886,5.573)--(7.956,5.573)--(8.026,5.573)--(8.096,5.573)%
  --(8.166,5.573)--(8.237,5.573)--(8.307,5.573)--(8.377,5.573)--(8.447,5.573);
\gpcolor{gp lt color 0}
\gpsetlinetype{gp lt plot 0}
\draw[gp path] (1.504,2.302)--(2.661,2.115)--(2.865,2.101)--(3.213,2.109)--(3.358,2.116)%
  --(3.417,2.123)--(3.562,2.136)--(3.765,2.031)--(3.818,2.068)--(4.976,2.427)--(6.133,8.631);
\gppoint{gp mark 2}{(1.504,2.302)}
\gppoint{gp mark 2}{(2.661,2.115)}
\gppoint{gp mark 2}{(2.865,2.101)}
\gppoint{gp mark 2}{(3.213,2.109)}
\gppoint{gp mark 2}{(3.358,2.116)}
\gppoint{gp mark 2}{(3.417,2.123)}
\gppoint{gp mark 2}{(3.562,2.136)}
\gppoint{gp mark 2}{(3.765,2.031)}
\gppoint{gp mark 2}{(3.818,2.068)}
\gppoint{gp mark 2}{(4.976,2.427)}
\gppoint{gp mark 2}{(6.133,8.631)}
\gpsetlinetype{gp lt axes}
\draw[gp path] (1.504,1.795)--(1.574,1.795)--(1.644,1.795)--(1.714,1.795)--(1.785,1.795)%
  --(1.855,1.795)--(1.925,1.795)--(1.995,1.795)--(2.065,1.795)--(2.135,1.795)--(2.205,1.795)%
  --(2.275,1.795)--(2.346,1.795)--(2.416,1.795)--(2.486,1.795)--(2.556,1.795)--(2.626,1.795)%
  --(2.696,1.795)--(2.766,1.795)--(2.836,1.795)--(2.907,1.795)--(2.977,1.795)--(3.047,1.795)%
  --(3.117,1.795)--(3.187,1.795)--(3.257,1.795)--(3.327,1.795)--(3.398,1.795)--(3.468,1.795)%
  --(3.538,1.795)--(3.608,1.795)--(3.678,1.795)--(3.748,1.795)--(3.818,1.795)--(3.888,1.795)%
  --(3.959,1.795)--(4.029,1.795)--(4.099,1.795)--(4.169,1.795)--(4.239,1.795)--(4.309,1.795)%
  --(4.379,1.795)--(4.450,1.795)--(4.520,1.795)--(4.590,1.795)--(4.660,1.795)--(4.730,1.795)%
  --(4.800,1.795)--(4.870,1.795)--(4.940,1.795)--(5.011,1.795)--(5.081,1.795)--(5.151,1.795)%
  --(5.221,1.795)--(5.291,1.795)--(5.361,1.795)--(5.431,1.795)--(5.501,1.795)--(5.572,1.795)%
  --(5.642,1.795)--(5.712,1.795)--(5.782,1.795)--(5.852,1.795)--(5.922,1.795)--(5.992,1.795)%
  --(6.063,1.795)--(6.133,1.795)--(6.203,1.795)--(6.273,1.795)--(6.343,1.795)--(6.413,1.795)%
  --(6.483,1.795)--(6.553,1.795)--(6.624,1.795)--(6.694,1.795)--(6.764,1.795)--(6.834,1.795)%
  --(6.904,1.795)--(6.974,1.795)--(7.044,1.795)--(7.115,1.795)--(7.185,1.795)--(7.255,1.795)%
  --(7.325,1.795)--(7.395,1.795)--(7.465,1.795)--(7.535,1.795)--(7.605,1.795)--(7.676,1.795)%
  --(7.746,1.795)--(7.816,1.795)--(7.886,1.795)--(7.956,1.795)--(8.026,1.795)--(8.096,1.795)%
  --(8.166,1.795)--(8.237,1.795)--(8.307,1.795)--(8.377,1.795)--(8.447,1.795);
\gpcolor{gp lt color 1}
\gpsetlinetype{gp lt plot 0}
\draw[gp path] (1.504,1.411)--(2.661,1.387)--(2.865,1.353)--(3.213,1.360)--(3.358,1.389)%
  --(3.417,1.381)--(3.562,1.354)--(3.765,1.348)--(3.818,1.356)--(4.976,1.560)--(6.133,4.752)%
  --(7.290,8.631)--(8.447,8.605);
\gppoint{gp mark 3}{(1.504,1.411)}
\gppoint{gp mark 3}{(2.661,1.387)}
\gppoint{gp mark 3}{(2.865,1.353)}
\gppoint{gp mark 3}{(3.213,1.360)}
\gppoint{gp mark 3}{(3.358,1.389)}
\gppoint{gp mark 3}{(3.417,1.381)}
\gppoint{gp mark 3}{(3.562,1.354)}
\gppoint{gp mark 3}{(3.765,1.348)}
\gppoint{gp mark 3}{(3.818,1.356)}
\gppoint{gp mark 3}{(4.976,1.560)}
\gppoint{gp mark 3}{(6.133,4.752)}
\gppoint{gp mark 3}{(7.290,8.631)}
\gppoint{gp mark 3}{(8.447,8.605)}
\gpsetlinetype{gp lt axes}
\draw[gp path] (1.504,1.330)--(1.574,1.330)--(1.644,1.330)--(1.714,1.330)--(1.785,1.330)%
  --(1.855,1.330)--(1.925,1.330)--(1.995,1.330)--(2.065,1.330)--(2.135,1.330)--(2.205,1.330)%
  --(2.275,1.330)--(2.346,1.330)--(2.416,1.330)--(2.486,1.330)--(2.556,1.330)--(2.626,1.330)%
  --(2.696,1.330)--(2.766,1.330)--(2.836,1.330)--(2.907,1.330)--(2.977,1.330)--(3.047,1.330)%
  --(3.117,1.330)--(3.187,1.330)--(3.257,1.330)--(3.327,1.330)--(3.398,1.330)--(3.468,1.330)%
  --(3.538,1.330)--(3.608,1.330)--(3.678,1.330)--(3.748,1.330)--(3.818,1.330)--(3.888,1.330)%
  --(3.959,1.330)--(4.029,1.330)--(4.099,1.330)--(4.169,1.330)--(4.239,1.330)--(4.309,1.330)%
  --(4.379,1.330)--(4.450,1.330)--(4.520,1.330)--(4.590,1.330)--(4.660,1.330)--(4.730,1.330)%
  --(4.800,1.330)--(4.870,1.330)--(4.940,1.330)--(5.011,1.330)--(5.081,1.330)--(5.151,1.330)%
  --(5.221,1.330)--(5.291,1.330)--(5.361,1.330)--(5.431,1.330)--(5.501,1.330)--(5.572,1.330)%
  --(5.642,1.330)--(5.712,1.330)--(5.782,1.330)--(5.852,1.330)--(5.922,1.330)--(5.992,1.330)%
  --(6.063,1.330)--(6.133,1.330)--(6.203,1.330)--(6.273,1.330)--(6.343,1.330)--(6.413,1.330)%
  --(6.483,1.330)--(6.553,1.330)--(6.624,1.330)--(6.694,1.330)--(6.764,1.330)--(6.834,1.330)%
  --(6.904,1.330)--(6.974,1.330)--(7.044,1.330)--(7.115,1.330)--(7.185,1.330)--(7.255,1.330)%
  --(7.325,1.330)--(7.395,1.330)--(7.465,1.330)--(7.535,1.330)--(7.605,1.330)--(7.676,1.330)%
  --(7.746,1.330)--(7.816,1.330)--(7.886,1.330)--(7.956,1.330)--(8.026,1.330)--(8.096,1.330)%
  --(8.166,1.330)--(8.237,1.330)--(8.307,1.330)--(8.377,1.330)--(8.447,1.330);
\gpcolor{gp lt color 2}
\gpsetlinetype{gp lt plot 0}
\draw[gp path] (1.504,1.267)--(2.661,1.209)--(2.865,1.205)--(3.213,1.218)--(3.358,1.199)%
  --(3.417,1.204)--(3.562,1.199)--(3.765,1.216)--(3.818,1.190)--(4.976,1.447)--(6.133,4.967)%
  --(7.290,8.593)--(8.447,8.631);
\gppoint{gp mark 4}{(1.504,1.267)}
\gppoint{gp mark 4}{(2.661,1.209)}
\gppoint{gp mark 4}{(2.865,1.205)}
\gppoint{gp mark 4}{(3.213,1.218)}
\gppoint{gp mark 4}{(3.358,1.199)}
\gppoint{gp mark 4}{(3.417,1.204)}
\gppoint{gp mark 4}{(3.562,1.199)}
\gppoint{gp mark 4}{(3.765,1.216)}
\gppoint{gp mark 4}{(3.818,1.190)}
\gppoint{gp mark 4}{(4.976,1.447)}
\gppoint{gp mark 4}{(6.133,4.967)}
\gppoint{gp mark 4}{(7.290,8.593)}
\gppoint{gp mark 4}{(8.447,8.631)}
\gpsetlinetype{gp lt axes}
\draw[gp path] (1.504,1.139)--(1.574,1.139)--(1.644,1.139)--(1.714,1.139)--(1.785,1.139)%
  --(1.855,1.139)--(1.925,1.139)--(1.995,1.139)--(2.065,1.139)--(2.135,1.139)--(2.205,1.139)%
  --(2.275,1.139)--(2.346,1.139)--(2.416,1.139)--(2.486,1.139)--(2.556,1.139)--(2.626,1.139)%
  --(2.696,1.139)--(2.766,1.139)--(2.836,1.139)--(2.907,1.139)--(2.977,1.139)--(3.047,1.139)%
  --(3.117,1.139)--(3.187,1.139)--(3.257,1.139)--(3.327,1.139)--(3.398,1.139)--(3.468,1.139)%
  --(3.538,1.139)--(3.608,1.139)--(3.678,1.139)--(3.748,1.139)--(3.818,1.139)--(3.888,1.139)%
  --(3.959,1.139)--(4.029,1.139)--(4.099,1.139)--(4.169,1.139)--(4.239,1.139)--(4.309,1.139)%
  --(4.379,1.139)--(4.450,1.139)--(4.520,1.139)--(4.590,1.139)--(4.660,1.139)--(4.730,1.139)%
  --(4.800,1.139)--(4.870,1.139)--(4.940,1.139)--(5.011,1.139)--(5.081,1.139)--(5.151,1.139)%
  --(5.221,1.139)--(5.291,1.139)--(5.361,1.139)--(5.431,1.139)--(5.501,1.139)--(5.572,1.139)%
  --(5.642,1.139)--(5.712,1.139)--(5.782,1.139)--(5.852,1.139)--(5.922,1.139)--(5.992,1.139)%
  --(6.063,1.139)--(6.133,1.139)--(6.203,1.139)--(6.273,1.139)--(6.343,1.139)--(6.413,1.139)%
  --(6.483,1.139)--(6.553,1.139)--(6.624,1.139)--(6.694,1.139)--(6.764,1.139)--(6.834,1.139)%
  --(6.904,1.139)--(6.974,1.139)--(7.044,1.139)--(7.115,1.139)--(7.185,1.139)--(7.255,1.139)%
  --(7.325,1.139)--(7.395,1.139)--(7.465,1.139)--(7.535,1.139)--(7.605,1.139)--(7.676,1.139)%
  --(7.746,1.139)--(7.816,1.139)--(7.886,1.139)--(7.956,1.139)--(8.026,1.139)--(8.096,1.139)%
  --(8.166,1.139)--(8.237,1.139)--(8.307,1.139)--(8.377,1.139)--(8.447,1.139);
\gpcolor{gp lt color 3}
\gpsetlinetype{gp lt plot 0}
\draw[gp path] (1.504,4.406)--(2.661,4.332)--(2.865,4.424)--(3.213,4.534)--(3.358,4.534)%
  --(3.417,4.698)--(3.562,4.442)--(3.765,4.680)--(3.818,4.515)--(4.976,8.027)--(6.133,8.631)%
  --(7.290,8.430)--(8.447,8.411);
\gppoint{gp mark 5}{(1.504,4.406)}
\gppoint{gp mark 5}{(2.661,4.332)}
\gppoint{gp mark 5}{(2.865,4.424)}
\gppoint{gp mark 5}{(3.213,4.534)}
\gppoint{gp mark 5}{(3.358,4.534)}
\gppoint{gp mark 5}{(3.417,4.698)}
\gppoint{gp mark 5}{(3.562,4.442)}
\gppoint{gp mark 5}{(3.765,4.680)}
\gppoint{gp mark 5}{(3.818,4.515)}
\gppoint{gp mark 5}{(4.976,8.027)}
\gppoint{gp mark 5}{(6.133,8.631)}
\gppoint{gp mark 5}{(7.290,8.430)}
\gppoint{gp mark 5}{(8.447,8.411)}
\gpsetlinetype{gp lt axes}
\draw[gp path] (1.504,4.406)--(1.574,4.406)--(1.644,4.406)--(1.714,4.406)--(1.785,4.406)%
  --(1.855,4.406)--(1.925,4.406)--(1.995,4.406)--(2.065,4.406)--(2.135,4.406)--(2.205,4.406)%
  --(2.275,4.406)--(2.346,4.406)--(2.416,4.406)--(2.486,4.406)--(2.556,4.406)--(2.626,4.406)%
  --(2.696,4.406)--(2.766,4.406)--(2.836,4.406)--(2.907,4.406)--(2.977,4.406)--(3.047,4.406)%
  --(3.117,4.406)--(3.187,4.406)--(3.257,4.406)--(3.327,4.406)--(3.398,4.406)--(3.468,4.406)%
  --(3.538,4.406)--(3.608,4.406)--(3.678,4.406)--(3.748,4.406)--(3.818,4.406)--(3.888,4.406)%
  --(3.959,4.406)--(4.029,4.406)--(4.099,4.406)--(4.169,4.406)--(4.239,4.406)--(4.309,4.406)%
  --(4.379,4.406)--(4.450,4.406)--(4.520,4.406)--(4.590,4.406)--(4.660,4.406)--(4.730,4.406)%
  --(4.800,4.406)--(4.870,4.406)--(4.940,4.406)--(5.011,4.406)--(5.081,4.406)--(5.151,4.406)%
  --(5.221,4.406)--(5.291,4.406)--(5.361,4.406)--(5.431,4.406)--(5.501,4.406)--(5.572,4.406)%
  --(5.642,4.406)--(5.712,4.406)--(5.782,4.406)--(5.852,4.406)--(5.922,4.406)--(5.992,4.406)%
  --(6.063,4.406)--(6.133,4.406)--(6.203,4.406)--(6.273,4.406)--(6.343,4.406)--(6.413,4.406)%
  --(6.483,4.406)--(6.553,4.406)--(6.624,4.406)--(6.694,4.406)--(6.764,4.406)--(6.834,4.406)%
  --(6.904,4.406)--(6.974,4.406)--(7.044,4.406)--(7.115,4.406)--(7.185,4.406)--(7.255,4.406)%
  --(7.325,4.406)--(7.395,4.406)--(7.465,4.406)--(7.535,4.406)--(7.605,4.406)--(7.676,4.406)%
  --(7.746,4.406)--(7.816,4.406)--(7.886,4.406)--(7.956,4.406)--(8.026,4.406)--(8.096,4.406)%
  --(8.166,4.406)--(8.237,4.406)--(8.307,4.406)--(8.377,4.406)--(8.447,4.406);
\gpcolor{gp lt color 4}
\gpsetlinetype{gp lt plot 0}
\draw[gp path] (1.504,2.387)--(2.661,2.278)--(2.865,2.472)--(3.213,2.454)--(3.358,2.456)%
  --(3.417,2.380)--(3.562,2.418)--(3.765,2.461)--(3.818,2.270)--(4.976,3.193)--(6.133,8.631)%
  --(7.290,8.477)--(8.447,8.527);
\gppoint{gp mark 6}{(1.504,2.387)}
\gppoint{gp mark 6}{(2.661,2.278)}
\gppoint{gp mark 6}{(2.865,2.472)}
\gppoint{gp mark 6}{(3.213,2.454)}
\gppoint{gp mark 6}{(3.358,2.456)}
\gppoint{gp mark 6}{(3.417,2.380)}
\gppoint{gp mark 6}{(3.562,2.418)}
\gppoint{gp mark 6}{(3.765,2.461)}
\gppoint{gp mark 6}{(3.818,2.270)}
\gppoint{gp mark 6}{(4.976,3.193)}
\gppoint{gp mark 6}{(6.133,8.631)}
\gppoint{gp mark 6}{(7.290,8.477)}
\gppoint{gp mark 6}{(8.447,8.527)}
\gpsetlinetype{gp lt axes}
\draw[gp path] (1.504,2.406)--(1.574,2.406)--(1.644,2.406)--(1.714,2.406)--(1.785,2.406)%
  --(1.855,2.406)--(1.925,2.406)--(1.995,2.406)--(2.065,2.406)--(2.135,2.406)--(2.205,2.406)%
  --(2.275,2.406)--(2.346,2.406)--(2.416,2.406)--(2.486,2.406)--(2.556,2.406)--(2.626,2.406)%
  --(2.696,2.406)--(2.766,2.406)--(2.836,2.406)--(2.907,2.406)--(2.977,2.406)--(3.047,2.406)%
  --(3.117,2.406)--(3.187,2.406)--(3.257,2.406)--(3.327,2.406)--(3.398,2.406)--(3.468,2.406)%
  --(3.538,2.406)--(3.608,2.406)--(3.678,2.406)--(3.748,2.406)--(3.818,2.406)--(3.888,2.406)%
  --(3.959,2.406)--(4.029,2.406)--(4.099,2.406)--(4.169,2.406)--(4.239,2.406)--(4.309,2.406)%
  --(4.379,2.406)--(4.450,2.406)--(4.520,2.406)--(4.590,2.406)--(4.660,2.406)--(4.730,2.406)%
  --(4.800,2.406)--(4.870,2.406)--(4.940,2.406)--(5.011,2.406)--(5.081,2.406)--(5.151,2.406)%
  --(5.221,2.406)--(5.291,2.406)--(5.361,2.406)--(5.431,2.406)--(5.501,2.406)--(5.572,2.406)%
  --(5.642,2.406)--(5.712,2.406)--(5.782,2.406)--(5.852,2.406)--(5.922,2.406)--(5.992,2.406)%
  --(6.063,2.406)--(6.133,2.406)--(6.203,2.406)--(6.273,2.406)--(6.343,2.406)--(6.413,2.406)%
  --(6.483,2.406)--(6.553,2.406)--(6.624,2.406)--(6.694,2.406)--(6.764,2.406)--(6.834,2.406)%
  --(6.904,2.406)--(6.974,2.406)--(7.044,2.406)--(7.115,2.406)--(7.185,2.406)--(7.255,2.406)%
  --(7.325,2.406)--(7.395,2.406)--(7.465,2.406)--(7.535,2.406)--(7.605,2.406)--(7.676,2.406)%
  --(7.746,2.406)--(7.816,2.406)--(7.886,2.406)--(7.956,2.406)--(8.026,2.406)--(8.096,2.406)%
  --(8.166,2.406)--(8.237,2.406)--(8.307,2.406)--(8.377,2.406)--(8.447,2.406);
\gpcolor{gp lt color 5}
\gpsetlinetype{gp lt plot 0}
\draw[gp path] (1.504,1.370)--(2.661,1.400)--(2.865,1.403)--(3.213,1.376)--(3.358,1.409)%
  --(3.417,1.388)--(3.562,1.348)--(3.765,1.326)--(3.818,1.328)--(4.976,1.855)--(6.133,7.875)%
  --(7.290,8.579)--(8.447,8.631);
\gppoint{gp mark 7}{(1.504,1.370)}
\gppoint{gp mark 7}{(2.661,1.400)}
\gppoint{gp mark 7}{(2.865,1.403)}
\gppoint{gp mark 7}{(3.213,1.376)}
\gppoint{gp mark 7}{(3.358,1.409)}
\gppoint{gp mark 7}{(3.417,1.388)}
\gppoint{gp mark 7}{(3.562,1.348)}
\gppoint{gp mark 7}{(3.765,1.326)}
\gppoint{gp mark 7}{(3.818,1.328)}
\gppoint{gp mark 7}{(4.976,1.855)}
\gppoint{gp mark 7}{(6.133,7.875)}
\gppoint{gp mark 7}{(7.290,8.579)}
\gppoint{gp mark 7}{(8.447,8.631)}
\gpsetlinetype{gp lt axes}
\draw[gp path] (1.504,1.316)--(1.574,1.316)--(1.644,1.316)--(1.714,1.316)--(1.785,1.316)%
  --(1.855,1.316)--(1.925,1.316)--(1.995,1.316)--(2.065,1.316)--(2.135,1.316)--(2.205,1.316)%
  --(2.275,1.316)--(2.346,1.316)--(2.416,1.316)--(2.486,1.316)--(2.556,1.316)--(2.626,1.316)%
  --(2.696,1.316)--(2.766,1.316)--(2.836,1.316)--(2.907,1.316)--(2.977,1.316)--(3.047,1.316)%
  --(3.117,1.316)--(3.187,1.316)--(3.257,1.316)--(3.327,1.316)--(3.398,1.316)--(3.468,1.316)%
  --(3.538,1.316)--(3.608,1.316)--(3.678,1.316)--(3.748,1.316)--(3.818,1.316)--(3.888,1.316)%
  --(3.959,1.316)--(4.029,1.316)--(4.099,1.316)--(4.169,1.316)--(4.239,1.316)--(4.309,1.316)%
  --(4.379,1.316)--(4.450,1.316)--(4.520,1.316)--(4.590,1.316)--(4.660,1.316)--(4.730,1.316)%
  --(4.800,1.316)--(4.870,1.316)--(4.940,1.316)--(5.011,1.316)--(5.081,1.316)--(5.151,1.316)%
  --(5.221,1.316)--(5.291,1.316)--(5.361,1.316)--(5.431,1.316)--(5.501,1.316)--(5.572,1.316)%
  --(5.642,1.316)--(5.712,1.316)--(5.782,1.316)--(5.852,1.316)--(5.922,1.316)--(5.992,1.316)%
  --(6.063,1.316)--(6.133,1.316)--(6.203,1.316)--(6.273,1.316)--(6.343,1.316)--(6.413,1.316)%
  --(6.483,1.316)--(6.553,1.316)--(6.624,1.316)--(6.694,1.316)--(6.764,1.316)--(6.834,1.316)%
  --(6.904,1.316)--(6.974,1.316)--(7.044,1.316)--(7.115,1.316)--(7.185,1.316)--(7.255,1.316)%
  --(7.325,1.316)--(7.395,1.316)--(7.465,1.316)--(7.535,1.316)--(7.605,1.316)--(7.676,1.316)%
  --(7.746,1.316)--(7.816,1.316)--(7.886,1.316)--(7.956,1.316)--(8.026,1.316)--(8.096,1.316)%
  --(8.166,1.316)--(8.237,1.316)--(8.307,1.316)--(8.377,1.316)--(8.447,1.316);
\gpcolor{gp lt color 6}
\gpsetlinetype{gp lt plot 0}
\draw[gp path] (1.504,4.127)--(2.661,3.656)--(2.865,3.719)--(3.213,3.730)--(3.358,3.690)%
  --(3.417,3.657)--(3.562,3.707)--(3.765,3.690)--(3.818,3.691)--(4.976,4.179)--(6.133,8.631)%
  --(7.290,8.590)--(8.447,8.550);
\gppoint{gp mark 8}{(1.504,4.127)}
\gppoint{gp mark 8}{(2.661,3.656)}
\gppoint{gp mark 8}{(2.865,3.719)}
\gppoint{gp mark 8}{(3.213,3.730)}
\gppoint{gp mark 8}{(3.358,3.690)}
\gppoint{gp mark 8}{(3.417,3.657)}
\gppoint{gp mark 8}{(3.562,3.707)}
\gppoint{gp mark 8}{(3.765,3.690)}
\gppoint{gp mark 8}{(3.818,3.691)}
\gppoint{gp mark 8}{(4.976,4.179)}
\gppoint{gp mark 8}{(6.133,8.631)}
\gppoint{gp mark 8}{(7.290,8.590)}
\gppoint{gp mark 8}{(8.447,8.550)}
\gpsetlinetype{gp lt axes}
\draw[gp path] (1.504,3.978)--(1.574,3.978)--(1.644,3.978)--(1.714,3.978)--(1.785,3.978)%
  --(1.855,3.978)--(1.925,3.978)--(1.995,3.978)--(2.065,3.978)--(2.135,3.978)--(2.205,3.978)%
  --(2.275,3.978)--(2.346,3.978)--(2.416,3.978)--(2.486,3.978)--(2.556,3.978)--(2.626,3.978)%
  --(2.696,3.978)--(2.766,3.978)--(2.836,3.978)--(2.907,3.978)--(2.977,3.978)--(3.047,3.978)%
  --(3.117,3.978)--(3.187,3.978)--(3.257,3.978)--(3.327,3.978)--(3.398,3.978)--(3.468,3.978)%
  --(3.538,3.978)--(3.608,3.978)--(3.678,3.978)--(3.748,3.978)--(3.818,3.978)--(3.888,3.978)%
  --(3.959,3.978)--(4.029,3.978)--(4.099,3.978)--(4.169,3.978)--(4.239,3.978)--(4.309,3.978)%
  --(4.379,3.978)--(4.450,3.978)--(4.520,3.978)--(4.590,3.978)--(4.660,3.978)--(4.730,3.978)%
  --(4.800,3.978)--(4.870,3.978)--(4.940,3.978)--(5.011,3.978)--(5.081,3.978)--(5.151,3.978)%
  --(5.221,3.978)--(5.291,3.978)--(5.361,3.978)--(5.431,3.978)--(5.501,3.978)--(5.572,3.978)%
  --(5.642,3.978)--(5.712,3.978)--(5.782,3.978)--(5.852,3.978)--(5.922,3.978)--(5.992,3.978)%
  --(6.063,3.978)--(6.133,3.978)--(6.203,3.978)--(6.273,3.978)--(6.343,3.978)--(6.413,3.978)%
  --(6.483,3.978)--(6.553,3.978)--(6.624,3.978)--(6.694,3.978)--(6.764,3.978)--(6.834,3.978)%
  --(6.904,3.978)--(6.974,3.978)--(7.044,3.978)--(7.115,3.978)--(7.185,3.978)--(7.255,3.978)%
  --(7.325,3.978)--(7.395,3.978)--(7.465,3.978)--(7.535,3.978)--(7.605,3.978)--(7.676,3.978)%
  --(7.746,3.978)--(7.816,3.978)--(7.886,3.978)--(7.956,3.978)--(8.026,3.978)--(8.096,3.978)%
  --(8.166,3.978)--(8.237,3.978)--(8.307,3.978)--(8.377,3.978)--(8.447,3.978);
\gpcolor{gp lt color 7}
\gpsetlinetype{gp lt plot 0}
\draw[gp path] (1.504,5.671)--(2.661,5.425)--(2.865,5.425)--(3.213,5.671)--(3.358,5.425)%
  --(3.417,5.671)--(3.562,5.425)--(3.765,5.425)--(3.818,5.671)--(4.976,8.606)--(6.133,8.631)%
  --(7.290,8.631)--(8.447,8.631);
\gppoint{gp mark 9}{(1.504,5.671)}
\gppoint{gp mark 9}{(2.661,5.425)}
\gppoint{gp mark 9}{(2.865,5.425)}
\gppoint{gp mark 9}{(3.213,5.671)}
\gppoint{gp mark 9}{(3.358,5.425)}
\gppoint{gp mark 9}{(3.417,5.671)}
\gppoint{gp mark 9}{(3.562,5.425)}
\gppoint{gp mark 9}{(3.765,5.425)}
\gppoint{gp mark 9}{(3.818,5.671)}
\gppoint{gp mark 9}{(4.976,8.606)}
\gppoint{gp mark 9}{(6.133,8.631)}
\gppoint{gp mark 9}{(7.290,8.631)}
\gppoint{gp mark 9}{(8.447,8.631)}
\gpsetlinetype{gp lt axes}
\draw[gp path] (1.504,5.425)--(1.574,5.425)--(1.644,5.425)--(1.714,5.425)--(1.785,5.425)%
  --(1.855,5.425)--(1.925,5.425)--(1.995,5.425)--(2.065,5.425)--(2.135,5.425)--(2.205,5.425)%
  --(2.275,5.425)--(2.346,5.425)--(2.416,5.425)--(2.486,5.425)--(2.556,5.425)--(2.626,5.425)%
  --(2.696,5.425)--(2.766,5.425)--(2.836,5.425)--(2.907,5.425)--(2.977,5.425)--(3.047,5.425)%
  --(3.117,5.425)--(3.187,5.425)--(3.257,5.425)--(3.327,5.425)--(3.398,5.425)--(3.468,5.425)%
  --(3.538,5.425)--(3.608,5.425)--(3.678,5.425)--(3.748,5.425)--(3.818,5.425)--(3.888,5.425)%
  --(3.959,5.425)--(4.029,5.425)--(4.099,5.425)--(4.169,5.425)--(4.239,5.425)--(4.309,5.425)%
  --(4.379,5.425)--(4.450,5.425)--(4.520,5.425)--(4.590,5.425)--(4.660,5.425)--(4.730,5.425)%
  --(4.800,5.425)--(4.870,5.425)--(4.940,5.425)--(5.011,5.425)--(5.081,5.425)--(5.151,5.425)%
  --(5.221,5.425)--(5.291,5.425)--(5.361,5.425)--(5.431,5.425)--(5.501,5.425)--(5.572,5.425)%
  --(5.642,5.425)--(5.712,5.425)--(5.782,5.425)--(5.852,5.425)--(5.922,5.425)--(5.992,5.425)%
  --(6.063,5.425)--(6.133,5.425)--(6.203,5.425)--(6.273,5.425)--(6.343,5.425)--(6.413,5.425)%
  --(6.483,5.425)--(6.553,5.425)--(6.624,5.425)--(6.694,5.425)--(6.764,5.425)--(6.834,5.425)%
  --(6.904,5.425)--(6.974,5.425)--(7.044,5.425)--(7.115,5.425)--(7.185,5.425)--(7.255,5.425)%
  --(7.325,5.425)--(7.395,5.425)--(7.465,5.425)--(7.535,5.425)--(7.605,5.425)--(7.676,5.425)%
  --(7.746,5.425)--(7.816,5.425)--(7.886,5.425)--(7.956,5.425)--(8.026,5.425)--(8.096,5.425)%
  --(8.166,5.425)--(8.237,5.425)--(8.307,5.425)--(8.377,5.425)--(8.447,5.425);
\gpcolor{gp lt color 0}
\gpsetlinetype{gp lt plot 0}
\draw[gp path] (1.504,4.046)--(2.661,3.586)--(2.865,3.590)--(3.213,3.578)--(3.358,3.613)%
  --(3.417,3.544)--(3.562,3.542)--(3.765,3.566)--(3.818,3.555)--(4.976,3.887)--(6.133,6.991)%
  --(7.290,8.264)--(8.447,8.631);
\gppoint{gp mark 10}{(1.504,4.046)}
\gppoint{gp mark 10}{(2.661,3.586)}
\gppoint{gp mark 10}{(2.865,3.590)}
\gppoint{gp mark 10}{(3.213,3.578)}
\gppoint{gp mark 10}{(3.358,3.613)}
\gppoint{gp mark 10}{(3.417,3.544)}
\gppoint{gp mark 10}{(3.562,3.542)}
\gppoint{gp mark 10}{(3.765,3.566)}
\gppoint{gp mark 10}{(3.818,3.555)}
\gppoint{gp mark 10}{(4.976,3.887)}
\gppoint{gp mark 10}{(6.133,6.991)}
\gppoint{gp mark 10}{(7.290,8.264)}
\gppoint{gp mark 10}{(8.447,8.631)}
\gpsetlinetype{gp lt axes}
\draw[gp path] (1.504,3.850)--(1.574,3.850)--(1.644,3.850)--(1.714,3.850)--(1.785,3.850)%
  --(1.855,3.850)--(1.925,3.850)--(1.995,3.850)--(2.065,3.850)--(2.135,3.850)--(2.205,3.850)%
  --(2.275,3.850)--(2.346,3.850)--(2.416,3.850)--(2.486,3.850)--(2.556,3.850)--(2.626,3.850)%
  --(2.696,3.850)--(2.766,3.850)--(2.836,3.850)--(2.907,3.850)--(2.977,3.850)--(3.047,3.850)%
  --(3.117,3.850)--(3.187,3.850)--(3.257,3.850)--(3.327,3.850)--(3.398,3.850)--(3.468,3.850)%
  --(3.538,3.850)--(3.608,3.850)--(3.678,3.850)--(3.748,3.850)--(3.818,3.850)--(3.888,3.850)%
  --(3.959,3.850)--(4.029,3.850)--(4.099,3.850)--(4.169,3.850)--(4.239,3.850)--(4.309,3.850)%
  --(4.379,3.850)--(4.450,3.850)--(4.520,3.850)--(4.590,3.850)--(4.660,3.850)--(4.730,3.850)%
  --(4.800,3.850)--(4.870,3.850)--(4.940,3.850)--(5.011,3.850)--(5.081,3.850)--(5.151,3.850)%
  --(5.221,3.850)--(5.291,3.850)--(5.361,3.850)--(5.431,3.850)--(5.501,3.850)--(5.572,3.850)%
  --(5.642,3.850)--(5.712,3.850)--(5.782,3.850)--(5.852,3.850)--(5.922,3.850)--(5.992,3.850)%
  --(6.063,3.850)--(6.133,3.850)--(6.203,3.850)--(6.273,3.850)--(6.343,3.850)--(6.413,3.850)%
  --(6.483,3.850)--(6.553,3.850)--(6.624,3.850)--(6.694,3.850)--(6.764,3.850)--(6.834,3.850)%
  --(6.904,3.850)--(6.974,3.850)--(7.044,3.850)--(7.115,3.850)--(7.185,3.850)--(7.255,3.850)%
  --(7.325,3.850)--(7.395,3.850)--(7.465,3.850)--(7.535,3.850)--(7.605,3.850)--(7.676,3.850)%
  --(7.746,3.850)--(7.816,3.850)--(7.886,3.850)--(7.956,3.850)--(8.026,3.850)--(8.096,3.850)%
  --(8.166,3.850)--(8.237,3.850)--(8.307,3.850)--(8.377,3.850)--(8.447,3.850);
\gpcolor{gp lt color 1}
\gpsetlinetype{gp lt plot 0}
\draw[gp path] (1.504,4.405)--(2.661,3.684)--(2.865,3.652)--(3.213,3.640)--(3.358,3.606)%
  --(3.417,3.639)--(3.562,3.604)--(3.765,3.621)--(3.818,3.605)--(4.976,3.827)--(6.133,6.374)%
  --(7.290,8.631)--(8.447,8.596);
\gppoint{gp mark 11}{(1.504,4.405)}
\gppoint{gp mark 11}{(2.661,3.684)}
\gppoint{gp mark 11}{(2.865,3.652)}
\gppoint{gp mark 11}{(3.213,3.640)}
\gppoint{gp mark 11}{(3.358,3.606)}
\gppoint{gp mark 11}{(3.417,3.639)}
\gppoint{gp mark 11}{(3.562,3.604)}
\gppoint{gp mark 11}{(3.765,3.621)}
\gppoint{gp mark 11}{(3.818,3.605)}
\gppoint{gp mark 11}{(4.976,3.827)}
\gppoint{gp mark 11}{(6.133,6.374)}
\gppoint{gp mark 11}{(7.290,8.631)}
\gppoint{gp mark 11}{(8.447,8.596)}
\gpsetlinetype{gp lt axes}
\draw[gp path] (1.504,3.906)--(1.574,3.906)--(1.644,3.906)--(1.714,3.906)--(1.785,3.906)%
  --(1.855,3.906)--(1.925,3.906)--(1.995,3.906)--(2.065,3.906)--(2.135,3.906)--(2.205,3.906)%
  --(2.275,3.906)--(2.346,3.906)--(2.416,3.906)--(2.486,3.906)--(2.556,3.906)--(2.626,3.906)%
  --(2.696,3.906)--(2.766,3.906)--(2.836,3.906)--(2.907,3.906)--(2.977,3.906)--(3.047,3.906)%
  --(3.117,3.906)--(3.187,3.906)--(3.257,3.906)--(3.327,3.906)--(3.398,3.906)--(3.468,3.906)%
  --(3.538,3.906)--(3.608,3.906)--(3.678,3.906)--(3.748,3.906)--(3.818,3.906)--(3.888,3.906)%
  --(3.959,3.906)--(4.029,3.906)--(4.099,3.906)--(4.169,3.906)--(4.239,3.906)--(4.309,3.906)%
  --(4.379,3.906)--(4.450,3.906)--(4.520,3.906)--(4.590,3.906)--(4.660,3.906)--(4.730,3.906)%
  --(4.800,3.906)--(4.870,3.906)--(4.940,3.906)--(5.011,3.906)--(5.081,3.906)--(5.151,3.906)%
  --(5.221,3.906)--(5.291,3.906)--(5.361,3.906)--(5.431,3.906)--(5.501,3.906)--(5.572,3.906)%
  --(5.642,3.906)--(5.712,3.906)--(5.782,3.906)--(5.852,3.906)--(5.922,3.906)--(5.992,3.906)%
  --(6.063,3.906)--(6.133,3.906)--(6.203,3.906)--(6.273,3.906)--(6.343,3.906)--(6.413,3.906)%
  --(6.483,3.906)--(6.553,3.906)--(6.624,3.906)--(6.694,3.906)--(6.764,3.906)--(6.834,3.906)%
  --(6.904,3.906)--(6.974,3.906)--(7.044,3.906)--(7.115,3.906)--(7.185,3.906)--(7.255,3.906)%
  --(7.325,3.906)--(7.395,3.906)--(7.465,3.906)--(7.535,3.906)--(7.605,3.906)--(7.676,3.906)%
  --(7.746,3.906)--(7.816,3.906)--(7.886,3.906)--(7.956,3.906)--(8.026,3.906)--(8.096,3.906)%
  --(8.166,3.906)--(8.237,3.906)--(8.307,3.906)--(8.377,3.906)--(8.447,3.906);
\gpcolor{gp lt color 2}
\gpsetlinetype{gp lt plot 0}
\draw[gp path] (1.504,5.357)--(2.661,4.776)--(2.865,4.826)--(3.213,4.783)--(3.358,4.743)%
  --(3.417,4.739)--(3.562,4.737)--(3.765,4.759)--(3.818,4.711)--(4.976,5.068)--(6.133,8.631);
\gppoint{gp mark 12}{(1.504,5.357)}
\gppoint{gp mark 12}{(2.661,4.776)}
\gppoint{gp mark 12}{(2.865,4.826)}
\gppoint{gp mark 12}{(3.213,4.783)}
\gppoint{gp mark 12}{(3.358,4.743)}
\gppoint{gp mark 12}{(3.417,4.739)}
\gppoint{gp mark 12}{(3.562,4.737)}
\gppoint{gp mark 12}{(3.765,4.759)}
\gppoint{gp mark 12}{(3.818,4.711)}
\gppoint{gp mark 12}{(4.976,5.068)}
\gppoint{gp mark 12}{(6.133,8.631)}
\gpsetlinetype{gp lt axes}
\draw[gp path] (1.504,5.109)--(1.574,5.109)--(1.644,5.109)--(1.714,5.109)--(1.785,5.109)%
  --(1.855,5.109)--(1.925,5.109)--(1.995,5.109)--(2.065,5.109)--(2.135,5.109)--(2.205,5.109)%
  --(2.275,5.109)--(2.346,5.109)--(2.416,5.109)--(2.486,5.109)--(2.556,5.109)--(2.626,5.109)%
  --(2.696,5.109)--(2.766,5.109)--(2.836,5.109)--(2.907,5.109)--(2.977,5.109)--(3.047,5.109)%
  --(3.117,5.109)--(3.187,5.109)--(3.257,5.109)--(3.327,5.109)--(3.398,5.109)--(3.468,5.109)%
  --(3.538,5.109)--(3.608,5.109)--(3.678,5.109)--(3.748,5.109)--(3.818,5.109)--(3.888,5.109)%
  --(3.959,5.109)--(4.029,5.109)--(4.099,5.109)--(4.169,5.109)--(4.239,5.109)--(4.309,5.109)%
  --(4.379,5.109)--(4.450,5.109)--(4.520,5.109)--(4.590,5.109)--(4.660,5.109)--(4.730,5.109)%
  --(4.800,5.109)--(4.870,5.109)--(4.940,5.109)--(5.011,5.109)--(5.081,5.109)--(5.151,5.109)%
  --(5.221,5.109)--(5.291,5.109)--(5.361,5.109)--(5.431,5.109)--(5.501,5.109)--(5.572,5.109)%
  --(5.642,5.109)--(5.712,5.109)--(5.782,5.109)--(5.852,5.109)--(5.922,5.109)--(5.992,5.109)%
  --(6.063,5.109)--(6.133,5.109)--(6.203,5.109)--(6.273,5.109)--(6.343,5.109)--(6.413,5.109)%
  --(6.483,5.109)--(6.553,5.109)--(6.624,5.109)--(6.694,5.109)--(6.764,5.109)--(6.834,5.109)%
  --(6.904,5.109)--(6.974,5.109)--(7.044,5.109)--(7.115,5.109)--(7.185,5.109)--(7.255,5.109)%
  --(7.325,5.109)--(7.395,5.109)--(7.465,5.109)--(7.535,5.109)--(7.605,5.109)--(7.676,5.109)%
  --(7.746,5.109)--(7.816,5.109)--(7.886,5.109)--(7.956,5.109)--(8.026,5.109)--(8.096,5.109)%
  --(8.166,5.109)--(8.237,5.109)--(8.307,5.109)--(8.377,5.109)--(8.447,5.109);
\gpcolor{gp lt color 3}
\gpsetlinetype{gp lt plot 0}
\draw[gp path] (1.504,6.045)--(2.661,5.391)--(2.865,5.340)--(3.213,5.338)--(3.358,5.341)%
  --(3.417,5.304)--(3.562,5.339)--(3.765,5.322)--(3.818,5.305)--(4.976,5.638)--(6.133,8.631);
\gppoint{gp mark 13}{(1.504,6.045)}
\gppoint{gp mark 13}{(2.661,5.391)}
\gppoint{gp mark 13}{(2.865,5.340)}
\gppoint{gp mark 13}{(3.213,5.338)}
\gppoint{gp mark 13}{(3.358,5.341)}
\gppoint{gp mark 13}{(3.417,5.304)}
\gppoint{gp mark 13}{(3.562,5.339)}
\gppoint{gp mark 13}{(3.765,5.322)}
\gppoint{gp mark 13}{(3.818,5.305)}
\gppoint{gp mark 13}{(4.976,5.638)}
\gppoint{gp mark 13}{(6.133,8.631)}
\gpsetlinetype{gp lt axes}
\draw[gp path] (1.504,5.852)--(1.574,5.852)--(1.644,5.852)--(1.714,5.852)--(1.785,5.852)%
  --(1.855,5.852)--(1.925,5.852)--(1.995,5.852)--(2.065,5.852)--(2.135,5.852)--(2.205,5.852)%
  --(2.275,5.852)--(2.346,5.852)--(2.416,5.852)--(2.486,5.852)--(2.556,5.852)--(2.626,5.852)%
  --(2.696,5.852)--(2.766,5.852)--(2.836,5.852)--(2.907,5.852)--(2.977,5.852)--(3.047,5.852)%
  --(3.117,5.852)--(3.187,5.852)--(3.257,5.852)--(3.327,5.852)--(3.398,5.852)--(3.468,5.852)%
  --(3.538,5.852)--(3.608,5.852)--(3.678,5.852)--(3.748,5.852)--(3.818,5.852)--(3.888,5.852)%
  --(3.959,5.852)--(4.029,5.852)--(4.099,5.852)--(4.169,5.852)--(4.239,5.852)--(4.309,5.852)%
  --(4.379,5.852)--(4.450,5.852)--(4.520,5.852)--(4.590,5.852)--(4.660,5.852)--(4.730,5.852)%
  --(4.800,5.852)--(4.870,5.852)--(4.940,5.852)--(5.011,5.852)--(5.081,5.852)--(5.151,5.852)%
  --(5.221,5.852)--(5.291,5.852)--(5.361,5.852)--(5.431,5.852)--(5.501,5.852)--(5.572,5.852)%
  --(5.642,5.852)--(5.712,5.852)--(5.782,5.852)--(5.852,5.852)--(5.922,5.852)--(5.992,5.852)%
  --(6.063,5.852)--(6.133,5.852)--(6.203,5.852)--(6.273,5.852)--(6.343,5.852)--(6.413,5.852)%
  --(6.483,5.852)--(6.553,5.852)--(6.624,5.852)--(6.694,5.852)--(6.764,5.852)--(6.834,5.852)%
  --(6.904,5.852)--(6.974,5.852)--(7.044,5.852)--(7.115,5.852)--(7.185,5.852)--(7.255,5.852)%
  --(7.325,5.852)--(7.395,5.852)--(7.465,5.852)--(7.535,5.852)--(7.605,5.852)--(7.676,5.852)%
  --(7.746,5.852)--(7.816,5.852)--(7.886,5.852)--(7.956,5.852)--(8.026,5.852)--(8.096,5.852)%
  --(8.166,5.852)--(8.237,5.852)--(8.307,5.852)--(8.377,5.852)--(8.447,5.852);
\gpcolor{gp lt color 4}
\gpsetlinetype{gp lt plot 0}
\draw[gp path] (1.504,2.757)--(2.661,2.698)--(2.865,2.752)--(3.213,2.746)--(3.358,2.762)%
  --(3.417,2.805)--(3.562,2.811)--(3.765,2.902)--(3.818,2.859)--(4.976,5.350)--(6.133,8.475)%
  --(7.290,8.491)--(8.447,8.631);
\gppoint{gp mark 14}{(1.504,2.757)}
\gppoint{gp mark 14}{(2.661,2.698)}
\gppoint{gp mark 14}{(2.865,2.752)}
\gppoint{gp mark 14}{(3.213,2.746)}
\gppoint{gp mark 14}{(3.358,2.762)}
\gppoint{gp mark 14}{(3.417,2.805)}
\gppoint{gp mark 14}{(3.562,2.811)}
\gppoint{gp mark 14}{(3.765,2.902)}
\gppoint{gp mark 14}{(3.818,2.859)}
\gppoint{gp mark 14}{(4.976,5.350)}
\gppoint{gp mark 14}{(6.133,8.475)}
\gppoint{gp mark 14}{(7.290,8.491)}
\gppoint{gp mark 14}{(8.447,8.631)}
\gpsetlinetype{gp lt axes}
\draw[gp path] (1.504,2.650)--(1.574,2.650)--(1.644,2.650)--(1.714,2.650)--(1.785,2.650)%
  --(1.855,2.650)--(1.925,2.650)--(1.995,2.650)--(2.065,2.650)--(2.135,2.650)--(2.205,2.650)%
  --(2.275,2.650)--(2.346,2.650)--(2.416,2.650)--(2.486,2.650)--(2.556,2.650)--(2.626,2.650)%
  --(2.696,2.650)--(2.766,2.650)--(2.836,2.650)--(2.907,2.650)--(2.977,2.650)--(3.047,2.650)%
  --(3.117,2.650)--(3.187,2.650)--(3.257,2.650)--(3.327,2.650)--(3.398,2.650)--(3.468,2.650)%
  --(3.538,2.650)--(3.608,2.650)--(3.678,2.650)--(3.748,2.650)--(3.818,2.650)--(3.888,2.650)%
  --(3.959,2.650)--(4.029,2.650)--(4.099,2.650)--(4.169,2.650)--(4.239,2.650)--(4.309,2.650)%
  --(4.379,2.650)--(4.450,2.650)--(4.520,2.650)--(4.590,2.650)--(4.660,2.650)--(4.730,2.650)%
  --(4.800,2.650)--(4.870,2.650)--(4.940,2.650)--(5.011,2.650)--(5.081,2.650)--(5.151,2.650)%
  --(5.221,2.650)--(5.291,2.650)--(5.361,2.650)--(5.431,2.650)--(5.501,2.650)--(5.572,2.650)%
  --(5.642,2.650)--(5.712,2.650)--(5.782,2.650)--(5.852,2.650)--(5.922,2.650)--(5.992,2.650)%
  --(6.063,2.650)--(6.133,2.650)--(6.203,2.650)--(6.273,2.650)--(6.343,2.650)--(6.413,2.650)%
  --(6.483,2.650)--(6.553,2.650)--(6.624,2.650)--(6.694,2.650)--(6.764,2.650)--(6.834,2.650)%
  --(6.904,2.650)--(6.974,2.650)--(7.044,2.650)--(7.115,2.650)--(7.185,2.650)--(7.255,2.650)%
  --(7.325,2.650)--(7.395,2.650)--(7.465,2.650)--(7.535,2.650)--(7.605,2.650)--(7.676,2.650)%
  --(7.746,2.650)--(7.816,2.650)--(7.886,2.650)--(7.956,2.650)--(8.026,2.650)--(8.096,2.650)%
  --(8.166,2.650)--(8.237,2.650)--(8.307,2.650)--(8.377,2.650)--(8.447,2.650);
\gpcolor{gp lt color 5}
\gpsetlinetype{gp lt plot 0}
\draw[gp path] (1.504,2.068)--(2.661,2.008)--(2.865,1.586)--(3.213,2.047)--(3.358,1.714)%
  --(3.417,1.732)--(3.562,1.828)--(3.765,1.954)--(3.818,1.669)--(4.976,2.870)--(6.133,8.631)%
  --(7.290,8.178)--(8.447,8.125);
\gppoint{gp mark 15}{(1.504,2.068)}
\gppoint{gp mark 15}{(2.661,2.008)}
\gppoint{gp mark 15}{(2.865,1.586)}
\gppoint{gp mark 15}{(3.213,2.047)}
\gppoint{gp mark 15}{(3.358,1.714)}
\gppoint{gp mark 15}{(3.417,1.732)}
\gppoint{gp mark 15}{(3.562,1.828)}
\gppoint{gp mark 15}{(3.765,1.954)}
\gppoint{gp mark 15}{(3.818,1.669)}
\gppoint{gp mark 15}{(4.976,2.870)}
\gppoint{gp mark 15}{(6.133,8.631)}
\gppoint{gp mark 15}{(7.290,8.178)}
\gppoint{gp mark 15}{(8.447,8.125)}
\gpsetlinetype{gp lt axes}
\draw[gp path] (1.504,2.024)--(1.574,2.024)--(1.644,2.024)--(1.714,2.024)--(1.785,2.024)%
  --(1.855,2.024)--(1.925,2.024)--(1.995,2.024)--(2.065,2.024)--(2.135,2.024)--(2.205,2.024)%
  --(2.275,2.024)--(2.346,2.024)--(2.416,2.024)--(2.486,2.024)--(2.556,2.024)--(2.626,2.024)%
  --(2.696,2.024)--(2.766,2.024)--(2.836,2.024)--(2.907,2.024)--(2.977,2.024)--(3.047,2.024)%
  --(3.117,2.024)--(3.187,2.024)--(3.257,2.024)--(3.327,2.024)--(3.398,2.024)--(3.468,2.024)%
  --(3.538,2.024)--(3.608,2.024)--(3.678,2.024)--(3.748,2.024)--(3.818,2.024)--(3.888,2.024)%
  --(3.959,2.024)--(4.029,2.024)--(4.099,2.024)--(4.169,2.024)--(4.239,2.024)--(4.309,2.024)%
  --(4.379,2.024)--(4.450,2.024)--(4.520,2.024)--(4.590,2.024)--(4.660,2.024)--(4.730,2.024)%
  --(4.800,2.024)--(4.870,2.024)--(4.940,2.024)--(5.011,2.024)--(5.081,2.024)--(5.151,2.024)%
  --(5.221,2.024)--(5.291,2.024)--(5.361,2.024)--(5.431,2.024)--(5.501,2.024)--(5.572,2.024)%
  --(5.642,2.024)--(5.712,2.024)--(5.782,2.024)--(5.852,2.024)--(5.922,2.024)--(5.992,2.024)%
  --(6.063,2.024)--(6.133,2.024)--(6.203,2.024)--(6.273,2.024)--(6.343,2.024)--(6.413,2.024)%
  --(6.483,2.024)--(6.553,2.024)--(6.624,2.024)--(6.694,2.024)--(6.764,2.024)--(6.834,2.024)%
  --(6.904,2.024)--(6.974,2.024)--(7.044,2.024)--(7.115,2.024)--(7.185,2.024)--(7.255,2.024)%
  --(7.325,2.024)--(7.395,2.024)--(7.465,2.024)--(7.535,2.024)--(7.605,2.024)--(7.676,2.024)%
  --(7.746,2.024)--(7.816,2.024)--(7.886,2.024)--(7.956,2.024)--(8.026,2.024)--(8.096,2.024)%
  --(8.166,2.024)--(8.237,2.024)--(8.307,2.024)--(8.377,2.024)--(8.447,2.024);
\gpcolor{gp lt color 6}
\gpsetlinetype{gp lt plot 0}
\draw[gp path] (1.504,1.797)--(2.661,2.058)--(2.865,1.793)--(3.213,1.680)--(3.358,2.013)%
  --(3.417,1.684)--(3.562,2.035)--(3.765,1.822)--(3.818,1.683)--(4.976,2.799)--(6.133,8.233)%
  --(7.290,8.538)--(8.447,8.631);
\gppoint{gp mark 1}{(1.504,1.797)}
\gppoint{gp mark 1}{(2.661,2.058)}
\gppoint{gp mark 1}{(2.865,1.793)}
\gppoint{gp mark 1}{(3.213,1.680)}
\gppoint{gp mark 1}{(3.358,2.013)}
\gppoint{gp mark 1}{(3.417,1.684)}
\gppoint{gp mark 1}{(3.562,2.035)}
\gppoint{gp mark 1}{(3.765,1.822)}
\gppoint{gp mark 1}{(3.818,1.683)}
\gppoint{gp mark 1}{(4.976,2.799)}
\gppoint{gp mark 1}{(6.133,8.233)}
\gppoint{gp mark 1}{(7.290,8.538)}
\gppoint{gp mark 1}{(8.447,8.631)}
\gpsetlinetype{gp lt axes}
\draw[gp path] (1.504,1.668)--(1.574,1.668)--(1.644,1.668)--(1.714,1.668)--(1.785,1.668)%
  --(1.855,1.668)--(1.925,1.668)--(1.995,1.668)--(2.065,1.668)--(2.135,1.668)--(2.205,1.668)%
  --(2.275,1.668)--(2.346,1.668)--(2.416,1.668)--(2.486,1.668)--(2.556,1.668)--(2.626,1.668)%
  --(2.696,1.668)--(2.766,1.668)--(2.836,1.668)--(2.907,1.668)--(2.977,1.668)--(3.047,1.668)%
  --(3.117,1.668)--(3.187,1.668)--(3.257,1.668)--(3.327,1.668)--(3.398,1.668)--(3.468,1.668)%
  --(3.538,1.668)--(3.608,1.668)--(3.678,1.668)--(3.748,1.668)--(3.818,1.668)--(3.888,1.668)%
  --(3.959,1.668)--(4.029,1.668)--(4.099,1.668)--(4.169,1.668)--(4.239,1.668)--(4.309,1.668)%
  --(4.379,1.668)--(4.450,1.668)--(4.520,1.668)--(4.590,1.668)--(4.660,1.668)--(4.730,1.668)%
  --(4.800,1.668)--(4.870,1.668)--(4.940,1.668)--(5.011,1.668)--(5.081,1.668)--(5.151,1.668)%
  --(5.221,1.668)--(5.291,1.668)--(5.361,1.668)--(5.431,1.668)--(5.501,1.668)--(5.572,1.668)%
  --(5.642,1.668)--(5.712,1.668)--(5.782,1.668)--(5.852,1.668)--(5.922,1.668)--(5.992,1.668)%
  --(6.063,1.668)--(6.133,1.668)--(6.203,1.668)--(6.273,1.668)--(6.343,1.668)--(6.413,1.668)%
  --(6.483,1.668)--(6.553,1.668)--(6.624,1.668)--(6.694,1.668)--(6.764,1.668)--(6.834,1.668)%
  --(6.904,1.668)--(6.974,1.668)--(7.044,1.668)--(7.115,1.668)--(7.185,1.668)--(7.255,1.668)%
  --(7.325,1.668)--(7.395,1.668)--(7.465,1.668)--(7.535,1.668)--(7.605,1.668)--(7.676,1.668)%
  --(7.746,1.668)--(7.816,1.668)--(7.886,1.668)--(7.956,1.668)--(8.026,1.668)--(8.096,1.668)%
  --(8.166,1.668)--(8.237,1.668)--(8.307,1.668)--(8.377,1.668)--(8.447,1.668);
\gpcolor{gp lt color 7}
\gpsetlinetype{gp lt plot 0}
\draw[gp path] (1.504,5.900)--(2.661,5.354)--(2.865,5.354)--(3.213,5.900)--(3.358,6.446)%
  --(3.417,5.900)--(3.562,5.900)--(3.765,6.446)--(3.818,6.446)--(4.976,8.085)--(6.133,8.085)%
  --(7.290,8.085)--(8.447,8.631);
\gppoint{gp mark 2}{(1.504,5.900)}
\gppoint{gp mark 2}{(2.661,5.354)}
\gppoint{gp mark 2}{(2.865,5.354)}
\gppoint{gp mark 2}{(3.213,5.900)}
\gppoint{gp mark 2}{(3.358,6.446)}
\gppoint{gp mark 2}{(3.417,5.900)}
\gppoint{gp mark 2}{(3.562,5.900)}
\gppoint{gp mark 2}{(3.765,6.446)}
\gppoint{gp mark 2}{(3.818,6.446)}
\gppoint{gp mark 2}{(4.976,8.085)}
\gppoint{gp mark 2}{(6.133,8.085)}
\gppoint{gp mark 2}{(7.290,8.085)}
\gppoint{gp mark 2}{(8.447,8.631)}
\gpsetlinetype{gp lt axes}
\draw[gp path] (1.504,4.808)--(1.574,4.808)--(1.644,4.808)--(1.714,4.808)--(1.785,4.808)%
  --(1.855,4.808)--(1.925,4.808)--(1.995,4.808)--(2.065,4.808)--(2.135,4.808)--(2.205,4.808)%
  --(2.275,4.808)--(2.346,4.808)--(2.416,4.808)--(2.486,4.808)--(2.556,4.808)--(2.626,4.808)%
  --(2.696,4.808)--(2.766,4.808)--(2.836,4.808)--(2.907,4.808)--(2.977,4.808)--(3.047,4.808)%
  --(3.117,4.808)--(3.187,4.808)--(3.257,4.808)--(3.327,4.808)--(3.398,4.808)--(3.468,4.808)%
  --(3.538,4.808)--(3.608,4.808)--(3.678,4.808)--(3.748,4.808)--(3.818,4.808)--(3.888,4.808)%
  --(3.959,4.808)--(4.029,4.808)--(4.099,4.808)--(4.169,4.808)--(4.239,4.808)--(4.309,4.808)%
  --(4.379,4.808)--(4.450,4.808)--(4.520,4.808)--(4.590,4.808)--(4.660,4.808)--(4.730,4.808)%
  --(4.800,4.808)--(4.870,4.808)--(4.940,4.808)--(5.011,4.808)--(5.081,4.808)--(5.151,4.808)%
  --(5.221,4.808)--(5.291,4.808)--(5.361,4.808)--(5.431,4.808)--(5.501,4.808)--(5.572,4.808)%
  --(5.642,4.808)--(5.712,4.808)--(5.782,4.808)--(5.852,4.808)--(5.922,4.808)--(5.992,4.808)%
  --(6.063,4.808)--(6.133,4.808)--(6.203,4.808)--(6.273,4.808)--(6.343,4.808)--(6.413,4.808)%
  --(6.483,4.808)--(6.553,4.808)--(6.624,4.808)--(6.694,4.808)--(6.764,4.808)--(6.834,4.808)%
  --(6.904,4.808)--(6.974,4.808)--(7.044,4.808)--(7.115,4.808)--(7.185,4.808)--(7.255,4.808)%
  --(7.325,4.808)--(7.395,4.808)--(7.465,4.808)--(7.535,4.808)--(7.605,4.808)--(7.676,4.808)%
  --(7.746,4.808)--(7.816,4.808)--(7.886,4.808)--(7.956,4.808)--(8.026,4.808)--(8.096,4.808)%
  --(8.166,4.808)--(8.237,4.808)--(8.307,4.808)--(8.377,4.808)--(8.447,4.808);
\gpcolor{gp lt color 0}
\gpsetlinetype{gp lt plot 0}
\draw[gp path] (1.504,3.748)--(2.661,3.649)--(2.865,3.649)--(3.213,3.698)--(3.358,3.748)%
  --(3.417,3.698)--(3.562,3.698)--(3.765,3.748)--(3.818,3.748)--(4.976,5.329)--(6.133,8.631)%
  --(7.290,7.692)--(8.447,7.252);
\gppoint{gp mark 3}{(1.504,3.748)}
\gppoint{gp mark 3}{(2.661,3.649)}
\gppoint{gp mark 3}{(2.865,3.649)}
\gppoint{gp mark 3}{(3.213,3.698)}
\gppoint{gp mark 3}{(3.358,3.748)}
\gppoint{gp mark 3}{(3.417,3.698)}
\gppoint{gp mark 3}{(3.562,3.698)}
\gppoint{gp mark 3}{(3.765,3.748)}
\gppoint{gp mark 3}{(3.818,3.748)}
\gppoint{gp mark 3}{(4.976,5.329)}
\gppoint{gp mark 3}{(6.133,8.631)}
\gppoint{gp mark 3}{(7.290,7.692)}
\gppoint{gp mark 3}{(8.447,7.252)}
\gpsetlinetype{gp lt axes}
\draw[gp path] (1.504,3.644)--(1.574,3.644)--(1.644,3.644)--(1.714,3.644)--(1.785,3.644)%
  --(1.855,3.644)--(1.925,3.644)--(1.995,3.644)--(2.065,3.644)--(2.135,3.644)--(2.205,3.644)%
  --(2.275,3.644)--(2.346,3.644)--(2.416,3.644)--(2.486,3.644)--(2.556,3.644)--(2.626,3.644)%
  --(2.696,3.644)--(2.766,3.644)--(2.836,3.644)--(2.907,3.644)--(2.977,3.644)--(3.047,3.644)%
  --(3.117,3.644)--(3.187,3.644)--(3.257,3.644)--(3.327,3.644)--(3.398,3.644)--(3.468,3.644)%
  --(3.538,3.644)--(3.608,3.644)--(3.678,3.644)--(3.748,3.644)--(3.818,3.644)--(3.888,3.644)%
  --(3.959,3.644)--(4.029,3.644)--(4.099,3.644)--(4.169,3.644)--(4.239,3.644)--(4.309,3.644)%
  --(4.379,3.644)--(4.450,3.644)--(4.520,3.644)--(4.590,3.644)--(4.660,3.644)--(4.730,3.644)%
  --(4.800,3.644)--(4.870,3.644)--(4.940,3.644)--(5.011,3.644)--(5.081,3.644)--(5.151,3.644)%
  --(5.221,3.644)--(5.291,3.644)--(5.361,3.644)--(5.431,3.644)--(5.501,3.644)--(5.572,3.644)%
  --(5.642,3.644)--(5.712,3.644)--(5.782,3.644)--(5.852,3.644)--(5.922,3.644)--(5.992,3.644)%
  --(6.063,3.644)--(6.133,3.644)--(6.203,3.644)--(6.273,3.644)--(6.343,3.644)--(6.413,3.644)%
  --(6.483,3.644)--(6.553,3.644)--(6.624,3.644)--(6.694,3.644)--(6.764,3.644)--(6.834,3.644)%
  --(6.904,3.644)--(6.974,3.644)--(7.044,3.644)--(7.115,3.644)--(7.185,3.644)--(7.255,3.644)%
  --(7.325,3.644)--(7.395,3.644)--(7.465,3.644)--(7.535,3.644)--(7.605,3.644)--(7.676,3.644)%
  --(7.746,3.644)--(7.816,3.644)--(7.886,3.644)--(7.956,3.644)--(8.026,3.644)--(8.096,3.644)%
  --(8.166,3.644)--(8.237,3.644)--(8.307,3.644)--(8.377,3.644)--(8.447,3.644);
\gpcolor{gp lt color 1}
\gpsetlinetype{gp lt plot 0}
\draw[gp path] (1.504,4.626)--(2.661,4.565)--(2.865,4.626)--(3.213,4.565)--(3.358,4.626)%
  --(3.417,4.565)--(3.562,4.565)--(3.765,4.686)--(3.818,4.686)--(4.976,6.443)--(6.133,8.625)%
  --(7.290,8.509)--(8.447,8.631);
\gppoint{gp mark 4}{(1.504,4.626)}
\gppoint{gp mark 4}{(2.661,4.565)}
\gppoint{gp mark 4}{(2.865,4.626)}
\gppoint{gp mark 4}{(3.213,4.565)}
\gppoint{gp mark 4}{(3.358,4.626)}
\gppoint{gp mark 4}{(3.417,4.565)}
\gppoint{gp mark 4}{(3.562,4.565)}
\gppoint{gp mark 4}{(3.765,4.686)}
\gppoint{gp mark 4}{(3.818,4.686)}
\gppoint{gp mark 4}{(4.976,6.443)}
\gppoint{gp mark 4}{(6.133,8.625)}
\gppoint{gp mark 4}{(7.290,8.509)}
\gppoint{gp mark 4}{(8.447,8.631)}
\gpsetlinetype{gp lt axes}
\draw[gp path] (1.504,4.686)--(1.574,4.686)--(1.644,4.686)--(1.714,4.686)--(1.785,4.686)%
  --(1.855,4.686)--(1.925,4.686)--(1.995,4.686)--(2.065,4.686)--(2.135,4.686)--(2.205,4.686)%
  --(2.275,4.686)--(2.346,4.686)--(2.416,4.686)--(2.486,4.686)--(2.556,4.686)--(2.626,4.686)%
  --(2.696,4.686)--(2.766,4.686)--(2.836,4.686)--(2.907,4.686)--(2.977,4.686)--(3.047,4.686)%
  --(3.117,4.686)--(3.187,4.686)--(3.257,4.686)--(3.327,4.686)--(3.398,4.686)--(3.468,4.686)%
  --(3.538,4.686)--(3.608,4.686)--(3.678,4.686)--(3.748,4.686)--(3.818,4.686)--(3.888,4.686)%
  --(3.959,4.686)--(4.029,4.686)--(4.099,4.686)--(4.169,4.686)--(4.239,4.686)--(4.309,4.686)%
  --(4.379,4.686)--(4.450,4.686)--(4.520,4.686)--(4.590,4.686)--(4.660,4.686)--(4.730,4.686)%
  --(4.800,4.686)--(4.870,4.686)--(4.940,4.686)--(5.011,4.686)--(5.081,4.686)--(5.151,4.686)%
  --(5.221,4.686)--(5.291,4.686)--(5.361,4.686)--(5.431,4.686)--(5.501,4.686)--(5.572,4.686)%
  --(5.642,4.686)--(5.712,4.686)--(5.782,4.686)--(5.852,4.686)--(5.922,4.686)--(5.992,4.686)%
  --(6.063,4.686)--(6.133,4.686)--(6.203,4.686)--(6.273,4.686)--(6.343,4.686)--(6.413,4.686)%
  --(6.483,4.686)--(6.553,4.686)--(6.624,4.686)--(6.694,4.686)--(6.764,4.686)--(6.834,4.686)%
  --(6.904,4.686)--(6.974,4.686)--(7.044,4.686)--(7.115,4.686)--(7.185,4.686)--(7.255,4.686)%
  --(7.325,4.686)--(7.395,4.686)--(7.465,4.686)--(7.535,4.686)--(7.605,4.686)--(7.676,4.686)%
  --(7.746,4.686)--(7.816,4.686)--(7.886,4.686)--(7.956,4.686)--(8.026,4.686)--(8.096,4.686)%
  --(8.166,4.686)--(8.237,4.686)--(8.307,4.686)--(8.377,4.686)--(8.447,4.686);
\gpcolor{gp lt color 2}
\gpsetlinetype{gp lt plot 0}
\draw[gp path] (1.504,4.125)--(2.661,4.010)--(2.865,4.125)--(3.213,4.125)--(3.358,4.182)%
  --(3.417,4.182)--(3.562,4.125)--(3.765,4.182)--(3.818,4.239)--(4.976,5.892)--(6.133,8.631)%
  --(7.290,8.351)--(8.447,8.459);
\gppoint{gp mark 5}{(1.504,4.125)}
\gppoint{gp mark 5}{(2.661,4.010)}
\gppoint{gp mark 5}{(2.865,4.125)}
\gppoint{gp mark 5}{(3.213,4.125)}
\gppoint{gp mark 5}{(3.358,4.182)}
\gppoint{gp mark 5}{(3.417,4.182)}
\gppoint{gp mark 5}{(3.562,4.125)}
\gppoint{gp mark 5}{(3.765,4.182)}
\gppoint{gp mark 5}{(3.818,4.239)}
\gppoint{gp mark 5}{(4.976,5.892)}
\gppoint{gp mark 5}{(6.133,8.631)}
\gppoint{gp mark 5}{(7.290,8.351)}
\gppoint{gp mark 5}{(8.447,8.459)}
\gpsetlinetype{gp lt axes}
\draw[gp path] (1.504,4.067)--(1.574,4.067)--(1.644,4.067)--(1.714,4.067)--(1.785,4.067)%
  --(1.855,4.067)--(1.925,4.067)--(1.995,4.067)--(2.065,4.067)--(2.135,4.067)--(2.205,4.067)%
  --(2.275,4.067)--(2.346,4.067)--(2.416,4.067)--(2.486,4.067)--(2.556,4.067)--(2.626,4.067)%
  --(2.696,4.067)--(2.766,4.067)--(2.836,4.067)--(2.907,4.067)--(2.977,4.067)--(3.047,4.067)%
  --(3.117,4.067)--(3.187,4.067)--(3.257,4.067)--(3.327,4.067)--(3.398,4.067)--(3.468,4.067)%
  --(3.538,4.067)--(3.608,4.067)--(3.678,4.067)--(3.748,4.067)--(3.818,4.067)--(3.888,4.067)%
  --(3.959,4.067)--(4.029,4.067)--(4.099,4.067)--(4.169,4.067)--(4.239,4.067)--(4.309,4.067)%
  --(4.379,4.067)--(4.450,4.067)--(4.520,4.067)--(4.590,4.067)--(4.660,4.067)--(4.730,4.067)%
  --(4.800,4.067)--(4.870,4.067)--(4.940,4.067)--(5.011,4.067)--(5.081,4.067)--(5.151,4.067)%
  --(5.221,4.067)--(5.291,4.067)--(5.361,4.067)--(5.431,4.067)--(5.501,4.067)--(5.572,4.067)%
  --(5.642,4.067)--(5.712,4.067)--(5.782,4.067)--(5.852,4.067)--(5.922,4.067)--(5.992,4.067)%
  --(6.063,4.067)--(6.133,4.067)--(6.203,4.067)--(6.273,4.067)--(6.343,4.067)--(6.413,4.067)%
  --(6.483,4.067)--(6.553,4.067)--(6.624,4.067)--(6.694,4.067)--(6.764,4.067)--(6.834,4.067)%
  --(6.904,4.067)--(6.974,4.067)--(7.044,4.067)--(7.115,4.067)--(7.185,4.067)--(7.255,4.067)%
  --(7.325,4.067)--(7.395,4.067)--(7.465,4.067)--(7.535,4.067)--(7.605,4.067)--(7.676,4.067)%
  --(7.746,4.067)--(7.816,4.067)--(7.886,4.067)--(7.956,4.067)--(8.026,4.067)--(8.096,4.067)%
  --(8.166,4.067)--(8.237,4.067)--(8.307,4.067)--(8.377,4.067)--(8.447,4.067);
\gpcolor{gp lt color 3}
\gpsetlinetype{gp lt plot 0}
\draw[gp path] (1.504,4.808)--(2.661,4.808)--(2.865,6.082)--(3.213,6.082)--(3.358,4.808)%
  --(3.417,6.082)--(3.562,6.082)--(3.765,6.082)--(3.818,7.357)--(4.976,8.631)--(6.133,8.631)%
  --(7.290,8.631)--(8.447,8.631);
\gppoint{gp mark 6}{(1.504,4.808)}
\gppoint{gp mark 6}{(2.661,4.808)}
\gppoint{gp mark 6}{(2.865,6.082)}
\gppoint{gp mark 6}{(3.213,6.082)}
\gppoint{gp mark 6}{(3.358,4.808)}
\gppoint{gp mark 6}{(3.417,6.082)}
\gppoint{gp mark 6}{(3.562,6.082)}
\gppoint{gp mark 6}{(3.765,6.082)}
\gppoint{gp mark 6}{(3.818,7.357)}
\gppoint{gp mark 6}{(4.976,8.631)}
\gppoint{gp mark 6}{(6.133,8.631)}
\gppoint{gp mark 6}{(7.290,8.631)}
\gppoint{gp mark 6}{(8.447,8.631)}
\gpsetlinetype{gp lt axes}
\draw[gp path] (1.504,6.082)--(1.574,6.082)--(1.644,6.082)--(1.714,6.082)--(1.785,6.082)%
  --(1.855,6.082)--(1.925,6.082)--(1.995,6.082)--(2.065,6.082)--(2.135,6.082)--(2.205,6.082)%
  --(2.275,6.082)--(2.346,6.082)--(2.416,6.082)--(2.486,6.082)--(2.556,6.082)--(2.626,6.082)%
  --(2.696,6.082)--(2.766,6.082)--(2.836,6.082)--(2.907,6.082)--(2.977,6.082)--(3.047,6.082)%
  --(3.117,6.082)--(3.187,6.082)--(3.257,6.082)--(3.327,6.082)--(3.398,6.082)--(3.468,6.082)%
  --(3.538,6.082)--(3.608,6.082)--(3.678,6.082)--(3.748,6.082)--(3.818,6.082)--(3.888,6.082)%
  --(3.959,6.082)--(4.029,6.082)--(4.099,6.082)--(4.169,6.082)--(4.239,6.082)--(4.309,6.082)%
  --(4.379,6.082)--(4.450,6.082)--(4.520,6.082)--(4.590,6.082)--(4.660,6.082)--(4.730,6.082)%
  --(4.800,6.082)--(4.870,6.082)--(4.940,6.082)--(5.011,6.082)--(5.081,6.082)--(5.151,6.082)%
  --(5.221,6.082)--(5.291,6.082)--(5.361,6.082)--(5.431,6.082)--(5.501,6.082)--(5.572,6.082)%
  --(5.642,6.082)--(5.712,6.082)--(5.782,6.082)--(5.852,6.082)--(5.922,6.082)--(5.992,6.082)%
  --(6.063,6.082)--(6.133,6.082)--(6.203,6.082)--(6.273,6.082)--(6.343,6.082)--(6.413,6.082)%
  --(6.483,6.082)--(6.553,6.082)--(6.624,6.082)--(6.694,6.082)--(6.764,6.082)--(6.834,6.082)%
  --(6.904,6.082)--(6.974,6.082)--(7.044,6.082)--(7.115,6.082)--(7.185,6.082)--(7.255,6.082)%
  --(7.325,6.082)--(7.395,6.082)--(7.465,6.082)--(7.535,6.082)--(7.605,6.082)--(7.676,6.082)%
  --(7.746,6.082)--(7.816,6.082)--(7.886,6.082)--(7.956,6.082)--(8.026,6.082)--(8.096,6.082)%
  --(8.166,6.082)--(8.237,6.082)--(8.307,6.082)--(8.377,6.082)--(8.447,6.082);
\gpcolor{gp lt color border}
\gpsetlinetype{gp lt border}
\draw[gp path] (1.504,8.631)--(1.504,0.985)--(8.447,0.985)--(8.447,8.631)--cycle;
%% coordinates of the plot area
\gpdefrectangularnode{gp plot 1}{\pgfpoint{1.504cm}{0.985cm}}{\pgfpoint{8.447cm}{8.631cm}}
\end{tikzpicture}
%% gnuplot variables

    \caption{Function execution time with various block size implementations}
    \label{profiling:blockSize:functionExecutionTime}
\end{figure}

\begin{figure}
    \centering
    \begin{tikzpicture}[gnuplot]
%% generated with GNUPLOT 4.4p3 (Lua 5.1.4; terminal rev. 97, script rev. 96a)
%% Wed 24 Oct 2012 11:48:41 EST
\gpcolor{gp lt color border}
\gpsetlinetype{gp lt border}
\gpsetlinewidth{1.00}
\draw[gp path] (1.504,0.985)--(1.684,0.985);
\draw[gp path] (8.447,0.985)--(8.267,0.985);
\node[gp node right] at (1.320,0.985) { 0};
\draw[gp path] (1.504,1.750)--(1.684,1.750);
\draw[gp path] (8.447,1.750)--(8.267,1.750);
\node[gp node right] at (1.320,1.750) { 0.1};
\draw[gp path] (1.504,2.514)--(1.684,2.514);
\draw[gp path] (8.447,2.514)--(8.267,2.514);
\node[gp node right] at (1.320,2.514) { 0.2};
\draw[gp path] (1.504,3.279)--(1.684,3.279);
\draw[gp path] (8.447,3.279)--(8.267,3.279);
\node[gp node right] at (1.320,3.279) { 0.3};
\draw[gp path] (1.504,4.043)--(1.684,4.043);
\draw[gp path] (8.447,4.043)--(8.267,4.043);
\node[gp node right] at (1.320,4.043) { 0.4};
\draw[gp path] (1.504,4.808)--(1.684,4.808);
\draw[gp path] (8.447,4.808)--(8.267,4.808);
\node[gp node right] at (1.320,4.808) { 0.5};
\draw[gp path] (1.504,5.573)--(1.684,5.573);
\draw[gp path] (8.447,5.573)--(8.267,5.573);
\node[gp node right] at (1.320,5.573) { 0.6};
\draw[gp path] (1.504,6.337)--(1.684,6.337);
\draw[gp path] (8.447,6.337)--(8.267,6.337);
\node[gp node right] at (1.320,6.337) { 0.7};
\draw[gp path] (1.504,7.102)--(1.684,7.102);
\draw[gp path] (8.447,7.102)--(8.267,7.102);
\node[gp node right] at (1.320,7.102) { 0.8};
\draw[gp path] (1.504,7.866)--(1.684,7.866);
\draw[gp path] (8.447,7.866)--(8.267,7.866);
\node[gp node right] at (1.320,7.866) { 0.9};
\draw[gp path] (1.504,8.631)--(1.684,8.631);
\draw[gp path] (8.447,8.631)--(8.267,8.631);
\node[gp node right] at (1.320,8.631) { 1};
\draw[gp path] (1.504,0.985)--(1.504,1.165);
\draw[gp path] (1.504,8.631)--(1.504,8.451);
\node[gp node center] at (1.504,0.677) {$10^{0}$};
\draw[gp path] (1.852,0.985)--(1.852,1.075);
\draw[gp path] (1.852,8.631)--(1.852,8.541);
\draw[gp path] (2.313,0.985)--(2.313,1.075);
\draw[gp path] (2.313,8.631)--(2.313,8.541);
\draw[gp path] (2.549,0.985)--(2.549,1.075);
\draw[gp path] (2.549,8.631)--(2.549,8.541);
\draw[gp path] (2.661,0.985)--(2.661,1.165);
\draw[gp path] (2.661,8.631)--(2.661,8.451);
\node[gp node center] at (2.661,0.677) {$10^{1}$};
\draw[gp path] (3.010,0.985)--(3.010,1.075);
\draw[gp path] (3.010,8.631)--(3.010,8.541);
\draw[gp path] (3.470,0.985)--(3.470,1.075);
\draw[gp path] (3.470,8.631)--(3.470,8.541);
\draw[gp path] (3.706,0.985)--(3.706,1.075);
\draw[gp path] (3.706,8.631)--(3.706,8.541);
\draw[gp path] (3.818,0.985)--(3.818,1.165);
\draw[gp path] (3.818,8.631)--(3.818,8.451);
\node[gp node center] at (3.818,0.677) {$10^{2}$};
\draw[gp path] (4.167,0.985)--(4.167,1.075);
\draw[gp path] (4.167,8.631)--(4.167,8.541);
\draw[gp path] (4.627,0.985)--(4.627,1.075);
\draw[gp path] (4.627,8.631)--(4.627,8.541);
\draw[gp path] (4.863,0.985)--(4.863,1.075);
\draw[gp path] (4.863,8.631)--(4.863,8.541);
\draw[gp path] (4.976,0.985)--(4.976,1.165);
\draw[gp path] (4.976,8.631)--(4.976,8.451);
\node[gp node center] at (4.976,0.677) {$10^{3}$};
\draw[gp path] (5.324,0.985)--(5.324,1.075);
\draw[gp path] (5.324,8.631)--(5.324,8.541);
\draw[gp path] (5.784,0.985)--(5.784,1.075);
\draw[gp path] (5.784,8.631)--(5.784,8.541);
\draw[gp path] (6.021,0.985)--(6.021,1.075);
\draw[gp path] (6.021,8.631)--(6.021,8.541);
\draw[gp path] (6.133,0.985)--(6.133,1.165);
\draw[gp path] (6.133,8.631)--(6.133,8.451);
\node[gp node center] at (6.133,0.677) {$10^{4}$};
\draw[gp path] (6.481,0.985)--(6.481,1.075);
\draw[gp path] (6.481,8.631)--(6.481,8.541);
\draw[gp path] (6.941,0.985)--(6.941,1.075);
\draw[gp path] (6.941,8.631)--(6.941,8.541);
\draw[gp path] (7.178,0.985)--(7.178,1.075);
\draw[gp path] (7.178,8.631)--(7.178,8.541);
\draw[gp path] (7.290,0.985)--(7.290,1.165);
\draw[gp path] (7.290,8.631)--(7.290,8.451);
\node[gp node center] at (7.290,0.677) {$10^{5}$};
\draw[gp path] (7.638,0.985)--(7.638,1.075);
\draw[gp path] (7.638,8.631)--(7.638,8.541);
\draw[gp path] (8.099,0.985)--(8.099,1.075);
\draw[gp path] (8.099,8.631)--(8.099,8.541);
\draw[gp path] (8.335,0.985)--(8.335,1.075);
\draw[gp path] (8.335,8.631)--(8.335,8.541);
\draw[gp path] (8.447,0.985)--(8.447,1.165);
\draw[gp path] (8.447,8.631)--(8.447,8.451);
\node[gp node center] at (8.447,0.677) {$10^{6}$};
\draw[gp path] (1.504,8.631)--(1.504,0.985)--(8.447,0.985)--(8.447,8.631)--cycle;
\node[gp node center,rotate=-270] at (0.246,4.808) {\textbf{Calls to the distance function (normalised)}};
\node[gp node center] at (4.975,0.215) {\textbf{Block size}};
\gpcolor{gp lt color axes}
\gpsetlinetype{gp lt plot 0}
\draw[gp path] (1.504,4.178)--(2.661,4.482)--(2.865,4.787)--(3.213,5.095)--(3.358,5.409)%
  --(3.417,5.722)--(3.562,6.037)--(3.765,6.360)--(3.818,6.688)--(4.976,7.174)--(6.133,7.659)%
  --(7.290,8.145)--(8.447,8.631);
\gpsetpointsize{4.00}
\gppoint{gp mark 1}{(1.504,4.178)}
\gppoint{gp mark 1}{(2.661,4.482)}
\gppoint{gp mark 1}{(2.865,4.787)}
\gppoint{gp mark 1}{(3.213,5.095)}
\gppoint{gp mark 1}{(3.358,5.409)}
\gppoint{gp mark 1}{(3.417,5.722)}
\gppoint{gp mark 1}{(3.562,6.037)}
\gppoint{gp mark 1}{(3.765,6.360)}
\gppoint{gp mark 1}{(3.818,6.688)}
\gppoint{gp mark 1}{(4.976,7.174)}
\gppoint{gp mark 1}{(6.133,7.659)}
\gppoint{gp mark 1}{(7.290,8.145)}
\gppoint{gp mark 1}{(8.447,8.631)}
\gpsetlinetype{gp lt axes}
\draw[gp path] (1.504,1.459)--(1.574,1.459)--(1.644,1.459)--(1.714,1.459)--(1.785,1.459)%
  --(1.855,1.459)--(1.925,1.459)--(1.995,1.459)--(2.065,1.459)--(2.135,1.459)--(2.205,1.459)%
  --(2.275,1.459)--(2.346,1.459)--(2.416,1.459)--(2.486,1.459)--(2.556,1.459)--(2.626,1.459)%
  --(2.696,1.459)--(2.766,1.459)--(2.836,1.459)--(2.907,1.459)--(2.977,1.459)--(3.047,1.459)%
  --(3.117,1.459)--(3.187,1.459)--(3.257,1.459)--(3.327,1.459)--(3.398,1.459)--(3.468,1.459)%
  --(3.538,1.459)--(3.608,1.459)--(3.678,1.459)--(3.748,1.459)--(3.818,1.459)--(3.888,1.459)%
  --(3.959,1.459)--(4.029,1.459)--(4.099,1.459)--(4.169,1.459)--(4.239,1.459)--(4.309,1.459)%
  --(4.379,1.459)--(4.450,1.459)--(4.520,1.459)--(4.590,1.459)--(4.660,1.459)--(4.730,1.459)%
  --(4.800,1.459)--(4.870,1.459)--(4.940,1.459)--(5.011,1.459)--(5.081,1.459)--(5.151,1.459)%
  --(5.221,1.459)--(5.291,1.459)--(5.361,1.459)--(5.431,1.459)--(5.501,1.459)--(5.572,1.459)%
  --(5.642,1.459)--(5.712,1.459)--(5.782,1.459)--(5.852,1.459)--(5.922,1.459)--(5.992,1.459)%
  --(6.063,1.459)--(6.133,1.459)--(6.203,1.459)--(6.273,1.459)--(6.343,1.459)--(6.413,1.459)%
  --(6.483,1.459)--(6.553,1.459)--(6.624,1.459)--(6.694,1.459)--(6.764,1.459)--(6.834,1.459)%
  --(6.904,1.459)--(6.974,1.459)--(7.044,1.459)--(7.115,1.459)--(7.185,1.459)--(7.255,1.459)%
  --(7.325,1.459)--(7.395,1.459)--(7.465,1.459)--(7.535,1.459)--(7.605,1.459)--(7.676,1.459)%
  --(7.746,1.459)--(7.816,1.459)--(7.886,1.459)--(7.956,1.459)--(8.026,1.459)--(8.096,1.459)%
  --(8.166,1.459)--(8.237,1.459)--(8.307,1.459)--(8.377,1.459)--(8.447,1.459);
\gpcolor{gp lt color 0}
\gpsetlinetype{gp lt plot 0}
\draw[gp path] (1.504,1.380)--(2.661,1.777)--(2.865,2.174)--(3.213,2.576)--(3.358,2.982)%
  --(3.417,3.387)--(3.562,3.793)--(3.765,4.206)--(3.818,4.619)--(4.976,5.260)--(6.133,8.631);
\gppoint{gp mark 2}{(1.504,1.380)}
\gppoint{gp mark 2}{(2.661,1.777)}
\gppoint{gp mark 2}{(2.865,2.174)}
\gppoint{gp mark 2}{(3.213,2.576)}
\gppoint{gp mark 2}{(3.358,2.982)}
\gppoint{gp mark 2}{(3.417,3.387)}
\gppoint{gp mark 2}{(3.562,3.793)}
\gppoint{gp mark 2}{(3.765,4.206)}
\gppoint{gp mark 2}{(3.818,4.619)}
\gppoint{gp mark 2}{(4.976,5.260)}
\gppoint{gp mark 2}{(6.133,8.631)}
\gpsetlinetype{gp lt axes}
\draw[gp path] (1.504,1.380)--(1.574,1.380)--(1.644,1.380)--(1.714,1.380)--(1.785,1.380)%
  --(1.855,1.380)--(1.925,1.380)--(1.995,1.380)--(2.065,1.380)--(2.135,1.380)--(2.205,1.380)%
  --(2.275,1.380)--(2.346,1.380)--(2.416,1.380)--(2.486,1.380)--(2.556,1.380)--(2.626,1.380)%
  --(2.696,1.380)--(2.766,1.380)--(2.836,1.380)--(2.907,1.380)--(2.977,1.380)--(3.047,1.380)%
  --(3.117,1.380)--(3.187,1.380)--(3.257,1.380)--(3.327,1.380)--(3.398,1.380)--(3.468,1.380)%
  --(3.538,1.380)--(3.608,1.380)--(3.678,1.380)--(3.748,1.380)--(3.818,1.380)--(3.888,1.380)%
  --(3.959,1.380)--(4.029,1.380)--(4.099,1.380)--(4.169,1.380)--(4.239,1.380)--(4.309,1.380)%
  --(4.379,1.380)--(4.450,1.380)--(4.520,1.380)--(4.590,1.380)--(4.660,1.380)--(4.730,1.380)%
  --(4.800,1.380)--(4.870,1.380)--(4.940,1.380)--(5.011,1.380)--(5.081,1.380)--(5.151,1.380)%
  --(5.221,1.380)--(5.291,1.380)--(5.361,1.380)--(5.431,1.380)--(5.501,1.380)--(5.572,1.380)%
  --(5.642,1.380)--(5.712,1.380)--(5.782,1.380)--(5.852,1.380)--(5.922,1.380)--(5.992,1.380)%
  --(6.063,1.380)--(6.133,1.380)--(6.203,1.380)--(6.273,1.380)--(6.343,1.380)--(6.413,1.380)%
  --(6.483,1.380)--(6.553,1.380)--(6.624,1.380)--(6.694,1.380)--(6.764,1.380)--(6.834,1.380)%
  --(6.904,1.380)--(6.974,1.380)--(7.044,1.380)--(7.115,1.380)--(7.185,1.380)--(7.255,1.380)%
  --(7.325,1.380)--(7.395,1.380)--(7.465,1.380)--(7.535,1.380)--(7.605,1.380)--(7.676,1.380)%
  --(7.746,1.380)--(7.816,1.380)--(7.886,1.380)--(7.956,1.380)--(8.026,1.380)--(8.096,1.380)%
  --(8.166,1.380)--(8.237,1.380)--(8.307,1.380)--(8.377,1.380)--(8.447,1.380);
\gpcolor{gp lt color 1}
\gpsetlinetype{gp lt plot 0}
\draw[gp path] (1.504,4.912)--(2.661,4.972)--(2.865,5.033)--(3.213,5.093)--(3.358,5.155)%
  --(3.417,5.216)--(3.562,5.278)--(3.765,5.340)--(3.818,5.403)--(4.976,5.510)--(6.133,6.149)%
  --(7.290,7.390)--(8.447,8.631);
\gppoint{gp mark 3}{(1.504,4.912)}
\gppoint{gp mark 3}{(2.661,4.972)}
\gppoint{gp mark 3}{(2.865,5.033)}
\gppoint{gp mark 3}{(3.213,5.093)}
\gppoint{gp mark 3}{(3.358,5.155)}
\gppoint{gp mark 3}{(3.417,5.216)}
\gppoint{gp mark 3}{(3.562,5.278)}
\gppoint{gp mark 3}{(3.765,5.340)}
\gppoint{gp mark 3}{(3.818,5.403)}
\gppoint{gp mark 3}{(4.976,5.510)}
\gppoint{gp mark 3}{(6.133,6.149)}
\gppoint{gp mark 3}{(7.290,7.390)}
\gppoint{gp mark 3}{(8.447,8.631)}
\gpsetlinetype{gp lt axes}
\draw[gp path] (1.504,1.097)--(1.574,1.097)--(1.644,1.097)--(1.714,1.097)--(1.785,1.097)%
  --(1.855,1.097)--(1.925,1.097)--(1.995,1.097)--(2.065,1.097)--(2.135,1.097)--(2.205,1.097)%
  --(2.275,1.097)--(2.346,1.097)--(2.416,1.097)--(2.486,1.097)--(2.556,1.097)--(2.626,1.097)%
  --(2.696,1.097)--(2.766,1.097)--(2.836,1.097)--(2.907,1.097)--(2.977,1.097)--(3.047,1.097)%
  --(3.117,1.097)--(3.187,1.097)--(3.257,1.097)--(3.327,1.097)--(3.398,1.097)--(3.468,1.097)%
  --(3.538,1.097)--(3.608,1.097)--(3.678,1.097)--(3.748,1.097)--(3.818,1.097)--(3.888,1.097)%
  --(3.959,1.097)--(4.029,1.097)--(4.099,1.097)--(4.169,1.097)--(4.239,1.097)--(4.309,1.097)%
  --(4.379,1.097)--(4.450,1.097)--(4.520,1.097)--(4.590,1.097)--(4.660,1.097)--(4.730,1.097)%
  --(4.800,1.097)--(4.870,1.097)--(4.940,1.097)--(5.011,1.097)--(5.081,1.097)--(5.151,1.097)%
  --(5.221,1.097)--(5.291,1.097)--(5.361,1.097)--(5.431,1.097)--(5.501,1.097)--(5.572,1.097)%
  --(5.642,1.097)--(5.712,1.097)--(5.782,1.097)--(5.852,1.097)--(5.922,1.097)--(5.992,1.097)%
  --(6.063,1.097)--(6.133,1.097)--(6.203,1.097)--(6.273,1.097)--(6.343,1.097)--(6.413,1.097)%
  --(6.483,1.097)--(6.553,1.097)--(6.624,1.097)--(6.694,1.097)--(6.764,1.097)--(6.834,1.097)%
  --(6.904,1.097)--(6.974,1.097)--(7.044,1.097)--(7.115,1.097)--(7.185,1.097)--(7.255,1.097)%
  --(7.325,1.097)--(7.395,1.097)--(7.465,1.097)--(7.535,1.097)--(7.605,1.097)--(7.676,1.097)%
  --(7.746,1.097)--(7.816,1.097)--(7.886,1.097)--(7.956,1.097)--(8.026,1.097)--(8.096,1.097)%
  --(8.166,1.097)--(8.237,1.097)--(8.307,1.097)--(8.377,1.097)--(8.447,1.097);
\gpcolor{gp lt color 2}
\gpsetlinetype{gp lt plot 0}
\draw[gp path] (1.504,2.381)--(2.661,2.432)--(2.865,2.482)--(3.213,2.534)--(3.358,2.586)%
  --(3.417,2.639)--(3.562,2.693)--(3.765,2.747)--(3.818,2.802)--(4.976,2.953)--(6.133,4.152)%
  --(7.290,6.391)--(8.447,8.631);
\gppoint{gp mark 4}{(1.504,2.381)}
\gppoint{gp mark 4}{(2.661,2.432)}
\gppoint{gp mark 4}{(2.865,2.482)}
\gppoint{gp mark 4}{(3.213,2.534)}
\gppoint{gp mark 4}{(3.358,2.586)}
\gppoint{gp mark 4}{(3.417,2.639)}
\gppoint{gp mark 4}{(3.562,2.693)}
\gppoint{gp mark 4}{(3.765,2.747)}
\gppoint{gp mark 4}{(3.818,2.802)}
\gppoint{gp mark 4}{(4.976,2.953)}
\gppoint{gp mark 4}{(6.133,4.152)}
\gppoint{gp mark 4}{(7.290,6.391)}
\gppoint{gp mark 4}{(8.447,8.631)}
\gpsetlinetype{gp lt axes}
\draw[gp path] (1.504,1.085)--(1.574,1.085)--(1.644,1.085)--(1.714,1.085)--(1.785,1.085)%
  --(1.855,1.085)--(1.925,1.085)--(1.995,1.085)--(2.065,1.085)--(2.135,1.085)--(2.205,1.085)%
  --(2.275,1.085)--(2.346,1.085)--(2.416,1.085)--(2.486,1.085)--(2.556,1.085)--(2.626,1.085)%
  --(2.696,1.085)--(2.766,1.085)--(2.836,1.085)--(2.907,1.085)--(2.977,1.085)--(3.047,1.085)%
  --(3.117,1.085)--(3.187,1.085)--(3.257,1.085)--(3.327,1.085)--(3.398,1.085)--(3.468,1.085)%
  --(3.538,1.085)--(3.608,1.085)--(3.678,1.085)--(3.748,1.085)--(3.818,1.085)--(3.888,1.085)%
  --(3.959,1.085)--(4.029,1.085)--(4.099,1.085)--(4.169,1.085)--(4.239,1.085)--(4.309,1.085)%
  --(4.379,1.085)--(4.450,1.085)--(4.520,1.085)--(4.590,1.085)--(4.660,1.085)--(4.730,1.085)%
  --(4.800,1.085)--(4.870,1.085)--(4.940,1.085)--(5.011,1.085)--(5.081,1.085)--(5.151,1.085)%
  --(5.221,1.085)--(5.291,1.085)--(5.361,1.085)--(5.431,1.085)--(5.501,1.085)--(5.572,1.085)%
  --(5.642,1.085)--(5.712,1.085)--(5.782,1.085)--(5.852,1.085)--(5.922,1.085)--(5.992,1.085)%
  --(6.063,1.085)--(6.133,1.085)--(6.203,1.085)--(6.273,1.085)--(6.343,1.085)--(6.413,1.085)%
  --(6.483,1.085)--(6.553,1.085)--(6.624,1.085)--(6.694,1.085)--(6.764,1.085)--(6.834,1.085)%
  --(6.904,1.085)--(6.974,1.085)--(7.044,1.085)--(7.115,1.085)--(7.185,1.085)--(7.255,1.085)%
  --(7.325,1.085)--(7.395,1.085)--(7.465,1.085)--(7.535,1.085)--(7.605,1.085)--(7.676,1.085)%
  --(7.746,1.085)--(7.816,1.085)--(7.886,1.085)--(7.956,1.085)--(8.026,1.085)--(8.096,1.085)%
  --(8.166,1.085)--(8.237,1.085)--(8.307,1.085)--(8.377,1.085)--(8.447,1.085);
\gpcolor{gp lt color 3}
\gpsetlinetype{gp lt plot 0}
\draw[gp path] (1.504,8.564)--(2.661,8.568)--(2.865,8.572)--(3.213,8.576)--(3.358,8.580)%
  --(3.417,8.584)--(3.562,8.588)--(3.765,8.592)--(3.818,8.596)--(4.976,8.605)--(6.133,8.614)%
  --(7.290,8.622)--(8.447,8.631);
\gppoint{gp mark 5}{(1.504,8.564)}
\gppoint{gp mark 5}{(2.661,8.568)}
\gppoint{gp mark 5}{(2.865,8.572)}
\gppoint{gp mark 5}{(3.213,8.576)}
\gppoint{gp mark 5}{(3.358,8.580)}
\gppoint{gp mark 5}{(3.417,8.584)}
\gppoint{gp mark 5}{(3.562,8.588)}
\gppoint{gp mark 5}{(3.765,8.592)}
\gppoint{gp mark 5}{(3.818,8.596)}
\gppoint{gp mark 5}{(4.976,8.605)}
\gppoint{gp mark 5}{(6.133,8.614)}
\gppoint{gp mark 5}{(7.290,8.622)}
\gppoint{gp mark 5}{(8.447,8.631)}
\gpsetlinetype{gp lt axes}
\draw[gp path] (1.504,1.088)--(1.574,1.088)--(1.644,1.088)--(1.714,1.088)--(1.785,1.088)%
  --(1.855,1.088)--(1.925,1.088)--(1.995,1.088)--(2.065,1.088)--(2.135,1.088)--(2.205,1.088)%
  --(2.275,1.088)--(2.346,1.088)--(2.416,1.088)--(2.486,1.088)--(2.556,1.088)--(2.626,1.088)%
  --(2.696,1.088)--(2.766,1.088)--(2.836,1.088)--(2.907,1.088)--(2.977,1.088)--(3.047,1.088)%
  --(3.117,1.088)--(3.187,1.088)--(3.257,1.088)--(3.327,1.088)--(3.398,1.088)--(3.468,1.088)%
  --(3.538,1.088)--(3.608,1.088)--(3.678,1.088)--(3.748,1.088)--(3.818,1.088)--(3.888,1.088)%
  --(3.959,1.088)--(4.029,1.088)--(4.099,1.088)--(4.169,1.088)--(4.239,1.088)--(4.309,1.088)%
  --(4.379,1.088)--(4.450,1.088)--(4.520,1.088)--(4.590,1.088)--(4.660,1.088)--(4.730,1.088)%
  --(4.800,1.088)--(4.870,1.088)--(4.940,1.088)--(5.011,1.088)--(5.081,1.088)--(5.151,1.088)%
  --(5.221,1.088)--(5.291,1.088)--(5.361,1.088)--(5.431,1.088)--(5.501,1.088)--(5.572,1.088)%
  --(5.642,1.088)--(5.712,1.088)--(5.782,1.088)--(5.852,1.088)--(5.922,1.088)--(5.992,1.088)%
  --(6.063,1.088)--(6.133,1.088)--(6.203,1.088)--(6.273,1.088)--(6.343,1.088)--(6.413,1.088)%
  --(6.483,1.088)--(6.553,1.088)--(6.624,1.088)--(6.694,1.088)--(6.764,1.088)--(6.834,1.088)%
  --(6.904,1.088)--(6.974,1.088)--(7.044,1.088)--(7.115,1.088)--(7.185,1.088)--(7.255,1.088)%
  --(7.325,1.088)--(7.395,1.088)--(7.465,1.088)--(7.535,1.088)--(7.605,1.088)--(7.676,1.088)%
  --(7.746,1.088)--(7.816,1.088)--(7.886,1.088)--(7.956,1.088)--(8.026,1.088)--(8.096,1.088)%
  --(8.166,1.088)--(8.237,1.088)--(8.307,1.088)--(8.377,1.088)--(8.447,1.088);
\gpcolor{gp lt color 4}
\gpsetlinetype{gp lt plot 0}
\draw[gp path] (1.504,1.269)--(2.661,1.554)--(2.865,1.840)--(3.213,2.126)--(3.358,2.416)%
  --(3.417,2.706)--(3.562,2.994)--(3.765,3.286)--(3.818,3.580)--(4.976,4.041)--(6.133,5.571)%
  --(7.290,7.101)--(8.447,8.631);
\gppoint{gp mark 6}{(1.504,1.269)}
\gppoint{gp mark 6}{(2.661,1.554)}
\gppoint{gp mark 6}{(2.865,1.840)}
\gppoint{gp mark 6}{(3.213,2.126)}
\gppoint{gp mark 6}{(3.358,2.416)}
\gppoint{gp mark 6}{(3.417,2.706)}
\gppoint{gp mark 6}{(3.562,2.994)}
\gppoint{gp mark 6}{(3.765,3.286)}
\gppoint{gp mark 6}{(3.818,3.580)}
\gppoint{gp mark 6}{(4.976,4.041)}
\gppoint{gp mark 6}{(6.133,5.571)}
\gppoint{gp mark 6}{(7.290,7.101)}
\gppoint{gp mark 6}{(8.447,8.631)}
\gpsetlinetype{gp lt axes}
\draw[gp path] (1.504,1.269)--(1.574,1.269)--(1.644,1.269)--(1.714,1.269)--(1.785,1.269)%
  --(1.855,1.269)--(1.925,1.269)--(1.995,1.269)--(2.065,1.269)--(2.135,1.269)--(2.205,1.269)%
  --(2.275,1.269)--(2.346,1.269)--(2.416,1.269)--(2.486,1.269)--(2.556,1.269)--(2.626,1.269)%
  --(2.696,1.269)--(2.766,1.269)--(2.836,1.269)--(2.907,1.269)--(2.977,1.269)--(3.047,1.269)%
  --(3.117,1.269)--(3.187,1.269)--(3.257,1.269)--(3.327,1.269)--(3.398,1.269)--(3.468,1.269)%
  --(3.538,1.269)--(3.608,1.269)--(3.678,1.269)--(3.748,1.269)--(3.818,1.269)--(3.888,1.269)%
  --(3.959,1.269)--(4.029,1.269)--(4.099,1.269)--(4.169,1.269)--(4.239,1.269)--(4.309,1.269)%
  --(4.379,1.269)--(4.450,1.269)--(4.520,1.269)--(4.590,1.269)--(4.660,1.269)--(4.730,1.269)%
  --(4.800,1.269)--(4.870,1.269)--(4.940,1.269)--(5.011,1.269)--(5.081,1.269)--(5.151,1.269)%
  --(5.221,1.269)--(5.291,1.269)--(5.361,1.269)--(5.431,1.269)--(5.501,1.269)--(5.572,1.269)%
  --(5.642,1.269)--(5.712,1.269)--(5.782,1.269)--(5.852,1.269)--(5.922,1.269)--(5.992,1.269)%
  --(6.063,1.269)--(6.133,1.269)--(6.203,1.269)--(6.273,1.269)--(6.343,1.269)--(6.413,1.269)%
  --(6.483,1.269)--(6.553,1.269)--(6.624,1.269)--(6.694,1.269)--(6.764,1.269)--(6.834,1.269)%
  --(6.904,1.269)--(6.974,1.269)--(7.044,1.269)--(7.115,1.269)--(7.185,1.269)--(7.255,1.269)%
  --(7.325,1.269)--(7.395,1.269)--(7.465,1.269)--(7.535,1.269)--(7.605,1.269)--(7.676,1.269)%
  --(7.746,1.269)--(7.816,1.269)--(7.886,1.269)--(7.956,1.269)--(8.026,1.269)--(8.096,1.269)%
  --(8.166,1.269)--(8.237,1.269)--(8.307,1.269)--(8.377,1.269)--(8.447,1.269);
\gpcolor{gp lt color 5}
\gpsetlinetype{gp lt plot 0}
\draw[gp path] (1.504,1.075)--(2.661,1.166)--(2.865,1.256)--(3.213,1.349)--(3.358,1.443)%
  --(3.417,1.538)--(3.562,1.634)--(3.765,1.733)--(3.818,1.833)--(4.976,2.088)--(6.133,4.136)%
  --(7.290,6.383)--(8.447,8.631);
\gppoint{gp mark 7}{(1.504,1.075)}
\gppoint{gp mark 7}{(2.661,1.166)}
\gppoint{gp mark 7}{(2.865,1.256)}
\gppoint{gp mark 7}{(3.213,1.349)}
\gppoint{gp mark 7}{(3.358,1.443)}
\gppoint{gp mark 7}{(3.417,1.538)}
\gppoint{gp mark 7}{(3.562,1.634)}
\gppoint{gp mark 7}{(3.765,1.733)}
\gppoint{gp mark 7}{(3.818,1.833)}
\gppoint{gp mark 7}{(4.976,2.088)}
\gppoint{gp mark 7}{(6.133,4.136)}
\gppoint{gp mark 7}{(7.290,6.383)}
\gppoint{gp mark 7}{(8.447,8.631)}
\gpsetlinetype{gp lt axes}
\draw[gp path] (1.504,1.075)--(1.574,1.075)--(1.644,1.075)--(1.714,1.075)--(1.785,1.075)%
  --(1.855,1.075)--(1.925,1.075)--(1.995,1.075)--(2.065,1.075)--(2.135,1.075)--(2.205,1.075)%
  --(2.275,1.075)--(2.346,1.075)--(2.416,1.075)--(2.486,1.075)--(2.556,1.075)--(2.626,1.075)%
  --(2.696,1.075)--(2.766,1.075)--(2.836,1.075)--(2.907,1.075)--(2.977,1.075)--(3.047,1.075)%
  --(3.117,1.075)--(3.187,1.075)--(3.257,1.075)--(3.327,1.075)--(3.398,1.075)--(3.468,1.075)%
  --(3.538,1.075)--(3.608,1.075)--(3.678,1.075)--(3.748,1.075)--(3.818,1.075)--(3.888,1.075)%
  --(3.959,1.075)--(4.029,1.075)--(4.099,1.075)--(4.169,1.075)--(4.239,1.075)--(4.309,1.075)%
  --(4.379,1.075)--(4.450,1.075)--(4.520,1.075)--(4.590,1.075)--(4.660,1.075)--(4.730,1.075)%
  --(4.800,1.075)--(4.870,1.075)--(4.940,1.075)--(5.011,1.075)--(5.081,1.075)--(5.151,1.075)%
  --(5.221,1.075)--(5.291,1.075)--(5.361,1.075)--(5.431,1.075)--(5.501,1.075)--(5.572,1.075)%
  --(5.642,1.075)--(5.712,1.075)--(5.782,1.075)--(5.852,1.075)--(5.922,1.075)--(5.992,1.075)%
  --(6.063,1.075)--(6.133,1.075)--(6.203,1.075)--(6.273,1.075)--(6.343,1.075)--(6.413,1.075)%
  --(6.483,1.075)--(6.553,1.075)--(6.624,1.075)--(6.694,1.075)--(6.764,1.075)--(6.834,1.075)%
  --(6.904,1.075)--(6.974,1.075)--(7.044,1.075)--(7.115,1.075)--(7.185,1.075)--(7.255,1.075)%
  --(7.325,1.075)--(7.395,1.075)--(7.465,1.075)--(7.535,1.075)--(7.605,1.075)--(7.676,1.075)%
  --(7.746,1.075)--(7.816,1.075)--(7.886,1.075)--(7.956,1.075)--(8.026,1.075)--(8.096,1.075)%
  --(8.166,1.075)--(8.237,1.075)--(8.307,1.075)--(8.377,1.075)--(8.447,1.075);
\gpcolor{gp lt color 6}
\gpsetlinetype{gp lt plot 0}
\draw[gp path] (1.504,2.387)--(2.661,2.763)--(2.865,3.139)--(3.213,3.515)--(3.358,3.892)%
  --(3.417,4.269)--(3.562,4.646)--(3.765,5.025)--(3.818,5.404)--(4.976,5.856)--(6.133,6.781)%
  --(7.290,7.706)--(8.447,8.631);
\gppoint{gp mark 8}{(1.504,2.387)}
\gppoint{gp mark 8}{(2.661,2.763)}
\gppoint{gp mark 8}{(2.865,3.139)}
\gppoint{gp mark 8}{(3.213,3.515)}
\gppoint{gp mark 8}{(3.358,3.892)}
\gppoint{gp mark 8}{(3.417,4.269)}
\gppoint{gp mark 8}{(3.562,4.646)}
\gppoint{gp mark 8}{(3.765,5.025)}
\gppoint{gp mark 8}{(3.818,5.404)}
\gppoint{gp mark 8}{(4.976,5.856)}
\gppoint{gp mark 8}{(6.133,6.781)}
\gppoint{gp mark 8}{(7.290,7.706)}
\gppoint{gp mark 8}{(8.447,8.631)}
\gpsetlinetype{gp lt axes}
\draw[gp path] (1.504,1.420)--(1.574,1.420)--(1.644,1.420)--(1.714,1.420)--(1.785,1.420)%
  --(1.855,1.420)--(1.925,1.420)--(1.995,1.420)--(2.065,1.420)--(2.135,1.420)--(2.205,1.420)%
  --(2.275,1.420)--(2.346,1.420)--(2.416,1.420)--(2.486,1.420)--(2.556,1.420)--(2.626,1.420)%
  --(2.696,1.420)--(2.766,1.420)--(2.836,1.420)--(2.907,1.420)--(2.977,1.420)--(3.047,1.420)%
  --(3.117,1.420)--(3.187,1.420)--(3.257,1.420)--(3.327,1.420)--(3.398,1.420)--(3.468,1.420)%
  --(3.538,1.420)--(3.608,1.420)--(3.678,1.420)--(3.748,1.420)--(3.818,1.420)--(3.888,1.420)%
  --(3.959,1.420)--(4.029,1.420)--(4.099,1.420)--(4.169,1.420)--(4.239,1.420)--(4.309,1.420)%
  --(4.379,1.420)--(4.450,1.420)--(4.520,1.420)--(4.590,1.420)--(4.660,1.420)--(4.730,1.420)%
  --(4.800,1.420)--(4.870,1.420)--(4.940,1.420)--(5.011,1.420)--(5.081,1.420)--(5.151,1.420)%
  --(5.221,1.420)--(5.291,1.420)--(5.361,1.420)--(5.431,1.420)--(5.501,1.420)--(5.572,1.420)%
  --(5.642,1.420)--(5.712,1.420)--(5.782,1.420)--(5.852,1.420)--(5.922,1.420)--(5.992,1.420)%
  --(6.063,1.420)--(6.133,1.420)--(6.203,1.420)--(6.273,1.420)--(6.343,1.420)--(6.413,1.420)%
  --(6.483,1.420)--(6.553,1.420)--(6.624,1.420)--(6.694,1.420)--(6.764,1.420)--(6.834,1.420)%
  --(6.904,1.420)--(6.974,1.420)--(7.044,1.420)--(7.115,1.420)--(7.185,1.420)--(7.255,1.420)%
  --(7.325,1.420)--(7.395,1.420)--(7.465,1.420)--(7.535,1.420)--(7.605,1.420)--(7.676,1.420)%
  --(7.746,1.420)--(7.816,1.420)--(7.886,1.420)--(7.956,1.420)--(8.026,1.420)--(8.096,1.420)%
  --(8.166,1.420)--(8.237,1.420)--(8.307,1.420)--(8.377,1.420)--(8.447,1.420);
\gpcolor{gp lt color 7}
\gpsetlinetype{gp lt plot 0}
\draw[gp path] (1.504,4.978)--(2.661,5.217)--(2.865,5.457)--(3.213,5.700)--(3.358,5.945)%
  --(3.417,6.190)--(3.562,6.436)--(3.765,6.685)--(3.818,6.937)--(4.976,7.361)--(6.133,7.784)%
  --(7.290,8.208)--(8.447,8.631);
\gppoint{gp mark 9}{(1.504,4.978)}
\gppoint{gp mark 9}{(2.661,5.217)}
\gppoint{gp mark 9}{(2.865,5.457)}
\gppoint{gp mark 9}{(3.213,5.700)}
\gppoint{gp mark 9}{(3.358,5.945)}
\gppoint{gp mark 9}{(3.417,6.190)}
\gppoint{gp mark 9}{(3.562,6.436)}
\gppoint{gp mark 9}{(3.765,6.685)}
\gppoint{gp mark 9}{(3.818,6.937)}
\gppoint{gp mark 9}{(4.976,7.361)}
\gppoint{gp mark 9}{(6.133,7.784)}
\gppoint{gp mark 9}{(7.290,8.208)}
\gppoint{gp mark 9}{(8.447,8.631)}
\gpsetlinetype{gp lt axes}
\draw[gp path] (1.504,1.459)--(1.574,1.459)--(1.644,1.459)--(1.714,1.459)--(1.785,1.459)%
  --(1.855,1.459)--(1.925,1.459)--(1.995,1.459)--(2.065,1.459)--(2.135,1.459)--(2.205,1.459)%
  --(2.275,1.459)--(2.346,1.459)--(2.416,1.459)--(2.486,1.459)--(2.556,1.459)--(2.626,1.459)%
  --(2.696,1.459)--(2.766,1.459)--(2.836,1.459)--(2.907,1.459)--(2.977,1.459)--(3.047,1.459)%
  --(3.117,1.459)--(3.187,1.459)--(3.257,1.459)--(3.327,1.459)--(3.398,1.459)--(3.468,1.459)%
  --(3.538,1.459)--(3.608,1.459)--(3.678,1.459)--(3.748,1.459)--(3.818,1.459)--(3.888,1.459)%
  --(3.959,1.459)--(4.029,1.459)--(4.099,1.459)--(4.169,1.459)--(4.239,1.459)--(4.309,1.459)%
  --(4.379,1.459)--(4.450,1.459)--(4.520,1.459)--(4.590,1.459)--(4.660,1.459)--(4.730,1.459)%
  --(4.800,1.459)--(4.870,1.459)--(4.940,1.459)--(5.011,1.459)--(5.081,1.459)--(5.151,1.459)%
  --(5.221,1.459)--(5.291,1.459)--(5.361,1.459)--(5.431,1.459)--(5.501,1.459)--(5.572,1.459)%
  --(5.642,1.459)--(5.712,1.459)--(5.782,1.459)--(5.852,1.459)--(5.922,1.459)--(5.992,1.459)%
  --(6.063,1.459)--(6.133,1.459)--(6.203,1.459)--(6.273,1.459)--(6.343,1.459)--(6.413,1.459)%
  --(6.483,1.459)--(6.553,1.459)--(6.624,1.459)--(6.694,1.459)--(6.764,1.459)--(6.834,1.459)%
  --(6.904,1.459)--(6.974,1.459)--(7.044,1.459)--(7.115,1.459)--(7.185,1.459)--(7.255,1.459)%
  --(7.325,1.459)--(7.395,1.459)--(7.465,1.459)--(7.535,1.459)--(7.605,1.459)--(7.676,1.459)%
  --(7.746,1.459)--(7.816,1.459)--(7.886,1.459)--(7.956,1.459)--(8.026,1.459)--(8.096,1.459)%
  --(8.166,1.459)--(8.237,1.459)--(8.307,1.459)--(8.377,1.459)--(8.447,1.459);
\gpcolor{gp lt color 0}
\gpsetlinetype{gp lt plot 0}
\draw[gp path] (1.504,3.050)--(2.661,3.404)--(2.865,3.758)--(3.213,4.113)--(3.358,4.467)%
  --(3.417,4.822)--(3.562,5.177)--(3.765,5.533)--(3.818,5.888)--(4.976,6.279)--(6.133,6.975)%
  --(7.290,7.803)--(8.447,8.631);
\gppoint{gp mark 10}{(1.504,3.050)}
\gppoint{gp mark 10}{(2.661,3.404)}
\gppoint{gp mark 10}{(2.865,3.758)}
\gppoint{gp mark 10}{(3.213,4.113)}
\gppoint{gp mark 10}{(3.358,4.467)}
\gppoint{gp mark 10}{(3.417,4.822)}
\gppoint{gp mark 10}{(3.562,5.177)}
\gppoint{gp mark 10}{(3.765,5.533)}
\gppoint{gp mark 10}{(3.818,5.888)}
\gppoint{gp mark 10}{(4.976,6.279)}
\gppoint{gp mark 10}{(6.133,6.975)}
\gppoint{gp mark 10}{(7.290,7.803)}
\gppoint{gp mark 10}{(8.447,8.631)}
\gpsetlinetype{gp lt axes}
\draw[gp path] (1.504,1.436)--(1.574,1.436)--(1.644,1.436)--(1.714,1.436)--(1.785,1.436)%
  --(1.855,1.436)--(1.925,1.436)--(1.995,1.436)--(2.065,1.436)--(2.135,1.436)--(2.205,1.436)%
  --(2.275,1.436)--(2.346,1.436)--(2.416,1.436)--(2.486,1.436)--(2.556,1.436)--(2.626,1.436)%
  --(2.696,1.436)--(2.766,1.436)--(2.836,1.436)--(2.907,1.436)--(2.977,1.436)--(3.047,1.436)%
  --(3.117,1.436)--(3.187,1.436)--(3.257,1.436)--(3.327,1.436)--(3.398,1.436)--(3.468,1.436)%
  --(3.538,1.436)--(3.608,1.436)--(3.678,1.436)--(3.748,1.436)--(3.818,1.436)--(3.888,1.436)%
  --(3.959,1.436)--(4.029,1.436)--(4.099,1.436)--(4.169,1.436)--(4.239,1.436)--(4.309,1.436)%
  --(4.379,1.436)--(4.450,1.436)--(4.520,1.436)--(4.590,1.436)--(4.660,1.436)--(4.730,1.436)%
  --(4.800,1.436)--(4.870,1.436)--(4.940,1.436)--(5.011,1.436)--(5.081,1.436)--(5.151,1.436)%
  --(5.221,1.436)--(5.291,1.436)--(5.361,1.436)--(5.431,1.436)--(5.501,1.436)--(5.572,1.436)%
  --(5.642,1.436)--(5.712,1.436)--(5.782,1.436)--(5.852,1.436)--(5.922,1.436)--(5.992,1.436)%
  --(6.063,1.436)--(6.133,1.436)--(6.203,1.436)--(6.273,1.436)--(6.343,1.436)--(6.413,1.436)%
  --(6.483,1.436)--(6.553,1.436)--(6.624,1.436)--(6.694,1.436)--(6.764,1.436)--(6.834,1.436)%
  --(6.904,1.436)--(6.974,1.436)--(7.044,1.436)--(7.115,1.436)--(7.185,1.436)--(7.255,1.436)%
  --(7.325,1.436)--(7.395,1.436)--(7.465,1.436)--(7.535,1.436)--(7.605,1.436)--(7.676,1.436)%
  --(7.746,1.436)--(7.816,1.436)--(7.886,1.436)--(7.956,1.436)--(8.026,1.436)--(8.096,1.436)%
  --(8.166,1.436)--(8.237,1.436)--(8.307,1.436)--(8.377,1.436)--(8.447,1.436);
\gpcolor{gp lt color 1}
\gpsetlinetype{gp lt plot 0}
\draw[gp path] (1.504,1.447)--(2.661,1.909)--(2.865,2.371)--(3.213,2.834)--(3.358,3.296)%
  --(3.417,3.759)--(3.562,4.222)--(3.765,4.685)--(3.818,5.149)--(4.976,5.641)--(6.133,6.433)%
  --(7.290,7.532)--(8.447,8.631);
\gppoint{gp mark 11}{(1.504,1.447)}
\gppoint{gp mark 11}{(2.661,1.909)}
\gppoint{gp mark 11}{(2.865,2.371)}
\gppoint{gp mark 11}{(3.213,2.834)}
\gppoint{gp mark 11}{(3.358,3.296)}
\gppoint{gp mark 11}{(3.417,3.759)}
\gppoint{gp mark 11}{(3.562,4.222)}
\gppoint{gp mark 11}{(3.765,4.685)}
\gppoint{gp mark 11}{(3.818,5.149)}
\gppoint{gp mark 11}{(4.976,5.641)}
\gppoint{gp mark 11}{(6.133,6.433)}
\gppoint{gp mark 11}{(7.290,7.532)}
\gppoint{gp mark 11}{(8.447,8.631)}
\gpsetlinetype{gp lt axes}
\draw[gp path] (1.504,1.447)--(1.574,1.447)--(1.644,1.447)--(1.714,1.447)--(1.785,1.447)%
  --(1.855,1.447)--(1.925,1.447)--(1.995,1.447)--(2.065,1.447)--(2.135,1.447)--(2.205,1.447)%
  --(2.275,1.447)--(2.346,1.447)--(2.416,1.447)--(2.486,1.447)--(2.556,1.447)--(2.626,1.447)%
  --(2.696,1.447)--(2.766,1.447)--(2.836,1.447)--(2.907,1.447)--(2.977,1.447)--(3.047,1.447)%
  --(3.117,1.447)--(3.187,1.447)--(3.257,1.447)--(3.327,1.447)--(3.398,1.447)--(3.468,1.447)%
  --(3.538,1.447)--(3.608,1.447)--(3.678,1.447)--(3.748,1.447)--(3.818,1.447)--(3.888,1.447)%
  --(3.959,1.447)--(4.029,1.447)--(4.099,1.447)--(4.169,1.447)--(4.239,1.447)--(4.309,1.447)%
  --(4.379,1.447)--(4.450,1.447)--(4.520,1.447)--(4.590,1.447)--(4.660,1.447)--(4.730,1.447)%
  --(4.800,1.447)--(4.870,1.447)--(4.940,1.447)--(5.011,1.447)--(5.081,1.447)--(5.151,1.447)%
  --(5.221,1.447)--(5.291,1.447)--(5.361,1.447)--(5.431,1.447)--(5.501,1.447)--(5.572,1.447)%
  --(5.642,1.447)--(5.712,1.447)--(5.782,1.447)--(5.852,1.447)--(5.922,1.447)--(5.992,1.447)%
  --(6.063,1.447)--(6.133,1.447)--(6.203,1.447)--(6.273,1.447)--(6.343,1.447)--(6.413,1.447)%
  --(6.483,1.447)--(6.553,1.447)--(6.624,1.447)--(6.694,1.447)--(6.764,1.447)--(6.834,1.447)%
  --(6.904,1.447)--(6.974,1.447)--(7.044,1.447)--(7.115,1.447)--(7.185,1.447)--(7.255,1.447)%
  --(7.325,1.447)--(7.395,1.447)--(7.465,1.447)--(7.535,1.447)--(7.605,1.447)--(7.676,1.447)%
  --(7.746,1.447)--(7.816,1.447)--(7.886,1.447)--(7.956,1.447)--(8.026,1.447)--(8.096,1.447)%
  --(8.166,1.447)--(8.237,1.447)--(8.307,1.447)--(8.377,1.447)--(8.447,1.447);
\gpcolor{gp lt color 2}
\gpsetlinetype{gp lt plot 0}
\draw[gp path] (1.504,1.633)--(2.661,2.281)--(2.865,2.929)--(3.213,3.577)--(3.358,4.226)%
  --(3.417,4.875)--(3.562,5.525)--(3.765,6.174)--(3.818,6.825)--(4.976,7.516)--(6.133,8.631);
\gppoint{gp mark 12}{(1.504,1.633)}
\gppoint{gp mark 12}{(2.661,2.281)}
\gppoint{gp mark 12}{(2.865,2.929)}
\gppoint{gp mark 12}{(3.213,3.577)}
\gppoint{gp mark 12}{(3.358,4.226)}
\gppoint{gp mark 12}{(3.417,4.875)}
\gppoint{gp mark 12}{(3.562,5.525)}
\gppoint{gp mark 12}{(3.765,6.174)}
\gppoint{gp mark 12}{(3.818,6.825)}
\gppoint{gp mark 12}{(4.976,7.516)}
\gppoint{gp mark 12}{(6.133,8.631)}
\gpsetlinetype{gp lt axes}
\draw[gp path] (1.504,1.633)--(1.574,1.633)--(1.644,1.633)--(1.714,1.633)--(1.785,1.633)%
  --(1.855,1.633)--(1.925,1.633)--(1.995,1.633)--(2.065,1.633)--(2.135,1.633)--(2.205,1.633)%
  --(2.275,1.633)--(2.346,1.633)--(2.416,1.633)--(2.486,1.633)--(2.556,1.633)--(2.626,1.633)%
  --(2.696,1.633)--(2.766,1.633)--(2.836,1.633)--(2.907,1.633)--(2.977,1.633)--(3.047,1.633)%
  --(3.117,1.633)--(3.187,1.633)--(3.257,1.633)--(3.327,1.633)--(3.398,1.633)--(3.468,1.633)%
  --(3.538,1.633)--(3.608,1.633)--(3.678,1.633)--(3.748,1.633)--(3.818,1.633)--(3.888,1.633)%
  --(3.959,1.633)--(4.029,1.633)--(4.099,1.633)--(4.169,1.633)--(4.239,1.633)--(4.309,1.633)%
  --(4.379,1.633)--(4.450,1.633)--(4.520,1.633)--(4.590,1.633)--(4.660,1.633)--(4.730,1.633)%
  --(4.800,1.633)--(4.870,1.633)--(4.940,1.633)--(5.011,1.633)--(5.081,1.633)--(5.151,1.633)%
  --(5.221,1.633)--(5.291,1.633)--(5.361,1.633)--(5.431,1.633)--(5.501,1.633)--(5.572,1.633)%
  --(5.642,1.633)--(5.712,1.633)--(5.782,1.633)--(5.852,1.633)--(5.922,1.633)--(5.992,1.633)%
  --(6.063,1.633)--(6.133,1.633)--(6.203,1.633)--(6.273,1.633)--(6.343,1.633)--(6.413,1.633)%
  --(6.483,1.633)--(6.553,1.633)--(6.624,1.633)--(6.694,1.633)--(6.764,1.633)--(6.834,1.633)%
  --(6.904,1.633)--(6.974,1.633)--(7.044,1.633)--(7.115,1.633)--(7.185,1.633)--(7.255,1.633)%
  --(7.325,1.633)--(7.395,1.633)--(7.465,1.633)--(7.535,1.633)--(7.605,1.633)--(7.676,1.633)%
  --(7.746,1.633)--(7.816,1.633)--(7.886,1.633)--(7.956,1.633)--(8.026,1.633)--(8.096,1.633)%
  --(8.166,1.633)--(8.237,1.633)--(8.307,1.633)--(8.377,1.633)--(8.447,1.633);
\gpcolor{gp lt color 3}
\gpsetlinetype{gp lt plot 0}
\draw[gp path] (1.504,1.649)--(2.661,2.313)--(2.865,2.977)--(3.213,3.641)--(3.358,4.305)%
  --(3.417,4.970)--(3.562,5.634)--(3.765,6.299)--(3.818,6.965)--(4.976,7.657)--(6.133,8.631);
\gppoint{gp mark 13}{(1.504,1.649)}
\gppoint{gp mark 13}{(2.661,2.313)}
\gppoint{gp mark 13}{(2.865,2.977)}
\gppoint{gp mark 13}{(3.213,3.641)}
\gppoint{gp mark 13}{(3.358,4.305)}
\gppoint{gp mark 13}{(3.417,4.970)}
\gppoint{gp mark 13}{(3.562,5.634)}
\gppoint{gp mark 13}{(3.765,6.299)}
\gppoint{gp mark 13}{(3.818,6.965)}
\gppoint{gp mark 13}{(4.976,7.657)}
\gppoint{gp mark 13}{(6.133,8.631)}
\gpsetlinetype{gp lt axes}
\draw[gp path] (1.504,1.649)--(1.574,1.649)--(1.644,1.649)--(1.714,1.649)--(1.785,1.649)%
  --(1.855,1.649)--(1.925,1.649)--(1.995,1.649)--(2.065,1.649)--(2.135,1.649)--(2.205,1.649)%
  --(2.275,1.649)--(2.346,1.649)--(2.416,1.649)--(2.486,1.649)--(2.556,1.649)--(2.626,1.649)%
  --(2.696,1.649)--(2.766,1.649)--(2.836,1.649)--(2.907,1.649)--(2.977,1.649)--(3.047,1.649)%
  --(3.117,1.649)--(3.187,1.649)--(3.257,1.649)--(3.327,1.649)--(3.398,1.649)--(3.468,1.649)%
  --(3.538,1.649)--(3.608,1.649)--(3.678,1.649)--(3.748,1.649)--(3.818,1.649)--(3.888,1.649)%
  --(3.959,1.649)--(4.029,1.649)--(4.099,1.649)--(4.169,1.649)--(4.239,1.649)--(4.309,1.649)%
  --(4.379,1.649)--(4.450,1.649)--(4.520,1.649)--(4.590,1.649)--(4.660,1.649)--(4.730,1.649)%
  --(4.800,1.649)--(4.870,1.649)--(4.940,1.649)--(5.011,1.649)--(5.081,1.649)--(5.151,1.649)%
  --(5.221,1.649)--(5.291,1.649)--(5.361,1.649)--(5.431,1.649)--(5.501,1.649)--(5.572,1.649)%
  --(5.642,1.649)--(5.712,1.649)--(5.782,1.649)--(5.852,1.649)--(5.922,1.649)--(5.992,1.649)%
  --(6.063,1.649)--(6.133,1.649)--(6.203,1.649)--(6.273,1.649)--(6.343,1.649)--(6.413,1.649)%
  --(6.483,1.649)--(6.553,1.649)--(6.624,1.649)--(6.694,1.649)--(6.764,1.649)--(6.834,1.649)%
  --(6.904,1.649)--(6.974,1.649)--(7.044,1.649)--(7.115,1.649)--(7.185,1.649)--(7.255,1.649)%
  --(7.325,1.649)--(7.395,1.649)--(7.465,1.649)--(7.535,1.649)--(7.605,1.649)--(7.676,1.649)%
  --(7.746,1.649)--(7.816,1.649)--(7.886,1.649)--(7.956,1.649)--(8.026,1.649)--(8.096,1.649)%
  --(8.166,1.649)--(8.237,1.649)--(8.307,1.649)--(8.377,1.649)--(8.447,1.649);
\gpcolor{gp lt color 4}
\gpsetlinetype{gp lt plot 0}
\draw[gp path] (1.504,8.581)--(2.661,8.583)--(2.865,8.585)--(3.213,8.587)--(3.358,8.590)%
  --(3.417,8.592)--(3.562,8.594)--(3.765,8.596)--(3.818,8.598)--(4.976,8.604)--(6.133,8.613)%
  --(7.290,8.622)--(8.447,8.631);
\gppoint{gp mark 14}{(1.504,8.581)}
\gppoint{gp mark 14}{(2.661,8.583)}
\gppoint{gp mark 14}{(2.865,8.585)}
\gppoint{gp mark 14}{(3.213,8.587)}
\gppoint{gp mark 14}{(3.358,8.590)}
\gppoint{gp mark 14}{(3.417,8.592)}
\gppoint{gp mark 14}{(3.562,8.594)}
\gppoint{gp mark 14}{(3.765,8.596)}
\gppoint{gp mark 14}{(3.818,8.598)}
\gppoint{gp mark 14}{(4.976,8.604)}
\gppoint{gp mark 14}{(6.133,8.613)}
\gppoint{gp mark 14}{(7.290,8.622)}
\gppoint{gp mark 14}{(8.447,8.631)}
\gpsetlinetype{gp lt axes}
\draw[gp path] (1.504,1.435)--(1.574,1.435)--(1.644,1.435)--(1.714,1.435)--(1.785,1.435)%
  --(1.855,1.435)--(1.925,1.435)--(1.995,1.435)--(2.065,1.435)--(2.135,1.435)--(2.205,1.435)%
  --(2.275,1.435)--(2.346,1.435)--(2.416,1.435)--(2.486,1.435)--(2.556,1.435)--(2.626,1.435)%
  --(2.696,1.435)--(2.766,1.435)--(2.836,1.435)--(2.907,1.435)--(2.977,1.435)--(3.047,1.435)%
  --(3.117,1.435)--(3.187,1.435)--(3.257,1.435)--(3.327,1.435)--(3.398,1.435)--(3.468,1.435)%
  --(3.538,1.435)--(3.608,1.435)--(3.678,1.435)--(3.748,1.435)--(3.818,1.435)--(3.888,1.435)%
  --(3.959,1.435)--(4.029,1.435)--(4.099,1.435)--(4.169,1.435)--(4.239,1.435)--(4.309,1.435)%
  --(4.379,1.435)--(4.450,1.435)--(4.520,1.435)--(4.590,1.435)--(4.660,1.435)--(4.730,1.435)%
  --(4.800,1.435)--(4.870,1.435)--(4.940,1.435)--(5.011,1.435)--(5.081,1.435)--(5.151,1.435)%
  --(5.221,1.435)--(5.291,1.435)--(5.361,1.435)--(5.431,1.435)--(5.501,1.435)--(5.572,1.435)%
  --(5.642,1.435)--(5.712,1.435)--(5.782,1.435)--(5.852,1.435)--(5.922,1.435)--(5.992,1.435)%
  --(6.063,1.435)--(6.133,1.435)--(6.203,1.435)--(6.273,1.435)--(6.343,1.435)--(6.413,1.435)%
  --(6.483,1.435)--(6.553,1.435)--(6.624,1.435)--(6.694,1.435)--(6.764,1.435)--(6.834,1.435)%
  --(6.904,1.435)--(6.974,1.435)--(7.044,1.435)--(7.115,1.435)--(7.185,1.435)--(7.255,1.435)%
  --(7.325,1.435)--(7.395,1.435)--(7.465,1.435)--(7.535,1.435)--(7.605,1.435)--(7.676,1.435)%
  --(7.746,1.435)--(7.816,1.435)--(7.886,1.435)--(7.956,1.435)--(8.026,1.435)--(8.096,1.435)%
  --(8.166,1.435)--(8.237,1.435)--(8.307,1.435)--(8.377,1.435)--(8.447,1.435);
\gpcolor{gp lt color 5}
\gpsetlinetype{gp lt plot 0}
\draw[gp path] (1.504,8.381)--(2.661,8.386)--(2.865,8.391)--(3.213,8.396)--(3.358,8.402)%
  --(3.417,8.407)--(3.562,8.413)--(3.765,8.419)--(3.818,8.425)--(4.976,8.441)--(6.133,8.504)%
  --(7.290,8.568)--(8.447,8.631);
\gppoint{gp mark 15}{(1.504,8.381)}
\gppoint{gp mark 15}{(2.661,8.386)}
\gppoint{gp mark 15}{(2.865,8.391)}
\gppoint{gp mark 15}{(3.213,8.396)}
\gppoint{gp mark 15}{(3.358,8.402)}
\gppoint{gp mark 15}{(3.417,8.407)}
\gppoint{gp mark 15}{(3.562,8.413)}
\gppoint{gp mark 15}{(3.765,8.419)}
\gppoint{gp mark 15}{(3.818,8.425)}
\gppoint{gp mark 15}{(4.976,8.441)}
\gppoint{gp mark 15}{(6.133,8.504)}
\gppoint{gp mark 15}{(7.290,8.568)}
\gppoint{gp mark 15}{(8.447,8.631)}
\gpsetlinetype{gp lt axes}
\draw[gp path] (1.504,1.098)--(1.574,1.098)--(1.644,1.098)--(1.714,1.098)--(1.785,1.098)%
  --(1.855,1.098)--(1.925,1.098)--(1.995,1.098)--(2.065,1.098)--(2.135,1.098)--(2.205,1.098)%
  --(2.275,1.098)--(2.346,1.098)--(2.416,1.098)--(2.486,1.098)--(2.556,1.098)--(2.626,1.098)%
  --(2.696,1.098)--(2.766,1.098)--(2.836,1.098)--(2.907,1.098)--(2.977,1.098)--(3.047,1.098)%
  --(3.117,1.098)--(3.187,1.098)--(3.257,1.098)--(3.327,1.098)--(3.398,1.098)--(3.468,1.098)%
  --(3.538,1.098)--(3.608,1.098)--(3.678,1.098)--(3.748,1.098)--(3.818,1.098)--(3.888,1.098)%
  --(3.959,1.098)--(4.029,1.098)--(4.099,1.098)--(4.169,1.098)--(4.239,1.098)--(4.309,1.098)%
  --(4.379,1.098)--(4.450,1.098)--(4.520,1.098)--(4.590,1.098)--(4.660,1.098)--(4.730,1.098)%
  --(4.800,1.098)--(4.870,1.098)--(4.940,1.098)--(5.011,1.098)--(5.081,1.098)--(5.151,1.098)%
  --(5.221,1.098)--(5.291,1.098)--(5.361,1.098)--(5.431,1.098)--(5.501,1.098)--(5.572,1.098)%
  --(5.642,1.098)--(5.712,1.098)--(5.782,1.098)--(5.852,1.098)--(5.922,1.098)--(5.992,1.098)%
  --(6.063,1.098)--(6.133,1.098)--(6.203,1.098)--(6.273,1.098)--(6.343,1.098)--(6.413,1.098)%
  --(6.483,1.098)--(6.553,1.098)--(6.624,1.098)--(6.694,1.098)--(6.764,1.098)--(6.834,1.098)%
  --(6.904,1.098)--(6.974,1.098)--(7.044,1.098)--(7.115,1.098)--(7.185,1.098)--(7.255,1.098)%
  --(7.325,1.098)--(7.395,1.098)--(7.465,1.098)--(7.535,1.098)--(7.605,1.098)--(7.676,1.098)%
  --(7.746,1.098)--(7.816,1.098)--(7.886,1.098)--(7.956,1.098)--(8.026,1.098)--(8.096,1.098)%
  --(8.166,1.098)--(8.237,1.098)--(8.307,1.098)--(8.377,1.098)--(8.447,1.098);
\gpcolor{gp lt color 6}
\gpsetlinetype{gp lt plot 0}
\draw[gp path] (1.504,1.135)--(2.661,1.287)--(2.865,1.440)--(3.213,1.595)--(3.358,1.753)%
  --(3.417,1.909)--(3.562,2.071)--(3.765,2.238)--(3.818,2.406)--(4.976,2.905)--(6.133,4.813)%
  --(7.290,6.722)--(8.447,8.631);
\gppoint{gp mark 1}{(1.504,1.135)}
\gppoint{gp mark 1}{(2.661,1.287)}
\gppoint{gp mark 1}{(2.865,1.440)}
\gppoint{gp mark 1}{(3.213,1.595)}
\gppoint{gp mark 1}{(3.358,1.753)}
\gppoint{gp mark 1}{(3.417,1.909)}
\gppoint{gp mark 1}{(3.562,2.071)}
\gppoint{gp mark 1}{(3.765,2.238)}
\gppoint{gp mark 1}{(3.818,2.406)}
\gppoint{gp mark 1}{(4.976,2.905)}
\gppoint{gp mark 1}{(6.133,4.813)}
\gppoint{gp mark 1}{(7.290,6.722)}
\gppoint{gp mark 1}{(8.447,8.631)}
\gpsetlinetype{gp lt axes}
\draw[gp path] (1.504,1.135)--(1.574,1.135)--(1.644,1.135)--(1.714,1.135)--(1.785,1.135)%
  --(1.855,1.135)--(1.925,1.135)--(1.995,1.135)--(2.065,1.135)--(2.135,1.135)--(2.205,1.135)%
  --(2.275,1.135)--(2.346,1.135)--(2.416,1.135)--(2.486,1.135)--(2.556,1.135)--(2.626,1.135)%
  --(2.696,1.135)--(2.766,1.135)--(2.836,1.135)--(2.907,1.135)--(2.977,1.135)--(3.047,1.135)%
  --(3.117,1.135)--(3.187,1.135)--(3.257,1.135)--(3.327,1.135)--(3.398,1.135)--(3.468,1.135)%
  --(3.538,1.135)--(3.608,1.135)--(3.678,1.135)--(3.748,1.135)--(3.818,1.135)--(3.888,1.135)%
  --(3.959,1.135)--(4.029,1.135)--(4.099,1.135)--(4.169,1.135)--(4.239,1.135)--(4.309,1.135)%
  --(4.379,1.135)--(4.450,1.135)--(4.520,1.135)--(4.590,1.135)--(4.660,1.135)--(4.730,1.135)%
  --(4.800,1.135)--(4.870,1.135)--(4.940,1.135)--(5.011,1.135)--(5.081,1.135)--(5.151,1.135)%
  --(5.221,1.135)--(5.291,1.135)--(5.361,1.135)--(5.431,1.135)--(5.501,1.135)--(5.572,1.135)%
  --(5.642,1.135)--(5.712,1.135)--(5.782,1.135)--(5.852,1.135)--(5.922,1.135)--(5.992,1.135)%
  --(6.063,1.135)--(6.133,1.135)--(6.203,1.135)--(6.273,1.135)--(6.343,1.135)--(6.413,1.135)%
  --(6.483,1.135)--(6.553,1.135)--(6.624,1.135)--(6.694,1.135)--(6.764,1.135)--(6.834,1.135)%
  --(6.904,1.135)--(6.974,1.135)--(7.044,1.135)--(7.115,1.135)--(7.185,1.135)--(7.255,1.135)%
  --(7.325,1.135)--(7.395,1.135)--(7.465,1.135)--(7.535,1.135)--(7.605,1.135)--(7.676,1.135)%
  --(7.746,1.135)--(7.816,1.135)--(7.886,1.135)--(7.956,1.135)--(8.026,1.135)--(8.096,1.135)%
  --(8.166,1.135)--(8.237,1.135)--(8.307,1.135)--(8.377,1.135)--(8.447,1.135);
\gpcolor{gp lt color 7}
\gpsetlinetype{gp lt plot 0}
\draw[gp path] (1.504,5.342)--(2.661,5.571)--(2.865,5.800)--(3.213,6.032)--(3.358,6.266)%
  --(3.417,6.499)--(3.562,6.737)--(3.765,6.976)--(3.818,7.219)--(4.976,7.572)--(6.133,7.925)%
  --(7.290,8.278)--(8.447,8.631);
\gppoint{gp mark 2}{(1.504,5.342)}
\gppoint{gp mark 2}{(2.661,5.571)}
\gppoint{gp mark 2}{(2.865,5.800)}
\gppoint{gp mark 2}{(3.213,6.032)}
\gppoint{gp mark 2}{(3.358,6.266)}
\gppoint{gp mark 2}{(3.417,6.499)}
\gppoint{gp mark 2}{(3.562,6.737)}
\gppoint{gp mark 2}{(3.765,6.976)}
\gppoint{gp mark 2}{(3.818,7.219)}
\gppoint{gp mark 2}{(4.976,7.572)}
\gppoint{gp mark 2}{(6.133,7.925)}
\gppoint{gp mark 2}{(7.290,8.278)}
\gppoint{gp mark 2}{(8.447,8.631)}
\gpsetlinetype{gp lt axes}
\draw[gp path] (1.504,1.467)--(1.574,1.467)--(1.644,1.467)--(1.714,1.467)--(1.785,1.467)%
  --(1.855,1.467)--(1.925,1.467)--(1.995,1.467)--(2.065,1.467)--(2.135,1.467)--(2.205,1.467)%
  --(2.275,1.467)--(2.346,1.467)--(2.416,1.467)--(2.486,1.467)--(2.556,1.467)--(2.626,1.467)%
  --(2.696,1.467)--(2.766,1.467)--(2.836,1.467)--(2.907,1.467)--(2.977,1.467)--(3.047,1.467)%
  --(3.117,1.467)--(3.187,1.467)--(3.257,1.467)--(3.327,1.467)--(3.398,1.467)--(3.468,1.467)%
  --(3.538,1.467)--(3.608,1.467)--(3.678,1.467)--(3.748,1.467)--(3.818,1.467)--(3.888,1.467)%
  --(3.959,1.467)--(4.029,1.467)--(4.099,1.467)--(4.169,1.467)--(4.239,1.467)--(4.309,1.467)%
  --(4.379,1.467)--(4.450,1.467)--(4.520,1.467)--(4.590,1.467)--(4.660,1.467)--(4.730,1.467)%
  --(4.800,1.467)--(4.870,1.467)--(4.940,1.467)--(5.011,1.467)--(5.081,1.467)--(5.151,1.467)%
  --(5.221,1.467)--(5.291,1.467)--(5.361,1.467)--(5.431,1.467)--(5.501,1.467)--(5.572,1.467)%
  --(5.642,1.467)--(5.712,1.467)--(5.782,1.467)--(5.852,1.467)--(5.922,1.467)--(5.992,1.467)%
  --(6.063,1.467)--(6.133,1.467)--(6.203,1.467)--(6.273,1.467)--(6.343,1.467)--(6.413,1.467)%
  --(6.483,1.467)--(6.553,1.467)--(6.624,1.467)--(6.694,1.467)--(6.764,1.467)--(6.834,1.467)%
  --(6.904,1.467)--(6.974,1.467)--(7.044,1.467)--(7.115,1.467)--(7.185,1.467)--(7.255,1.467)%
  --(7.325,1.467)--(7.395,1.467)--(7.465,1.467)--(7.535,1.467)--(7.605,1.467)--(7.676,1.467)%
  --(7.746,1.467)--(7.816,1.467)--(7.886,1.467)--(7.956,1.467)--(8.026,1.467)--(8.096,1.467)%
  --(8.166,1.467)--(8.237,1.467)--(8.307,1.467)--(8.377,1.467)--(8.447,1.467);
\gpcolor{gp lt color 0}
\gpsetlinetype{gp lt plot 0}
\draw[gp path] (1.504,6.647)--(2.661,6.766)--(2.865,6.885)--(3.213,7.005)--(3.358,7.126)%
  --(3.417,7.246)--(3.562,7.367)--(3.765,7.490)--(3.818,7.613)--(4.976,7.805)--(6.133,8.080)%
  --(7.290,8.356)--(8.447,8.631);
\gppoint{gp mark 3}{(1.504,6.647)}
\gppoint{gp mark 3}{(2.661,6.766)}
\gppoint{gp mark 3}{(2.865,6.885)}
\gppoint{gp mark 3}{(3.213,7.005)}
\gppoint{gp mark 3}{(3.358,7.126)}
\gppoint{gp mark 3}{(3.417,7.246)}
\gppoint{gp mark 3}{(3.562,7.367)}
\gppoint{gp mark 3}{(3.765,7.490)}
\gppoint{gp mark 3}{(3.818,7.613)}
\gppoint{gp mark 3}{(4.976,7.805)}
\gppoint{gp mark 3}{(6.133,8.080)}
\gppoint{gp mark 3}{(7.290,8.356)}
\gppoint{gp mark 3}{(8.447,8.631)}
\gpsetlinetype{gp lt axes}
\draw[gp path] (1.504,1.428)--(1.574,1.428)--(1.644,1.428)--(1.714,1.428)--(1.785,1.428)%
  --(1.855,1.428)--(1.925,1.428)--(1.995,1.428)--(2.065,1.428)--(2.135,1.428)--(2.205,1.428)%
  --(2.275,1.428)--(2.346,1.428)--(2.416,1.428)--(2.486,1.428)--(2.556,1.428)--(2.626,1.428)%
  --(2.696,1.428)--(2.766,1.428)--(2.836,1.428)--(2.907,1.428)--(2.977,1.428)--(3.047,1.428)%
  --(3.117,1.428)--(3.187,1.428)--(3.257,1.428)--(3.327,1.428)--(3.398,1.428)--(3.468,1.428)%
  --(3.538,1.428)--(3.608,1.428)--(3.678,1.428)--(3.748,1.428)--(3.818,1.428)--(3.888,1.428)%
  --(3.959,1.428)--(4.029,1.428)--(4.099,1.428)--(4.169,1.428)--(4.239,1.428)--(4.309,1.428)%
  --(4.379,1.428)--(4.450,1.428)--(4.520,1.428)--(4.590,1.428)--(4.660,1.428)--(4.730,1.428)%
  --(4.800,1.428)--(4.870,1.428)--(4.940,1.428)--(5.011,1.428)--(5.081,1.428)--(5.151,1.428)%
  --(5.221,1.428)--(5.291,1.428)--(5.361,1.428)--(5.431,1.428)--(5.501,1.428)--(5.572,1.428)%
  --(5.642,1.428)--(5.712,1.428)--(5.782,1.428)--(5.852,1.428)--(5.922,1.428)--(5.992,1.428)%
  --(6.063,1.428)--(6.133,1.428)--(6.203,1.428)--(6.273,1.428)--(6.343,1.428)--(6.413,1.428)%
  --(6.483,1.428)--(6.553,1.428)--(6.624,1.428)--(6.694,1.428)--(6.764,1.428)--(6.834,1.428)%
  --(6.904,1.428)--(6.974,1.428)--(7.044,1.428)--(7.115,1.428)--(7.185,1.428)--(7.255,1.428)%
  --(7.325,1.428)--(7.395,1.428)--(7.465,1.428)--(7.535,1.428)--(7.605,1.428)--(7.676,1.428)%
  --(7.746,1.428)--(7.816,1.428)--(7.886,1.428)--(7.956,1.428)--(8.026,1.428)--(8.096,1.428)%
  --(8.166,1.428)--(8.237,1.428)--(8.307,1.428)--(8.377,1.428)--(8.447,1.428);
\gpcolor{gp lt color 1}
\gpsetlinetype{gp lt plot 0}
\draw[gp path] (1.504,4.031)--(2.661,4.318)--(2.865,4.606)--(3.213,4.896)--(3.358,5.186)%
  --(3.417,5.478)--(3.562,5.771)--(3.765,6.066)--(3.818,6.363)--(4.976,6.797)--(6.133,7.408)%
  --(7.290,8.020)--(8.447,8.631);
\gppoint{gp mark 4}{(1.504,4.031)}
\gppoint{gp mark 4}{(2.661,4.318)}
\gppoint{gp mark 4}{(2.865,4.606)}
\gppoint{gp mark 4}{(3.213,4.896)}
\gppoint{gp mark 4}{(3.358,5.186)}
\gppoint{gp mark 4}{(3.417,5.478)}
\gppoint{gp mark 4}{(3.562,5.771)}
\gppoint{gp mark 4}{(3.765,6.066)}
\gppoint{gp mark 4}{(3.818,6.363)}
\gppoint{gp mark 4}{(4.976,6.797)}
\gppoint{gp mark 4}{(6.133,7.408)}
\gppoint{gp mark 4}{(7.290,8.020)}
\gppoint{gp mark 4}{(8.447,8.631)}
\gpsetlinetype{gp lt axes}
\draw[gp path] (1.504,1.443)--(1.574,1.443)--(1.644,1.443)--(1.714,1.443)--(1.785,1.443)%
  --(1.855,1.443)--(1.925,1.443)--(1.995,1.443)--(2.065,1.443)--(2.135,1.443)--(2.205,1.443)%
  --(2.275,1.443)--(2.346,1.443)--(2.416,1.443)--(2.486,1.443)--(2.556,1.443)--(2.626,1.443)%
  --(2.696,1.443)--(2.766,1.443)--(2.836,1.443)--(2.907,1.443)--(2.977,1.443)--(3.047,1.443)%
  --(3.117,1.443)--(3.187,1.443)--(3.257,1.443)--(3.327,1.443)--(3.398,1.443)--(3.468,1.443)%
  --(3.538,1.443)--(3.608,1.443)--(3.678,1.443)--(3.748,1.443)--(3.818,1.443)--(3.888,1.443)%
  --(3.959,1.443)--(4.029,1.443)--(4.099,1.443)--(4.169,1.443)--(4.239,1.443)--(4.309,1.443)%
  --(4.379,1.443)--(4.450,1.443)--(4.520,1.443)--(4.590,1.443)--(4.660,1.443)--(4.730,1.443)%
  --(4.800,1.443)--(4.870,1.443)--(4.940,1.443)--(5.011,1.443)--(5.081,1.443)--(5.151,1.443)%
  --(5.221,1.443)--(5.291,1.443)--(5.361,1.443)--(5.431,1.443)--(5.501,1.443)--(5.572,1.443)%
  --(5.642,1.443)--(5.712,1.443)--(5.782,1.443)--(5.852,1.443)--(5.922,1.443)--(5.992,1.443)%
  --(6.063,1.443)--(6.133,1.443)--(6.203,1.443)--(6.273,1.443)--(6.343,1.443)--(6.413,1.443)%
  --(6.483,1.443)--(6.553,1.443)--(6.624,1.443)--(6.694,1.443)--(6.764,1.443)--(6.834,1.443)%
  --(6.904,1.443)--(6.974,1.443)--(7.044,1.443)--(7.115,1.443)--(7.185,1.443)--(7.255,1.443)%
  --(7.325,1.443)--(7.395,1.443)--(7.465,1.443)--(7.535,1.443)--(7.605,1.443)--(7.676,1.443)%
  --(7.746,1.443)--(7.816,1.443)--(7.886,1.443)--(7.956,1.443)--(8.026,1.443)--(8.096,1.443)%
  --(8.166,1.443)--(8.237,1.443)--(8.307,1.443)--(8.377,1.443)--(8.447,1.443);
\gpcolor{gp lt color 2}
\gpsetlinetype{gp lt plot 0}
\draw[gp path] (1.504,5.896)--(2.661,6.060)--(2.865,6.225)--(3.213,6.390)--(3.358,6.557)%
  --(3.417,6.724)--(3.562,6.892)--(3.765,7.061)--(3.818,7.231)--(4.976,7.492)--(6.133,7.872)%
  --(7.290,8.251)--(8.447,8.631);
\gppoint{gp mark 5}{(1.504,5.896)}
\gppoint{gp mark 5}{(2.661,6.060)}
\gppoint{gp mark 5}{(2.865,6.225)}
\gppoint{gp mark 5}{(3.213,6.390)}
\gppoint{gp mark 5}{(3.358,6.557)}
\gppoint{gp mark 5}{(3.417,6.724)}
\gppoint{gp mark 5}{(3.562,6.892)}
\gppoint{gp mark 5}{(3.765,7.061)}
\gppoint{gp mark 5}{(3.818,7.231)}
\gppoint{gp mark 5}{(4.976,7.492)}
\gppoint{gp mark 5}{(6.133,7.872)}
\gppoint{gp mark 5}{(7.290,8.251)}
\gppoint{gp mark 5}{(8.447,8.631)}
\gpsetlinetype{gp lt axes}
\draw[gp path] (1.504,1.433)--(1.574,1.433)--(1.644,1.433)--(1.714,1.433)--(1.785,1.433)%
  --(1.855,1.433)--(1.925,1.433)--(1.995,1.433)--(2.065,1.433)--(2.135,1.433)--(2.205,1.433)%
  --(2.275,1.433)--(2.346,1.433)--(2.416,1.433)--(2.486,1.433)--(2.556,1.433)--(2.626,1.433)%
  --(2.696,1.433)--(2.766,1.433)--(2.836,1.433)--(2.907,1.433)--(2.977,1.433)--(3.047,1.433)%
  --(3.117,1.433)--(3.187,1.433)--(3.257,1.433)--(3.327,1.433)--(3.398,1.433)--(3.468,1.433)%
  --(3.538,1.433)--(3.608,1.433)--(3.678,1.433)--(3.748,1.433)--(3.818,1.433)--(3.888,1.433)%
  --(3.959,1.433)--(4.029,1.433)--(4.099,1.433)--(4.169,1.433)--(4.239,1.433)--(4.309,1.433)%
  --(4.379,1.433)--(4.450,1.433)--(4.520,1.433)--(4.590,1.433)--(4.660,1.433)--(4.730,1.433)%
  --(4.800,1.433)--(4.870,1.433)--(4.940,1.433)--(5.011,1.433)--(5.081,1.433)--(5.151,1.433)%
  --(5.221,1.433)--(5.291,1.433)--(5.361,1.433)--(5.431,1.433)--(5.501,1.433)--(5.572,1.433)%
  --(5.642,1.433)--(5.712,1.433)--(5.782,1.433)--(5.852,1.433)--(5.922,1.433)--(5.992,1.433)%
  --(6.063,1.433)--(6.133,1.433)--(6.203,1.433)--(6.273,1.433)--(6.343,1.433)--(6.413,1.433)%
  --(6.483,1.433)--(6.553,1.433)--(6.624,1.433)--(6.694,1.433)--(6.764,1.433)--(6.834,1.433)%
  --(6.904,1.433)--(6.974,1.433)--(7.044,1.433)--(7.115,1.433)--(7.185,1.433)--(7.255,1.433)%
  --(7.325,1.433)--(7.395,1.433)--(7.465,1.433)--(7.535,1.433)--(7.605,1.433)--(7.676,1.433)%
  --(7.746,1.433)--(7.816,1.433)--(7.886,1.433)--(7.956,1.433)--(8.026,1.433)--(8.096,1.433)%
  --(8.166,1.433)--(8.237,1.433)--(8.307,1.433)--(8.377,1.433)--(8.447,1.433);
\gpcolor{gp lt color 3}
\gpsetlinetype{gp lt plot 0}
\draw[gp path] (1.504,1.442)--(2.661,1.900)--(2.865,2.364)--(3.213,2.851)--(3.358,3.311)%
  --(3.417,3.778)--(3.562,4.266)--(3.765,4.802)--(3.818,5.354)--(4.976,6.173)--(6.133,6.993)%
  --(7.290,7.812)--(8.447,8.631);
\gppoint{gp mark 6}{(1.504,1.442)}
\gppoint{gp mark 6}{(2.661,1.900)}
\gppoint{gp mark 6}{(2.865,2.364)}
\gppoint{gp mark 6}{(3.213,2.851)}
\gppoint{gp mark 6}{(3.358,3.311)}
\gppoint{gp mark 6}{(3.417,3.778)}
\gppoint{gp mark 6}{(3.562,4.266)}
\gppoint{gp mark 6}{(3.765,4.802)}
\gppoint{gp mark 6}{(3.818,5.354)}
\gppoint{gp mark 6}{(4.976,6.173)}
\gppoint{gp mark 6}{(6.133,6.993)}
\gppoint{gp mark 6}{(7.290,7.812)}
\gppoint{gp mark 6}{(8.447,8.631)}
\gpsetlinetype{gp lt axes}
\draw[gp path] (1.504,1.442)--(1.574,1.442)--(1.644,1.442)--(1.714,1.442)--(1.785,1.442)%
  --(1.855,1.442)--(1.925,1.442)--(1.995,1.442)--(2.065,1.442)--(2.135,1.442)--(2.205,1.442)%
  --(2.275,1.442)--(2.346,1.442)--(2.416,1.442)--(2.486,1.442)--(2.556,1.442)--(2.626,1.442)%
  --(2.696,1.442)--(2.766,1.442)--(2.836,1.442)--(2.907,1.442)--(2.977,1.442)--(3.047,1.442)%
  --(3.117,1.442)--(3.187,1.442)--(3.257,1.442)--(3.327,1.442)--(3.398,1.442)--(3.468,1.442)%
  --(3.538,1.442)--(3.608,1.442)--(3.678,1.442)--(3.748,1.442)--(3.818,1.442)--(3.888,1.442)%
  --(3.959,1.442)--(4.029,1.442)--(4.099,1.442)--(4.169,1.442)--(4.239,1.442)--(4.309,1.442)%
  --(4.379,1.442)--(4.450,1.442)--(4.520,1.442)--(4.590,1.442)--(4.660,1.442)--(4.730,1.442)%
  --(4.800,1.442)--(4.870,1.442)--(4.940,1.442)--(5.011,1.442)--(5.081,1.442)--(5.151,1.442)%
  --(5.221,1.442)--(5.291,1.442)--(5.361,1.442)--(5.431,1.442)--(5.501,1.442)--(5.572,1.442)%
  --(5.642,1.442)--(5.712,1.442)--(5.782,1.442)--(5.852,1.442)--(5.922,1.442)--(5.992,1.442)%
  --(6.063,1.442)--(6.133,1.442)--(6.203,1.442)--(6.273,1.442)--(6.343,1.442)--(6.413,1.442)%
  --(6.483,1.442)--(6.553,1.442)--(6.624,1.442)--(6.694,1.442)--(6.764,1.442)--(6.834,1.442)%
  --(6.904,1.442)--(6.974,1.442)--(7.044,1.442)--(7.115,1.442)--(7.185,1.442)--(7.255,1.442)%
  --(7.325,1.442)--(7.395,1.442)--(7.465,1.442)--(7.535,1.442)--(7.605,1.442)--(7.676,1.442)%
  --(7.746,1.442)--(7.816,1.442)--(7.886,1.442)--(7.956,1.442)--(8.026,1.442)--(8.096,1.442)%
  --(8.166,1.442)--(8.237,1.442)--(8.307,1.442)--(8.377,1.442)--(8.447,1.442);
\gpcolor{gp lt color border}
\gpsetlinetype{gp lt border}
\draw[gp path] (1.504,8.631)--(1.504,0.985)--(8.447,0.985)--(8.447,8.631)--cycle;
%% coordinates of the plot area
\gpdefrectangularnode{gp plot 1}{\pgfpoint{1.504cm}{0.985cm}}{\pgfpoint{8.447cm}{8.631cm}}
\end{tikzpicture}
%% gnuplot variables

    \caption{Number of calls to the \command{distance} function with various
        block size implementations}
    \label{profiling:blockSize:distanceCalls}
\end{figure}

\begin{figure}
    \centering
    \begin{tikzpicture}[gnuplot]
%% generated with GNUPLOT 4.4p3 (Lua 5.1.4; terminal rev. 97, script rev. 96a)
%% Mon 22 Oct 2012 22:31:49 EST
\gpcolor{\gprgb{502}{502}{502}}
\gpsetlinetype{gp lt plot 0}
\gpsetlinewidth{1.00}
\draw[gp path] (1.504,0.985)--(1.684,0.985);
\draw[gp path] (9.447,0.985)--(9.267,0.985);
\node[gp node right] at (1.320,0.985) { 0};
\draw[gp path] (1.504,1.850)--(1.684,1.850);
\draw[gp path] (9.447,1.850)--(9.267,1.850);
\node[gp node right] at (1.320,1.850) { 0.1};
\draw[gp path] (1.504,2.714)--(1.684,2.714);
\draw[gp path] (9.447,2.714)--(9.267,2.714);
\node[gp node right] at (1.320,2.714) { 0.2};
\draw[gp path] (1.504,3.579)--(1.684,3.579);
\draw[gp path] (9.447,3.579)--(9.267,3.579);
\node[gp node right] at (1.320,3.579) { 0.3};
\draw[gp path] (1.504,4.443)--(1.684,4.443);
\draw[gp path] (9.447,4.443)--(9.267,4.443);
\node[gp node right] at (1.320,4.443) { 0.4};
\draw[gp path] (1.504,5.308)--(1.684,5.308);
\draw[gp path] (9.447,5.308)--(9.267,5.308);
\node[gp node right] at (1.320,5.308) { 0.5};
\draw[gp path] (1.504,6.173)--(1.684,6.173);
\draw[gp path] (9.447,6.173)--(9.267,6.173);
\node[gp node right] at (1.320,6.173) { 0.6};
\draw[gp path] (1.504,7.037)--(1.684,7.037);
\draw[gp path] (9.447,7.037)--(9.267,7.037);
\node[gp node right] at (1.320,7.037) { 0.7};
\draw[gp path] (1.504,7.902)--(1.684,7.902);
\draw[gp path] (9.447,7.902)--(9.267,7.902);
\node[gp node right] at (1.320,7.902) { 0.8};
\draw[gp path] (1.504,8.766)--(1.684,8.766);
\draw[gp path] (9.447,8.766)--(9.267,8.766);
\node[gp node right] at (1.320,8.766) { 0.9};
\draw[gp path] (1.504,9.631)--(1.684,9.631);
\draw[gp path] (9.447,9.631)--(9.267,9.631);
\node[gp node right] at (1.320,9.631) { 1};
\draw[gp path] (1.504,0.985)--(1.504,1.165);
\draw[gp path] (1.504,9.631)--(1.504,9.451);
\node[gp node center] at (1.504,0.677) {$10^{0}$};
\draw[gp path] (1.903,0.985)--(1.903,1.075);
\draw[gp path] (1.903,9.631)--(1.903,9.541);
\draw[gp path] (2.429,0.985)--(2.429,1.075);
\draw[gp path] (2.429,9.631)--(2.429,9.541);
\draw[gp path] (2.700,0.985)--(2.700,1.075);
\draw[gp path] (2.700,9.631)--(2.700,9.541);
\draw[gp path] (2.828,0.985)--(2.828,1.165);
\draw[gp path] (2.828,9.631)--(2.828,9.451);
\node[gp node center] at (2.828,0.677) {$10^{1}$};
\draw[gp path] (3.226,0.985)--(3.226,1.075);
\draw[gp path] (3.226,9.631)--(3.226,9.541);
\draw[gp path] (3.753,0.985)--(3.753,1.075);
\draw[gp path] (3.753,9.631)--(3.753,9.541);
\draw[gp path] (4.023,0.985)--(4.023,1.075);
\draw[gp path] (4.023,9.631)--(4.023,9.541);
\draw[gp path] (4.152,0.985)--(4.152,1.165);
\draw[gp path] (4.152,9.631)--(4.152,9.451);
\node[gp node center] at (4.152,0.677) {$10^{2}$};
\draw[gp path] (4.550,0.985)--(4.550,1.075);
\draw[gp path] (4.550,9.631)--(4.550,9.541);
\draw[gp path] (5.077,0.985)--(5.077,1.075);
\draw[gp path] (5.077,9.631)--(5.077,9.541);
\draw[gp path] (5.347,0.985)--(5.347,1.075);
\draw[gp path] (5.347,9.631)--(5.347,9.541);
\draw[gp path] (5.476,0.985)--(5.476,1.165);
\draw[gp path] (5.476,9.631)--(5.476,9.451);
\node[gp node center] at (5.476,0.677) {$10^{3}$};
\draw[gp path] (5.874,0.985)--(5.874,1.075);
\draw[gp path] (5.874,9.631)--(5.874,9.541);
\draw[gp path] (6.401,0.985)--(6.401,1.075);
\draw[gp path] (6.401,9.631)--(6.401,9.541);
\draw[gp path] (6.671,0.985)--(6.671,1.075);
\draw[gp path] (6.671,9.631)--(6.671,9.541);
\draw[gp path] (6.799,0.985)--(6.799,1.165);
\draw[gp path] (6.799,9.631)--(6.799,9.451);
\node[gp node center] at (6.799,0.677) {$10^{4}$};
\draw[gp path] (7.198,0.985)--(7.198,1.075);
\draw[gp path] (7.198,9.631)--(7.198,9.541);
\draw[gp path] (7.725,0.985)--(7.725,1.075);
\draw[gp path] (7.725,9.631)--(7.725,9.541);
\draw[gp path] (7.995,0.985)--(7.995,1.075);
\draw[gp path] (7.995,9.631)--(7.995,9.541);
\draw[gp path] (8.123,0.985)--(8.123,1.165);
\draw[gp path] (8.123,9.631)--(8.123,9.451);
\node[gp node center] at (8.123,0.677) {$10^{5}$};
\draw[gp path] (8.522,0.985)--(8.522,1.075);
\draw[gp path] (8.522,9.631)--(8.522,9.541);
\draw[gp path] (9.048,0.985)--(9.048,1.075);
\draw[gp path] (9.048,9.631)--(9.048,9.541);
\draw[gp path] (9.319,0.985)--(9.319,1.075);
\draw[gp path] (9.319,9.631)--(9.319,9.541);
\draw[gp path] (9.447,0.985)--(9.447,1.165);
\draw[gp path] (9.447,9.631)--(9.447,9.451);
\node[gp node center] at (9.447,0.677) {$10^{6}$};
\draw[gp path] (1.504,9.631)--(1.504,0.985)--(9.447,0.985);
\gpcolor{gp lt color border}
\node[gp node center,rotate=-270] at (0.246,5.308) {Number of vectors pruned (normalised)};
\node[gp node center] at (5.475,0.215) {Block size};
\gpcolor{gp lt color axes}
\draw[gp path] (1.504,5.921)--(2.828,6.405)--(3.061,6.883)--(3.459,7.356)--(3.625,7.820)%
  --(3.693,8.287)--(3.858,8.748)--(4.091,9.198)--(4.152,9.631)--(5.475,9.631)--(6.799,9.631)%
  --(8.123,9.631)--(9.447,9.631);
\gpsetpointsize{4.00}
\gppoint{gp mark 1}{(1.504,5.921)}
\gppoint{gp mark 1}{(2.828,6.405)}
\gppoint{gp mark 1}{(3.061,6.883)}
\gppoint{gp mark 1}{(3.459,7.356)}
\gppoint{gp mark 1}{(3.625,7.820)}
\gppoint{gp mark 1}{(3.693,8.287)}
\gppoint{gp mark 1}{(3.858,8.748)}
\gppoint{gp mark 1}{(4.091,9.198)}
\gppoint{gp mark 1}{(4.152,9.631)}
\gppoint{gp mark 1}{(5.475,9.631)}
\gppoint{gp mark 1}{(6.799,9.631)}
\gppoint{gp mark 1}{(8.123,9.631)}
\gppoint{gp mark 1}{(9.447,9.631)}
\gpsetlinetype{gp lt axes}
\draw[gp path] (1.504,1.997)--(1.584,1.997)--(1.664,1.997)--(1.745,1.997)--(1.825,1.997)%
  --(1.905,1.997)--(1.985,1.997)--(2.066,1.997)--(2.146,1.997)--(2.226,1.997)--(2.306,1.997)%
  --(2.387,1.997)--(2.467,1.997)--(2.547,1.997)--(2.627,1.997)--(2.707,1.997)--(2.788,1.997)%
  --(2.868,1.997)--(2.948,1.997)--(3.028,1.997)--(3.109,1.997)--(3.189,1.997)--(3.269,1.997)%
  --(3.349,1.997)--(3.430,1.997)--(3.510,1.997)--(3.590,1.997)--(3.670,1.997)--(3.751,1.997)%
  --(3.831,1.997)--(3.911,1.997)--(3.991,1.997)--(4.071,1.997)--(4.152,1.997)--(4.232,1.997)%
  --(4.312,1.997)--(4.392,1.997)--(4.473,1.997)--(4.553,1.997)--(4.633,1.997)--(4.713,1.997)%
  --(4.794,1.997)--(4.874,1.997)--(4.954,1.997)--(5.034,1.997)--(5.114,1.997)--(5.195,1.997)%
  --(5.275,1.997)--(5.355,1.997)--(5.435,1.997)--(5.516,1.997)--(5.596,1.997)--(5.676,1.997)%
  --(5.756,1.997)--(5.837,1.997)--(5.917,1.997)--(5.997,1.997)--(6.077,1.997)--(6.157,1.997)%
  --(6.238,1.997)--(6.318,1.997)--(6.398,1.997)--(6.478,1.997)--(6.559,1.997)--(6.639,1.997)%
  --(6.719,1.997)--(6.799,1.997)--(6.880,1.997)--(6.960,1.997)--(7.040,1.997)--(7.120,1.997)%
  --(7.200,1.997)--(7.281,1.997)--(7.361,1.997)--(7.441,1.997)--(7.521,1.997)--(7.602,1.997)%
  --(7.682,1.997)--(7.762,1.997)--(7.842,1.997)--(7.923,1.997)--(8.003,1.997)--(8.083,1.997)%
  --(8.163,1.997)--(8.244,1.997)--(8.324,1.997)--(8.404,1.997)--(8.484,1.997)--(8.564,1.997)%
  --(8.645,1.997)--(8.725,1.997)--(8.805,1.997)--(8.885,1.997)--(8.966,1.997)--(9.046,1.997)%
  --(9.126,1.997)--(9.206,1.997)--(9.287,1.997)--(9.367,1.997)--(9.447,1.997);
\gpcolor{gp lt color 0}
\gpsetlinetype{gp lt plot 0}
\draw[gp path] (1.504,1.783)--(2.828,2.580)--(3.061,3.378)--(3.459,4.175)--(3.625,4.972)%
  --(3.693,5.770)--(3.858,6.567)--(4.091,7.364)--(4.152,8.161)--(5.475,8.949)--(6.799,9.631);
\gppoint{gp mark 2}{(1.504,1.783)}
\gppoint{gp mark 2}{(2.828,2.580)}
\gppoint{gp mark 2}{(3.061,3.378)}
\gppoint{gp mark 2}{(3.459,4.175)}
\gppoint{gp mark 2}{(3.625,4.972)}
\gppoint{gp mark 2}{(3.693,5.770)}
\gppoint{gp mark 2}{(3.858,6.567)}
\gppoint{gp mark 2}{(4.091,7.364)}
\gppoint{gp mark 2}{(4.152,8.161)}
\gppoint{gp mark 2}{(5.475,8.949)}
\gppoint{gp mark 2}{(6.799,9.631)}
\gpsetlinetype{gp lt axes}
\draw[gp path] (1.504,1.783)--(1.584,1.783)--(1.664,1.783)--(1.745,1.783)--(1.825,1.783)%
  --(1.905,1.783)--(1.985,1.783)--(2.066,1.783)--(2.146,1.783)--(2.226,1.783)--(2.306,1.783)%
  --(2.387,1.783)--(2.467,1.783)--(2.547,1.783)--(2.627,1.783)--(2.707,1.783)--(2.788,1.783)%
  --(2.868,1.783)--(2.948,1.783)--(3.028,1.783)--(3.109,1.783)--(3.189,1.783)--(3.269,1.783)%
  --(3.349,1.783)--(3.430,1.783)--(3.510,1.783)--(3.590,1.783)--(3.670,1.783)--(3.751,1.783)%
  --(3.831,1.783)--(3.911,1.783)--(3.991,1.783)--(4.071,1.783)--(4.152,1.783)--(4.232,1.783)%
  --(4.312,1.783)--(4.392,1.783)--(4.473,1.783)--(4.553,1.783)--(4.633,1.783)--(4.713,1.783)%
  --(4.794,1.783)--(4.874,1.783)--(4.954,1.783)--(5.034,1.783)--(5.114,1.783)--(5.195,1.783)%
  --(5.275,1.783)--(5.355,1.783)--(5.435,1.783)--(5.516,1.783)--(5.596,1.783)--(5.676,1.783)%
  --(5.756,1.783)--(5.837,1.783)--(5.917,1.783)--(5.997,1.783)--(6.077,1.783)--(6.157,1.783)%
  --(6.238,1.783)--(6.318,1.783)--(6.398,1.783)--(6.478,1.783)--(6.559,1.783)--(6.639,1.783)%
  --(6.719,1.783)--(6.799,1.783)--(6.880,1.783)--(6.960,1.783)--(7.040,1.783)--(7.120,1.783)%
  --(7.200,1.783)--(7.281,1.783)--(7.361,1.783)--(7.441,1.783)--(7.521,1.783)--(7.602,1.783)%
  --(7.682,1.783)--(7.762,1.783)--(7.842,1.783)--(7.923,1.783)--(8.003,1.783)--(8.083,1.783)%
  --(8.163,1.783)--(8.244,1.783)--(8.324,1.783)--(8.404,1.783)--(8.484,1.783)--(8.564,1.783)%
  --(8.645,1.783)--(8.725,1.783)--(8.805,1.783)--(8.885,1.783)--(8.966,1.783)--(9.046,1.783)%
  --(9.126,1.783)--(9.206,1.783)--(9.287,1.783)--(9.367,1.783)--(9.447,1.783);
\gpcolor{gp lt color 1}
\gpsetlinetype{gp lt plot 0}
\draw[gp path] (1.504,6.218)--(2.828,6.579)--(3.061,6.940)--(3.459,7.301)--(3.625,7.661)%
  --(3.693,8.022)--(3.858,8.383)--(4.091,8.743)--(4.152,9.103)--(5.475,9.449)--(6.799,9.631)%
  --(8.123,9.631)--(9.447,9.631);
\gppoint{gp mark 3}{(1.504,6.218)}
\gppoint{gp mark 3}{(2.828,6.579)}
\gppoint{gp mark 3}{(3.061,6.940)}
\gppoint{gp mark 3}{(3.459,7.301)}
\gppoint{gp mark 3}{(3.625,7.661)}
\gppoint{gp mark 3}{(3.693,8.022)}
\gppoint{gp mark 3}{(3.858,8.383)}
\gppoint{gp mark 3}{(4.091,8.743)}
\gppoint{gp mark 3}{(4.152,9.103)}
\gppoint{gp mark 3}{(5.475,9.449)}
\gppoint{gp mark 3}{(6.799,9.631)}
\gppoint{gp mark 3}{(8.123,9.631)}
\gppoint{gp mark 3}{(9.447,9.631)}
\gpsetlinetype{gp lt axes}
\draw[gp path] (1.504,1.822)--(1.584,1.822)--(1.664,1.822)--(1.745,1.822)--(1.825,1.822)%
  --(1.905,1.822)--(1.985,1.822)--(2.066,1.822)--(2.146,1.822)--(2.226,1.822)--(2.306,1.822)%
  --(2.387,1.822)--(2.467,1.822)--(2.547,1.822)--(2.627,1.822)--(2.707,1.822)--(2.788,1.822)%
  --(2.868,1.822)--(2.948,1.822)--(3.028,1.822)--(3.109,1.822)--(3.189,1.822)--(3.269,1.822)%
  --(3.349,1.822)--(3.430,1.822)--(3.510,1.822)--(3.590,1.822)--(3.670,1.822)--(3.751,1.822)%
  --(3.831,1.822)--(3.911,1.822)--(3.991,1.822)--(4.071,1.822)--(4.152,1.822)--(4.232,1.822)%
  --(4.312,1.822)--(4.392,1.822)--(4.473,1.822)--(4.553,1.822)--(4.633,1.822)--(4.713,1.822)%
  --(4.794,1.822)--(4.874,1.822)--(4.954,1.822)--(5.034,1.822)--(5.114,1.822)--(5.195,1.822)%
  --(5.275,1.822)--(5.355,1.822)--(5.435,1.822)--(5.516,1.822)--(5.596,1.822)--(5.676,1.822)%
  --(5.756,1.822)--(5.837,1.822)--(5.917,1.822)--(5.997,1.822)--(6.077,1.822)--(6.157,1.822)%
  --(6.238,1.822)--(6.318,1.822)--(6.398,1.822)--(6.478,1.822)--(6.559,1.822)--(6.639,1.822)%
  --(6.719,1.822)--(6.799,1.822)--(6.880,1.822)--(6.960,1.822)--(7.040,1.822)--(7.120,1.822)%
  --(7.200,1.822)--(7.281,1.822)--(7.361,1.822)--(7.441,1.822)--(7.521,1.822)--(7.602,1.822)%
  --(7.682,1.822)--(7.762,1.822)--(7.842,1.822)--(7.923,1.822)--(8.003,1.822)--(8.083,1.822)%
  --(8.163,1.822)--(8.244,1.822)--(8.324,1.822)--(8.404,1.822)--(8.484,1.822)--(8.564,1.822)%
  --(8.645,1.822)--(8.725,1.822)--(8.805,1.822)--(8.885,1.822)--(8.966,1.822)--(9.046,1.822)%
  --(9.126,1.822)--(9.206,1.822)--(9.287,1.822)--(9.367,1.822)--(9.447,1.822);
\gpcolor{gp lt color 2}
\gpsetlinetype{gp lt plot 0}
\draw[gp path] (1.504,3.711)--(2.828,4.340)--(3.061,4.968)--(3.459,5.596)--(3.625,6.224)%
  --(3.693,6.851)--(3.858,7.479)--(4.091,8.106)--(4.152,8.733)--(5.475,9.332)--(6.799,9.631)%
  --(8.123,9.631)--(9.447,9.631);
\gppoint{gp mark 4}{(1.504,3.711)}
\gppoint{gp mark 4}{(2.828,4.340)}
\gppoint{gp mark 4}{(3.061,4.968)}
\gppoint{gp mark 4}{(3.459,5.596)}
\gppoint{gp mark 4}{(3.625,6.224)}
\gppoint{gp mark 4}{(3.693,6.851)}
\gppoint{gp mark 4}{(3.858,7.479)}
\gppoint{gp mark 4}{(4.091,8.106)}
\gppoint{gp mark 4}{(4.152,8.733)}
\gppoint{gp mark 4}{(5.475,9.332)}
\gppoint{gp mark 4}{(6.799,9.631)}
\gppoint{gp mark 4}{(8.123,9.631)}
\gppoint{gp mark 4}{(9.447,9.631)}
\gpsetlinetype{gp lt axes}
\draw[gp path] (1.504,1.827)--(1.584,1.827)--(1.664,1.827)--(1.745,1.827)--(1.825,1.827)%
  --(1.905,1.827)--(1.985,1.827)--(2.066,1.827)--(2.146,1.827)--(2.226,1.827)--(2.306,1.827)%
  --(2.387,1.827)--(2.467,1.827)--(2.547,1.827)--(2.627,1.827)--(2.707,1.827)--(2.788,1.827)%
  --(2.868,1.827)--(2.948,1.827)--(3.028,1.827)--(3.109,1.827)--(3.189,1.827)--(3.269,1.827)%
  --(3.349,1.827)--(3.430,1.827)--(3.510,1.827)--(3.590,1.827)--(3.670,1.827)--(3.751,1.827)%
  --(3.831,1.827)--(3.911,1.827)--(3.991,1.827)--(4.071,1.827)--(4.152,1.827)--(4.232,1.827)%
  --(4.312,1.827)--(4.392,1.827)--(4.473,1.827)--(4.553,1.827)--(4.633,1.827)--(4.713,1.827)%
  --(4.794,1.827)--(4.874,1.827)--(4.954,1.827)--(5.034,1.827)--(5.114,1.827)--(5.195,1.827)%
  --(5.275,1.827)--(5.355,1.827)--(5.435,1.827)--(5.516,1.827)--(5.596,1.827)--(5.676,1.827)%
  --(5.756,1.827)--(5.837,1.827)--(5.917,1.827)--(5.997,1.827)--(6.077,1.827)--(6.157,1.827)%
  --(6.238,1.827)--(6.318,1.827)--(6.398,1.827)--(6.478,1.827)--(6.559,1.827)--(6.639,1.827)%
  --(6.719,1.827)--(6.799,1.827)--(6.880,1.827)--(6.960,1.827)--(7.040,1.827)--(7.120,1.827)%
  --(7.200,1.827)--(7.281,1.827)--(7.361,1.827)--(7.441,1.827)--(7.521,1.827)--(7.602,1.827)%
  --(7.682,1.827)--(7.762,1.827)--(7.842,1.827)--(7.923,1.827)--(8.003,1.827)--(8.083,1.827)%
  --(8.163,1.827)--(8.244,1.827)--(8.324,1.827)--(8.404,1.827)--(8.484,1.827)--(8.564,1.827)%
  --(8.645,1.827)--(8.725,1.827)--(8.805,1.827)--(8.885,1.827)--(8.966,1.827)--(9.046,1.827)%
  --(9.126,1.827)--(9.206,1.827)--(9.287,1.827)--(9.367,1.827)--(9.447,1.827);
\gpcolor{gp lt color 3}
\gpsetlinetype{gp lt plot 0}
\draw[gp path] (1.504,9.376)--(2.828,9.407)--(3.061,9.439)--(3.459,9.470)--(3.625,9.502)%
  --(3.693,9.533)--(3.858,9.564)--(4.091,9.594)--(4.152,9.625)--(5.475,9.631)--(6.799,9.631)%
  --(8.123,9.631)--(9.447,9.631);
\gppoint{gp mark 5}{(1.504,9.376)}
\gppoint{gp mark 5}{(2.828,9.407)}
\gppoint{gp mark 5}{(3.061,9.439)}
\gppoint{gp mark 5}{(3.459,9.470)}
\gppoint{gp mark 5}{(3.625,9.502)}
\gppoint{gp mark 5}{(3.693,9.533)}
\gppoint{gp mark 5}{(3.858,9.564)}
\gppoint{gp mark 5}{(4.091,9.594)}
\gppoint{gp mark 5}{(4.152,9.625)}
\gppoint{gp mark 5}{(5.475,9.631)}
\gppoint{gp mark 5}{(6.799,9.631)}
\gppoint{gp mark 5}{(8.123,9.631)}
\gppoint{gp mark 5}{(9.447,9.631)}
\gpsetlinetype{gp lt axes}
\draw[gp path] (1.504,1.831)--(1.584,1.831)--(1.664,1.831)--(1.745,1.831)--(1.825,1.831)%
  --(1.905,1.831)--(1.985,1.831)--(2.066,1.831)--(2.146,1.831)--(2.226,1.831)--(2.306,1.831)%
  --(2.387,1.831)--(2.467,1.831)--(2.547,1.831)--(2.627,1.831)--(2.707,1.831)--(2.788,1.831)%
  --(2.868,1.831)--(2.948,1.831)--(3.028,1.831)--(3.109,1.831)--(3.189,1.831)--(3.269,1.831)%
  --(3.349,1.831)--(3.430,1.831)--(3.510,1.831)--(3.590,1.831)--(3.670,1.831)--(3.751,1.831)%
  --(3.831,1.831)--(3.911,1.831)--(3.991,1.831)--(4.071,1.831)--(4.152,1.831)--(4.232,1.831)%
  --(4.312,1.831)--(4.392,1.831)--(4.473,1.831)--(4.553,1.831)--(4.633,1.831)--(4.713,1.831)%
  --(4.794,1.831)--(4.874,1.831)--(4.954,1.831)--(5.034,1.831)--(5.114,1.831)--(5.195,1.831)%
  --(5.275,1.831)--(5.355,1.831)--(5.435,1.831)--(5.516,1.831)--(5.596,1.831)--(5.676,1.831)%
  --(5.756,1.831)--(5.837,1.831)--(5.917,1.831)--(5.997,1.831)--(6.077,1.831)--(6.157,1.831)%
  --(6.238,1.831)--(6.318,1.831)--(6.398,1.831)--(6.478,1.831)--(6.559,1.831)--(6.639,1.831)%
  --(6.719,1.831)--(6.799,1.831)--(6.880,1.831)--(6.960,1.831)--(7.040,1.831)--(7.120,1.831)%
  --(7.200,1.831)--(7.281,1.831)--(7.361,1.831)--(7.441,1.831)--(7.521,1.831)--(7.602,1.831)%
  --(7.682,1.831)--(7.762,1.831)--(7.842,1.831)--(7.923,1.831)--(8.003,1.831)--(8.083,1.831)%
  --(8.163,1.831)--(8.244,1.831)--(8.324,1.831)--(8.404,1.831)--(8.484,1.831)--(8.564,1.831)%
  --(8.645,1.831)--(8.725,1.831)--(8.805,1.831)--(8.885,1.831)--(8.966,1.831)--(9.046,1.831)%
  --(9.126,1.831)--(9.206,1.831)--(9.287,1.831)--(9.367,1.831)--(9.447,1.831);
\gpcolor{gp lt color 4}
\gpsetlinetype{gp lt plot 0}
\draw[gp path] (1.504,1.864)--(2.828,2.742)--(3.061,3.619)--(3.459,4.496)--(3.625,5.371)%
  --(3.693,6.246)--(3.858,7.121)--(4.091,7.995)--(4.152,8.867)--(5.475,9.631)--(6.799,9.631)%
  --(8.123,9.631)--(9.447,9.631);
\gppoint{gp mark 6}{(1.504,1.864)}
\gppoint{gp mark 6}{(2.828,2.742)}
\gppoint{gp mark 6}{(3.061,3.619)}
\gppoint{gp mark 6}{(3.459,4.496)}
\gppoint{gp mark 6}{(3.625,5.371)}
\gppoint{gp mark 6}{(3.693,6.246)}
\gppoint{gp mark 6}{(3.858,7.121)}
\gppoint{gp mark 6}{(4.091,7.995)}
\gppoint{gp mark 6}{(4.152,8.867)}
\gppoint{gp mark 6}{(5.475,9.631)}
\gppoint{gp mark 6}{(6.799,9.631)}
\gppoint{gp mark 6}{(8.123,9.631)}
\gppoint{gp mark 6}{(9.447,9.631)}
\gpsetlinetype{gp lt axes}
\draw[gp path] (1.504,1.864)--(1.584,1.864)--(1.664,1.864)--(1.745,1.864)--(1.825,1.864)%
  --(1.905,1.864)--(1.985,1.864)--(2.066,1.864)--(2.146,1.864)--(2.226,1.864)--(2.306,1.864)%
  --(2.387,1.864)--(2.467,1.864)--(2.547,1.864)--(2.627,1.864)--(2.707,1.864)--(2.788,1.864)%
  --(2.868,1.864)--(2.948,1.864)--(3.028,1.864)--(3.109,1.864)--(3.189,1.864)--(3.269,1.864)%
  --(3.349,1.864)--(3.430,1.864)--(3.510,1.864)--(3.590,1.864)--(3.670,1.864)--(3.751,1.864)%
  --(3.831,1.864)--(3.911,1.864)--(3.991,1.864)--(4.071,1.864)--(4.152,1.864)--(4.232,1.864)%
  --(4.312,1.864)--(4.392,1.864)--(4.473,1.864)--(4.553,1.864)--(4.633,1.864)--(4.713,1.864)%
  --(4.794,1.864)--(4.874,1.864)--(4.954,1.864)--(5.034,1.864)--(5.114,1.864)--(5.195,1.864)%
  --(5.275,1.864)--(5.355,1.864)--(5.435,1.864)--(5.516,1.864)--(5.596,1.864)--(5.676,1.864)%
  --(5.756,1.864)--(5.837,1.864)--(5.917,1.864)--(5.997,1.864)--(6.077,1.864)--(6.157,1.864)%
  --(6.238,1.864)--(6.318,1.864)--(6.398,1.864)--(6.478,1.864)--(6.559,1.864)--(6.639,1.864)%
  --(6.719,1.864)--(6.799,1.864)--(6.880,1.864)--(6.960,1.864)--(7.040,1.864)--(7.120,1.864)%
  --(7.200,1.864)--(7.281,1.864)--(7.361,1.864)--(7.441,1.864)--(7.521,1.864)--(7.602,1.864)%
  --(7.682,1.864)--(7.762,1.864)--(7.842,1.864)--(7.923,1.864)--(8.003,1.864)--(8.083,1.864)%
  --(8.163,1.864)--(8.244,1.864)--(8.324,1.864)--(8.404,1.864)--(8.484,1.864)--(8.564,1.864)%
  --(8.645,1.864)--(8.725,1.864)--(8.805,1.864)--(8.885,1.864)--(8.966,1.864)--(9.046,1.864)%
  --(9.126,1.864)--(9.206,1.864)--(9.287,1.864)--(9.367,1.864)--(9.447,1.864);
\gpcolor{gp lt color 5}
\gpsetlinetype{gp lt plot 0}
\draw[gp path] (1.504,1.850)--(2.828,2.715)--(3.061,3.579)--(3.459,4.443)--(3.625,5.307)%
  --(3.693,6.169)--(3.858,7.032)--(4.091,7.893)--(4.152,8.754)--(5.475,9.551)--(6.799,9.631)%
  --(8.123,9.631)--(9.447,9.631);
\gppoint{gp mark 7}{(1.504,1.850)}
\gppoint{gp mark 7}{(2.828,2.715)}
\gppoint{gp mark 7}{(3.061,3.579)}
\gppoint{gp mark 7}{(3.459,4.443)}
\gppoint{gp mark 7}{(3.625,5.307)}
\gppoint{gp mark 7}{(3.693,6.169)}
\gppoint{gp mark 7}{(3.858,7.032)}
\gppoint{gp mark 7}{(4.091,7.893)}
\gppoint{gp mark 7}{(4.152,8.754)}
\gppoint{gp mark 7}{(5.475,9.551)}
\gppoint{gp mark 7}{(6.799,9.631)}
\gppoint{gp mark 7}{(8.123,9.631)}
\gppoint{gp mark 7}{(9.447,9.631)}
\gpsetlinetype{gp lt axes}
\draw[gp path] (1.504,1.850)--(1.584,1.850)--(1.664,1.850)--(1.745,1.850)--(1.825,1.850)%
  --(1.905,1.850)--(1.985,1.850)--(2.066,1.850)--(2.146,1.850)--(2.226,1.850)--(2.306,1.850)%
  --(2.387,1.850)--(2.467,1.850)--(2.547,1.850)--(2.627,1.850)--(2.707,1.850)--(2.788,1.850)%
  --(2.868,1.850)--(2.948,1.850)--(3.028,1.850)--(3.109,1.850)--(3.189,1.850)--(3.269,1.850)%
  --(3.349,1.850)--(3.430,1.850)--(3.510,1.850)--(3.590,1.850)--(3.670,1.850)--(3.751,1.850)%
  --(3.831,1.850)--(3.911,1.850)--(3.991,1.850)--(4.071,1.850)--(4.152,1.850)--(4.232,1.850)%
  --(4.312,1.850)--(4.392,1.850)--(4.473,1.850)--(4.553,1.850)--(4.633,1.850)--(4.713,1.850)%
  --(4.794,1.850)--(4.874,1.850)--(4.954,1.850)--(5.034,1.850)--(5.114,1.850)--(5.195,1.850)%
  --(5.275,1.850)--(5.355,1.850)--(5.435,1.850)--(5.516,1.850)--(5.596,1.850)--(5.676,1.850)%
  --(5.756,1.850)--(5.837,1.850)--(5.917,1.850)--(5.997,1.850)--(6.077,1.850)--(6.157,1.850)%
  --(6.238,1.850)--(6.318,1.850)--(6.398,1.850)--(6.478,1.850)--(6.559,1.850)--(6.639,1.850)%
  --(6.719,1.850)--(6.799,1.850)--(6.880,1.850)--(6.960,1.850)--(7.040,1.850)--(7.120,1.850)%
  --(7.200,1.850)--(7.281,1.850)--(7.361,1.850)--(7.441,1.850)--(7.521,1.850)--(7.602,1.850)%
  --(7.682,1.850)--(7.762,1.850)--(7.842,1.850)--(7.923,1.850)--(8.003,1.850)--(8.083,1.850)%
  --(8.163,1.850)--(8.244,1.850)--(8.324,1.850)--(8.404,1.850)--(8.484,1.850)--(8.564,1.850)%
  --(8.645,1.850)--(8.725,1.850)--(8.805,1.850)--(8.885,1.850)--(8.966,1.850)--(9.046,1.850)%
  --(9.126,1.850)--(9.206,1.850)--(9.287,1.850)--(9.367,1.850)--(9.447,1.850);
\gpcolor{gp lt color 6}
\gpsetlinetype{gp lt plot 0}
\draw[gp path] (1.504,5.065)--(2.828,5.578)--(3.061,6.091)--(3.459,6.604)--(3.625,7.116)%
  --(3.693,7.628)--(3.858,8.140)--(4.091,8.652)--(4.152,9.163)--(5.475,9.631)--(6.799,9.631)%
  --(8.123,9.631)--(9.447,9.631);
\gppoint{gp mark 8}{(1.504,5.065)}
\gppoint{gp mark 8}{(2.828,5.578)}
\gppoint{gp mark 8}{(3.061,6.091)}
\gppoint{gp mark 8}{(3.459,6.604)}
\gppoint{gp mark 8}{(3.625,7.116)}
\gppoint{gp mark 8}{(3.693,7.628)}
\gppoint{gp mark 8}{(3.858,8.140)}
\gppoint{gp mark 8}{(4.091,8.652)}
\gppoint{gp mark 8}{(4.152,9.163)}
\gppoint{gp mark 8}{(5.475,9.631)}
\gppoint{gp mark 8}{(6.799,9.631)}
\gppoint{gp mark 8}{(8.123,9.631)}
\gppoint{gp mark 8}{(9.447,9.631)}
\gpsetlinetype{gp lt axes}
\draw[gp path] (1.504,1.886)--(1.584,1.886)--(1.664,1.886)--(1.745,1.886)--(1.825,1.886)%
  --(1.905,1.886)--(1.985,1.886)--(2.066,1.886)--(2.146,1.886)--(2.226,1.886)--(2.306,1.886)%
  --(2.387,1.886)--(2.467,1.886)--(2.547,1.886)--(2.627,1.886)--(2.707,1.886)--(2.788,1.886)%
  --(2.868,1.886)--(2.948,1.886)--(3.028,1.886)--(3.109,1.886)--(3.189,1.886)--(3.269,1.886)%
  --(3.349,1.886)--(3.430,1.886)--(3.510,1.886)--(3.590,1.886)--(3.670,1.886)--(3.751,1.886)%
  --(3.831,1.886)--(3.911,1.886)--(3.991,1.886)--(4.071,1.886)--(4.152,1.886)--(4.232,1.886)%
  --(4.312,1.886)--(4.392,1.886)--(4.473,1.886)--(4.553,1.886)--(4.633,1.886)--(4.713,1.886)%
  --(4.794,1.886)--(4.874,1.886)--(4.954,1.886)--(5.034,1.886)--(5.114,1.886)--(5.195,1.886)%
  --(5.275,1.886)--(5.355,1.886)--(5.435,1.886)--(5.516,1.886)--(5.596,1.886)--(5.676,1.886)%
  --(5.756,1.886)--(5.837,1.886)--(5.917,1.886)--(5.997,1.886)--(6.077,1.886)--(6.157,1.886)%
  --(6.238,1.886)--(6.318,1.886)--(6.398,1.886)--(6.478,1.886)--(6.559,1.886)--(6.639,1.886)%
  --(6.719,1.886)--(6.799,1.886)--(6.880,1.886)--(6.960,1.886)--(7.040,1.886)--(7.120,1.886)%
  --(7.200,1.886)--(7.281,1.886)--(7.361,1.886)--(7.441,1.886)--(7.521,1.886)--(7.602,1.886)%
  --(7.682,1.886)--(7.762,1.886)--(7.842,1.886)--(7.923,1.886)--(8.003,1.886)--(8.083,1.886)%
  --(8.163,1.886)--(8.244,1.886)--(8.324,1.886)--(8.404,1.886)--(8.484,1.886)--(8.564,1.886)%
  --(8.645,1.886)--(8.725,1.886)--(8.805,1.886)--(8.885,1.886)--(8.966,1.886)--(9.046,1.886)%
  --(9.126,1.886)--(9.206,1.886)--(9.287,1.886)--(9.367,1.886)--(9.447,1.886);
\gpcolor{gp lt color 7}
\gpsetlinetype{gp lt plot 0}
\draw[gp path] (1.504,6.324)--(2.828,6.746)--(3.061,7.166)--(3.459,7.583)--(3.625,7.996)%
  --(3.693,8.409)--(3.858,8.822)--(4.091,9.229)--(4.152,9.631)--(5.475,9.631)--(6.799,9.631)%
  --(8.123,9.631)--(9.447,9.631);
\gppoint{gp mark 9}{(1.504,6.324)}
\gppoint{gp mark 9}{(2.828,6.746)}
\gppoint{gp mark 9}{(3.061,7.166)}
\gppoint{gp mark 9}{(3.459,7.583)}
\gppoint{gp mark 9}{(3.625,7.996)}
\gppoint{gp mark 9}{(3.693,8.409)}
\gppoint{gp mark 9}{(3.858,8.822)}
\gppoint{gp mark 9}{(4.091,9.229)}
\gppoint{gp mark 9}{(4.152,9.631)}
\gppoint{gp mark 9}{(5.475,9.631)}
\gppoint{gp mark 9}{(6.799,9.631)}
\gppoint{gp mark 9}{(8.123,9.631)}
\gppoint{gp mark 9}{(9.447,9.631)}
\gpsetlinetype{gp lt axes}
\draw[gp path] (1.504,1.990)--(1.584,1.990)--(1.664,1.990)--(1.745,1.990)--(1.825,1.990)%
  --(1.905,1.990)--(1.985,1.990)--(2.066,1.990)--(2.146,1.990)--(2.226,1.990)--(2.306,1.990)%
  --(2.387,1.990)--(2.467,1.990)--(2.547,1.990)--(2.627,1.990)--(2.707,1.990)--(2.788,1.990)%
  --(2.868,1.990)--(2.948,1.990)--(3.028,1.990)--(3.109,1.990)--(3.189,1.990)--(3.269,1.990)%
  --(3.349,1.990)--(3.430,1.990)--(3.510,1.990)--(3.590,1.990)--(3.670,1.990)--(3.751,1.990)%
  --(3.831,1.990)--(3.911,1.990)--(3.991,1.990)--(4.071,1.990)--(4.152,1.990)--(4.232,1.990)%
  --(4.312,1.990)--(4.392,1.990)--(4.473,1.990)--(4.553,1.990)--(4.633,1.990)--(4.713,1.990)%
  --(4.794,1.990)--(4.874,1.990)--(4.954,1.990)--(5.034,1.990)--(5.114,1.990)--(5.195,1.990)%
  --(5.275,1.990)--(5.355,1.990)--(5.435,1.990)--(5.516,1.990)--(5.596,1.990)--(5.676,1.990)%
  --(5.756,1.990)--(5.837,1.990)--(5.917,1.990)--(5.997,1.990)--(6.077,1.990)--(6.157,1.990)%
  --(6.238,1.990)--(6.318,1.990)--(6.398,1.990)--(6.478,1.990)--(6.559,1.990)--(6.639,1.990)%
  --(6.719,1.990)--(6.799,1.990)--(6.880,1.990)--(6.960,1.990)--(7.040,1.990)--(7.120,1.990)%
  --(7.200,1.990)--(7.281,1.990)--(7.361,1.990)--(7.441,1.990)--(7.521,1.990)--(7.602,1.990)%
  --(7.682,1.990)--(7.762,1.990)--(7.842,1.990)--(7.923,1.990)--(8.003,1.990)--(8.083,1.990)%
  --(8.163,1.990)--(8.244,1.990)--(8.324,1.990)--(8.404,1.990)--(8.484,1.990)--(8.564,1.990)%
  --(8.645,1.990)--(8.725,1.990)--(8.805,1.990)--(8.885,1.990)--(8.966,1.990)--(9.046,1.990)%
  --(9.126,1.990)--(9.206,1.990)--(9.287,1.990)--(9.367,1.990)--(9.447,1.990);
\gpcolor{gp lt color 0}
\gpsetlinetype{gp lt plot 0}
\draw[gp path] (1.504,5.285)--(2.828,5.745)--(3.061,6.205)--(3.459,6.664)--(3.625,7.123)%
  --(3.693,7.583)--(3.858,8.042)--(4.091,8.501)--(4.152,8.960)--(5.475,9.398)--(6.799,9.631)%
  --(8.123,9.631)--(9.447,9.631);
\gppoint{gp mark 10}{(1.504,5.285)}
\gppoint{gp mark 10}{(2.828,5.745)}
\gppoint{gp mark 10}{(3.061,6.205)}
\gppoint{gp mark 10}{(3.459,6.664)}
\gppoint{gp mark 10}{(3.625,7.123)}
\gppoint{gp mark 10}{(3.693,7.583)}
\gppoint{gp mark 10}{(3.858,8.042)}
\gppoint{gp mark 10}{(4.091,8.501)}
\gppoint{gp mark 10}{(4.152,8.960)}
\gppoint{gp mark 10}{(5.475,9.398)}
\gppoint{gp mark 10}{(6.799,9.631)}
\gppoint{gp mark 10}{(8.123,9.631)}
\gppoint{gp mark 10}{(9.447,9.631)}
\gpsetlinetype{gp lt axes}
\draw[gp path] (1.504,1.845)--(1.584,1.845)--(1.664,1.845)--(1.745,1.845)--(1.825,1.845)%
  --(1.905,1.845)--(1.985,1.845)--(2.066,1.845)--(2.146,1.845)--(2.226,1.845)--(2.306,1.845)%
  --(2.387,1.845)--(2.467,1.845)--(2.547,1.845)--(2.627,1.845)--(2.707,1.845)--(2.788,1.845)%
  --(2.868,1.845)--(2.948,1.845)--(3.028,1.845)--(3.109,1.845)--(3.189,1.845)--(3.269,1.845)%
  --(3.349,1.845)--(3.430,1.845)--(3.510,1.845)--(3.590,1.845)--(3.670,1.845)--(3.751,1.845)%
  --(3.831,1.845)--(3.911,1.845)--(3.991,1.845)--(4.071,1.845)--(4.152,1.845)--(4.232,1.845)%
  --(4.312,1.845)--(4.392,1.845)--(4.473,1.845)--(4.553,1.845)--(4.633,1.845)--(4.713,1.845)%
  --(4.794,1.845)--(4.874,1.845)--(4.954,1.845)--(5.034,1.845)--(5.114,1.845)--(5.195,1.845)%
  --(5.275,1.845)--(5.355,1.845)--(5.435,1.845)--(5.516,1.845)--(5.596,1.845)--(5.676,1.845)%
  --(5.756,1.845)--(5.837,1.845)--(5.917,1.845)--(5.997,1.845)--(6.077,1.845)--(6.157,1.845)%
  --(6.238,1.845)--(6.318,1.845)--(6.398,1.845)--(6.478,1.845)--(6.559,1.845)--(6.639,1.845)%
  --(6.719,1.845)--(6.799,1.845)--(6.880,1.845)--(6.960,1.845)--(7.040,1.845)--(7.120,1.845)%
  --(7.200,1.845)--(7.281,1.845)--(7.361,1.845)--(7.441,1.845)--(7.521,1.845)--(7.602,1.845)%
  --(7.682,1.845)--(7.762,1.845)--(7.842,1.845)--(7.923,1.845)--(8.003,1.845)--(8.083,1.845)%
  --(8.163,1.845)--(8.244,1.845)--(8.324,1.845)--(8.404,1.845)--(8.484,1.845)--(8.564,1.845)%
  --(8.645,1.845)--(8.725,1.845)--(8.805,1.845)--(8.885,1.845)--(8.966,1.845)--(9.046,1.845)%
  --(9.126,1.845)--(9.206,1.845)--(9.287,1.845)--(9.367,1.845)--(9.447,1.845);
\gpcolor{gp lt color 1}
\gpsetlinetype{gp lt plot 0}
\draw[gp path] (1.504,1.798)--(2.828,2.611)--(3.061,3.423)--(3.459,4.236)--(3.625,5.048)%
  --(3.693,5.860)--(3.858,6.672)--(4.091,7.484)--(4.152,8.296)--(5.475,9.086)--(6.799,9.631)%
  --(8.123,9.631)--(9.447,9.631);
\gppoint{gp mark 11}{(1.504,1.798)}
\gppoint{gp mark 11}{(2.828,2.611)}
\gppoint{gp mark 11}{(3.061,3.423)}
\gppoint{gp mark 11}{(3.459,4.236)}
\gppoint{gp mark 11}{(3.625,5.048)}
\gppoint{gp mark 11}{(3.693,5.860)}
\gppoint{gp mark 11}{(3.858,6.672)}
\gppoint{gp mark 11}{(4.091,7.484)}
\gppoint{gp mark 11}{(4.152,8.296)}
\gppoint{gp mark 11}{(5.475,9.086)}
\gppoint{gp mark 11}{(6.799,9.631)}
\gppoint{gp mark 11}{(8.123,9.631)}
\gppoint{gp mark 11}{(9.447,9.631)}
\gpsetlinetype{gp lt axes}
\draw[gp path] (1.504,1.798)--(1.584,1.798)--(1.664,1.798)--(1.745,1.798)--(1.825,1.798)%
  --(1.905,1.798)--(1.985,1.798)--(2.066,1.798)--(2.146,1.798)--(2.226,1.798)--(2.306,1.798)%
  --(2.387,1.798)--(2.467,1.798)--(2.547,1.798)--(2.627,1.798)--(2.707,1.798)--(2.788,1.798)%
  --(2.868,1.798)--(2.948,1.798)--(3.028,1.798)--(3.109,1.798)--(3.189,1.798)--(3.269,1.798)%
  --(3.349,1.798)--(3.430,1.798)--(3.510,1.798)--(3.590,1.798)--(3.670,1.798)--(3.751,1.798)%
  --(3.831,1.798)--(3.911,1.798)--(3.991,1.798)--(4.071,1.798)--(4.152,1.798)--(4.232,1.798)%
  --(4.312,1.798)--(4.392,1.798)--(4.473,1.798)--(4.553,1.798)--(4.633,1.798)--(4.713,1.798)%
  --(4.794,1.798)--(4.874,1.798)--(4.954,1.798)--(5.034,1.798)--(5.114,1.798)--(5.195,1.798)%
  --(5.275,1.798)--(5.355,1.798)--(5.435,1.798)--(5.516,1.798)--(5.596,1.798)--(5.676,1.798)%
  --(5.756,1.798)--(5.837,1.798)--(5.917,1.798)--(5.997,1.798)--(6.077,1.798)--(6.157,1.798)%
  --(6.238,1.798)--(6.318,1.798)--(6.398,1.798)--(6.478,1.798)--(6.559,1.798)--(6.639,1.798)%
  --(6.719,1.798)--(6.799,1.798)--(6.880,1.798)--(6.960,1.798)--(7.040,1.798)--(7.120,1.798)%
  --(7.200,1.798)--(7.281,1.798)--(7.361,1.798)--(7.441,1.798)--(7.521,1.798)--(7.602,1.798)%
  --(7.682,1.798)--(7.762,1.798)--(7.842,1.798)--(7.923,1.798)--(8.003,1.798)--(8.083,1.798)%
  --(8.163,1.798)--(8.244,1.798)--(8.324,1.798)--(8.404,1.798)--(8.484,1.798)--(8.564,1.798)%
  --(8.645,1.798)--(8.725,1.798)--(8.805,1.798)--(8.885,1.798)--(8.966,1.798)--(9.046,1.798)%
  --(9.126,1.798)--(9.206,1.798)--(9.287,1.798)--(9.367,1.798)--(9.447,1.798);
\gpcolor{gp lt color 2}
\gpsetlinetype{gp lt plot 0}
\draw[gp path] (1.504,1.791)--(2.828,2.597)--(3.061,3.403)--(3.459,4.208)--(3.625,5.014)%
  --(3.693,5.819)--(3.858,6.625)--(4.091,7.430)--(4.152,8.235)--(5.475,9.023)--(6.799,9.631);
\gppoint{gp mark 12}{(1.504,1.791)}
\gppoint{gp mark 12}{(2.828,2.597)}
\gppoint{gp mark 12}{(3.061,3.403)}
\gppoint{gp mark 12}{(3.459,4.208)}
\gppoint{gp mark 12}{(3.625,5.014)}
\gppoint{gp mark 12}{(3.693,5.819)}
\gppoint{gp mark 12}{(3.858,6.625)}
\gppoint{gp mark 12}{(4.091,7.430)}
\gppoint{gp mark 12}{(4.152,8.235)}
\gppoint{gp mark 12}{(5.475,9.023)}
\gppoint{gp mark 12}{(6.799,9.631)}
\gpsetlinetype{gp lt axes}
\draw[gp path] (1.504,1.791)--(1.584,1.791)--(1.664,1.791)--(1.745,1.791)--(1.825,1.791)%
  --(1.905,1.791)--(1.985,1.791)--(2.066,1.791)--(2.146,1.791)--(2.226,1.791)--(2.306,1.791)%
  --(2.387,1.791)--(2.467,1.791)--(2.547,1.791)--(2.627,1.791)--(2.707,1.791)--(2.788,1.791)%
  --(2.868,1.791)--(2.948,1.791)--(3.028,1.791)--(3.109,1.791)--(3.189,1.791)--(3.269,1.791)%
  --(3.349,1.791)--(3.430,1.791)--(3.510,1.791)--(3.590,1.791)--(3.670,1.791)--(3.751,1.791)%
  --(3.831,1.791)--(3.911,1.791)--(3.991,1.791)--(4.071,1.791)--(4.152,1.791)--(4.232,1.791)%
  --(4.312,1.791)--(4.392,1.791)--(4.473,1.791)--(4.553,1.791)--(4.633,1.791)--(4.713,1.791)%
  --(4.794,1.791)--(4.874,1.791)--(4.954,1.791)--(5.034,1.791)--(5.114,1.791)--(5.195,1.791)%
  --(5.275,1.791)--(5.355,1.791)--(5.435,1.791)--(5.516,1.791)--(5.596,1.791)--(5.676,1.791)%
  --(5.756,1.791)--(5.837,1.791)--(5.917,1.791)--(5.997,1.791)--(6.077,1.791)--(6.157,1.791)%
  --(6.238,1.791)--(6.318,1.791)--(6.398,1.791)--(6.478,1.791)--(6.559,1.791)--(6.639,1.791)%
  --(6.719,1.791)--(6.799,1.791)--(6.880,1.791)--(6.960,1.791)--(7.040,1.791)--(7.120,1.791)%
  --(7.200,1.791)--(7.281,1.791)--(7.361,1.791)--(7.441,1.791)--(7.521,1.791)--(7.602,1.791)%
  --(7.682,1.791)--(7.762,1.791)--(7.842,1.791)--(7.923,1.791)--(8.003,1.791)--(8.083,1.791)%
  --(8.163,1.791)--(8.244,1.791)--(8.324,1.791)--(8.404,1.791)--(8.484,1.791)--(8.564,1.791)%
  --(8.645,1.791)--(8.725,1.791)--(8.805,1.791)--(8.885,1.791)--(8.966,1.791)--(9.046,1.791)%
  --(9.126,1.791)--(9.206,1.791)--(9.287,1.791)--(9.367,1.791)--(9.447,1.791);
\gpcolor{gp lt color 3}
\gpsetlinetype{gp lt plot 0}
\draw[gp path] (1.504,1.787)--(2.828,2.589)--(3.061,3.391)--(3.459,4.193)--(3.625,4.994)%
  --(3.693,5.796)--(3.858,6.598)--(4.091,7.399)--(4.152,8.200)--(5.475,8.988)--(6.799,9.631);
\gppoint{gp mark 13}{(1.504,1.787)}
\gppoint{gp mark 13}{(2.828,2.589)}
\gppoint{gp mark 13}{(3.061,3.391)}
\gppoint{gp mark 13}{(3.459,4.193)}
\gppoint{gp mark 13}{(3.625,4.994)}
\gppoint{gp mark 13}{(3.693,5.796)}
\gppoint{gp mark 13}{(3.858,6.598)}
\gppoint{gp mark 13}{(4.091,7.399)}
\gppoint{gp mark 13}{(4.152,8.200)}
\gppoint{gp mark 13}{(5.475,8.988)}
\gppoint{gp mark 13}{(6.799,9.631)}
\gpsetlinetype{gp lt axes}
\draw[gp path] (1.504,1.787)--(1.584,1.787)--(1.664,1.787)--(1.745,1.787)--(1.825,1.787)%
  --(1.905,1.787)--(1.985,1.787)--(2.066,1.787)--(2.146,1.787)--(2.226,1.787)--(2.306,1.787)%
  --(2.387,1.787)--(2.467,1.787)--(2.547,1.787)--(2.627,1.787)--(2.707,1.787)--(2.788,1.787)%
  --(2.868,1.787)--(2.948,1.787)--(3.028,1.787)--(3.109,1.787)--(3.189,1.787)--(3.269,1.787)%
  --(3.349,1.787)--(3.430,1.787)--(3.510,1.787)--(3.590,1.787)--(3.670,1.787)--(3.751,1.787)%
  --(3.831,1.787)--(3.911,1.787)--(3.991,1.787)--(4.071,1.787)--(4.152,1.787)--(4.232,1.787)%
  --(4.312,1.787)--(4.392,1.787)--(4.473,1.787)--(4.553,1.787)--(4.633,1.787)--(4.713,1.787)%
  --(4.794,1.787)--(4.874,1.787)--(4.954,1.787)--(5.034,1.787)--(5.114,1.787)--(5.195,1.787)%
  --(5.275,1.787)--(5.355,1.787)--(5.435,1.787)--(5.516,1.787)--(5.596,1.787)--(5.676,1.787)%
  --(5.756,1.787)--(5.837,1.787)--(5.917,1.787)--(5.997,1.787)--(6.077,1.787)--(6.157,1.787)%
  --(6.238,1.787)--(6.318,1.787)--(6.398,1.787)--(6.478,1.787)--(6.559,1.787)--(6.639,1.787)%
  --(6.719,1.787)--(6.799,1.787)--(6.880,1.787)--(6.960,1.787)--(7.040,1.787)--(7.120,1.787)%
  --(7.200,1.787)--(7.281,1.787)--(7.361,1.787)--(7.441,1.787)--(7.521,1.787)--(7.602,1.787)%
  --(7.682,1.787)--(7.762,1.787)--(7.842,1.787)--(7.923,1.787)--(8.003,1.787)--(8.083,1.787)%
  --(8.163,1.787)--(8.244,1.787)--(8.324,1.787)--(8.404,1.787)--(8.484,1.787)--(8.564,1.787)%
  --(8.645,1.787)--(8.725,1.787)--(8.805,1.787)--(8.885,1.787)--(8.966,1.787)--(9.046,1.787)%
  --(9.126,1.787)--(9.206,1.787)--(9.287,1.787)--(9.367,1.787)--(9.447,1.787);
\gpcolor{gp lt color 4}
\gpsetlinetype{gp lt plot 0}
\draw[gp path] (1.504,9.279)--(2.828,9.320)--(3.061,9.362)--(3.459,9.404)--(3.625,9.445)%
  --(3.693,9.486)--(3.858,9.527)--(4.091,9.568)--(4.152,9.608)--(5.475,9.631)--(6.799,9.631)%
  --(8.123,9.631)--(9.447,9.631);
\gppoint{gp mark 14}{(1.504,9.279)}
\gppoint{gp mark 14}{(2.828,9.320)}
\gppoint{gp mark 14}{(3.061,9.362)}
\gppoint{gp mark 14}{(3.459,9.404)}
\gppoint{gp mark 14}{(3.625,9.445)}
\gppoint{gp mark 14}{(3.693,9.486)}
\gppoint{gp mark 14}{(3.858,9.527)}
\gppoint{gp mark 14}{(4.091,9.568)}
\gppoint{gp mark 14}{(4.152,9.608)}
\gppoint{gp mark 14}{(5.475,9.631)}
\gppoint{gp mark 14}{(6.799,9.631)}
\gppoint{gp mark 14}{(8.123,9.631)}
\gppoint{gp mark 14}{(9.447,9.631)}
\gpsetlinetype{gp lt axes}
\draw[gp path] (1.504,1.848)--(1.584,1.848)--(1.664,1.848)--(1.745,1.848)--(1.825,1.848)%
  --(1.905,1.848)--(1.985,1.848)--(2.066,1.848)--(2.146,1.848)--(2.226,1.848)--(2.306,1.848)%
  --(2.387,1.848)--(2.467,1.848)--(2.547,1.848)--(2.627,1.848)--(2.707,1.848)--(2.788,1.848)%
  --(2.868,1.848)--(2.948,1.848)--(3.028,1.848)--(3.109,1.848)--(3.189,1.848)--(3.269,1.848)%
  --(3.349,1.848)--(3.430,1.848)--(3.510,1.848)--(3.590,1.848)--(3.670,1.848)--(3.751,1.848)%
  --(3.831,1.848)--(3.911,1.848)--(3.991,1.848)--(4.071,1.848)--(4.152,1.848)--(4.232,1.848)%
  --(4.312,1.848)--(4.392,1.848)--(4.473,1.848)--(4.553,1.848)--(4.633,1.848)--(4.713,1.848)%
  --(4.794,1.848)--(4.874,1.848)--(4.954,1.848)--(5.034,1.848)--(5.114,1.848)--(5.195,1.848)%
  --(5.275,1.848)--(5.355,1.848)--(5.435,1.848)--(5.516,1.848)--(5.596,1.848)--(5.676,1.848)%
  --(5.756,1.848)--(5.837,1.848)--(5.917,1.848)--(5.997,1.848)--(6.077,1.848)--(6.157,1.848)%
  --(6.238,1.848)--(6.318,1.848)--(6.398,1.848)--(6.478,1.848)--(6.559,1.848)--(6.639,1.848)%
  --(6.719,1.848)--(6.799,1.848)--(6.880,1.848)--(6.960,1.848)--(7.040,1.848)--(7.120,1.848)%
  --(7.200,1.848)--(7.281,1.848)--(7.361,1.848)--(7.441,1.848)--(7.521,1.848)--(7.602,1.848)%
  --(7.682,1.848)--(7.762,1.848)--(7.842,1.848)--(7.923,1.848)--(8.003,1.848)--(8.083,1.848)%
  --(8.163,1.848)--(8.244,1.848)--(8.324,1.848)--(8.404,1.848)--(8.484,1.848)--(8.564,1.848)%
  --(8.645,1.848)--(8.725,1.848)--(8.805,1.848)--(8.885,1.848)--(8.966,1.848)--(9.046,1.848)%
  --(9.126,1.848)--(9.206,1.848)--(9.287,1.848)--(9.367,1.848)--(9.447,1.848);
\gpcolor{gp lt color 5}
\gpsetlinetype{gp lt plot 0}
\draw[gp path] (1.504,8.992)--(2.828,9.065)--(3.061,9.138)--(3.459,9.211)--(3.625,9.283)%
  --(3.693,9.355)--(3.858,9.428)--(4.091,9.500)--(4.152,9.572)--(5.475,9.631)--(6.799,9.631)%
  --(8.123,9.631)--(9.447,9.631);
\gppoint{gp mark 15}{(1.504,8.992)}
\gppoint{gp mark 15}{(2.828,9.065)}
\gppoint{gp mark 15}{(3.061,9.138)}
\gppoint{gp mark 15}{(3.459,9.211)}
\gppoint{gp mark 15}{(3.625,9.283)}
\gppoint{gp mark 15}{(3.693,9.355)}
\gppoint{gp mark 15}{(3.858,9.428)}
\gppoint{gp mark 15}{(4.091,9.500)}
\gppoint{gp mark 15}{(4.152,9.572)}
\gppoint{gp mark 15}{(5.475,9.631)}
\gppoint{gp mark 15}{(6.799,9.631)}
\gppoint{gp mark 15}{(8.123,9.631)}
\gppoint{gp mark 15}{(9.447,9.631)}
\gpsetlinetype{gp lt axes}
\draw[gp path] (1.504,1.826)--(1.584,1.826)--(1.664,1.826)--(1.745,1.826)--(1.825,1.826)%
  --(1.905,1.826)--(1.985,1.826)--(2.066,1.826)--(2.146,1.826)--(2.226,1.826)--(2.306,1.826)%
  --(2.387,1.826)--(2.467,1.826)--(2.547,1.826)--(2.627,1.826)--(2.707,1.826)--(2.788,1.826)%
  --(2.868,1.826)--(2.948,1.826)--(3.028,1.826)--(3.109,1.826)--(3.189,1.826)--(3.269,1.826)%
  --(3.349,1.826)--(3.430,1.826)--(3.510,1.826)--(3.590,1.826)--(3.670,1.826)--(3.751,1.826)%
  --(3.831,1.826)--(3.911,1.826)--(3.991,1.826)--(4.071,1.826)--(4.152,1.826)--(4.232,1.826)%
  --(4.312,1.826)--(4.392,1.826)--(4.473,1.826)--(4.553,1.826)--(4.633,1.826)--(4.713,1.826)%
  --(4.794,1.826)--(4.874,1.826)--(4.954,1.826)--(5.034,1.826)--(5.114,1.826)--(5.195,1.826)%
  --(5.275,1.826)--(5.355,1.826)--(5.435,1.826)--(5.516,1.826)--(5.596,1.826)--(5.676,1.826)%
  --(5.756,1.826)--(5.837,1.826)--(5.917,1.826)--(5.997,1.826)--(6.077,1.826)--(6.157,1.826)%
  --(6.238,1.826)--(6.318,1.826)--(6.398,1.826)--(6.478,1.826)--(6.559,1.826)--(6.639,1.826)%
  --(6.719,1.826)--(6.799,1.826)--(6.880,1.826)--(6.960,1.826)--(7.040,1.826)--(7.120,1.826)%
  --(7.200,1.826)--(7.281,1.826)--(7.361,1.826)--(7.441,1.826)--(7.521,1.826)--(7.602,1.826)%
  --(7.682,1.826)--(7.762,1.826)--(7.842,1.826)--(7.923,1.826)--(8.003,1.826)--(8.083,1.826)%
  --(8.163,1.826)--(8.244,1.826)--(8.324,1.826)--(8.404,1.826)--(8.484,1.826)--(8.564,1.826)%
  --(8.645,1.826)--(8.725,1.826)--(8.805,1.826)--(8.885,1.826)--(8.966,1.826)--(9.046,1.826)%
  --(9.126,1.826)--(9.206,1.826)--(9.287,1.826)--(9.367,1.826)--(9.447,1.826);
\gpcolor{gp lt color 6}
\gpsetlinetype{gp lt plot 0}
\draw[gp path] (1.504,1.872)--(2.828,2.757)--(3.061,3.641)--(3.459,4.524)--(3.625,5.406)%
  --(3.693,6.290)--(3.858,7.170)--(4.091,8.048)--(4.152,8.924)--(5.475,9.631)--(6.799,9.631)%
  --(8.123,9.631)--(9.447,9.631);
\gppoint{gp mark 1}{(1.504,1.872)}
\gppoint{gp mark 1}{(2.828,2.757)}
\gppoint{gp mark 1}{(3.061,3.641)}
\gppoint{gp mark 1}{(3.459,4.524)}
\gppoint{gp mark 1}{(3.625,5.406)}
\gppoint{gp mark 1}{(3.693,6.290)}
\gppoint{gp mark 1}{(3.858,7.170)}
\gppoint{gp mark 1}{(4.091,8.048)}
\gppoint{gp mark 1}{(4.152,8.924)}
\gppoint{gp mark 1}{(5.475,9.631)}
\gppoint{gp mark 1}{(6.799,9.631)}
\gppoint{gp mark 1}{(8.123,9.631)}
\gppoint{gp mark 1}{(9.447,9.631)}
\gpsetlinetype{gp lt axes}
\draw[gp path] (1.504,1.872)--(1.584,1.872)--(1.664,1.872)--(1.745,1.872)--(1.825,1.872)%
  --(1.905,1.872)--(1.985,1.872)--(2.066,1.872)--(2.146,1.872)--(2.226,1.872)--(2.306,1.872)%
  --(2.387,1.872)--(2.467,1.872)--(2.547,1.872)--(2.627,1.872)--(2.707,1.872)--(2.788,1.872)%
  --(2.868,1.872)--(2.948,1.872)--(3.028,1.872)--(3.109,1.872)--(3.189,1.872)--(3.269,1.872)%
  --(3.349,1.872)--(3.430,1.872)--(3.510,1.872)--(3.590,1.872)--(3.670,1.872)--(3.751,1.872)%
  --(3.831,1.872)--(3.911,1.872)--(3.991,1.872)--(4.071,1.872)--(4.152,1.872)--(4.232,1.872)%
  --(4.312,1.872)--(4.392,1.872)--(4.473,1.872)--(4.553,1.872)--(4.633,1.872)--(4.713,1.872)%
  --(4.794,1.872)--(4.874,1.872)--(4.954,1.872)--(5.034,1.872)--(5.114,1.872)--(5.195,1.872)%
  --(5.275,1.872)--(5.355,1.872)--(5.435,1.872)--(5.516,1.872)--(5.596,1.872)--(5.676,1.872)%
  --(5.756,1.872)--(5.837,1.872)--(5.917,1.872)--(5.997,1.872)--(6.077,1.872)--(6.157,1.872)%
  --(6.238,1.872)--(6.318,1.872)--(6.398,1.872)--(6.478,1.872)--(6.559,1.872)--(6.639,1.872)%
  --(6.719,1.872)--(6.799,1.872)--(6.880,1.872)--(6.960,1.872)--(7.040,1.872)--(7.120,1.872)%
  --(7.200,1.872)--(7.281,1.872)--(7.361,1.872)--(7.441,1.872)--(7.521,1.872)--(7.602,1.872)%
  --(7.682,1.872)--(7.762,1.872)--(7.842,1.872)--(7.923,1.872)--(8.003,1.872)--(8.083,1.872)%
  --(8.163,1.872)--(8.244,1.872)--(8.324,1.872)--(8.404,1.872)--(8.484,1.872)--(8.564,1.872)%
  --(8.645,1.872)--(8.725,1.872)--(8.805,1.872)--(8.885,1.872)--(8.966,1.872)--(9.046,1.872)%
  --(9.126,1.872)--(9.206,1.872)--(9.287,1.872)--(9.367,1.872)--(9.447,1.872);
\gpcolor{gp lt color 7}
\gpsetlinetype{gp lt plot 0}
\draw[gp path] (1.504,6.952)--(2.828,7.308)--(3.061,7.663)--(3.459,8.007)--(3.625,8.344)%
  --(3.693,8.685)--(3.858,9.014)--(4.091,9.338)--(4.152,9.631)--(5.475,9.631)--(6.799,9.631)%
  --(8.123,9.631)--(9.447,9.631);
\gppoint{gp mark 2}{(1.504,6.952)}
\gppoint{gp mark 2}{(2.828,7.308)}
\gppoint{gp mark 2}{(3.061,7.663)}
\gppoint{gp mark 2}{(3.459,8.007)}
\gppoint{gp mark 2}{(3.625,8.344)}
\gppoint{gp mark 2}{(3.693,8.685)}
\gppoint{gp mark 2}{(3.858,9.014)}
\gppoint{gp mark 2}{(4.091,9.338)}
\gppoint{gp mark 2}{(4.152,9.631)}
\gppoint{gp mark 2}{(5.475,9.631)}
\gppoint{gp mark 2}{(6.799,9.631)}
\gppoint{gp mark 2}{(8.123,9.631)}
\gppoint{gp mark 2}{(9.447,9.631)}
\gpsetlinetype{gp lt axes}
\draw[gp path] (1.504,2.006)--(1.584,2.006)--(1.664,2.006)--(1.745,2.006)--(1.825,2.006)%
  --(1.905,2.006)--(1.985,2.006)--(2.066,2.006)--(2.146,2.006)--(2.226,2.006)--(2.306,2.006)%
  --(2.387,2.006)--(2.467,2.006)--(2.547,2.006)--(2.627,2.006)--(2.707,2.006)--(2.788,2.006)%
  --(2.868,2.006)--(2.948,2.006)--(3.028,2.006)--(3.109,2.006)--(3.189,2.006)--(3.269,2.006)%
  --(3.349,2.006)--(3.430,2.006)--(3.510,2.006)--(3.590,2.006)--(3.670,2.006)--(3.751,2.006)%
  --(3.831,2.006)--(3.911,2.006)--(3.991,2.006)--(4.071,2.006)--(4.152,2.006)--(4.232,2.006)%
  --(4.312,2.006)--(4.392,2.006)--(4.473,2.006)--(4.553,2.006)--(4.633,2.006)--(4.713,2.006)%
  --(4.794,2.006)--(4.874,2.006)--(4.954,2.006)--(5.034,2.006)--(5.114,2.006)--(5.195,2.006)%
  --(5.275,2.006)--(5.355,2.006)--(5.435,2.006)--(5.516,2.006)--(5.596,2.006)--(5.676,2.006)%
  --(5.756,2.006)--(5.837,2.006)--(5.917,2.006)--(5.997,2.006)--(6.077,2.006)--(6.157,2.006)%
  --(6.238,2.006)--(6.318,2.006)--(6.398,2.006)--(6.478,2.006)--(6.559,2.006)--(6.639,2.006)%
  --(6.719,2.006)--(6.799,2.006)--(6.880,2.006)--(6.960,2.006)--(7.040,2.006)--(7.120,2.006)%
  --(7.200,2.006)--(7.281,2.006)--(7.361,2.006)--(7.441,2.006)--(7.521,2.006)--(7.602,2.006)%
  --(7.682,2.006)--(7.762,2.006)--(7.842,2.006)--(7.923,2.006)--(8.003,2.006)--(8.083,2.006)%
  --(8.163,2.006)--(8.244,2.006)--(8.324,2.006)--(8.404,2.006)--(8.484,2.006)--(8.564,2.006)%
  --(8.645,2.006)--(8.725,2.006)--(8.805,2.006)--(8.885,2.006)--(8.966,2.006)--(9.046,2.006)%
  --(9.126,2.006)--(9.206,2.006)--(9.287,2.006)--(9.367,2.006)--(9.447,2.006);
\gpcolor{gp lt color 0}
\gpsetlinetype{gp lt plot 0}
\draw[gp path] (1.504,7.665)--(2.828,7.897)--(3.061,8.130)--(3.459,8.361)--(3.625,8.591)%
  --(3.693,8.822)--(3.858,9.052)--(4.091,9.280)--(4.152,9.508)--(5.475,9.631)--(6.799,9.631)%
  --(8.123,9.631)--(9.447,9.631);
\gppoint{gp mark 3}{(1.504,7.665)}
\gppoint{gp mark 3}{(2.828,7.897)}
\gppoint{gp mark 3}{(3.061,8.130)}
\gppoint{gp mark 3}{(3.459,8.361)}
\gppoint{gp mark 3}{(3.625,8.591)}
\gppoint{gp mark 3}{(3.693,8.822)}
\gppoint{gp mark 3}{(3.858,9.052)}
\gppoint{gp mark 3}{(4.091,9.280)}
\gppoint{gp mark 3}{(4.152,9.508)}
\gppoint{gp mark 3}{(5.475,9.631)}
\gppoint{gp mark 3}{(6.799,9.631)}
\gppoint{gp mark 3}{(8.123,9.631)}
\gppoint{gp mark 3}{(9.447,9.631)}
\gpsetlinetype{gp lt axes}
\draw[gp path] (1.504,1.924)--(1.584,1.924)--(1.664,1.924)--(1.745,1.924)--(1.825,1.924)%
  --(1.905,1.924)--(1.985,1.924)--(2.066,1.924)--(2.146,1.924)--(2.226,1.924)--(2.306,1.924)%
  --(2.387,1.924)--(2.467,1.924)--(2.547,1.924)--(2.627,1.924)--(2.707,1.924)--(2.788,1.924)%
  --(2.868,1.924)--(2.948,1.924)--(3.028,1.924)--(3.109,1.924)--(3.189,1.924)--(3.269,1.924)%
  --(3.349,1.924)--(3.430,1.924)--(3.510,1.924)--(3.590,1.924)--(3.670,1.924)--(3.751,1.924)%
  --(3.831,1.924)--(3.911,1.924)--(3.991,1.924)--(4.071,1.924)--(4.152,1.924)--(4.232,1.924)%
  --(4.312,1.924)--(4.392,1.924)--(4.473,1.924)--(4.553,1.924)--(4.633,1.924)--(4.713,1.924)%
  --(4.794,1.924)--(4.874,1.924)--(4.954,1.924)--(5.034,1.924)--(5.114,1.924)--(5.195,1.924)%
  --(5.275,1.924)--(5.355,1.924)--(5.435,1.924)--(5.516,1.924)--(5.596,1.924)--(5.676,1.924)%
  --(5.756,1.924)--(5.837,1.924)--(5.917,1.924)--(5.997,1.924)--(6.077,1.924)--(6.157,1.924)%
  --(6.238,1.924)--(6.318,1.924)--(6.398,1.924)--(6.478,1.924)--(6.559,1.924)--(6.639,1.924)%
  --(6.719,1.924)--(6.799,1.924)--(6.880,1.924)--(6.960,1.924)--(7.040,1.924)--(7.120,1.924)%
  --(7.200,1.924)--(7.281,1.924)--(7.361,1.924)--(7.441,1.924)--(7.521,1.924)--(7.602,1.924)%
  --(7.682,1.924)--(7.762,1.924)--(7.842,1.924)--(7.923,1.924)--(8.003,1.924)--(8.083,1.924)%
  --(8.163,1.924)--(8.244,1.924)--(8.324,1.924)--(8.404,1.924)--(8.484,1.924)--(8.564,1.924)%
  --(8.645,1.924)--(8.725,1.924)--(8.805,1.924)--(8.885,1.924)--(8.966,1.924)--(9.046,1.924)%
  --(9.126,1.924)--(9.206,1.924)--(9.287,1.924)--(9.367,1.924)--(9.447,1.924);
\gpcolor{gp lt color 1}
\gpsetlinetype{gp lt plot 0}
\draw[gp path] (1.504,5.738)--(2.828,6.199)--(3.061,6.659)--(3.459,7.118)--(3.625,7.574)%
  --(3.693,8.030)--(3.858,8.485)--(4.091,8.936)--(4.152,9.386)--(5.475,9.631)--(6.799,9.631)%
  --(8.123,9.631)--(9.447,9.631);
\gppoint{gp mark 4}{(1.504,5.738)}
\gppoint{gp mark 4}{(2.828,6.199)}
\gppoint{gp mark 4}{(3.061,6.659)}
\gppoint{gp mark 4}{(3.459,7.118)}
\gppoint{gp mark 4}{(3.625,7.574)}
\gppoint{gp mark 4}{(3.693,8.030)}
\gppoint{gp mark 4}{(3.858,8.485)}
\gppoint{gp mark 4}{(4.091,8.936)}
\gppoint{gp mark 4}{(4.152,9.386)}
\gppoint{gp mark 4}{(5.475,9.631)}
\gppoint{gp mark 4}{(6.799,9.631)}
\gppoint{gp mark 4}{(8.123,9.631)}
\gppoint{gp mark 4}{(9.447,9.631)}
\gpsetlinetype{gp lt axes}
\draw[gp path] (1.504,1.946)--(1.584,1.946)--(1.664,1.946)--(1.745,1.946)--(1.825,1.946)%
  --(1.905,1.946)--(1.985,1.946)--(2.066,1.946)--(2.146,1.946)--(2.226,1.946)--(2.306,1.946)%
  --(2.387,1.946)--(2.467,1.946)--(2.547,1.946)--(2.627,1.946)--(2.707,1.946)--(2.788,1.946)%
  --(2.868,1.946)--(2.948,1.946)--(3.028,1.946)--(3.109,1.946)--(3.189,1.946)--(3.269,1.946)%
  --(3.349,1.946)--(3.430,1.946)--(3.510,1.946)--(3.590,1.946)--(3.670,1.946)--(3.751,1.946)%
  --(3.831,1.946)--(3.911,1.946)--(3.991,1.946)--(4.071,1.946)--(4.152,1.946)--(4.232,1.946)%
  --(4.312,1.946)--(4.392,1.946)--(4.473,1.946)--(4.553,1.946)--(4.633,1.946)--(4.713,1.946)%
  --(4.794,1.946)--(4.874,1.946)--(4.954,1.946)--(5.034,1.946)--(5.114,1.946)--(5.195,1.946)%
  --(5.275,1.946)--(5.355,1.946)--(5.435,1.946)--(5.516,1.946)--(5.596,1.946)--(5.676,1.946)%
  --(5.756,1.946)--(5.837,1.946)--(5.917,1.946)--(5.997,1.946)--(6.077,1.946)--(6.157,1.946)%
  --(6.238,1.946)--(6.318,1.946)--(6.398,1.946)--(6.478,1.946)--(6.559,1.946)--(6.639,1.946)%
  --(6.719,1.946)--(6.799,1.946)--(6.880,1.946)--(6.960,1.946)--(7.040,1.946)--(7.120,1.946)%
  --(7.200,1.946)--(7.281,1.946)--(7.361,1.946)--(7.441,1.946)--(7.521,1.946)--(7.602,1.946)%
  --(7.682,1.946)--(7.762,1.946)--(7.842,1.946)--(7.923,1.946)--(8.003,1.946)--(8.083,1.946)%
  --(8.163,1.946)--(8.244,1.946)--(8.324,1.946)--(8.404,1.946)--(8.484,1.946)--(8.564,1.946)%
  --(8.645,1.946)--(8.725,1.946)--(8.805,1.946)--(8.885,1.946)--(8.966,1.946)--(9.046,1.946)%
  --(9.126,1.946)--(9.206,1.946)--(9.287,1.946)--(9.367,1.946)--(9.447,1.946);
\gpcolor{gp lt color 2}
\gpsetlinetype{gp lt plot 0}
\draw[gp path] (1.504,7.027)--(2.828,7.336)--(3.061,7.644)--(3.459,7.951)--(3.625,8.256)%
  --(3.693,8.560)--(3.858,8.865)--(4.091,9.167)--(4.152,9.467)--(5.475,9.631)--(6.799,9.631)%
  --(8.123,9.631)--(9.447,9.631);
\gppoint{gp mark 5}{(1.504,7.027)}
\gppoint{gp mark 5}{(2.828,7.336)}
\gppoint{gp mark 5}{(3.061,7.644)}
\gppoint{gp mark 5}{(3.459,7.951)}
\gppoint{gp mark 5}{(3.625,8.256)}
\gppoint{gp mark 5}{(3.693,8.560)}
\gppoint{gp mark 5}{(3.858,8.865)}
\gppoint{gp mark 5}{(4.091,9.167)}
\gppoint{gp mark 5}{(4.152,9.467)}
\gppoint{gp mark 5}{(5.475,9.631)}
\gppoint{gp mark 5}{(6.799,9.631)}
\gppoint{gp mark 5}{(8.123,9.631)}
\gppoint{gp mark 5}{(9.447,9.631)}
\gpsetlinetype{gp lt axes}
\draw[gp path] (1.504,1.932)--(1.584,1.932)--(1.664,1.932)--(1.745,1.932)--(1.825,1.932)%
  --(1.905,1.932)--(1.985,1.932)--(2.066,1.932)--(2.146,1.932)--(2.226,1.932)--(2.306,1.932)%
  --(2.387,1.932)--(2.467,1.932)--(2.547,1.932)--(2.627,1.932)--(2.707,1.932)--(2.788,1.932)%
  --(2.868,1.932)--(2.948,1.932)--(3.028,1.932)--(3.109,1.932)--(3.189,1.932)--(3.269,1.932)%
  --(3.349,1.932)--(3.430,1.932)--(3.510,1.932)--(3.590,1.932)--(3.670,1.932)--(3.751,1.932)%
  --(3.831,1.932)--(3.911,1.932)--(3.991,1.932)--(4.071,1.932)--(4.152,1.932)--(4.232,1.932)%
  --(4.312,1.932)--(4.392,1.932)--(4.473,1.932)--(4.553,1.932)--(4.633,1.932)--(4.713,1.932)%
  --(4.794,1.932)--(4.874,1.932)--(4.954,1.932)--(5.034,1.932)--(5.114,1.932)--(5.195,1.932)%
  --(5.275,1.932)--(5.355,1.932)--(5.435,1.932)--(5.516,1.932)--(5.596,1.932)--(5.676,1.932)%
  --(5.756,1.932)--(5.837,1.932)--(5.917,1.932)--(5.997,1.932)--(6.077,1.932)--(6.157,1.932)%
  --(6.238,1.932)--(6.318,1.932)--(6.398,1.932)--(6.478,1.932)--(6.559,1.932)--(6.639,1.932)%
  --(6.719,1.932)--(6.799,1.932)--(6.880,1.932)--(6.960,1.932)--(7.040,1.932)--(7.120,1.932)%
  --(7.200,1.932)--(7.281,1.932)--(7.361,1.932)--(7.441,1.932)--(7.521,1.932)--(7.602,1.932)%
  --(7.682,1.932)--(7.762,1.932)--(7.842,1.932)--(7.923,1.932)--(8.003,1.932)--(8.083,1.932)%
  --(8.163,1.932)--(8.244,1.932)--(8.324,1.932)--(8.404,1.932)--(8.484,1.932)--(8.564,1.932)%
  --(8.645,1.932)--(8.725,1.932)--(8.805,1.932)--(8.885,1.932)--(8.966,1.932)--(9.046,1.932)%
  --(9.126,1.932)--(9.206,1.932)--(9.287,1.932)--(9.367,1.932)--(9.447,1.932);
\gpcolor{gp lt color 3}
\gpsetlinetype{gp lt plot 0}
\draw[gp path] (1.504,1.993)--(2.828,3.001)--(3.061,3.999)--(3.459,4.957)--(3.625,5.963)%
  --(3.693,6.957)--(3.858,7.913)--(4.091,8.786)--(4.152,9.631)--(5.475,9.631)--(6.799,9.631)%
  --(8.123,9.631)--(9.447,9.631);
\gppoint{gp mark 6}{(1.504,1.993)}
\gppoint{gp mark 6}{(2.828,3.001)}
\gppoint{gp mark 6}{(3.061,3.999)}
\gppoint{gp mark 6}{(3.459,4.957)}
\gppoint{gp mark 6}{(3.625,5.963)}
\gppoint{gp mark 6}{(3.693,6.957)}
\gppoint{gp mark 6}{(3.858,7.913)}
\gppoint{gp mark 6}{(4.091,8.786)}
\gppoint{gp mark 6}{(4.152,9.631)}
\gppoint{gp mark 6}{(5.475,9.631)}
\gppoint{gp mark 6}{(6.799,9.631)}
\gppoint{gp mark 6}{(8.123,9.631)}
\gppoint{gp mark 6}{(9.447,9.631)}
\gpsetlinetype{gp lt axes}
\draw[gp path] (1.504,1.993)--(1.584,1.993)--(1.664,1.993)--(1.745,1.993)--(1.825,1.993)%
  --(1.905,1.993)--(1.985,1.993)--(2.066,1.993)--(2.146,1.993)--(2.226,1.993)--(2.306,1.993)%
  --(2.387,1.993)--(2.467,1.993)--(2.547,1.993)--(2.627,1.993)--(2.707,1.993)--(2.788,1.993)%
  --(2.868,1.993)--(2.948,1.993)--(3.028,1.993)--(3.109,1.993)--(3.189,1.993)--(3.269,1.993)%
  --(3.349,1.993)--(3.430,1.993)--(3.510,1.993)--(3.590,1.993)--(3.670,1.993)--(3.751,1.993)%
  --(3.831,1.993)--(3.911,1.993)--(3.991,1.993)--(4.071,1.993)--(4.152,1.993)--(4.232,1.993)%
  --(4.312,1.993)--(4.392,1.993)--(4.473,1.993)--(4.553,1.993)--(4.633,1.993)--(4.713,1.993)%
  --(4.794,1.993)--(4.874,1.993)--(4.954,1.993)--(5.034,1.993)--(5.114,1.993)--(5.195,1.993)%
  --(5.275,1.993)--(5.355,1.993)--(5.435,1.993)--(5.516,1.993)--(5.596,1.993)--(5.676,1.993)%
  --(5.756,1.993)--(5.837,1.993)--(5.917,1.993)--(5.997,1.993)--(6.077,1.993)--(6.157,1.993)%
  --(6.238,1.993)--(6.318,1.993)--(6.398,1.993)--(6.478,1.993)--(6.559,1.993)--(6.639,1.993)%
  --(6.719,1.993)--(6.799,1.993)--(6.880,1.993)--(6.960,1.993)--(7.040,1.993)--(7.120,1.993)%
  --(7.200,1.993)--(7.281,1.993)--(7.361,1.993)--(7.441,1.993)--(7.521,1.993)--(7.602,1.993)%
  --(7.682,1.993)--(7.762,1.993)--(7.842,1.993)--(7.923,1.993)--(8.003,1.993)--(8.083,1.993)%
  --(8.163,1.993)--(8.244,1.993)--(8.324,1.993)--(8.404,1.993)--(8.484,1.993)--(8.564,1.993)%
  --(8.645,1.993)--(8.725,1.993)--(8.805,1.993)--(8.885,1.993)--(8.966,1.993)--(9.046,1.993)%
  --(9.126,1.993)--(9.206,1.993)--(9.287,1.993)--(9.367,1.993)--(9.447,1.993);
%% coordinates of the plot area
\gpdefrectangularnode{gp plot 1}{\pgfpoint{1.504cm}{0.985cm}}{\pgfpoint{9.447cm}{9.631cm}}
\end{tikzpicture}
%% gnuplot variables

    \caption[Number of vectors pruned with various block size implementations]{
        Number of vectors pruned (for the \command{TopN_Outlier_Pruning_Block}
        function) with various block size implementations}
    \label{profiling:blockSize:vectorsPruned}
\end{figure}

\begin{figure}
    \centering
    \begin{tikzpicture}[gnuplot]
\gpsetlinetype{gp lt plot 2}
\gpsetlinewidth{1.00}
\node[gp node right] at (4.129,19.666) {block\_size=0};
\gpcolor{gp lt color axes}
\gpsetlinetype{gp lt plot 0}
\draw[gp path] (4.313,19.666)--(5.229,19.666);
\node[gp node right] at (4.129,19.358) {block\_size=1};
\gpcolor{gp lt color 0}
\draw[gp path] (4.313,19.358)--(5.229,19.358);
\node[gp node right] at (4.129,19.050) {block\_size=10};
\gpcolor{gp lt color 1}
\draw[gp path] (4.313,19.050)--(5.229,19.050);
\node[gp node right] at (4.129,18.742) {block\_size=15};
\gpcolor{gp lt color 2}
\draw[gp path] (4.313,18.742)--(5.229,18.742);
\node[gp node right] at (4.129,18.434) {block\_size=30};
\gpcolor{gp lt color 3}
\draw[gp path] (4.313,18.434)--(5.229,18.434);
\node[gp node right] at (4.129,18.126) {block\_size=40};
\gpcolor{gp lt color 4}
\draw[gp path] (4.313,18.126)--(5.229,18.126);
\node[gp node right] at (4.129,17.818) {block\_size=45};
\gpcolor{gp lt color 5}
\draw[gp path] (4.313,17.818)--(5.229,17.818);
\node[gp node right] at (8.725,19.666) {block\_size=60};
\gpcolor{gp lt color 6}
\draw[gp path] (8.909,19.666)--(9.825,19.666);
\node[gp node right] at (8.725,19.358) {block\_size=90};
\gpcolor{gp lt color 7}
\draw[gp path] (8.909,19.358)--(9.825,19.358);
\node[gp node right] at (8.725,19.050) {block\_size=100};
\gpcolor{gp lt color 0}
\draw[gp path] (8.909,19.050)--(9.825,19.050);
\node[gp node right] at (8.725,18.742) {block\_size=1000};
\gpcolor{gp lt color 1}
\draw[gp path] (8.909,18.742)--(9.825,18.742);
\node[gp node right] at (8.725,18.434) {block\_size=10000};
\gpcolor{gp lt color 2}
\draw[gp path] (8.909,18.434)--(9.825,18.434);
\node[gp node right] at (8.725,18.126) {block\_size=100000};
\gpcolor{gp lt color 3}
\draw[gp path] (8.909,18.126)--(9.825,18.126);
\node[gp node right] at (8.725,17.818) {block\_size=1000000};
\gpcolor{gp lt color 4}
\draw[gp path] (8.909,17.818)--(9.825,17.818);
\end{tikzpicture}
    \caption[Block size profiling legend]{The block sizes legend for all figures
        in \autoref{profiling:blockSize}}
    \label{profiling:blockSize:legend:block_sizes}
\end{figure}

\begin{figure}
    \centering
    \begin{minipage}{\textwidth}
        \centering
        \begin{tikzpicture}[gnuplot]
%% generated with GNUPLOT 4.4p3 (Lua 5.1.4; terminal rev. 97, script rev. 96a)
%% Wed 24 Oct 2012 11:48:45 EST
\gpcolor{gp lt color border}
\gpsetlinetype{gp lt border}
\gpsetlinewidth{1.00}
\draw[gp path] (1.688,0.985)--(1.868,0.985);
\draw[gp path] (8.447,0.985)--(8.267,0.985);
\node[gp node right] at (1.504,0.985) { 0};
\draw[gp path] (1.688,2.077)--(1.868,2.077);
\draw[gp path] (8.447,2.077)--(8.267,2.077);
\node[gp node right] at (1.504,2.077) { 1000};
\draw[gp path] (1.688,3.170)--(1.868,3.170);
\draw[gp path] (8.447,3.170)--(8.267,3.170);
\node[gp node right] at (1.504,3.170) { 2000};
\draw[gp path] (1.688,4.262)--(1.868,4.262);
\draw[gp path] (8.447,4.262)--(8.267,4.262);
\node[gp node right] at (1.504,4.262) { 3000};
\draw[gp path] (1.688,5.354)--(1.868,5.354);
\draw[gp path] (8.447,5.354)--(8.267,5.354);
\node[gp node right] at (1.504,5.354) { 4000};
\draw[gp path] (1.688,6.446)--(1.868,6.446);
\draw[gp path] (8.447,6.446)--(8.267,6.446);
\node[gp node right] at (1.504,6.446) { 5000};
\draw[gp path] (1.688,7.539)--(1.868,7.539);
\draw[gp path] (8.447,7.539)--(8.267,7.539);
\node[gp node right] at (1.504,7.539) { 6000};
\draw[gp path] (1.688,8.631)--(1.868,8.631);
\draw[gp path] (8.447,8.631)--(8.267,8.631);
\node[gp node right] at (1.504,8.631) { 7000};
\draw[gp path] (1.688,0.985)--(1.688,1.165);
\draw[gp path] (1.688,8.631)--(1.688,8.451);
\node[gp node center] at (1.688,0.677) {$10^{2}$};
\draw[gp path] (2.366,0.985)--(2.366,1.075);
\draw[gp path] (2.366,8.631)--(2.366,8.541);
\draw[gp path] (2.763,0.985)--(2.763,1.075);
\draw[gp path] (2.763,8.631)--(2.763,8.541);
\draw[gp path] (3.044,0.985)--(3.044,1.075);
\draw[gp path] (3.044,8.631)--(3.044,8.541);
\draw[gp path] (3.263,0.985)--(3.263,1.075);
\draw[gp path] (3.263,8.631)--(3.263,8.541);
\draw[gp path] (3.441,0.985)--(3.441,1.075);
\draw[gp path] (3.441,8.631)--(3.441,8.541);
\draw[gp path] (3.592,0.985)--(3.592,1.075);
\draw[gp path] (3.592,8.631)--(3.592,8.541);
\draw[gp path] (3.723,0.985)--(3.723,1.075);
\draw[gp path] (3.723,8.631)--(3.723,8.541);
\draw[gp path] (3.838,0.985)--(3.838,1.075);
\draw[gp path] (3.838,8.631)--(3.838,8.541);
\draw[gp path] (3.941,0.985)--(3.941,1.165);
\draw[gp path] (3.941,8.631)--(3.941,8.451);
\node[gp node center] at (3.941,0.677) {$10^{3}$};
\draw[gp path] (4.619,0.985)--(4.619,1.075);
\draw[gp path] (4.619,8.631)--(4.619,8.541);
\draw[gp path] (5.016,0.985)--(5.016,1.075);
\draw[gp path] (5.016,8.631)--(5.016,8.541);
\draw[gp path] (5.297,0.985)--(5.297,1.075);
\draw[gp path] (5.297,8.631)--(5.297,8.541);
\draw[gp path] (5.516,0.985)--(5.516,1.075);
\draw[gp path] (5.516,8.631)--(5.516,8.541);
\draw[gp path] (5.694,0.985)--(5.694,1.075);
\draw[gp path] (5.694,8.631)--(5.694,8.541);
\draw[gp path] (5.845,0.985)--(5.845,1.075);
\draw[gp path] (5.845,8.631)--(5.845,8.541);
\draw[gp path] (5.976,0.985)--(5.976,1.075);
\draw[gp path] (5.976,8.631)--(5.976,8.541);
\draw[gp path] (6.091,0.985)--(6.091,1.075);
\draw[gp path] (6.091,8.631)--(6.091,8.541);
\draw[gp path] (6.194,0.985)--(6.194,1.165);
\draw[gp path] (6.194,8.631)--(6.194,8.451);
\node[gp node center] at (6.194,0.677) {$10^{4}$};
\draw[gp path] (6.872,0.985)--(6.872,1.075);
\draw[gp path] (6.872,8.631)--(6.872,8.541);
\draw[gp path] (7.269,0.985)--(7.269,1.075);
\draw[gp path] (7.269,8.631)--(7.269,8.541);
\draw[gp path] (7.550,0.985)--(7.550,1.075);
\draw[gp path] (7.550,8.631)--(7.550,8.541);
\draw[gp path] (7.769,0.985)--(7.769,1.075);
\draw[gp path] (7.769,8.631)--(7.769,8.541);
\draw[gp path] (7.947,0.985)--(7.947,1.075);
\draw[gp path] (7.947,8.631)--(7.947,8.541);
\draw[gp path] (8.098,0.985)--(8.098,1.075);
\draw[gp path] (8.098,8.631)--(8.098,8.541);
\draw[gp path] (8.229,0.985)--(8.229,1.075);
\draw[gp path] (8.229,8.631)--(8.229,8.541);
\draw[gp path] (8.344,0.985)--(8.344,1.075);
\draw[gp path] (8.344,8.631)--(8.344,8.541);
\draw[gp path] (8.447,0.985)--(8.447,1.165);
\draw[gp path] (8.447,8.631)--(8.447,8.451);
\node[gp node center] at (8.447,0.677) {$10^{5}$};
\draw[gp path] (1.688,8.631)--(1.688,0.985)--(8.447,0.985)--(8.447,8.631)--cycle;
\node[gp node center,rotate=-270] at (0.246,4.808) {\textbf{Total execution time}};
\node[gp node center] at (5.067,0.215) {\textbf{Problem size}};
\gpcolor{gp lt color axes}
\gpsetlinetype{gp lt plot 0}
\draw[gp path] (3.140,1.004)--(3.360,0.996)--(3.504,0.997)--(3.941,1.002)--(4.114,1.017)%
  --(4.619,1.020)--(4.667,1.032)--(5.413,1.088)--(5.434,1.095)--(5.787,1.194)--(6.194,1.238)%
  --(6.287,1.232)--(6.823,1.495)--(6.872,1.607)--(7.269,2.118)--(7.550,2.661)--(7.769,3.477)%
  --(8.063,6.965);
\gpsetpointsize{4.00}
\gppoint{gp mark 1}{(3.140,1.004)}
\gppoint{gp mark 1}{(3.360,0.996)}
\gppoint{gp mark 1}{(3.504,0.997)}
\gppoint{gp mark 1}{(3.941,1.002)}
\gppoint{gp mark 1}{(4.114,1.017)}
\gppoint{gp mark 1}{(4.619,1.020)}
\gppoint{gp mark 1}{(4.667,1.032)}
\gppoint{gp mark 1}{(5.413,1.088)}
\gppoint{gp mark 1}{(5.434,1.095)}
\gppoint{gp mark 1}{(5.787,1.194)}
\gppoint{gp mark 1}{(6.194,1.238)}
\gppoint{gp mark 1}{(6.287,1.232)}
\gppoint{gp mark 1}{(6.823,1.495)}
\gppoint{gp mark 1}{(6.872,1.607)}
\gppoint{gp mark 1}{(7.269,2.118)}
\gppoint{gp mark 1}{(7.550,2.661)}
\gppoint{gp mark 1}{(7.769,3.477)}
\gppoint{gp mark 1}{(8.063,6.965)}
\gpcolor{gp lt color 0}
\draw[gp path] (3.140,1.003)--(3.360,0.996)--(3.504,0.997)--(3.941,1.002)--(4.114,1.018)%
  --(4.619,1.020)--(4.667,1.031)--(5.413,1.084)--(5.434,1.095)--(5.787,1.200)--(6.194,1.240)%
  --(6.287,1.233)--(6.823,1.511)--(6.872,1.608)--(7.269,2.191)--(7.550,2.731)--(7.769,3.525)%
  --(8.063,6.822);
\gppoint{gp mark 2}{(3.140,1.003)}
\gppoint{gp mark 2}{(3.360,0.996)}
\gppoint{gp mark 2}{(3.504,0.997)}
\gppoint{gp mark 2}{(3.941,1.002)}
\gppoint{gp mark 2}{(4.114,1.018)}
\gppoint{gp mark 2}{(4.619,1.020)}
\gppoint{gp mark 2}{(4.667,1.031)}
\gppoint{gp mark 2}{(5.413,1.084)}
\gppoint{gp mark 2}{(5.434,1.095)}
\gppoint{gp mark 2}{(5.787,1.200)}
\gppoint{gp mark 2}{(6.194,1.240)}
\gppoint{gp mark 2}{(6.287,1.233)}
\gppoint{gp mark 2}{(6.823,1.511)}
\gppoint{gp mark 2}{(6.872,1.608)}
\gppoint{gp mark 2}{(7.269,2.191)}
\gppoint{gp mark 2}{(7.550,2.731)}
\gppoint{gp mark 2}{(7.769,3.525)}
\gppoint{gp mark 2}{(8.063,6.822)}
\gpcolor{gp lt color 1}
\draw[gp path] (3.140,1.003)--(3.360,0.996)--(3.504,0.997)--(3.941,1.002)--(4.114,1.018)%
  --(4.619,1.020)--(4.667,1.031)--(5.413,1.085)--(5.434,1.095)--(5.787,1.194)--(6.194,1.238)%
  --(6.287,1.239)--(6.823,1.512)--(6.872,1.601)--(7.269,2.155)--(7.550,2.720)--(7.769,3.473)%
  --(8.063,6.646);
\gppoint{gp mark 3}{(3.140,1.003)}
\gppoint{gp mark 3}{(3.360,0.996)}
\gppoint{gp mark 3}{(3.504,0.997)}
\gppoint{gp mark 3}{(3.941,1.002)}
\gppoint{gp mark 3}{(4.114,1.018)}
\gppoint{gp mark 3}{(4.619,1.020)}
\gppoint{gp mark 3}{(4.667,1.031)}
\gppoint{gp mark 3}{(5.413,1.085)}
\gppoint{gp mark 3}{(5.434,1.095)}
\gppoint{gp mark 3}{(5.787,1.194)}
\gppoint{gp mark 3}{(6.194,1.238)}
\gppoint{gp mark 3}{(6.287,1.239)}
\gppoint{gp mark 3}{(6.823,1.512)}
\gppoint{gp mark 3}{(6.872,1.601)}
\gppoint{gp mark 3}{(7.269,2.155)}
\gppoint{gp mark 3}{(7.550,2.720)}
\gppoint{gp mark 3}{(7.769,3.473)}
\gppoint{gp mark 3}{(8.063,6.646)}
\gpcolor{gp lt color 2}
\draw[gp path] (3.140,1.003)--(3.360,0.996)--(3.504,0.997)--(3.941,1.002)--(4.114,1.018)%
  --(4.619,1.020)--(4.667,1.031)--(5.413,1.086)--(5.434,1.095)--(5.787,1.194)--(6.194,1.238)%
  --(6.287,1.236)--(6.823,1.499)--(6.872,1.613)--(7.269,2.148)--(7.550,2.727)--(7.769,3.474)%
  --(8.063,6.612);
\gppoint{gp mark 4}{(3.140,1.003)}
\gppoint{gp mark 4}{(3.360,0.996)}
\gppoint{gp mark 4}{(3.504,0.997)}
\gppoint{gp mark 4}{(3.941,1.002)}
\gppoint{gp mark 4}{(4.114,1.018)}
\gppoint{gp mark 4}{(4.619,1.020)}
\gppoint{gp mark 4}{(4.667,1.031)}
\gppoint{gp mark 4}{(5.413,1.086)}
\gppoint{gp mark 4}{(5.434,1.095)}
\gppoint{gp mark 4}{(5.787,1.194)}
\gppoint{gp mark 4}{(6.194,1.238)}
\gppoint{gp mark 4}{(6.287,1.236)}
\gppoint{gp mark 4}{(6.823,1.499)}
\gppoint{gp mark 4}{(6.872,1.613)}
\gppoint{gp mark 4}{(7.269,2.148)}
\gppoint{gp mark 4}{(7.550,2.727)}
\gppoint{gp mark 4}{(7.769,3.474)}
\gppoint{gp mark 4}{(8.063,6.612)}
\gpcolor{gp lt color 3}
\draw[gp path] (3.140,1.003)--(3.360,0.996)--(3.504,0.997)--(3.941,1.002)--(4.114,1.018)%
  --(4.619,1.021)--(4.667,1.031)--(5.413,1.085)--(5.434,1.097)--(5.787,1.193)--(6.194,1.238)%
  --(6.287,1.237)--(6.823,1.507)--(6.872,1.614)--(7.269,2.150)--(7.550,2.728)--(7.769,3.481)%
  --(8.063,6.785);
\gppoint{gp mark 5}{(3.140,1.003)}
\gppoint{gp mark 5}{(3.360,0.996)}
\gppoint{gp mark 5}{(3.504,0.997)}
\gppoint{gp mark 5}{(3.941,1.002)}
\gppoint{gp mark 5}{(4.114,1.018)}
\gppoint{gp mark 5}{(4.619,1.021)}
\gppoint{gp mark 5}{(4.667,1.031)}
\gppoint{gp mark 5}{(5.413,1.085)}
\gppoint{gp mark 5}{(5.434,1.097)}
\gppoint{gp mark 5}{(5.787,1.193)}
\gppoint{gp mark 5}{(6.194,1.238)}
\gppoint{gp mark 5}{(6.287,1.237)}
\gppoint{gp mark 5}{(6.823,1.507)}
\gppoint{gp mark 5}{(6.872,1.614)}
\gppoint{gp mark 5}{(7.269,2.150)}
\gppoint{gp mark 5}{(7.550,2.728)}
\gppoint{gp mark 5}{(7.769,3.481)}
\gppoint{gp mark 5}{(8.063,6.785)}
\gpcolor{gp lt color 4}
\draw[gp path] (3.140,1.003)--(3.360,0.996)--(3.504,0.997)--(3.941,1.002)--(4.114,1.018)%
  --(4.619,1.021)--(4.667,1.031)--(5.413,1.086)--(5.434,1.103)--(5.787,1.196)--(6.194,1.238)%
  --(6.287,1.236)--(6.823,1.511)--(6.872,1.607)--(7.269,2.143)--(7.550,2.737)--(7.769,3.491)%
  --(8.063,6.791);
\gppoint{gp mark 6}{(3.140,1.003)}
\gppoint{gp mark 6}{(3.360,0.996)}
\gppoint{gp mark 6}{(3.504,0.997)}
\gppoint{gp mark 6}{(3.941,1.002)}
\gppoint{gp mark 6}{(4.114,1.018)}
\gppoint{gp mark 6}{(4.619,1.021)}
\gppoint{gp mark 6}{(4.667,1.031)}
\gppoint{gp mark 6}{(5.413,1.086)}
\gppoint{gp mark 6}{(5.434,1.103)}
\gppoint{gp mark 6}{(5.787,1.196)}
\gppoint{gp mark 6}{(6.194,1.238)}
\gppoint{gp mark 6}{(6.287,1.236)}
\gppoint{gp mark 6}{(6.823,1.511)}
\gppoint{gp mark 6}{(6.872,1.607)}
\gppoint{gp mark 6}{(7.269,2.143)}
\gppoint{gp mark 6}{(7.550,2.737)}
\gppoint{gp mark 6}{(7.769,3.491)}
\gppoint{gp mark 6}{(8.063,6.791)}
\gpcolor{gp lt color 5}
\draw[gp path] (3.140,1.003)--(3.360,0.996)--(3.504,0.997)--(3.941,1.002)--(4.114,1.018)%
  --(4.619,1.021)--(4.667,1.031)--(5.413,1.087)--(5.434,1.095)--(5.787,1.197)--(6.194,1.239)%
  --(6.287,1.241)--(6.823,1.517)--(6.872,1.601)--(7.269,2.154)--(7.550,2.724)--(7.769,3.507)%
  --(8.063,6.814);
\gppoint{gp mark 7}{(3.140,1.003)}
\gppoint{gp mark 7}{(3.360,0.996)}
\gppoint{gp mark 7}{(3.504,0.997)}
\gppoint{gp mark 7}{(3.941,1.002)}
\gppoint{gp mark 7}{(4.114,1.018)}
\gppoint{gp mark 7}{(4.619,1.021)}
\gppoint{gp mark 7}{(4.667,1.031)}
\gppoint{gp mark 7}{(5.413,1.087)}
\gppoint{gp mark 7}{(5.434,1.095)}
\gppoint{gp mark 7}{(5.787,1.197)}
\gppoint{gp mark 7}{(6.194,1.239)}
\gppoint{gp mark 7}{(6.287,1.241)}
\gppoint{gp mark 7}{(6.823,1.517)}
\gppoint{gp mark 7}{(6.872,1.601)}
\gppoint{gp mark 7}{(7.269,2.154)}
\gppoint{gp mark 7}{(7.550,2.724)}
\gppoint{gp mark 7}{(7.769,3.507)}
\gppoint{gp mark 7}{(8.063,6.814)}
\gpcolor{gp lt color 6}
\draw[gp path] (3.140,1.003)--(3.360,0.996)--(3.504,0.997)--(3.941,1.002)--(4.114,1.018)%
  --(4.619,1.021)--(4.667,1.031)--(5.413,1.086)--(5.434,1.096)--(5.787,1.195)--(6.194,1.239)%
  --(6.287,1.240)--(6.823,1.510)--(6.872,1.609)--(7.269,2.148)--(7.550,2.726)--(7.769,3.493)%
  --(8.063,7.256);
\gppoint{gp mark 8}{(3.140,1.003)}
\gppoint{gp mark 8}{(3.360,0.996)}
\gppoint{gp mark 8}{(3.504,0.997)}
\gppoint{gp mark 8}{(3.941,1.002)}
\gppoint{gp mark 8}{(4.114,1.018)}
\gppoint{gp mark 8}{(4.619,1.021)}
\gppoint{gp mark 8}{(4.667,1.031)}
\gppoint{gp mark 8}{(5.413,1.086)}
\gppoint{gp mark 8}{(5.434,1.096)}
\gppoint{gp mark 8}{(5.787,1.195)}
\gppoint{gp mark 8}{(6.194,1.239)}
\gppoint{gp mark 8}{(6.287,1.240)}
\gppoint{gp mark 8}{(6.823,1.510)}
\gppoint{gp mark 8}{(6.872,1.609)}
\gppoint{gp mark 8}{(7.269,2.148)}
\gppoint{gp mark 8}{(7.550,2.726)}
\gppoint{gp mark 8}{(7.769,3.493)}
\gppoint{gp mark 8}{(8.063,7.256)}
\gpcolor{gp lt color 7}
\draw[gp path] (3.140,1.003)--(3.360,0.996)--(3.504,0.997)--(3.941,1.002)--(4.114,1.018)%
  --(4.619,1.021)--(4.667,1.031)--(5.413,1.088)--(5.434,1.095)--(5.787,1.196)--(6.194,1.240)%
  --(6.287,1.241)--(6.823,1.508)--(6.872,1.613)--(7.269,2.151)--(7.550,2.722)--(7.769,3.508)%
  --(8.063,6.629);
\gppoint{gp mark 9}{(3.140,1.003)}
\gppoint{gp mark 9}{(3.360,0.996)}
\gppoint{gp mark 9}{(3.504,0.997)}
\gppoint{gp mark 9}{(3.941,1.002)}
\gppoint{gp mark 9}{(4.114,1.018)}
\gppoint{gp mark 9}{(4.619,1.021)}
\gppoint{gp mark 9}{(4.667,1.031)}
\gppoint{gp mark 9}{(5.413,1.088)}
\gppoint{gp mark 9}{(5.434,1.095)}
\gppoint{gp mark 9}{(5.787,1.196)}
\gppoint{gp mark 9}{(6.194,1.240)}
\gppoint{gp mark 9}{(6.287,1.241)}
\gppoint{gp mark 9}{(6.823,1.508)}
\gppoint{gp mark 9}{(6.872,1.613)}
\gppoint{gp mark 9}{(7.269,2.151)}
\gppoint{gp mark 9}{(7.550,2.722)}
\gppoint{gp mark 9}{(7.769,3.508)}
\gppoint{gp mark 9}{(8.063,6.629)}
\gpcolor{gp lt color 0}
\draw[gp path] (3.140,1.003)--(3.360,0.996)--(3.504,0.997)--(3.941,1.002)--(4.114,1.018)%
  --(4.619,1.021)--(4.667,1.032)--(5.413,1.087)--(5.434,1.095)--(5.787,1.197)--(6.194,1.240)%
  --(6.287,1.235)--(6.823,1.504)--(6.872,1.608)--(7.269,2.153)--(7.550,2.727)--(7.769,3.478)%
  --(8.063,7.254);
\gppoint{gp mark 10}{(3.140,1.003)}
\gppoint{gp mark 10}{(3.360,0.996)}
\gppoint{gp mark 10}{(3.504,0.997)}
\gppoint{gp mark 10}{(3.941,1.002)}
\gppoint{gp mark 10}{(4.114,1.018)}
\gppoint{gp mark 10}{(4.619,1.021)}
\gppoint{gp mark 10}{(4.667,1.032)}
\gppoint{gp mark 10}{(5.413,1.087)}
\gppoint{gp mark 10}{(5.434,1.095)}
\gppoint{gp mark 10}{(5.787,1.197)}
\gppoint{gp mark 10}{(6.194,1.240)}
\gppoint{gp mark 10}{(6.287,1.235)}
\gppoint{gp mark 10}{(6.823,1.504)}
\gppoint{gp mark 10}{(6.872,1.608)}
\gppoint{gp mark 10}{(7.269,2.153)}
\gppoint{gp mark 10}{(7.550,2.727)}
\gppoint{gp mark 10}{(7.769,3.478)}
\gppoint{gp mark 10}{(8.063,7.254)}
\gpcolor{gp lt color 1}
\draw[gp path] (3.140,1.003)--(3.360,0.996)--(3.504,0.997)--(3.941,1.003)--(4.114,1.018)%
  --(4.619,1.021)--(4.667,1.032)--(5.413,1.088)--(5.434,1.096)--(5.787,1.205)--(6.194,1.243)%
  --(6.287,1.238)--(6.823,1.512)--(6.872,1.616)--(7.269,2.163)--(7.550,2.738)--(7.769,3.503)%
  --(8.063,6.714);
\gppoint{gp mark 11}{(3.140,1.003)}
\gppoint{gp mark 11}{(3.360,0.996)}
\gppoint{gp mark 11}{(3.504,0.997)}
\gppoint{gp mark 11}{(3.941,1.003)}
\gppoint{gp mark 11}{(4.114,1.018)}
\gppoint{gp mark 11}{(4.619,1.021)}
\gppoint{gp mark 11}{(4.667,1.032)}
\gppoint{gp mark 11}{(5.413,1.088)}
\gppoint{gp mark 11}{(5.434,1.096)}
\gppoint{gp mark 11}{(5.787,1.205)}
\gppoint{gp mark 11}{(6.194,1.243)}
\gppoint{gp mark 11}{(6.287,1.238)}
\gppoint{gp mark 11}{(6.823,1.512)}
\gppoint{gp mark 11}{(6.872,1.616)}
\gppoint{gp mark 11}{(7.269,2.163)}
\gppoint{gp mark 11}{(7.550,2.738)}
\gppoint{gp mark 11}{(7.769,3.503)}
\gppoint{gp mark 11}{(8.063,6.714)}
\gpcolor{gp lt color 2}
\draw[gp path] (3.140,1.003)--(3.360,0.996)--(3.504,0.997)--(3.941,1.003)--(4.114,1.018)%
  --(4.619,1.022)--(4.667,1.033)--(5.413,1.092)--(5.434,1.105)--(5.787,1.210)--(6.194,1.265)%
  --(6.287,1.274)--(6.823,1.572)--(6.872,1.674)--(7.269,2.275)--(7.550,2.909)--(7.769,3.764)%
  --(8.063,7.616);
\gppoint{gp mark 12}{(3.140,1.003)}
\gppoint{gp mark 12}{(3.360,0.996)}
\gppoint{gp mark 12}{(3.504,0.997)}
\gppoint{gp mark 12}{(3.941,1.003)}
\gppoint{gp mark 12}{(4.114,1.018)}
\gppoint{gp mark 12}{(4.619,1.022)}
\gppoint{gp mark 12}{(4.667,1.033)}
\gppoint{gp mark 12}{(5.413,1.092)}
\gppoint{gp mark 12}{(5.434,1.105)}
\gppoint{gp mark 12}{(5.787,1.210)}
\gppoint{gp mark 12}{(6.194,1.265)}
\gppoint{gp mark 12}{(6.287,1.274)}
\gppoint{gp mark 12}{(6.823,1.572)}
\gppoint{gp mark 12}{(6.872,1.674)}
\gppoint{gp mark 12}{(7.269,2.275)}
\gppoint{gp mark 12}{(7.550,2.909)}
\gppoint{gp mark 12}{(7.769,3.764)}
\gppoint{gp mark 12}{(8.063,7.616)}
\gpcolor{gp lt color 3}
\draw[gp path] (3.140,1.003)--(3.360,0.996)--(3.504,0.997)--(3.941,1.003)--(4.114,1.018)%
  --(4.619,1.022)--(4.667,1.033)--(5.413,1.091)--(5.434,1.102)--(5.787,1.210)--(6.194,1.265)%
  --(6.287,1.277)--(6.823,1.631)--(6.872,1.731)--(7.269,2.366);
\gppoint{gp mark 13}{(3.140,1.003)}
\gppoint{gp mark 13}{(3.360,0.996)}
\gppoint{gp mark 13}{(3.504,0.997)}
\gppoint{gp mark 13}{(3.941,1.003)}
\gppoint{gp mark 13}{(4.114,1.018)}
\gppoint{gp mark 13}{(4.619,1.022)}
\gppoint{gp mark 13}{(4.667,1.033)}
\gppoint{gp mark 13}{(5.413,1.091)}
\gppoint{gp mark 13}{(5.434,1.102)}
\gppoint{gp mark 13}{(5.787,1.210)}
\gppoint{gp mark 13}{(6.194,1.265)}
\gppoint{gp mark 13}{(6.287,1.277)}
\gppoint{gp mark 13}{(6.823,1.631)}
\gppoint{gp mark 13}{(6.872,1.731)}
\gppoint{gp mark 13}{(7.269,2.366)}
\gpcolor{gp lt color 4}
\draw[gp path] (3.140,1.003)--(3.360,0.996)--(3.504,0.997)--(3.941,1.003)--(4.114,1.018)%
  --(4.619,1.022)--(4.667,1.033)--(5.413,1.091)--(5.434,1.102)--(5.787,1.209)--(6.194,1.266)%
  --(6.287,1.282)--(6.823,1.631)--(6.872,1.741)--(7.269,2.365);
\gppoint{gp mark 14}{(3.140,1.003)}
\gppoint{gp mark 14}{(3.360,0.996)}
\gppoint{gp mark 14}{(3.504,0.997)}
\gppoint{gp mark 14}{(3.941,1.003)}
\gppoint{gp mark 14}{(4.114,1.018)}
\gppoint{gp mark 14}{(4.619,1.022)}
\gppoint{gp mark 14}{(4.667,1.033)}
\gppoint{gp mark 14}{(5.413,1.091)}
\gppoint{gp mark 14}{(5.434,1.102)}
\gppoint{gp mark 14}{(5.787,1.209)}
\gppoint{gp mark 14}{(6.194,1.266)}
\gppoint{gp mark 14}{(6.287,1.282)}
\gppoint{gp mark 14}{(6.823,1.631)}
\gppoint{gp mark 14}{(6.872,1.741)}
\gppoint{gp mark 14}{(7.269,2.365)}
\gpcolor{gp lt color border}
\gpsetlinetype{gp lt border}
\draw[gp path] (1.688,8.631)--(1.688,0.985)--(8.447,0.985)--(8.447,8.631)--cycle;
%% coordinates of the plot area
\gpdefrectangularnode{gp plot 1}{\pgfpoint{1.688cm}{0.985cm}}{\pgfpoint{8.447cm}{8.631cm}}
\end{tikzpicture}
%% gnuplot variables

        \caption[Total run time complexity with various block size
            implementations (linear)]{Total run time complexity (for the
            \command{TopN_Outlier_Pruning_Block} function) with various block
            size implementations}
        \label{profiling:blockSize:totalRunTimeComplexity:linear}
    \end{minipage}
    \begin{minipage}{\textwidth}
        \centering
        \begin{tikzpicture}[gnuplot]
%% generated with GNUPLOT 4.4p3 (Lua 5.1.4; terminal rev. 97, script rev. 96a)
%% Wed 24 Oct 2012 11:33:53 EST
\gpcolor{\gprgb{502}{502}{502}}
\gpsetlinetype{gp lt plot 0}
\gpsetlinewidth{1.00}
\draw[gp path] (1.872,0.985)--(2.052,0.985);
\draw[gp path] (8.447,0.985)--(8.267,0.985);
\node[gp node right] at (1.688,0.985) { 1};
\draw[gp path] (1.872,1.560)--(1.962,1.560);
\draw[gp path] (8.447,1.560)--(8.357,1.560);
\draw[gp path] (1.872,1.897)--(1.962,1.897);
\draw[gp path] (8.447,1.897)--(8.357,1.897);
\draw[gp path] (1.872,2.136)--(1.962,2.136);
\draw[gp path] (8.447,2.136)--(8.357,2.136);
\draw[gp path] (1.872,2.321)--(1.962,2.321);
\draw[gp path] (8.447,2.321)--(8.357,2.321);
\draw[gp path] (1.872,2.472)--(1.962,2.472);
\draw[gp path] (8.447,2.472)--(8.357,2.472);
\draw[gp path] (1.872,2.600)--(1.962,2.600);
\draw[gp path] (8.447,2.600)--(8.357,2.600);
\draw[gp path] (1.872,2.711)--(1.962,2.711);
\draw[gp path] (8.447,2.711)--(8.357,2.711);
\draw[gp path] (1.872,2.809)--(1.962,2.809);
\draw[gp path] (8.447,2.809)--(8.357,2.809);
\draw[gp path] (1.872,2.897)--(2.052,2.897);
\draw[gp path] (8.447,2.897)--(8.267,2.897);
\node[gp node right] at (1.688,2.897) { 10};
\draw[gp path] (1.872,3.472)--(1.962,3.472);
\draw[gp path] (8.447,3.472)--(8.357,3.472);
\draw[gp path] (1.872,3.809)--(1.962,3.809);
\draw[gp path] (8.447,3.809)--(8.357,3.809);
\draw[gp path] (1.872,4.047)--(1.962,4.047);
\draw[gp path] (8.447,4.047)--(8.357,4.047);
\draw[gp path] (1.872,4.233)--(1.962,4.233);
\draw[gp path] (8.447,4.233)--(8.357,4.233);
\draw[gp path] (1.872,4.384)--(1.962,4.384);
\draw[gp path] (8.447,4.384)--(8.357,4.384);
\draw[gp path] (1.872,4.512)--(1.962,4.512);
\draw[gp path] (8.447,4.512)--(8.357,4.512);
\draw[gp path] (1.872,4.623)--(1.962,4.623);
\draw[gp path] (8.447,4.623)--(8.357,4.623);
\draw[gp path] (1.872,4.721)--(1.962,4.721);
\draw[gp path] (8.447,4.721)--(8.357,4.721);
\draw[gp path] (1.872,4.808)--(2.052,4.808);
\draw[gp path] (8.447,4.808)--(8.267,4.808);
\node[gp node right] at (1.688,4.808) { 100};
\draw[gp path] (1.872,5.383)--(1.962,5.383);
\draw[gp path] (8.447,5.383)--(8.357,5.383);
\draw[gp path] (1.872,5.720)--(1.962,5.720);
\draw[gp path] (8.447,5.720)--(8.357,5.720);
\draw[gp path] (1.872,5.959)--(1.962,5.959);
\draw[gp path] (8.447,5.959)--(8.357,5.959);
\draw[gp path] (1.872,6.144)--(1.962,6.144);
\draw[gp path] (8.447,6.144)--(8.357,6.144);
\draw[gp path] (1.872,6.295)--(1.962,6.295);
\draw[gp path] (8.447,6.295)--(8.357,6.295);
\draw[gp path] (1.872,6.423)--(1.962,6.423);
\draw[gp path] (8.447,6.423)--(8.357,6.423);
\draw[gp path] (1.872,6.534)--(1.962,6.534);
\draw[gp path] (8.447,6.534)--(8.357,6.534);
\draw[gp path] (1.872,6.632)--(1.962,6.632);
\draw[gp path] (8.447,6.632)--(8.357,6.632);
\draw[gp path] (1.872,6.720)--(2.052,6.720);
\draw[gp path] (8.447,6.720)--(8.267,6.720);
\node[gp node right] at (1.688,6.720) { 1000};
\draw[gp path] (1.872,7.295)--(1.962,7.295);
\draw[gp path] (8.447,7.295)--(8.357,7.295);
\draw[gp path] (1.872,7.632)--(1.962,7.632);
\draw[gp path] (8.447,7.632)--(8.357,7.632);
\draw[gp path] (1.872,7.870)--(1.962,7.870);
\draw[gp path] (8.447,7.870)--(8.357,7.870);
\draw[gp path] (1.872,8.056)--(1.962,8.056);
\draw[gp path] (8.447,8.056)--(8.357,8.056);
\draw[gp path] (1.872,8.207)--(1.962,8.207);
\draw[gp path] (8.447,8.207)--(8.357,8.207);
\draw[gp path] (1.872,8.335)--(1.962,8.335);
\draw[gp path] (8.447,8.335)--(8.357,8.335);
\draw[gp path] (1.872,8.446)--(1.962,8.446);
\draw[gp path] (8.447,8.446)--(8.357,8.446);
\draw[gp path] (1.872,8.544)--(1.962,8.544);
\draw[gp path] (8.447,8.544)--(8.357,8.544);
\draw[gp path] (1.872,8.631)--(2.052,8.631);
\draw[gp path] (8.447,8.631)--(8.267,8.631);
\node[gp node right] at (1.688,8.631) { 10000};
\draw[gp path] (1.872,0.985)--(1.872,1.165);
\draw[gp path] (1.872,8.631)--(1.872,8.451);
\node[gp node center] at (1.872,0.677) {$10^{2}$};
\draw[gp path] (2.532,0.985)--(2.532,1.075);
\draw[gp path] (2.532,8.631)--(2.532,8.541);
\draw[gp path] (2.918,0.985)--(2.918,1.075);
\draw[gp path] (2.918,8.631)--(2.918,8.541);
\draw[gp path] (3.192,0.985)--(3.192,1.075);
\draw[gp path] (3.192,8.631)--(3.192,8.541);
\draw[gp path] (3.404,0.985)--(3.404,1.075);
\draw[gp path] (3.404,8.631)--(3.404,8.541);
\draw[gp path] (3.577,0.985)--(3.577,1.075);
\draw[gp path] (3.577,8.631)--(3.577,8.541);
\draw[gp path] (3.724,0.985)--(3.724,1.075);
\draw[gp path] (3.724,8.631)--(3.724,8.541);
\draw[gp path] (3.851,0.985)--(3.851,1.075);
\draw[gp path] (3.851,8.631)--(3.851,8.541);
\draw[gp path] (3.963,0.985)--(3.963,1.075);
\draw[gp path] (3.963,8.631)--(3.963,8.541);
\draw[gp path] (4.064,0.985)--(4.064,1.165);
\draw[gp path] (4.064,8.631)--(4.064,8.451);
\node[gp node center] at (4.064,0.677) {$10^{3}$};
\draw[gp path] (4.723,0.985)--(4.723,1.075);
\draw[gp path] (4.723,8.631)--(4.723,8.541);
\draw[gp path] (5.109,0.985)--(5.109,1.075);
\draw[gp path] (5.109,8.631)--(5.109,8.541);
\draw[gp path] (5.383,0.985)--(5.383,1.075);
\draw[gp path] (5.383,8.631)--(5.383,8.541);
\draw[gp path] (5.596,0.985)--(5.596,1.075);
\draw[gp path] (5.596,8.631)--(5.596,8.541);
\draw[gp path] (5.769,0.985)--(5.769,1.075);
\draw[gp path] (5.769,8.631)--(5.769,8.541);
\draw[gp path] (5.916,0.985)--(5.916,1.075);
\draw[gp path] (5.916,8.631)--(5.916,8.541);
\draw[gp path] (6.043,0.985)--(6.043,1.075);
\draw[gp path] (6.043,8.631)--(6.043,8.541);
\draw[gp path] (6.155,0.985)--(6.155,1.075);
\draw[gp path] (6.155,8.631)--(6.155,8.541);
\draw[gp path] (6.255,0.985)--(6.255,1.165);
\draw[gp path] (6.255,8.631)--(6.255,8.451);
\node[gp node center] at (6.255,0.677) {$10^{4}$};
\draw[gp path] (6.915,0.985)--(6.915,1.075);
\draw[gp path] (6.915,8.631)--(6.915,8.541);
\draw[gp path] (7.301,0.985)--(7.301,1.075);
\draw[gp path] (7.301,8.631)--(7.301,8.541);
\draw[gp path] (7.575,0.985)--(7.575,1.075);
\draw[gp path] (7.575,8.631)--(7.575,8.541);
\draw[gp path] (7.787,0.985)--(7.787,1.075);
\draw[gp path] (7.787,8.631)--(7.787,8.541);
\draw[gp path] (7.961,0.985)--(7.961,1.075);
\draw[gp path] (7.961,8.631)--(7.961,8.541);
\draw[gp path] (8.108,0.985)--(8.108,1.075);
\draw[gp path] (8.108,8.631)--(8.108,8.541);
\draw[gp path] (8.235,0.985)--(8.235,1.075);
\draw[gp path] (8.235,8.631)--(8.235,8.541);
\draw[gp path] (8.347,0.985)--(8.347,1.075);
\draw[gp path] (8.347,8.631)--(8.347,8.541);
\draw[gp path] (8.447,0.985)--(8.447,1.165);
\draw[gp path] (8.447,8.631)--(8.447,8.451);
\node[gp node center] at (8.447,0.677) {$10^{5}$};
\draw[gp path] (1.872,8.631)--(1.872,0.985)--(8.447,0.985);
\gpcolor{gp lt color border}
\node[gp node center,rotate=-270] at (0.246,4.808) {\textbf{Total execution time}};
\node[gp node center] at (5.159,0.215) {\textbf{Problem size}};
\gpcolor{gp lt color axes}
\draw[gp path] (3.284,3.345)--(3.498,2.871)--(3.639,2.965)--(4.064,3.275)--(4.232,3.799)%
  --(4.723,3.873)--(4.770,4.103)--(5.496,4.761)--(5.516,4.817)--(5.860,5.347)--(6.255,5.506)%
  --(6.345,5.484)--(6.867,6.087)--(6.915,6.250)--(7.301,6.750)--(7.575,7.075)--(7.787,7.404)%
  --(8.074,8.131);
\gpsetpointsize{4.00}
\gppoint{gp mark 1}{(3.284,3.345)}
\gppoint{gp mark 1}{(3.498,2.871)}
\gppoint{gp mark 1}{(3.639,2.965)}
\gppoint{gp mark 1}{(4.064,3.275)}
\gppoint{gp mark 1}{(4.232,3.799)}
\gppoint{gp mark 1}{(4.723,3.873)}
\gppoint{gp mark 1}{(4.770,4.103)}
\gppoint{gp mark 1}{(5.496,4.761)}
\gppoint{gp mark 1}{(5.516,4.817)}
\gppoint{gp mark 1}{(5.860,5.347)}
\gppoint{gp mark 1}{(6.255,5.506)}
\gppoint{gp mark 1}{(6.345,5.484)}
\gppoint{gp mark 1}{(6.867,6.087)}
\gppoint{gp mark 1}{(6.915,6.250)}
\gppoint{gp mark 1}{(7.301,6.750)}
\gppoint{gp mark 1}{(7.575,7.075)}
\gppoint{gp mark 1}{(7.787,7.404)}
\gppoint{gp mark 1}{(8.074,8.131)}
\gpcolor{gp lt color 0}
\draw[gp path] (3.284,3.314)--(3.498,2.871)--(3.639,2.965)--(4.064,3.276)--(4.232,3.802)%
  --(4.723,3.873)--(4.770,4.096)--(5.496,4.726)--(5.516,4.814)--(5.860,5.369)--(6.255,5.513)%
  --(6.345,5.488)--(6.867,6.113)--(6.915,6.252)--(7.301,6.802)--(7.575,7.109)--(7.787,7.420)%
  --(8.074,8.111);
\gppoint{gp mark 2}{(3.284,3.314)}
\gppoint{gp mark 2}{(3.498,2.871)}
\gppoint{gp mark 2}{(3.639,2.965)}
\gppoint{gp mark 2}{(4.064,3.276)}
\gppoint{gp mark 2}{(4.232,3.802)}
\gppoint{gp mark 2}{(4.723,3.873)}
\gppoint{gp mark 2}{(4.770,4.096)}
\gppoint{gp mark 2}{(5.496,4.726)}
\gppoint{gp mark 2}{(5.516,4.814)}
\gppoint{gp mark 2}{(5.860,5.369)}
\gppoint{gp mark 2}{(6.255,5.513)}
\gppoint{gp mark 2}{(6.345,5.488)}
\gppoint{gp mark 2}{(6.867,6.113)}
\gppoint{gp mark 2}{(6.915,6.252)}
\gppoint{gp mark 2}{(7.301,6.802)}
\gppoint{gp mark 2}{(7.575,7.109)}
\gppoint{gp mark 2}{(7.787,7.420)}
\gppoint{gp mark 2}{(8.074,8.111)}
\gpcolor{gp lt color 1}
\draw[gp path] (3.284,3.299)--(3.498,2.870)--(3.639,2.965)--(4.064,3.276)--(4.232,3.804)%
  --(4.723,3.874)--(4.770,4.092)--(5.496,4.732)--(5.516,4.812)--(5.860,5.348)--(6.255,5.507)%
  --(6.345,5.507)--(6.867,6.115)--(6.915,6.242)--(7.301,6.777)--(7.575,7.104)--(7.787,7.403)%
  --(8.074,8.085);
\gppoint{gp mark 3}{(3.284,3.299)}
\gppoint{gp mark 3}{(3.498,2.870)}
\gppoint{gp mark 3}{(3.639,2.965)}
\gppoint{gp mark 3}{(4.064,3.276)}
\gppoint{gp mark 3}{(4.232,3.804)}
\gppoint{gp mark 3}{(4.723,3.874)}
\gppoint{gp mark 3}{(4.770,4.092)}
\gppoint{gp mark 3}{(5.496,4.732)}
\gppoint{gp mark 3}{(5.516,4.812)}
\gppoint{gp mark 3}{(5.860,5.348)}
\gppoint{gp mark 3}{(6.255,5.507)}
\gppoint{gp mark 3}{(6.345,5.507)}
\gppoint{gp mark 3}{(6.867,6.115)}
\gppoint{gp mark 3}{(6.915,6.242)}
\gppoint{gp mark 3}{(7.301,6.777)}
\gppoint{gp mark 3}{(7.575,7.104)}
\gppoint{gp mark 3}{(7.787,7.403)}
\gppoint{gp mark 3}{(8.074,8.085)}
\gpcolor{gp lt color 2}
\draw[gp path] (3.284,3.296)--(3.498,2.873)--(3.639,2.967)--(4.064,3.277)--(4.232,3.808)%
  --(4.723,3.875)--(4.770,4.096)--(5.496,4.741)--(5.516,4.813)--(5.860,5.346)--(6.255,5.507)%
  --(6.345,5.500)--(6.867,6.094)--(6.915,6.259)--(7.301,6.771)--(7.575,7.107)--(7.787,7.403)%
  --(8.074,8.080);
\gppoint{gp mark 4}{(3.284,3.296)}
\gppoint{gp mark 4}{(3.498,2.873)}
\gppoint{gp mark 4}{(3.639,2.967)}
\gppoint{gp mark 4}{(4.064,3.277)}
\gppoint{gp mark 4}{(4.232,3.808)}
\gppoint{gp mark 4}{(4.723,3.875)}
\gppoint{gp mark 4}{(4.770,4.096)}
\gppoint{gp mark 4}{(5.496,4.741)}
\gppoint{gp mark 4}{(5.516,4.813)}
\gppoint{gp mark 4}{(5.860,5.346)}
\gppoint{gp mark 4}{(6.255,5.507)}
\gppoint{gp mark 4}{(6.345,5.500)}
\gppoint{gp mark 4}{(6.867,6.094)}
\gppoint{gp mark 4}{(6.915,6.259)}
\gppoint{gp mark 4}{(7.301,6.771)}
\gppoint{gp mark 4}{(7.575,7.107)}
\gppoint{gp mark 4}{(7.787,7.403)}
\gppoint{gp mark 4}{(8.074,8.080)}
\gpcolor{gp lt color 3}
\draw[gp path] (3.284,3.299)--(3.498,2.873)--(3.639,2.968)--(4.064,3.276)--(4.232,3.805)%
  --(4.723,3.876)--(4.770,4.094)--(5.496,4.737)--(5.516,4.827)--(5.860,5.343)--(6.255,5.506)%
  --(6.345,5.501)--(6.867,6.107)--(6.915,6.260)--(7.301,6.773)--(7.575,7.107)--(7.787,7.405)%
  --(8.074,8.105);
\gppoint{gp mark 5}{(3.284,3.299)}
\gppoint{gp mark 5}{(3.498,2.873)}
\gppoint{gp mark 5}{(3.639,2.968)}
\gppoint{gp mark 5}{(4.064,3.276)}
\gppoint{gp mark 5}{(4.232,3.805)}
\gppoint{gp mark 5}{(4.723,3.876)}
\gppoint{gp mark 5}{(4.770,4.094)}
\gppoint{gp mark 5}{(5.496,4.737)}
\gppoint{gp mark 5}{(5.516,4.827)}
\gppoint{gp mark 5}{(5.860,5.343)}
\gppoint{gp mark 5}{(6.255,5.506)}
\gppoint{gp mark 5}{(6.345,5.501)}
\gppoint{gp mark 5}{(6.867,6.107)}
\gppoint{gp mark 5}{(6.915,6.260)}
\gppoint{gp mark 5}{(7.301,6.773)}
\gppoint{gp mark 5}{(7.575,7.107)}
\gppoint{gp mark 5}{(7.787,7.405)}
\gppoint{gp mark 5}{(8.074,8.105)}
\gpcolor{gp lt color 4}
\draw[gp path] (3.284,3.295)--(3.498,2.874)--(3.639,2.968)--(4.064,3.280)--(4.232,3.804)%
  --(4.723,3.879)--(4.770,4.097)--(5.496,4.745)--(5.516,4.871)--(5.860,5.356)--(6.255,5.506)%
  --(6.345,5.498)--(6.867,6.113)--(6.915,6.250)--(7.301,6.768)--(7.575,7.112)--(7.787,7.409)%
  --(8.074,8.106);
\gppoint{gp mark 6}{(3.284,3.295)}
\gppoint{gp mark 6}{(3.498,2.874)}
\gppoint{gp mark 6}{(3.639,2.968)}
\gppoint{gp mark 6}{(4.064,3.280)}
\gppoint{gp mark 6}{(4.232,3.804)}
\gppoint{gp mark 6}{(4.723,3.879)}
\gppoint{gp mark 6}{(4.770,4.097)}
\gppoint{gp mark 6}{(5.496,4.745)}
\gppoint{gp mark 6}{(5.516,4.871)}
\gppoint{gp mark 6}{(5.860,5.356)}
\gppoint{gp mark 6}{(6.255,5.506)}
\gppoint{gp mark 6}{(6.345,5.498)}
\gppoint{gp mark 6}{(6.867,6.113)}
\gppoint{gp mark 6}{(6.915,6.250)}
\gppoint{gp mark 6}{(7.301,6.768)}
\gppoint{gp mark 6}{(7.575,7.112)}
\gppoint{gp mark 6}{(7.787,7.409)}
\gppoint{gp mark 6}{(8.074,8.106)}
\gpcolor{gp lt color 5}
\draw[gp path] (3.284,3.298)--(3.498,2.873)--(3.639,2.970)--(4.064,3.282)--(4.232,3.807)%
  --(4.723,3.880)--(4.770,4.097)--(5.496,4.749)--(5.516,4.812)--(5.860,5.359)--(6.255,5.509)%
  --(6.345,5.514)--(6.867,6.122)--(6.915,6.243)--(7.301,6.776)--(7.575,7.106)--(7.787,7.414)%
  --(8.074,8.110);
\gppoint{gp mark 7}{(3.284,3.298)}
\gppoint{gp mark 7}{(3.498,2.873)}
\gppoint{gp mark 7}{(3.639,2.970)}
\gppoint{gp mark 7}{(4.064,3.282)}
\gppoint{gp mark 7}{(4.232,3.807)}
\gppoint{gp mark 7}{(4.723,3.880)}
\gppoint{gp mark 7}{(4.770,4.097)}
\gppoint{gp mark 7}{(5.496,4.749)}
\gppoint{gp mark 7}{(5.516,4.812)}
\gppoint{gp mark 7}{(5.860,5.359)}
\gppoint{gp mark 7}{(6.255,5.509)}
\gppoint{gp mark 7}{(6.345,5.514)}
\gppoint{gp mark 7}{(6.867,6.122)}
\gppoint{gp mark 7}{(6.915,6.243)}
\gppoint{gp mark 7}{(7.301,6.776)}
\gppoint{gp mark 7}{(7.575,7.106)}
\gppoint{gp mark 7}{(7.787,7.414)}
\gppoint{gp mark 7}{(8.074,8.110)}
\gpcolor{gp lt color 6}
\draw[gp path] (3.284,3.297)--(3.498,2.874)--(3.639,2.970)--(4.064,3.283)--(4.232,3.808)%
  --(4.723,3.878)--(4.770,4.096)--(5.496,4.741)--(5.516,4.824)--(5.860,5.352)--(6.255,5.509)%
  --(6.345,5.511)--(6.867,6.111)--(6.915,6.253)--(7.301,6.771)--(7.575,7.106)--(7.787,7.410)%
  --(8.074,8.170);
\gppoint{gp mark 8}{(3.284,3.297)}
\gppoint{gp mark 8}{(3.498,2.874)}
\gppoint{gp mark 8}{(3.639,2.970)}
\gppoint{gp mark 8}{(4.064,3.283)}
\gppoint{gp mark 8}{(4.232,3.808)}
\gppoint{gp mark 8}{(4.723,3.878)}
\gppoint{gp mark 8}{(4.770,4.096)}
\gppoint{gp mark 8}{(5.496,4.741)}
\gppoint{gp mark 8}{(5.516,4.824)}
\gppoint{gp mark 8}{(5.860,5.352)}
\gppoint{gp mark 8}{(6.255,5.509)}
\gppoint{gp mark 8}{(6.345,5.511)}
\gppoint{gp mark 8}{(6.867,6.111)}
\gppoint{gp mark 8}{(6.915,6.253)}
\gppoint{gp mark 8}{(7.301,6.771)}
\gppoint{gp mark 8}{(7.575,7.106)}
\gppoint{gp mark 8}{(7.787,7.410)}
\gppoint{gp mark 8}{(8.074,8.170)}
\gpcolor{gp lt color 7}
\draw[gp path] (3.284,3.298)--(3.498,2.876)--(3.639,2.972)--(4.064,3.282)--(4.232,3.804)%
  --(4.723,3.879)--(4.770,4.094)--(5.496,4.755)--(5.516,4.813)--(5.860,5.356)--(6.255,5.511)%
  --(6.345,5.514)--(6.867,6.108)--(6.915,6.259)--(7.301,6.774)--(7.575,7.105)--(7.787,7.414)%
  --(8.074,8.083);
\gppoint{gp mark 9}{(3.284,3.298)}
\gppoint{gp mark 9}{(3.498,2.876)}
\gppoint{gp mark 9}{(3.639,2.972)}
\gppoint{gp mark 9}{(4.064,3.282)}
\gppoint{gp mark 9}{(4.232,3.804)}
\gppoint{gp mark 9}{(4.723,3.879)}
\gppoint{gp mark 9}{(4.770,4.094)}
\gppoint{gp mark 9}{(5.496,4.755)}
\gppoint{gp mark 9}{(5.516,4.813)}
\gppoint{gp mark 9}{(5.860,5.356)}
\gppoint{gp mark 9}{(6.255,5.511)}
\gppoint{gp mark 9}{(6.345,5.514)}
\gppoint{gp mark 9}{(6.867,6.108)}
\gppoint{gp mark 9}{(6.915,6.259)}
\gppoint{gp mark 9}{(7.301,6.774)}
\gppoint{gp mark 9}{(7.575,7.105)}
\gppoint{gp mark 9}{(7.787,7.414)}
\gppoint{gp mark 9}{(8.074,8.083)}
\gpcolor{gp lt color 0}
\draw[gp path] (3.284,3.299)--(3.498,2.876)--(3.639,2.972)--(4.064,3.283)--(4.232,3.810)%
  --(4.723,3.880)--(4.770,4.100)--(5.496,4.753)--(5.516,4.816)--(5.860,5.358)--(6.255,5.511)%
  --(6.345,5.497)--(6.867,6.102)--(6.915,6.252)--(7.301,6.775)--(7.575,7.107)--(7.787,7.404)%
  --(8.074,8.170);
\gppoint{gp mark 10}{(3.284,3.299)}
\gppoint{gp mark 10}{(3.498,2.876)}
\gppoint{gp mark 10}{(3.639,2.972)}
\gppoint{gp mark 10}{(4.064,3.283)}
\gppoint{gp mark 10}{(4.232,3.810)}
\gppoint{gp mark 10}{(4.723,3.880)}
\gppoint{gp mark 10}{(4.770,4.100)}
\gppoint{gp mark 10}{(5.496,4.753)}
\gppoint{gp mark 10}{(5.516,4.816)}
\gppoint{gp mark 10}{(5.860,5.358)}
\gppoint{gp mark 10}{(6.255,5.511)}
\gppoint{gp mark 10}{(6.345,5.497)}
\gppoint{gp mark 10}{(6.867,6.102)}
\gppoint{gp mark 10}{(6.915,6.252)}
\gppoint{gp mark 10}{(7.301,6.775)}
\gppoint{gp mark 10}{(7.575,7.107)}
\gppoint{gp mark 10}{(7.787,7.404)}
\gppoint{gp mark 10}{(8.074,8.170)}
\gpcolor{gp lt color 1}
\draw[gp path] (3.284,3.301)--(3.498,2.880)--(3.639,2.977)--(4.064,3.291)--(4.232,3.813)%
  --(4.723,3.890)--(4.770,4.110)--(5.496,4.758)--(5.516,4.822)--(5.860,5.391)--(6.255,5.522)%
  --(6.345,5.504)--(6.867,6.115)--(6.915,6.263)--(7.301,6.782)--(7.575,7.112)--(7.787,7.413)%
  --(8.074,8.095);
\gppoint{gp mark 11}{(3.284,3.301)}
\gppoint{gp mark 11}{(3.498,2.880)}
\gppoint{gp mark 11}{(3.639,2.977)}
\gppoint{gp mark 11}{(4.064,3.291)}
\gppoint{gp mark 11}{(4.232,3.813)}
\gppoint{gp mark 11}{(4.723,3.890)}
\gppoint{gp mark 11}{(4.770,4.110)}
\gppoint{gp mark 11}{(5.496,4.758)}
\gppoint{gp mark 11}{(5.516,4.822)}
\gppoint{gp mark 11}{(5.860,5.391)}
\gppoint{gp mark 11}{(6.255,5.522)}
\gppoint{gp mark 11}{(6.345,5.504)}
\gppoint{gp mark 11}{(6.867,6.115)}
\gppoint{gp mark 11}{(6.915,6.263)}
\gppoint{gp mark 11}{(7.301,6.782)}
\gppoint{gp mark 11}{(7.575,7.112)}
\gppoint{gp mark 11}{(7.787,7.413)}
\gppoint{gp mark 11}{(8.074,8.095)}
\gpcolor{gp lt color 2}
\draw[gp path] (3.284,3.305)--(3.498,2.880)--(3.639,2.976)--(4.064,3.292)--(4.232,3.803)%
  --(4.723,3.903)--(4.770,4.119)--(5.496,4.791)--(5.516,4.883)--(5.860,5.408)--(6.255,5.589)%
  --(6.345,5.615)--(6.867,6.204)--(6.915,6.336)--(7.301,6.857)--(7.575,7.189)--(7.787,7.495)%
  --(8.074,8.217);
\gppoint{gp mark 12}{(3.284,3.305)}
\gppoint{gp mark 12}{(3.498,2.880)}
\gppoint{gp mark 12}{(3.639,2.976)}
\gppoint{gp mark 12}{(4.064,3.292)}
\gppoint{gp mark 12}{(4.232,3.803)}
\gppoint{gp mark 12}{(4.723,3.903)}
\gppoint{gp mark 12}{(4.770,4.119)}
\gppoint{gp mark 12}{(5.496,4.791)}
\gppoint{gp mark 12}{(5.516,4.883)}
\gppoint{gp mark 12}{(5.860,5.408)}
\gppoint{gp mark 12}{(6.255,5.589)}
\gppoint{gp mark 12}{(6.345,5.615)}
\gppoint{gp mark 12}{(6.867,6.204)}
\gppoint{gp mark 12}{(6.915,6.336)}
\gppoint{gp mark 12}{(7.301,6.857)}
\gppoint{gp mark 12}{(7.575,7.189)}
\gppoint{gp mark 12}{(7.787,7.495)}
\gppoint{gp mark 12}{(8.074,8.217)}
\gpcolor{gp lt color 3}
\draw[gp path] (3.284,3.303)--(3.498,2.883)--(3.639,2.977)--(4.064,3.293)--(4.232,3.805)%
  --(4.723,3.900)--(4.770,4.123)--(5.496,4.782)--(5.516,4.868)--(5.860,5.408)--(6.255,5.589)%
  --(6.345,5.624)--(6.867,6.284)--(6.915,6.403)--(7.301,6.914);
\gppoint{gp mark 13}{(3.284,3.303)}
\gppoint{gp mark 13}{(3.498,2.883)}
\gppoint{gp mark 13}{(3.639,2.977)}
\gppoint{gp mark 13}{(4.064,3.293)}
\gppoint{gp mark 13}{(4.232,3.805)}
\gppoint{gp mark 13}{(4.723,3.900)}
\gppoint{gp mark 13}{(4.770,4.123)}
\gppoint{gp mark 13}{(5.496,4.782)}
\gppoint{gp mark 13}{(5.516,4.868)}
\gppoint{gp mark 13}{(5.860,5.408)}
\gppoint{gp mark 13}{(6.255,5.589)}
\gppoint{gp mark 13}{(6.345,5.624)}
\gppoint{gp mark 13}{(6.867,6.284)}
\gppoint{gp mark 13}{(6.915,6.403)}
\gppoint{gp mark 13}{(7.301,6.914)}
\gpcolor{gp lt color 4}
\draw[gp path] (3.284,3.302)--(3.498,2.883)--(3.639,2.978)--(4.064,3.293)--(4.232,3.812)%
  --(4.723,3.900)--(4.770,4.122)--(5.496,4.781)--(5.516,4.867)--(5.860,5.406)--(6.255,5.591)%
  --(6.345,5.637)--(6.867,6.284)--(6.915,6.414)--(7.301,6.913);
\gppoint{gp mark 14}{(3.284,3.302)}
\gppoint{gp mark 14}{(3.498,2.883)}
\gppoint{gp mark 14}{(3.639,2.978)}
\gppoint{gp mark 14}{(4.064,3.293)}
\gppoint{gp mark 14}{(4.232,3.812)}
\gppoint{gp mark 14}{(4.723,3.900)}
\gppoint{gp mark 14}{(4.770,4.122)}
\gppoint{gp mark 14}{(5.496,4.781)}
\gppoint{gp mark 14}{(5.516,4.867)}
\gppoint{gp mark 14}{(5.860,5.406)}
\gppoint{gp mark 14}{(6.255,5.591)}
\gppoint{gp mark 14}{(6.345,5.637)}
\gppoint{gp mark 14}{(6.867,6.284)}
\gppoint{gp mark 14}{(6.915,6.414)}
\gppoint{gp mark 14}{(7.301,6.913)}
%% coordinates of the plot area
\gpdefrectangularnode{gp plot 1}{\pgfpoint{1.872cm}{0.985cm}}{\pgfpoint{8.447cm}{8.631cm}}
\end{tikzpicture}
%% gnuplot variables

        \caption[Total run time complexity with various block size
            implementations (logarithmic)]{Total run time complexity (for the
            \command{TopN_Outlier_Pruning_Block} function) with various block
            size implementations}
        \label{profiling:blockSize:totalRunTimeComplexity:logarithmic}
    \end{minipage}
\end{figure}

\begin{figure}
    \centering
    \begin{minipage}{\textwidth}
        \centering
        \begin{tikzpicture}[gnuplot]
%% generated with GNUPLOT 4.4p3 (Lua 5.1.4; terminal rev. 97, script rev. 96a)
%% Wed 24 Oct 2012 11:48:43 EST
\gpcolor{gp lt color border}
\gpsetlinetype{gp lt border}
\gpsetlinewidth{1.00}
\draw[gp path] (1.504,0.985)--(1.684,0.985);
\draw[gp path] (8.447,0.985)--(8.267,0.985);
\node[gp node right] at (1.320,0.985) { 0};
\draw[gp path] (1.504,2.077)--(1.684,2.077);
\draw[gp path] (8.447,2.077)--(8.267,2.077);
\node[gp node right] at (1.320,2.077) { 50};
\draw[gp path] (1.504,3.170)--(1.684,3.170);
\draw[gp path] (8.447,3.170)--(8.267,3.170);
\node[gp node right] at (1.320,3.170) { 100};
\draw[gp path] (1.504,4.262)--(1.684,4.262);
\draw[gp path] (8.447,4.262)--(8.267,4.262);
\node[gp node right] at (1.320,4.262) { 150};
\draw[gp path] (1.504,5.354)--(1.684,5.354);
\draw[gp path] (8.447,5.354)--(8.267,5.354);
\node[gp node right] at (1.320,5.354) { 200};
\draw[gp path] (1.504,6.446)--(1.684,6.446);
\draw[gp path] (8.447,6.446)--(8.267,6.446);
\node[gp node right] at (1.320,6.446) { 250};
\draw[gp path] (1.504,7.539)--(1.684,7.539);
\draw[gp path] (8.447,7.539)--(8.267,7.539);
\node[gp node right] at (1.320,7.539) { 300};
\draw[gp path] (1.504,8.631)--(1.684,8.631);
\draw[gp path] (8.447,8.631)--(8.267,8.631);
\node[gp node right] at (1.320,8.631) { 350};
\draw[gp path] (1.504,0.985)--(1.504,1.165);
\draw[gp path] (1.504,8.631)--(1.504,8.451);
\node[gp node center] at (1.504,0.677) {$10^{2}$};
\draw[gp path] (2.201,0.985)--(2.201,1.075);
\draw[gp path] (2.201,8.631)--(2.201,8.541);
\draw[gp path] (2.608,0.985)--(2.608,1.075);
\draw[gp path] (2.608,8.631)--(2.608,8.541);
\draw[gp path] (2.897,0.985)--(2.897,1.075);
\draw[gp path] (2.897,8.631)--(2.897,8.541);
\draw[gp path] (3.122,0.985)--(3.122,1.075);
\draw[gp path] (3.122,8.631)--(3.122,8.541);
\draw[gp path] (3.305,0.985)--(3.305,1.075);
\draw[gp path] (3.305,8.631)--(3.305,8.541);
\draw[gp path] (3.460,0.985)--(3.460,1.075);
\draw[gp path] (3.460,8.631)--(3.460,8.541);
\draw[gp path] (3.594,0.985)--(3.594,1.075);
\draw[gp path] (3.594,8.631)--(3.594,8.541);
\draw[gp path] (3.712,0.985)--(3.712,1.075);
\draw[gp path] (3.712,8.631)--(3.712,8.541);
\draw[gp path] (3.818,0.985)--(3.818,1.165);
\draw[gp path] (3.818,8.631)--(3.818,8.451);
\node[gp node center] at (3.818,0.677) {$10^{3}$};
\draw[gp path] (4.515,0.985)--(4.515,1.075);
\draw[gp path] (4.515,8.631)--(4.515,8.541);
\draw[gp path] (4.923,0.985)--(4.923,1.075);
\draw[gp path] (4.923,8.631)--(4.923,8.541);
\draw[gp path] (5.212,0.985)--(5.212,1.075);
\draw[gp path] (5.212,8.631)--(5.212,8.541);
\draw[gp path] (5.436,0.985)--(5.436,1.075);
\draw[gp path] (5.436,8.631)--(5.436,8.541);
\draw[gp path] (5.619,0.985)--(5.619,1.075);
\draw[gp path] (5.619,8.631)--(5.619,8.541);
\draw[gp path] (5.774,0.985)--(5.774,1.075);
\draw[gp path] (5.774,8.631)--(5.774,8.541);
\draw[gp path] (5.908,0.985)--(5.908,1.075);
\draw[gp path] (5.908,8.631)--(5.908,8.541);
\draw[gp path] (6.027,0.985)--(6.027,1.075);
\draw[gp path] (6.027,8.631)--(6.027,8.541);
\draw[gp path] (6.133,0.985)--(6.133,1.165);
\draw[gp path] (6.133,8.631)--(6.133,8.451);
\node[gp node center] at (6.133,0.677) {$10^{4}$};
\draw[gp path] (6.829,0.985)--(6.829,1.075);
\draw[gp path] (6.829,8.631)--(6.829,8.541);
\draw[gp path] (7.237,0.985)--(7.237,1.075);
\draw[gp path] (7.237,8.631)--(7.237,8.541);
\draw[gp path] (7.526,0.985)--(7.526,1.075);
\draw[gp path] (7.526,8.631)--(7.526,8.541);
\draw[gp path] (7.750,0.985)--(7.750,1.075);
\draw[gp path] (7.750,8.631)--(7.750,8.541);
\draw[gp path] (7.934,0.985)--(7.934,1.075);
\draw[gp path] (7.934,8.631)--(7.934,8.541);
\draw[gp path] (8.089,0.985)--(8.089,1.075);
\draw[gp path] (8.089,8.631)--(8.089,8.541);
\draw[gp path] (8.223,0.985)--(8.223,1.075);
\draw[gp path] (8.223,8.631)--(8.223,8.541);
\draw[gp path] (8.341,0.985)--(8.341,1.075);
\draw[gp path] (8.341,8.631)--(8.341,8.541);
\draw[gp path] (8.447,0.985)--(8.447,1.165);
\draw[gp path] (8.447,8.631)--(8.447,8.451);
\node[gp node center] at (8.447,0.677) {$10^{5}$};
\draw[gp path] (1.504,8.631)--(1.504,0.985)--(8.447,0.985)--(8.447,8.631)--cycle;
\node[gp node center,rotate=-270] at (0.246,4.808) {\textbf{Function execution time}};
\node[gp node center] at (4.975,0.215) {\textbf{Problem size}};
\gpcolor{gp lt color axes}
\gpsetlinetype{gp lt plot 0}
\draw[gp path] (2.995,0.986)--(3.221,0.986)--(3.370,0.987)--(3.818,0.989)--(3.996,0.989)%
  --(4.515,0.997)--(4.564,0.992)--(5.330,0.996)--(5.352,1.003)--(5.715,1.041)--(6.133,1.243)%
  --(6.228,1.013)--(6.779,1.024)--(6.829,1.408)--(7.237,2.651)--(7.526,3.275)--(7.750,5.420)%
  --(8.053,1.391);
\gpsetpointsize{4.00}
\gppoint{gp mark 1}{(2.995,0.986)}
\gppoint{gp mark 1}{(3.221,0.986)}
\gppoint{gp mark 1}{(3.370,0.987)}
\gppoint{gp mark 1}{(3.818,0.989)}
\gppoint{gp mark 1}{(3.996,0.989)}
\gppoint{gp mark 1}{(4.515,0.997)}
\gppoint{gp mark 1}{(4.564,0.992)}
\gppoint{gp mark 1}{(5.330,0.996)}
\gppoint{gp mark 1}{(5.352,1.003)}
\gppoint{gp mark 1}{(5.715,1.041)}
\gppoint{gp mark 1}{(6.133,1.243)}
\gppoint{gp mark 1}{(6.228,1.013)}
\gppoint{gp mark 1}{(6.779,1.024)}
\gppoint{gp mark 1}{(6.829,1.408)}
\gppoint{gp mark 1}{(7.237,2.651)}
\gppoint{gp mark 1}{(7.526,3.275)}
\gppoint{gp mark 1}{(7.750,5.420)}
\gppoint{gp mark 1}{(8.053,1.391)}
\gpcolor{gp lt color 0}
\draw[gp path] (2.995,0.986)--(3.221,0.986)--(3.370,0.987)--(3.818,0.989)--(3.996,0.989)%
  --(4.515,0.997)--(4.564,0.992)--(5.330,0.998)--(5.352,1.003)--(5.715,1.040)--(6.133,1.256)%
  --(6.228,1.018)--(6.779,1.056)--(6.829,1.444)--(7.237,2.936)--(7.526,3.413)--(7.750,5.596)%
  --(8.053,1.646);
\gppoint{gp mark 2}{(2.995,0.986)}
\gppoint{gp mark 2}{(3.221,0.986)}
\gppoint{gp mark 2}{(3.370,0.987)}
\gppoint{gp mark 2}{(3.818,0.989)}
\gppoint{gp mark 2}{(3.996,0.989)}
\gppoint{gp mark 2}{(4.515,0.997)}
\gppoint{gp mark 2}{(4.564,0.992)}
\gppoint{gp mark 2}{(5.330,0.998)}
\gppoint{gp mark 2}{(5.352,1.003)}
\gppoint{gp mark 2}{(5.715,1.040)}
\gppoint{gp mark 2}{(6.133,1.256)}
\gppoint{gp mark 2}{(6.228,1.018)}
\gppoint{gp mark 2}{(6.779,1.056)}
\gppoint{gp mark 2}{(6.829,1.444)}
\gppoint{gp mark 2}{(7.237,2.936)}
\gppoint{gp mark 2}{(7.526,3.413)}
\gppoint{gp mark 2}{(7.750,5.596)}
\gppoint{gp mark 2}{(8.053,1.646)}
\gpcolor{gp lt color 1}
\draw[gp path] (2.995,0.986)--(3.221,0.986)--(3.370,0.987)--(3.818,0.989)--(3.996,0.989)%
  --(4.515,0.997)--(4.564,0.992)--(5.330,1.002)--(5.352,1.002)--(5.715,1.036)--(6.133,1.215)%
  --(6.228,1.020)--(6.779,1.042)--(6.829,1.380)--(7.237,2.524)--(7.526,3.091)--(7.750,5.000)%
  --(8.053,1.552);
\gppoint{gp mark 3}{(2.995,0.986)}
\gppoint{gp mark 3}{(3.221,0.986)}
\gppoint{gp mark 3}{(3.370,0.987)}
\gppoint{gp mark 3}{(3.818,0.989)}
\gppoint{gp mark 3}{(3.996,0.989)}
\gppoint{gp mark 3}{(4.515,0.997)}
\gppoint{gp mark 3}{(4.564,0.992)}
\gppoint{gp mark 3}{(5.330,1.002)}
\gppoint{gp mark 3}{(5.352,1.002)}
\gppoint{gp mark 3}{(5.715,1.036)}
\gppoint{gp mark 3}{(6.133,1.215)}
\gppoint{gp mark 3}{(6.228,1.020)}
\gppoint{gp mark 3}{(6.779,1.042)}
\gppoint{gp mark 3}{(6.829,1.380)}
\gppoint{gp mark 3}{(7.237,2.524)}
\gppoint{gp mark 3}{(7.526,3.091)}
\gppoint{gp mark 3}{(7.750,5.000)}
\gppoint{gp mark 3}{(8.053,1.552)}
\gpcolor{gp lt color 2}
\draw[gp path] (2.995,0.986)--(3.221,0.986)--(3.370,0.987)--(3.818,0.989)--(3.996,0.989)%
  --(4.515,0.997)--(4.564,0.992)--(5.330,0.998)--(5.352,0.995)--(5.715,1.043)--(6.133,1.221)%
  --(6.228,1.021)--(6.779,1.041)--(6.829,1.376)--(7.237,2.506)--(7.526,3.119)--(7.750,4.953)%
  --(8.053,1.545);
\gppoint{gp mark 4}{(2.995,0.986)}
\gppoint{gp mark 4}{(3.221,0.986)}
\gppoint{gp mark 4}{(3.370,0.987)}
\gppoint{gp mark 4}{(3.818,0.989)}
\gppoint{gp mark 4}{(3.996,0.989)}
\gppoint{gp mark 4}{(4.515,0.997)}
\gppoint{gp mark 4}{(4.564,0.992)}
\gppoint{gp mark 4}{(5.330,0.998)}
\gppoint{gp mark 4}{(5.352,0.995)}
\gppoint{gp mark 4}{(5.715,1.043)}
\gppoint{gp mark 4}{(6.133,1.221)}
\gppoint{gp mark 4}{(6.228,1.021)}
\gppoint{gp mark 4}{(6.779,1.041)}
\gppoint{gp mark 4}{(6.829,1.376)}
\gppoint{gp mark 4}{(7.237,2.506)}
\gppoint{gp mark 4}{(7.526,3.119)}
\gppoint{gp mark 4}{(7.750,4.953)}
\gppoint{gp mark 4}{(8.053,1.545)}
\gpcolor{gp lt color 3}
\draw[gp path] (2.995,0.986)--(3.221,0.986)--(3.370,0.987)--(3.818,0.989)--(3.996,0.989)%
  --(4.515,0.997)--(4.564,0.992)--(5.330,0.996)--(5.352,1.003)--(5.715,1.042)--(6.133,1.222)%
  --(6.228,1.018)--(6.779,1.044)--(6.829,1.376)--(7.237,2.499)--(7.526,3.094)--(7.750,4.952)%
  --(8.053,1.549);
\gppoint{gp mark 5}{(2.995,0.986)}
\gppoint{gp mark 5}{(3.221,0.986)}
\gppoint{gp mark 5}{(3.370,0.987)}
\gppoint{gp mark 5}{(3.818,0.989)}
\gppoint{gp mark 5}{(3.996,0.989)}
\gppoint{gp mark 5}{(4.515,0.997)}
\gppoint{gp mark 5}{(4.564,0.992)}
\gppoint{gp mark 5}{(5.330,0.996)}
\gppoint{gp mark 5}{(5.352,1.003)}
\gppoint{gp mark 5}{(5.715,1.042)}
\gppoint{gp mark 5}{(6.133,1.222)}
\gppoint{gp mark 5}{(6.228,1.018)}
\gppoint{gp mark 5}{(6.779,1.044)}
\gppoint{gp mark 5}{(6.829,1.376)}
\gppoint{gp mark 5}{(7.237,2.499)}
\gppoint{gp mark 5}{(7.526,3.094)}
\gppoint{gp mark 5}{(7.750,4.952)}
\gppoint{gp mark 5}{(8.053,1.549)}
\gpcolor{gp lt color 4}
\draw[gp path] (2.995,0.986)--(3.221,0.986)--(3.370,0.987)--(3.818,0.989)--(3.996,0.989)%
  --(4.515,0.998)--(4.564,0.992)--(5.330,1.001)--(5.352,0.997)--(5.715,1.043)--(6.133,1.218)%
  --(6.228,1.021)--(6.779,1.039)--(6.829,1.384)--(7.237,2.480)--(7.526,3.072)--(7.750,4.954)%
  --(8.053,1.552);
\gppoint{gp mark 6}{(2.995,0.986)}
\gppoint{gp mark 6}{(3.221,0.986)}
\gppoint{gp mark 6}{(3.370,0.987)}
\gppoint{gp mark 6}{(3.818,0.989)}
\gppoint{gp mark 6}{(3.996,0.989)}
\gppoint{gp mark 6}{(4.515,0.998)}
\gppoint{gp mark 6}{(4.564,0.992)}
\gppoint{gp mark 6}{(5.330,1.001)}
\gppoint{gp mark 6}{(5.352,0.997)}
\gppoint{gp mark 6}{(5.715,1.043)}
\gppoint{gp mark 6}{(6.133,1.218)}
\gppoint{gp mark 6}{(6.228,1.021)}
\gppoint{gp mark 6}{(6.779,1.039)}
\gppoint{gp mark 6}{(6.829,1.384)}
\gppoint{gp mark 6}{(7.237,2.480)}
\gppoint{gp mark 6}{(7.526,3.072)}
\gppoint{gp mark 6}{(7.750,4.954)}
\gppoint{gp mark 6}{(8.053,1.552)}
\gpcolor{gp lt color 5}
\draw[gp path] (2.995,0.986)--(3.221,0.986)--(3.370,0.987)--(3.818,0.989)--(3.996,0.989)%
  --(4.515,0.997)--(4.564,0.992)--(5.330,0.996)--(5.352,0.998)--(5.715,1.040)--(6.133,1.215)%
  --(6.228,1.019)--(6.779,1.040)--(6.829,1.374)--(7.237,2.499)--(7.526,3.070)--(7.750,4.921)%
  --(8.053,1.556);
\gppoint{gp mark 7}{(2.995,0.986)}
\gppoint{gp mark 7}{(3.221,0.986)}
\gppoint{gp mark 7}{(3.370,0.987)}
\gppoint{gp mark 7}{(3.818,0.989)}
\gppoint{gp mark 7}{(3.996,0.989)}
\gppoint{gp mark 7}{(4.515,0.997)}
\gppoint{gp mark 7}{(4.564,0.992)}
\gppoint{gp mark 7}{(5.330,0.996)}
\gppoint{gp mark 7}{(5.352,0.998)}
\gppoint{gp mark 7}{(5.715,1.040)}
\gppoint{gp mark 7}{(6.133,1.215)}
\gppoint{gp mark 7}{(6.228,1.019)}
\gppoint{gp mark 7}{(6.779,1.040)}
\gppoint{gp mark 7}{(6.829,1.374)}
\gppoint{gp mark 7}{(7.237,2.499)}
\gppoint{gp mark 7}{(7.526,3.070)}
\gppoint{gp mark 7}{(7.750,4.921)}
\gppoint{gp mark 7}{(8.053,1.556)}
\gpcolor{gp lt color 6}
\draw[gp path] (2.995,0.986)--(3.221,0.987)--(3.370,0.987)--(3.818,0.989)--(3.996,0.989)%
  --(4.515,0.997)--(4.564,0.992)--(5.330,1.001)--(5.352,0.999)--(5.715,1.041)--(6.133,1.220)%
  --(6.228,1.016)--(6.779,1.039)--(6.829,1.370)--(7.237,2.479)--(7.526,3.069)--(7.750,4.953)%
  --(8.053,1.563);
\gppoint{gp mark 8}{(2.995,0.986)}
\gppoint{gp mark 8}{(3.221,0.987)}
\gppoint{gp mark 8}{(3.370,0.987)}
\gppoint{gp mark 8}{(3.818,0.989)}
\gppoint{gp mark 8}{(3.996,0.989)}
\gppoint{gp mark 8}{(4.515,0.997)}
\gppoint{gp mark 8}{(4.564,0.992)}
\gppoint{gp mark 8}{(5.330,1.001)}
\gppoint{gp mark 8}{(5.352,0.999)}
\gppoint{gp mark 8}{(5.715,1.041)}
\gppoint{gp mark 8}{(6.133,1.220)}
\gppoint{gp mark 8}{(6.228,1.016)}
\gppoint{gp mark 8}{(6.779,1.039)}
\gppoint{gp mark 8}{(6.829,1.370)}
\gppoint{gp mark 8}{(7.237,2.479)}
\gppoint{gp mark 8}{(7.526,3.069)}
\gppoint{gp mark 8}{(7.750,4.953)}
\gppoint{gp mark 8}{(8.053,1.563)}
\gpcolor{gp lt color 7}
\draw[gp path] (2.995,0.986)--(3.221,0.987)--(3.370,0.987)--(3.818,0.989)--(3.996,0.989)%
  --(4.515,0.998)--(4.564,0.993)--(5.330,0.998)--(5.352,1.002)--(5.715,1.043)--(6.133,1.218)%
  --(6.228,1.014)--(6.779,1.044)--(6.829,1.372)--(7.237,2.488)--(7.526,3.081)--(7.750,4.938)%
  --(8.053,1.510);
\gppoint{gp mark 9}{(2.995,0.986)}
\gppoint{gp mark 9}{(3.221,0.987)}
\gppoint{gp mark 9}{(3.370,0.987)}
\gppoint{gp mark 9}{(3.818,0.989)}
\gppoint{gp mark 9}{(3.996,0.989)}
\gppoint{gp mark 9}{(4.515,0.998)}
\gppoint{gp mark 9}{(4.564,0.993)}
\gppoint{gp mark 9}{(5.330,0.998)}
\gppoint{gp mark 9}{(5.352,1.002)}
\gppoint{gp mark 9}{(5.715,1.043)}
\gppoint{gp mark 9}{(6.133,1.218)}
\gppoint{gp mark 9}{(6.228,1.014)}
\gppoint{gp mark 9}{(6.779,1.044)}
\gppoint{gp mark 9}{(6.829,1.372)}
\gppoint{gp mark 9}{(7.237,2.488)}
\gppoint{gp mark 9}{(7.526,3.081)}
\gppoint{gp mark 9}{(7.750,4.938)}
\gppoint{gp mark 9}{(8.053,1.510)}
\gpcolor{gp lt color 0}
\draw[gp path] (2.995,0.986)--(3.221,0.987)--(3.370,0.987)--(3.818,0.989)--(3.996,0.989)%
  --(4.515,0.998)--(4.564,0.993)--(5.330,0.996)--(5.352,0.997)--(5.715,1.035)--(6.133,1.218)%
  --(6.228,1.014)--(6.779,1.037)--(6.829,1.372)--(7.237,2.480)--(7.526,3.055)--(7.750,4.922)%
  --(8.053,1.529);
\gppoint{gp mark 10}{(2.995,0.986)}
\gppoint{gp mark 10}{(3.221,0.987)}
\gppoint{gp mark 10}{(3.370,0.987)}
\gppoint{gp mark 10}{(3.818,0.989)}
\gppoint{gp mark 10}{(3.996,0.989)}
\gppoint{gp mark 10}{(4.515,0.998)}
\gppoint{gp mark 10}{(4.564,0.993)}
\gppoint{gp mark 10}{(5.330,0.996)}
\gppoint{gp mark 10}{(5.352,0.997)}
\gppoint{gp mark 10}{(5.715,1.035)}
\gppoint{gp mark 10}{(6.133,1.218)}
\gppoint{gp mark 10}{(6.228,1.014)}
\gppoint{gp mark 10}{(6.779,1.037)}
\gppoint{gp mark 10}{(6.829,1.372)}
\gppoint{gp mark 10}{(7.237,2.480)}
\gppoint{gp mark 10}{(7.526,3.055)}
\gppoint{gp mark 10}{(7.750,4.922)}
\gppoint{gp mark 10}{(8.053,1.529)}
\gpcolor{gp lt color 1}
\draw[gp path] (2.995,0.986)--(3.221,0.987)--(3.370,0.988)--(3.818,0.992)--(3.996,0.993)%
  --(4.515,1.004)--(4.564,1.003)--(5.330,1.013)--(5.352,1.017)--(5.715,1.071)--(6.133,1.260)%
  --(6.228,1.059)--(6.779,1.102)--(6.829,1.440)--(7.237,2.606)--(7.526,3.253)--(7.750,5.225)%
  --(8.053,1.709);
\gppoint{gp mark 11}{(2.995,0.986)}
\gppoint{gp mark 11}{(3.221,0.987)}
\gppoint{gp mark 11}{(3.370,0.988)}
\gppoint{gp mark 11}{(3.818,0.992)}
\gppoint{gp mark 11}{(3.996,0.993)}
\gppoint{gp mark 11}{(4.515,1.004)}
\gppoint{gp mark 11}{(4.564,1.003)}
\gppoint{gp mark 11}{(5.330,1.013)}
\gppoint{gp mark 11}{(5.352,1.017)}
\gppoint{gp mark 11}{(5.715,1.071)}
\gppoint{gp mark 11}{(6.133,1.260)}
\gppoint{gp mark 11}{(6.228,1.059)}
\gppoint{gp mark 11}{(6.779,1.102)}
\gppoint{gp mark 11}{(6.829,1.440)}
\gppoint{gp mark 11}{(7.237,2.606)}
\gppoint{gp mark 11}{(7.526,3.253)}
\gppoint{gp mark 11}{(7.750,5.225)}
\gppoint{gp mark 11}{(8.053,1.709)}
\gpcolor{gp lt color 2}
\draw[gp path] (2.995,0.986)--(3.221,0.987)--(3.370,0.988)--(3.818,0.992)--(3.996,0.994)%
  --(4.515,1.015)--(4.564,1.015)--(5.330,1.099)--(5.352,1.115)--(5.715,1.284)--(6.133,1.644)%
  --(6.228,1.572)--(6.779,1.992)--(6.829,2.230)--(7.237,4.058)--(7.526,5.232)--(7.750,7.953)%
  --(8.053,4.822);
\gppoint{gp mark 12}{(2.995,0.986)}
\gppoint{gp mark 12}{(3.221,0.987)}
\gppoint{gp mark 12}{(3.370,0.988)}
\gppoint{gp mark 12}{(3.818,0.992)}
\gppoint{gp mark 12}{(3.996,0.994)}
\gppoint{gp mark 12}{(4.515,1.015)}
\gppoint{gp mark 12}{(4.564,1.015)}
\gppoint{gp mark 12}{(5.330,1.099)}
\gppoint{gp mark 12}{(5.352,1.115)}
\gppoint{gp mark 12}{(5.715,1.284)}
\gppoint{gp mark 12}{(6.133,1.644)}
\gppoint{gp mark 12}{(6.228,1.572)}
\gppoint{gp mark 12}{(6.779,1.992)}
\gppoint{gp mark 12}{(6.829,2.230)}
\gppoint{gp mark 12}{(7.237,4.058)}
\gppoint{gp mark 12}{(7.526,5.232)}
\gppoint{gp mark 12}{(7.750,7.953)}
\gppoint{gp mark 12}{(8.053,4.822)}
\gpcolor{gp lt color 3}
\draw[gp path] (2.995,0.986)--(3.221,0.987)--(3.370,0.988)--(3.818,0.992)--(3.996,0.994)%
  --(4.515,1.013)--(4.564,1.016)--(5.330,1.103)--(5.352,1.108)--(5.715,1.278)--(6.133,1.640)%
  --(6.228,1.632)--(6.779,2.910)--(6.829,2.857)--(7.237,5.346);
\gppoint{gp mark 13}{(2.995,0.986)}
\gppoint{gp mark 13}{(3.221,0.987)}
\gppoint{gp mark 13}{(3.370,0.988)}
\gppoint{gp mark 13}{(3.818,0.992)}
\gppoint{gp mark 13}{(3.996,0.994)}
\gppoint{gp mark 13}{(4.515,1.013)}
\gppoint{gp mark 13}{(4.564,1.016)}
\gppoint{gp mark 13}{(5.330,1.103)}
\gppoint{gp mark 13}{(5.352,1.108)}
\gppoint{gp mark 13}{(5.715,1.278)}
\gppoint{gp mark 13}{(6.133,1.640)}
\gppoint{gp mark 13}{(6.228,1.632)}
\gppoint{gp mark 13}{(6.779,2.910)}
\gppoint{gp mark 13}{(6.829,2.857)}
\gppoint{gp mark 13}{(7.237,5.346)}
\gpcolor{gp lt color 4}
\draw[gp path] (2.995,0.986)--(3.221,0.987)--(3.370,0.988)--(3.818,0.992)--(3.996,0.994)%
  --(4.515,1.013)--(4.564,1.016)--(5.330,1.105)--(5.352,1.107)--(5.715,1.280)--(6.133,1.637)%
  --(6.228,1.636)--(6.779,2.920)--(6.829,2.902)--(7.237,5.326);
\gppoint{gp mark 14}{(2.995,0.986)}
\gppoint{gp mark 14}{(3.221,0.987)}
\gppoint{gp mark 14}{(3.370,0.988)}
\gppoint{gp mark 14}{(3.818,0.992)}
\gppoint{gp mark 14}{(3.996,0.994)}
\gppoint{gp mark 14}{(4.515,1.013)}
\gppoint{gp mark 14}{(4.564,1.016)}
\gppoint{gp mark 14}{(5.330,1.105)}
\gppoint{gp mark 14}{(5.352,1.107)}
\gppoint{gp mark 14}{(5.715,1.280)}
\gppoint{gp mark 14}{(6.133,1.637)}
\gppoint{gp mark 14}{(6.228,1.636)}
\gppoint{gp mark 14}{(6.779,2.920)}
\gppoint{gp mark 14}{(6.829,2.902)}
\gppoint{gp mark 14}{(7.237,5.326)}
\gpcolor{gp lt color border}
\gpsetlinetype{gp lt border}
\draw[gp path] (1.504,8.631)--(1.504,0.985)--(8.447,0.985)--(8.447,8.631)--cycle;
%% coordinates of the plot area
\gpdefrectangularnode{gp plot 1}{\pgfpoint{1.504cm}{0.985cm}}{\pgfpoint{8.447cm}{8.631cm}}
\end{tikzpicture}
%% gnuplot variables

        \caption[Function run time complexity with various block size
            implementations (linear)]{Function run time complexity (for the
            \command{TopN_Outlier_Pruning_Block} function) with various block
            size implementations}
        \label{profiling:blockSize:functionRunTimeComplexity:linear}
    \end{minipage}
    \begin{minipage}{\textwidth}
       \centering
        \begin{tikzpicture}[gnuplot]
%% generated with GNUPLOT 4.4p3 (Lua 5.1.4; terminal rev. 97, script rev. 96a)
%% Wed 24 Oct 2012 11:33:51 EST
\gpcolor{\gprgb{502}{502}{502}}
\gpsetlinetype{gp lt plot 0}
\gpsetlinewidth{1.00}
\draw[gp path] (1.688,0.985)--(1.868,0.985);
\draw[gp path] (8.447,0.985)--(8.267,0.985);
\node[gp node right] at (1.504,0.985) { 0.01};
\draw[gp path] (1.688,1.445)--(1.778,1.445);
\draw[gp path] (8.447,1.445)--(8.357,1.445);
\draw[gp path] (1.688,2.054)--(1.778,2.054);
\draw[gp path] (8.447,2.054)--(8.357,2.054);
\draw[gp path] (1.688,2.366)--(1.778,2.366);
\draw[gp path] (8.447,2.366)--(8.357,2.366);
\draw[gp path] (1.688,2.514)--(1.868,2.514);
\draw[gp path] (8.447,2.514)--(8.267,2.514);
\node[gp node right] at (1.504,2.514) { 0.1};
\draw[gp path] (1.688,2.975)--(1.778,2.975);
\draw[gp path] (8.447,2.975)--(8.357,2.975);
\draw[gp path] (1.688,3.583)--(1.778,3.583);
\draw[gp path] (8.447,3.583)--(8.357,3.583);
\draw[gp path] (1.688,3.895)--(1.778,3.895);
\draw[gp path] (8.447,3.895)--(8.357,3.895);
\draw[gp path] (1.688,4.043)--(1.868,4.043);
\draw[gp path] (8.447,4.043)--(8.267,4.043);
\node[gp node right] at (1.504,4.043) { 1};
\draw[gp path] (1.688,4.504)--(1.778,4.504);
\draw[gp path] (8.447,4.504)--(8.357,4.504);
\draw[gp path] (1.688,5.112)--(1.778,5.112);
\draw[gp path] (8.447,5.112)--(8.357,5.112);
\draw[gp path] (1.688,5.424)--(1.778,5.424);
\draw[gp path] (8.447,5.424)--(8.357,5.424);
\draw[gp path] (1.688,5.573)--(1.868,5.573);
\draw[gp path] (8.447,5.573)--(8.267,5.573);
\node[gp node right] at (1.504,5.573) { 10};
\draw[gp path] (1.688,6.033)--(1.778,6.033);
\draw[gp path] (8.447,6.033)--(8.357,6.033);
\draw[gp path] (1.688,6.641)--(1.778,6.641);
\draw[gp path] (8.447,6.641)--(8.357,6.641);
\draw[gp path] (1.688,6.954)--(1.778,6.954);
\draw[gp path] (8.447,6.954)--(8.357,6.954);
\draw[gp path] (1.688,7.102)--(1.868,7.102);
\draw[gp path] (8.447,7.102)--(8.267,7.102);
\node[gp node right] at (1.504,7.102) { 100};
\draw[gp path] (1.688,7.562)--(1.778,7.562);
\draw[gp path] (8.447,7.562)--(8.357,7.562);
\draw[gp path] (1.688,8.171)--(1.778,8.171);
\draw[gp path] (8.447,8.171)--(8.357,8.171);
\draw[gp path] (1.688,8.483)--(1.778,8.483);
\draw[gp path] (8.447,8.483)--(8.357,8.483);
\draw[gp path] (1.688,8.631)--(1.868,8.631);
\draw[gp path] (8.447,8.631)--(8.267,8.631);
\node[gp node right] at (1.504,8.631) { 1000};
\draw[gp path] (1.688,0.985)--(1.688,1.165);
\draw[gp path] (1.688,8.631)--(1.688,8.451);
\node[gp node center] at (1.688,0.677) {$10^{2}$};
\draw[gp path] (2.366,0.985)--(2.366,1.075);
\draw[gp path] (2.366,8.631)--(2.366,8.541);
\draw[gp path] (2.763,0.985)--(2.763,1.075);
\draw[gp path] (2.763,8.631)--(2.763,8.541);
\draw[gp path] (3.044,0.985)--(3.044,1.075);
\draw[gp path] (3.044,8.631)--(3.044,8.541);
\draw[gp path] (3.263,0.985)--(3.263,1.075);
\draw[gp path] (3.263,8.631)--(3.263,8.541);
\draw[gp path] (3.441,0.985)--(3.441,1.075);
\draw[gp path] (3.441,8.631)--(3.441,8.541);
\draw[gp path] (3.592,0.985)--(3.592,1.075);
\draw[gp path] (3.592,8.631)--(3.592,8.541);
\draw[gp path] (3.723,0.985)--(3.723,1.075);
\draw[gp path] (3.723,8.631)--(3.723,8.541);
\draw[gp path] (3.838,0.985)--(3.838,1.075);
\draw[gp path] (3.838,8.631)--(3.838,8.541);
\draw[gp path] (3.941,0.985)--(3.941,1.165);
\draw[gp path] (3.941,8.631)--(3.941,8.451);
\node[gp node center] at (3.941,0.677) {$10^{3}$};
\draw[gp path] (4.619,0.985)--(4.619,1.075);
\draw[gp path] (4.619,8.631)--(4.619,8.541);
\draw[gp path] (5.016,0.985)--(5.016,1.075);
\draw[gp path] (5.016,8.631)--(5.016,8.541);
\draw[gp path] (5.297,0.985)--(5.297,1.075);
\draw[gp path] (5.297,8.631)--(5.297,8.541);
\draw[gp path] (5.516,0.985)--(5.516,1.075);
\draw[gp path] (5.516,8.631)--(5.516,8.541);
\draw[gp path] (5.694,0.985)--(5.694,1.075);
\draw[gp path] (5.694,8.631)--(5.694,8.541);
\draw[gp path] (5.845,0.985)--(5.845,1.075);
\draw[gp path] (5.845,8.631)--(5.845,8.541);
\draw[gp path] (5.976,0.985)--(5.976,1.075);
\draw[gp path] (5.976,8.631)--(5.976,8.541);
\draw[gp path] (6.091,0.985)--(6.091,1.075);
\draw[gp path] (6.091,8.631)--(6.091,8.541);
\draw[gp path] (6.194,0.985)--(6.194,1.165);
\draw[gp path] (6.194,8.631)--(6.194,8.451);
\node[gp node center] at (6.194,0.677) {$10^{4}$};
\draw[gp path] (6.872,0.985)--(6.872,1.075);
\draw[gp path] (6.872,8.631)--(6.872,8.541);
\draw[gp path] (7.269,0.985)--(7.269,1.075);
\draw[gp path] (7.269,8.631)--(7.269,8.541);
\draw[gp path] (7.550,0.985)--(7.550,1.075);
\draw[gp path] (7.550,8.631)--(7.550,8.541);
\draw[gp path] (7.769,0.985)--(7.769,1.075);
\draw[gp path] (7.769,8.631)--(7.769,8.541);
\draw[gp path] (7.947,0.985)--(7.947,1.075);
\draw[gp path] (7.947,8.631)--(7.947,8.541);
\draw[gp path] (8.098,0.985)--(8.098,1.075);
\draw[gp path] (8.098,8.631)--(8.098,8.541);
\draw[gp path] (8.229,0.985)--(8.229,1.075);
\draw[gp path] (8.229,8.631)--(8.229,8.541);
\draw[gp path] (8.344,0.985)--(8.344,1.075);
\draw[gp path] (8.344,8.631)--(8.344,8.541);
\draw[gp path] (8.447,0.985)--(8.447,1.165);
\draw[gp path] (8.447,8.631)--(8.447,8.451);
\node[gp node center] at (8.447,0.677) {$10^{5}$};
\draw[gp path] (1.688,8.631)--(1.688,0.985)--(8.447,0.985);
\gpcolor{gp lt color border}
\node[gp node center,rotate=-270] at (0.246,4.808) {\textbf{Function execution time}};
\node[gp node center] at (5.067,0.215) {\textbf{Problem size}};
\gpcolor{gp lt color axes}
\draw[gp path] (3.140,1.906)--(3.360,2.175)--(3.504,2.277)--(3.941,2.905)--(4.114,2.930)%
  --(4.619,3.660)--(4.667,3.266)--(5.413,3.570)--(5.434,3.904)--(5.787,4.664)--(6.194,5.683)%
  --(6.287,4.213)--(6.823,4.427)--(6.872,5.660)--(7.269,6.922)--(7.550,7.133)--(7.769,7.572)%
  --(8.063,5.985);
\gpsetpointsize{4.00}
\gppoint{gp mark 1}{(3.140,1.906)}
\gppoint{gp mark 1}{(3.360,2.175)}
\gppoint{gp mark 1}{(3.504,2.277)}
\gppoint{gp mark 1}{(3.941,2.905)}
\gppoint{gp mark 1}{(4.114,2.930)}
\gppoint{gp mark 1}{(4.619,3.660)}
\gppoint{gp mark 1}{(4.667,3.266)}
\gppoint{gp mark 1}{(5.413,3.570)}
\gppoint{gp mark 1}{(5.434,3.904)}
\gppoint{gp mark 1}{(5.787,4.664)}
\gppoint{gp mark 1}{(6.194,5.683)}
\gppoint{gp mark 1}{(6.287,4.213)}
\gppoint{gp mark 1}{(6.823,4.427)}
\gppoint{gp mark 1}{(6.872,5.660)}
\gppoint{gp mark 1}{(7.269,6.922)}
\gppoint{gp mark 1}{(7.550,7.133)}
\gppoint{gp mark 1}{(7.769,7.572)}
\gppoint{gp mark 1}{(8.063,5.985)}
\gpcolor{gp lt color 0}
\draw[gp path] (3.140,1.715)--(3.360,2.175)--(3.504,2.444)--(3.941,2.940)--(4.114,2.930)%
  --(4.619,3.668)--(4.667,3.307)--(5.413,3.685)--(5.434,3.932)--(5.787,4.654)--(6.194,5.715)%
  --(6.287,4.314)--(6.823,4.828)--(6.872,5.752)--(7.269,7.027)--(7.550,7.172)--(7.769,7.598)%
  --(8.063,6.308);
\gppoint{gp mark 2}{(3.140,1.715)}
\gppoint{gp mark 2}{(3.360,2.175)}
\gppoint{gp mark 2}{(3.504,2.444)}
\gppoint{gp mark 2}{(3.941,2.940)}
\gppoint{gp mark 2}{(4.114,2.930)}
\gppoint{gp mark 2}{(4.619,3.668)}
\gppoint{gp mark 2}{(4.667,3.307)}
\gppoint{gp mark 2}{(5.413,3.685)}
\gppoint{gp mark 2}{(5.434,3.932)}
\gppoint{gp mark 2}{(5.787,4.654)}
\gppoint{gp mark 2}{(6.194,5.715)}
\gppoint{gp mark 2}{(6.287,4.314)}
\gppoint{gp mark 2}{(6.823,4.828)}
\gppoint{gp mark 2}{(6.872,5.752)}
\gppoint{gp mark 2}{(7.269,7.027)}
\gppoint{gp mark 2}{(7.550,7.172)}
\gppoint{gp mark 2}{(7.769,7.598)}
\gppoint{gp mark 2}{(8.063,6.308)}
\gpcolor{gp lt color 1}
\draw[gp path] (3.140,1.715)--(3.360,2.175)--(3.504,2.366)--(3.941,2.905)--(4.114,2.916)%
  --(4.619,3.648)--(4.667,3.285)--(5.413,3.870)--(5.434,3.894)--(5.787,4.601)--(6.194,5.607)%
  --(6.287,4.363)--(6.823,4.675)--(6.872,5.679)--(7.269,6.869)--(7.550,7.077)--(7.769,7.506)%
  --(8.063,6.206);
\gppoint{gp mark 3}{(3.140,1.715)}
\gppoint{gp mark 3}{(3.360,2.175)}
\gppoint{gp mark 3}{(3.504,2.366)}
\gppoint{gp mark 3}{(3.941,2.905)}
\gppoint{gp mark 3}{(4.114,2.916)}
\gppoint{gp mark 3}{(4.619,3.648)}
\gppoint{gp mark 3}{(4.667,3.285)}
\gppoint{gp mark 3}{(5.413,3.870)}
\gppoint{gp mark 3}{(5.434,3.894)}
\gppoint{gp mark 3}{(5.787,4.601)}
\gppoint{gp mark 3}{(6.194,5.607)}
\gppoint{gp mark 3}{(6.287,4.363)}
\gppoint{gp mark 3}{(6.823,4.675)}
\gppoint{gp mark 3}{(6.872,5.679)}
\gppoint{gp mark 3}{(7.269,6.869)}
\gppoint{gp mark 3}{(7.550,7.077)}
\gppoint{gp mark 3}{(7.769,7.506)}
\gppoint{gp mark 3}{(8.063,6.206)}
\gpcolor{gp lt color 2}
\draw[gp path] (3.140,1.906)--(3.360,2.175)--(3.504,2.366)--(3.941,2.905)--(4.114,2.933)%
  --(4.619,3.660)--(4.667,3.305)--(5.413,3.682)--(5.434,3.541)--(5.787,4.694)--(6.194,5.623)%
  --(6.287,4.368)--(6.823,4.666)--(6.872,5.650)--(7.269,6.861)--(7.550,7.086)--(7.769,7.498)%
  --(8.063,6.198);
\gppoint{gp mark 4}{(3.140,1.906)}
\gppoint{gp mark 4}{(3.360,2.175)}
\gppoint{gp mark 4}{(3.504,2.366)}
\gppoint{gp mark 4}{(3.941,2.905)}
\gppoint{gp mark 4}{(4.114,2.933)}
\gppoint{gp mark 4}{(4.619,3.660)}
\gppoint{gp mark 4}{(4.667,3.305)}
\gppoint{gp mark 4}{(5.413,3.682)}
\gppoint{gp mark 4}{(5.434,3.541)}
\gppoint{gp mark 4}{(5.787,4.694)}
\gppoint{gp mark 4}{(6.194,5.623)}
\gppoint{gp mark 4}{(6.287,4.368)}
\gppoint{gp mark 4}{(6.823,4.666)}
\gppoint{gp mark 4}{(6.872,5.650)}
\gppoint{gp mark 4}{(7.269,6.861)}
\gppoint{gp mark 4}{(7.550,7.086)}
\gppoint{gp mark 4}{(7.769,7.498)}
\gppoint{gp mark 4}{(8.063,6.198)}
\gpcolor{gp lt color 3}
\draw[gp path] (3.140,1.906)--(3.360,2.175)--(3.504,2.444)--(3.941,2.940)--(4.114,2.954)%
  --(4.619,3.661)--(4.667,3.303)--(5.413,3.582)--(5.434,3.919)--(5.787,4.685)--(6.194,5.625)%
  --(6.287,4.323)--(6.823,4.704)--(6.872,5.655)--(7.269,6.858)--(7.550,7.079)--(7.769,7.498)%
  --(8.063,6.203);
\gppoint{gp mark 5}{(3.140,1.906)}
\gppoint{gp mark 5}{(3.360,2.175)}
\gppoint{gp mark 5}{(3.504,2.444)}
\gppoint{gp mark 5}{(3.941,2.940)}
\gppoint{gp mark 5}{(4.114,2.954)}
\gppoint{gp mark 5}{(4.619,3.661)}
\gppoint{gp mark 5}{(4.667,3.303)}
\gppoint{gp mark 5}{(5.413,3.582)}
\gppoint{gp mark 5}{(5.434,3.919)}
\gppoint{gp mark 5}{(5.787,4.685)}
\gppoint{gp mark 5}{(6.194,5.625)}
\gppoint{gp mark 5}{(6.287,4.323)}
\gppoint{gp mark 5}{(6.823,4.704)}
\gppoint{gp mark 5}{(6.872,5.655)}
\gppoint{gp mark 5}{(7.269,6.858)}
\gppoint{gp mark 5}{(7.550,7.079)}
\gppoint{gp mark 5}{(7.769,7.498)}
\gppoint{gp mark 5}{(8.063,6.203)}
\gpcolor{gp lt color 4}
\draw[gp path] (3.140,1.715)--(3.360,2.175)--(3.504,2.514)--(3.941,2.905)--(4.114,2.954)%
  --(4.619,3.672)--(4.667,3.309)--(5.413,3.842)--(5.434,3.669)--(5.787,4.686)--(6.194,5.616)%
  --(6.287,4.378)--(6.823,4.646)--(6.872,5.684)--(7.269,6.850)--(7.550,7.072)--(7.769,7.498)%
  --(8.063,6.207);
\gppoint{gp mark 6}{(3.140,1.715)}
\gppoint{gp mark 6}{(3.360,2.175)}
\gppoint{gp mark 6}{(3.504,2.514)}
\gppoint{gp mark 6}{(3.941,2.905)}
\gppoint{gp mark 6}{(4.114,2.954)}
\gppoint{gp mark 6}{(4.619,3.672)}
\gppoint{gp mark 6}{(4.667,3.309)}
\gppoint{gp mark 6}{(5.413,3.842)}
\gppoint{gp mark 6}{(5.434,3.669)}
\gppoint{gp mark 6}{(5.787,4.686)}
\gppoint{gp mark 6}{(6.194,5.616)}
\gppoint{gp mark 6}{(6.287,4.378)}
\gppoint{gp mark 6}{(6.823,4.646)}
\gppoint{gp mark 6}{(6.872,5.684)}
\gppoint{gp mark 6}{(7.269,6.850)}
\gppoint{gp mark 6}{(7.550,7.072)}
\gppoint{gp mark 6}{(7.769,7.498)}
\gppoint{gp mark 6}{(8.063,6.207)}
\gpcolor{gp lt color 5}
\draw[gp path] (3.140,1.906)--(3.360,2.175)--(3.504,2.444)--(3.941,2.940)--(4.114,2.984)%
  --(4.619,3.665)--(4.667,3.325)--(5.413,3.586)--(5.434,3.685)--(5.787,4.651)--(6.194,5.608)%
  --(6.287,4.344)--(6.823,4.662)--(6.872,5.668)--(7.269,6.858)--(7.550,7.071)--(7.769,7.493)%
  --(8.063,6.211);
\gppoint{gp mark 7}{(3.140,1.906)}
\gppoint{gp mark 7}{(3.360,2.175)}
\gppoint{gp mark 7}{(3.504,2.444)}
\gppoint{gp mark 7}{(3.941,2.940)}
\gppoint{gp mark 7}{(4.114,2.984)}
\gppoint{gp mark 7}{(4.619,3.665)}
\gppoint{gp mark 7}{(4.667,3.325)}
\gppoint{gp mark 7}{(5.413,3.586)}
\gppoint{gp mark 7}{(5.434,3.685)}
\gppoint{gp mark 7}{(5.787,4.651)}
\gppoint{gp mark 7}{(6.194,5.608)}
\gppoint{gp mark 7}{(6.287,4.344)}
\gppoint{gp mark 7}{(6.823,4.662)}
\gppoint{gp mark 7}{(6.872,5.668)}
\gppoint{gp mark 7}{(7.269,6.858)}
\gppoint{gp mark 7}{(7.550,7.071)}
\gppoint{gp mark 7}{(7.769,7.493)}
\gppoint{gp mark 7}{(8.063,6.211)}
\gpcolor{gp lt color 6}
\draw[gp path] (3.140,1.906)--(3.360,2.277)--(3.504,2.444)--(3.941,2.905)--(4.114,2.937)%
  --(4.619,3.661)--(4.667,3.327)--(5.413,3.856)--(5.434,3.765)--(5.787,4.669)--(6.194,5.620)%
  --(6.287,4.274)--(6.823,4.647)--(6.872,5.645)--(7.269,6.849)--(7.550,7.071)--(7.769,7.498)%
  --(8.063,6.219);
\gppoint{gp mark 8}{(3.140,1.906)}
\gppoint{gp mark 8}{(3.360,2.277)}
\gppoint{gp mark 8}{(3.504,2.444)}
\gppoint{gp mark 8}{(3.941,2.905)}
\gppoint{gp mark 8}{(4.114,2.937)}
\gppoint{gp mark 8}{(4.619,3.661)}
\gppoint{gp mark 8}{(4.667,3.327)}
\gppoint{gp mark 8}{(5.413,3.856)}
\gppoint{gp mark 8}{(5.434,3.765)}
\gppoint{gp mark 8}{(5.787,4.669)}
\gppoint{gp mark 8}{(6.194,5.620)}
\gppoint{gp mark 8}{(6.287,4.274)}
\gppoint{gp mark 8}{(6.823,4.647)}
\gppoint{gp mark 8}{(6.872,5.645)}
\gppoint{gp mark 8}{(7.269,6.849)}
\gppoint{gp mark 8}{(7.550,7.071)}
\gppoint{gp mark 8}{(7.769,7.498)}
\gppoint{gp mark 8}{(8.063,6.219)}
\gpcolor{gp lt color 7}
\draw[gp path] (3.140,1.906)--(3.360,2.277)--(3.504,2.514)--(3.941,2.905)--(4.114,2.981)%
  --(4.619,3.676)--(4.667,3.359)--(5.413,3.705)--(5.434,3.859)--(5.787,4.689)--(6.194,5.616)%
  --(6.287,4.233)--(6.823,4.698)--(6.872,5.642)--(7.269,6.854)--(7.550,7.074)--(7.769,7.496)%
  --(8.063,6.155);
\gppoint{gp mark 9}{(3.140,1.906)}
\gppoint{gp mark 9}{(3.360,2.277)}
\gppoint{gp mark 9}{(3.504,2.514)}
\gppoint{gp mark 9}{(3.941,2.905)}
\gppoint{gp mark 9}{(4.114,2.981)}
\gppoint{gp mark 9}{(4.619,3.676)}
\gppoint{gp mark 9}{(4.667,3.359)}
\gppoint{gp mark 9}{(5.413,3.705)}
\gppoint{gp mark 9}{(5.434,3.859)}
\gppoint{gp mark 9}{(5.787,4.689)}
\gppoint{gp mark 9}{(6.194,5.616)}
\gppoint{gp mark 9}{(6.287,4.233)}
\gppoint{gp mark 9}{(6.823,4.698)}
\gppoint{gp mark 9}{(6.872,5.642)}
\gppoint{gp mark 9}{(7.269,6.854)}
\gppoint{gp mark 9}{(7.550,7.074)}
\gppoint{gp mark 9}{(7.769,7.496)}
\gppoint{gp mark 9}{(8.063,6.155)}
\gpcolor{gp lt color 0}
\draw[gp path] (3.140,2.054)--(3.360,2.366)--(3.504,2.514)--(3.941,2.940)--(4.114,2.951)%
  --(4.619,3.680)--(4.667,3.344)--(5.413,3.584)--(5.434,3.627)--(5.787,4.597)--(6.194,5.616)%
  --(6.287,4.236)--(6.823,4.618)--(6.872,5.648)--(7.269,6.850)--(7.550,7.066)--(7.769,7.493)%
  --(8.063,6.178);
\gppoint{gp mark 10}{(3.140,2.054)}
\gppoint{gp mark 10}{(3.360,2.366)}
\gppoint{gp mark 10}{(3.504,2.514)}
\gppoint{gp mark 10}{(3.941,2.940)}
\gppoint{gp mark 10}{(4.114,2.951)}
\gppoint{gp mark 10}{(4.619,3.680)}
\gppoint{gp mark 10}{(4.667,3.344)}
\gppoint{gp mark 10}{(5.413,3.584)}
\gppoint{gp mark 10}{(5.434,3.627)}
\gppoint{gp mark 10}{(5.787,4.597)}
\gppoint{gp mark 10}{(6.194,5.616)}
\gppoint{gp mark 10}{(6.287,4.236)}
\gppoint{gp mark 10}{(6.823,4.618)}
\gppoint{gp mark 10}{(6.872,5.648)}
\gppoint{gp mark 10}{(7.269,6.850)}
\gppoint{gp mark 10}{(7.550,7.066)}
\gppoint{gp mark 10}{(7.769,7.493)}
\gppoint{gp mark 10}{(8.063,6.178)}
\gpcolor{gp lt color 1}
\draw[gp path] (3.140,2.175)--(3.360,2.444)--(3.504,2.688)--(3.941,3.263)--(4.114,3.409)%
  --(4.619,3.957)--(4.667,3.906)--(5.413,4.219)--(5.434,4.300)--(5.787,4.956)--(6.194,5.726)%
  --(6.287,4.855)--(6.823,5.158)--(6.872,5.834)--(7.269,6.904)--(7.550,7.127)--(7.769,7.542)%
  --(8.063,6.368);
\gppoint{gp mark 11}{(3.140,2.175)}
\gppoint{gp mark 11}{(3.360,2.444)}
\gppoint{gp mark 11}{(3.504,2.688)}
\gppoint{gp mark 11}{(3.941,3.263)}
\gppoint{gp mark 11}{(4.114,3.409)}
\gppoint{gp mark 11}{(4.619,3.957)}
\gppoint{gp mark 11}{(4.667,3.906)}
\gppoint{gp mark 11}{(5.413,4.219)}
\gppoint{gp mark 11}{(5.434,4.300)}
\gppoint{gp mark 11}{(5.787,4.956)}
\gppoint{gp mark 11}{(6.194,5.726)}
\gppoint{gp mark 11}{(6.287,4.855)}
\gppoint{gp mark 11}{(6.823,5.158)}
\gppoint{gp mark 11}{(6.872,5.834)}
\gppoint{gp mark 11}{(7.269,6.904)}
\gppoint{gp mark 11}{(7.550,7.127)}
\gppoint{gp mark 11}{(7.769,7.542)}
\gppoint{gp mark 11}{(8.063,6.368)}
\gpcolor{gp lt color 2}
\draw[gp path] (3.140,2.175)--(3.360,2.514)--(3.504,2.688)--(3.941,3.266)--(4.114,3.464)%
  --(4.619,4.255)--(4.667,4.264)--(5.413,5.139)--(5.434,5.230)--(5.787,5.781)--(6.194,6.306)%
  --(6.287,6.229)--(6.823,6.588)--(6.872,6.700)--(7.269,7.329)--(7.550,7.543)--(7.769,7.872)%
  --(8.063,7.476);
\gppoint{gp mark 12}{(3.140,2.175)}
\gppoint{gp mark 12}{(3.360,2.514)}
\gppoint{gp mark 12}{(3.504,2.688)}
\gppoint{gp mark 12}{(3.941,3.266)}
\gppoint{gp mark 12}{(4.114,3.464)}
\gppoint{gp mark 12}{(4.619,4.255)}
\gppoint{gp mark 12}{(4.667,4.264)}
\gppoint{gp mark 12}{(5.413,5.139)}
\gppoint{gp mark 12}{(5.434,5.230)}
\gppoint{gp mark 12}{(5.787,5.781)}
\gppoint{gp mark 12}{(6.194,6.306)}
\gppoint{gp mark 12}{(6.287,6.229)}
\gppoint{gp mark 12}{(6.823,6.588)}
\gppoint{gp mark 12}{(6.872,6.700)}
\gppoint{gp mark 12}{(7.269,7.329)}
\gppoint{gp mark 12}{(7.550,7.543)}
\gppoint{gp mark 12}{(7.769,7.872)}
\gppoint{gp mark 12}{(8.063,7.476)}
\gpcolor{gp lt color 3}
\draw[gp path] (3.140,2.175)--(3.360,2.514)--(3.504,2.688)--(3.941,3.266)--(4.114,3.446)%
  --(4.619,4.214)--(4.667,4.266)--(5.413,5.166)--(5.434,5.190)--(5.787,5.767)--(6.194,6.302)%
  --(6.287,6.294)--(6.823,7.018)--(6.872,6.999)--(7.269,7.561);
\gppoint{gp mark 13}{(3.140,2.175)}
\gppoint{gp mark 13}{(3.360,2.514)}
\gppoint{gp mark 13}{(3.504,2.688)}
\gppoint{gp mark 13}{(3.941,3.266)}
\gppoint{gp mark 13}{(4.114,3.446)}
\gppoint{gp mark 13}{(4.619,4.214)}
\gppoint{gp mark 13}{(4.667,4.266)}
\gppoint{gp mark 13}{(5.413,5.166)}
\gppoint{gp mark 13}{(5.434,5.190)}
\gppoint{gp mark 13}{(5.787,5.767)}
\gppoint{gp mark 13}{(6.194,6.302)}
\gppoint{gp mark 13}{(6.287,6.294)}
\gppoint{gp mark 13}{(6.823,7.018)}
\gppoint{gp mark 13}{(6.872,6.999)}
\gppoint{gp mark 13}{(7.269,7.561)}
\gpcolor{gp lt color 4}
\draw[gp path] (3.140,2.175)--(3.360,2.514)--(3.504,2.738)--(3.941,3.266)--(4.114,3.445)%
  --(4.619,4.206)--(4.667,4.278)--(5.413,5.174)--(5.434,5.185)--(5.787,5.772)--(6.194,6.299)%
  --(6.287,6.298)--(6.823,7.021)--(6.872,7.014)--(7.269,7.558);
\gppoint{gp mark 14}{(3.140,2.175)}
\gppoint{gp mark 14}{(3.360,2.514)}
\gppoint{gp mark 14}{(3.504,2.738)}
\gppoint{gp mark 14}{(3.941,3.266)}
\gppoint{gp mark 14}{(4.114,3.445)}
\gppoint{gp mark 14}{(4.619,4.206)}
\gppoint{gp mark 14}{(4.667,4.278)}
\gppoint{gp mark 14}{(5.413,5.174)}
\gppoint{gp mark 14}{(5.434,5.185)}
\gppoint{gp mark 14}{(5.787,5.772)}
\gppoint{gp mark 14}{(6.194,6.299)}
\gppoint{gp mark 14}{(6.287,6.298)}
\gppoint{gp mark 14}{(6.823,7.021)}
\gppoint{gp mark 14}{(6.872,7.014)}
\gppoint{gp mark 14}{(7.269,7.558)}
%% coordinates of the plot area
\gpdefrectangularnode{gp plot 1}{\pgfpoint{1.688cm}{0.985cm}}{\pgfpoint{8.447cm}{8.631cm}}
\end{tikzpicture}
%% gnuplot variables

        \caption[Function run time complexity with various block size
            implementations (logarithmic)]{Function run time complexity (for the
            \command{TopN_Outlier_Pruning_Block} function) with various block
            size implementations}
        \label{profiling:blockSize:functionRunTimeComplexity:logarithmic}
    \end{minipage}
\end{figure}

%%%%%%%%%%%%%%%%%%%%%%%%%%%%%%%%%%%%%%%%%%%%%%%%%%%%%%%%%%%%%%%%%%%%%%%%%%%%%%%%
% Discussion
%%%%%%%%%%%%%%%%%%%%%%%%%%%%%%%%%%%%%%%%%%%%%%%%%%%%%%%%%%%%%%%%%%%%%%%%%%%%%%%%
\subsection{Discussion}
\label{algorithmPerformance:discussion}
The results collected above all support the hypothesis that blocking had little
effect on the \command{TopN_Outlier_Pruning_Block} algorithm's performance. In
fact, large block sizes had a negative impact on the algorithm's performance,
although this could have been due to a limitation of hardware resources on the
testing hosts. The following remarks regarding the effect of blocking were
inferred from the test results gathered:
\begin{itemize}
    \item Most data sets seem to perform better without blocking than with,
        although the performance difference between a small block size (less
        than 1000) when compared to no blocking, is not significant.
    \item Most data sets seems to perform significantly worse with large block
        sizes (greater than 1000), when compared with smaller block sizes (less
        than 1000).
    \item Block sizes larger than 100000 did not seem to further affect the
        algorithm's performance. However. this was likely to be due to the fact
        that the largest data set that was used for testing was only 67557
        (< 100000) vectors.
    \item Seven of the test data sets (namely \dataset{letter-recognition},
        \dataset{magicgamma}, \dataset{connect4}, \dataset{musk},
        \dataset{spam_train}, \dataset{spam} and \dataset{runningex30k})
        performed significantly better without blocking. These data sets
        performed at least 5 times faster when compared to a block size of
        greater than 100000.
    \item The total number of calls to the distance function is nearly linear in
        the block size.
    \item Four of the test data sets (namely \dataset{segmentation},
        \dataset{pendigits}, \dataset{spam} and \dataset{mesh_network}) had no
        change in the number of calls to the \command{distance_squared} function
        compared with block size.
    \item Generally, the smaller the block size was, the faster the algorithm
        executed.
    \item The larger the block size, the more the number of vectors that are
        pruned during the algorithm's execution. Conversely, without blocking at
        all, very few vectors were pruned.
\end{itemize}

Fitting lines of best fit to the data points shown in
\autoref{profiling:blockSize:functionRunTimeComplexity:logarithmic} provides a
validation of the scaling performance of the
\command{TopN_Outlier_Pruning_Block} algorithm. The data shown in
\autoref{tbl:topn_outlier_pruning_block:lobf} shows that, for our test data
sets, the runtime performance of the \command{TopN_Outlier_Pruning_Block}
algorithm varied between $O(n^1.57)$ and $O(n^2.18)$, depending on the value
chosen for the block size.

% Lines of best fit
\begin{table}
    \begin{longtable}{|+>{\bfseries}c|^l|^c|}
        \tableHeader{Block size & Fit & Coefficient of determination ($R^2$)}
        0 &         $2.95232268646359 \times 10^-6 \times n^{1.5701731748}$ &   0.8252617823 \\
        1 &         $1.99974741419878 \times 10^-6 \times n^{1.6279693701}$ &   0.8472712848 \\
        10 &        $2.47146286428157 \times 10^-6 \times n^{1.5942823477}$ &   0.8514101635 \\
        15 &        $2.61779443859478 \times 10^-6 \times n^{1.5882193828}$ &   0.8273032456 \\
        30 &        $2.85347906672221 \times 10^-6 \times n^{1.5810782275}$ &   0.8407365739 \\
        40 &        $2.393            \times 10^-6 \times n^{1.6026893824}$ &   0.8369025885 \\
        45 &        $2.11550069078546 \times 10^-6 \times n^{1.6215256016}$ &   0.8176455619 \\
        60 &        $1.90246021613607 \times 10^-6 \times n^{1.6408351928}$ &   0.807660717 \\
        90 &        $1.8552473060756  \times 10^-6 \times n^{1.6472982903}$ &   0.7956635499 \\
        100 &       $2.45648959267628 \times 10^-6 \times n^{1.5966255392}$ &   0.8105129767 \\
        1000 &      $4.80646490593615 \times 10^-6 \times n^{1.5689092708}$ &   0.8328463309 \\\
        10000 &     $1.07142672605121 \times 10^-6 \times n^{1.8339855181}$ &   0.9182785472 \\
        100000 &    $8.59282468122866 \times 10^-8 \times n^{2.172997051}$  &   0.933976207 \\
        1000000 &   $8.04573229523765 \times 10^-8 \times n^{2.1829343104}$ &   0.9298342132 \\\hline
    \end{longtable}
    \caption{Scaling performance of the \command{TopN_Outlier_Pruning_Block}
        algorithm}
    \label{tbl:topn_outlier_pruning_block:lobf}
\end{table}
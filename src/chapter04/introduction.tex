In this chapter, I recount the design process that was adopted in order to
transform the \algm{anomaly detection using commute time} algorithm into a
hardware design. Specifically, this chapter analyses software implementations of
the algorithm (in both \progLang{MATLAB} and \progLang{C}), in order to identify
performance bottlenecks that the algorithm faces on a typical
microprocessor-based system.

In order to choose which step of the outlier detection algorithm would be best
suited for implementation on \gls{FPGA}-based hardware, it was first necessary
to \emph{profile} the execution of the algorithm under various test cases. This
task was performed using \software{MATLAB} with the source code listed in
\autoref{sourceCode:matlab}. Using \software{MATLAB}'s \command{profile}
command, I was able to analyse the algorithm and make an assessment of the
running time of the algorithm. The results of the algorithm profiling appear in
the following sections.

Essentially, this chapter details a range of tests and benchmarks that were
performed at various stages along the development process in order to evaluate
and compare the performance of several variations of the basic algorithm
implementation.

%%%%%%%%%%%%%%%%%%%%%%%%%%%%%%%%%%%%%%%%%%%%%%%%%%%%%%%%%%%%%%%%%%%%%%%%%%%%%%%%
% Base Implementation and Design Flow
%%%%%%%%%%%%%%%%%%%%%%%%%%%%%%%%%%%%%%%%%%%%%%%%%%%%%%%%%%%%%%%%%%%%%%%%%%%%%%%%
\subsection{Base Implementation and Design Flow}
\label{software:baseImplementation}
The base implementation of the \algm{anomaly detection using commute time}
algorithm, on which much of the research of this \thesis{} is based, was
generously provided by \getPerson{Khoa}. \citeauthor{Khoa:2012} had designed a
\progLang{MATLAB} implementation of the algorithm as a part of his research for
\citetitle{Khoa:2012} \cite{Khoa:2012}.

This \progLang{MATLAB} code formed the foundation on which all successive code
of this \thesis{} eventuates. The design flow that was adopted for the
development process is illustrated in \autoref{thesis:designFlow}.

\begin{figure}
    \centering
    % TODO: Finish this

\begin{tikzpicture}[
        >=latex,
        block/.style = {rectangle, draw, text width = 5em, text centered, rounded corners, minimum height = 4em},
        arrow/.style = {draw, -latex'},
        node distance = 3cm,
        auto]
    % Place nodes
    \node [block]                   (matlab)    {\software{MATLAB}};
    \node [block, below of=matlab]  (mex)       {\progLang{MEX}};
    \node [block, below of=mex]     (cpp)       {\progLang{C/C++}};
    \node [block, below of=cpp]     (autoesl)   {\software{AutoESL}};
    \node [block, below of=autoesl] (pcore)     {pcore};
    \node [block, below of=pcore]   (bitstream) {bitstream};

    % Edges
    \path [arrow] (matlab.south)  -| (mex.north);
    \path [arrow] (mex.south)     -| (cpp.north);
    \path [arrow] (cpp.south)     -| (autoesl.north);
    \path [arrow] (autoesl.south) -| (pcore.north);
    \path [arrow] (pcore.south)   -| (bitstream.north);
\end{tikzpicture}
    \caption{Design flow for this \thesis{}}
    \label{thesis:designFlow}
\end{figure}

%%%%%%%%%%%%%%%%%%%%%%%%%%%%%%%%%%%%%%%%%%%%%%%%%%%%%%%%%%%%%%%%%%%%%%%%%%%%%%%%
% Data Sets Used For Testing, Profiling and Benchmarking
%%%%%%%%%%%%%%%%%%%%%%%%%%%%%%%%%%%%%%%%%%%%%%%%%%%%%%%%%%%%%%%%%%%%%%%%%%%%%%%%
\subsection{Data Sets Used For Testing, Profiling and Benchmarking}
\label{software:datasets}
A total of 21 data sets were used for testing, profiling and benchmarking the
commute distance using anomaly detection algorithm. A brief description of each
data set is provided in \autoref{dataSets:brief}. These data sets are
described in detail in \autoref{dataSets}. Further information regarding the
data sets can be found in \autoref{dataSets}.

\begin{table}
    \centering
    \begin{datasets}
        \dataSetBrief{testoutrank}{441}{2}
        \dataSetBrief{ball1}{551}{2}
        \dataSetBrief{testCD}{640}{2}
        \dataSetBrief{runningex1k}{1000}{2}
        \dataSetBrief{testCDST}{2000}{2}
        \dataSetBrief{testCDST2}{2000}{2}
        \dataSetBrief{testCDST3}{2000}{2}
        \dataSetBrief{runningex10k}{10000}{2}
        \dataSetBrief{runningex20k}{20000}{2}
        \dataSetBrief{segmentation}{2100}{20}
        \dataSetBrief{spam_train}{4501}{58}
        \dataSetBrief{runningex30k}{30000}{2}
        \dataSetBrief{pendigits}{4601}{58}
        \dataSetBrief{runningex40k}{40000}{2}
        \dataSetBrief{runningex50k}{50000}{2}
        \dataSetBrief{spam}{4601}{58}
        \dataSetBrief{letter-recognition}{20000}{17}
        \dataSetBrief{mesh_network}{1193}{392}
        \dataSetBrief{magicgamma}{19020}{11}
        \dataSetBrief{musk}{6598}{167}
        \dataSetBrief{connect4}{67557}{43}
    \end{datasets}
    \caption{Brief description of the data sets}
    \label{dataSets:brief}
\end{table}

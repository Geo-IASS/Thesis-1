In this chapter, I recount the design process that was adopted in order to
transform the anomaly detection using commute time algorithm into a hardware
design. Specifically, this chapter analyses software implementations of the
algorithm (in both \programmingLanguage{MATLAB} and \programmingLanguage{C}), in
order to identify performance bottlenecks that the algorithm faces on a typical
microprocess-based system.

Essentially, this chapter details a range of tests that were performed at
various stages along the development process in order to evaluate and compare
the performance of several variations of the basic algorithm implementation.

%%%%%%%%%%%%%%%%%%%%%%%%%%%%%%%%%%%%%%%%%%%%%%%%%
% Base Implementation and Design Flow
%%%%%%%%%%%%%%%%%%%%%%%%%%%%%%%%%%%%%%%%%%%%%%%%%
\subsection{Base Implementation and Design Flow}
\label{software:baseImplementation}
The base implementation of the anomaly detection using commute time algorithm,
on which much of the research of this Thesis is based, was generously provided
by \citeauthor{Khoa:2012}. \citeauthor{Khoa:2012} had designed a
\programmingLanguage{MATLAB} implementation of the algorithm as a part of his
research for \fullcite{Khoa:2012} \cite{Khoa:2012}.

This \programmingLanguage{MATLAB} code formed the foundation on which all
successive code of this Thesis eventuates. The design flow that was adopted for
the development process is as follows:
\begin{enumerate}
    \item \programmingLanguage{MATLAB} $\longrightarrow$
        \programmingLanguage{MATLAB}+\programmingLanguage{C} (using
        \programmingLanguage{MEX} files).
    \item \programmingLanguage{MATLAB}+\programmingLanguage{C} $\longrightarrow$
        Pure \programmingLanguage{C}.
    \item Pure \programmingLanguage{C} $\longrightarrow$ Pure
        \programmingLanguage{C} with \software{AutoESL} directives for \gls{HDL}
        synthesis.
    \item  Pure \programmingLanguage{C} with \software{AutoESL} directives
        $\longrightarrow$ pcore.
    % TODO
\end{enumerate}

% TODO: Flow chart

%%%%%%%%%%%%%%%%%%%%%%%%%%%%%%%%%%%%%%%%%%%%%%%%%
% Data Sets Used For Testing and Profiling
%%%%%%%%%%%%%%%%%%%%%%%%%%%%%%%%%%%%%%%%%%%%%%%%%
\subsection{Data Sets Used For Testing and Profiling}
\label{software:datasets}

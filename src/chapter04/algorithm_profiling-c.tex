%%%%%%%%%%%%%%%%%%%%%%%%%%%%%%%%%%%%%%%%%%%%%%%%%%%%%%%%%%%%%%%%%%%%%%%%%%%%%%%%
% Introduction
%%%%%%%%%%%%%%%%%%%%%%%%%%%%%%%%%%%%%%%%%%%%%%%%%%%%%%%%%%%%%%%%%%%%%%%%%%%%%%%%
\subsection{Introduction}
\label{profiling:c:introduction}
Once the \progLang{MATLAB} code had been profiled and a candidate function had
been selected for hardware implementation, the next step in my design process
was to translate the \progLang{MATLAB} code for this function into, initially, a
\software{MATLAB} MEX file written in \progLang{C} and, later, a standalone
\progLang{C} application.

In order to profile the \progLang{C} application, I utilised the
\software{GNU Profiler}[gprof], which was linked in with the executable using
the \software{GCC Compiler Collection}[gcc] \progLang{C} compiler.

%%%%%%%%%%%%%%%%%%%%%%%%%%%%%%%%%%%%%%%%%%%%%%%%%%%%%%%%%%%%%%%%%%%%%%%%%%%%%%%%
% Method
%%%%%%%%%%%%%%%%%%%%%%%%%%%%%%%%%%%%%%%%%%%%%%%%%%%%%%%%%%%%%%%%%%%%%%%%%%%%%%%%
\section{Method}
\label{profiling:c:method}
In order to allow for a standalone \progLang{C} application to be developed,
without necessitating the translation of the complete \algm{anomaly detection
using commute distance} algorithm from \progLang{MATLAB} to \progLang{C}, I
developed \command{vardump} (variable dump) functions to store the arguments to
the \command{TopN_Outlier_Pruning_Block} function into a binary file, together
with the expected outliers and outlier scores. In this way, I was able to
incrementally test the standalone \progLang{C} program and ensure that the
detected outliers matched those of the original \software{MATLAB}
implementation.

The compiler flags that were used for the \software{MATLAB} MEX file compilation
are as follows:
\begin{itemize}[noitemsep]
    \item \lstinline|-glnxa64|
    \item \lstinline|-D__MEX__|
    \item \lstinline|-I/usr/local/MATLAB/R2012a/extern/include|
    \item \lstinline|-largeArrayDims|
    \item \lstinline|-O|
    \item \lstinline|-DNDEBUG|
    \item \lstinline|-DSORTED_INSERT|
    \item \lstinline|-DBLOCKING|
    \item \lstinline|-DUSE_DYNAMIC_ARRAY_SIZE=0|
    \item \lstinline|CFLAGS=|
        \begin{itemize}[noitemsep]
            \item \lstinline|-D_GNU_SOURCE|
            \item \lstinline|-fexceptions|
            \item \lstinline|-fPIC|
            \item \lstinline|-fno-omit-frame-pointer|
            \item \lstinline|-pthread|
            \item \lstinline|-std=c99|
            \item \lstinline|-W|
            \item \lstinline|-Wall|
            \item \lstinline|-Wextra|
            \item \lstinline|-pedantic|
            \item \lstinline|-fmessage-length=0|
            \item \lstinline|-Wno-unused-label|
            \item \lstinline|-O3|
            \item \lstinline|-DNDEBUG|
            \item \lstinline|-DSORTED_INSERT|
            \item \lstinline|-DBLOCKING|
            \item \lstinline|-DUSE_DYNAMIC_ARRAY_SIZE=0|
        \end{itemize}
    \item \lstinline|CXXFLAGS=|
        \begin{itemize}[noitemsep]
            \item \lstinline|-D_GNU_SOURCE|
            \item \lstinline|-fPIC|
            \item \lstinline|-fno-omit-frame-pointer|
            \item \lstinline|-pthread|
            \item \lstinline|-std=c++0x|
            \item \lstinline|-W|
            \item \lstinline|-Wall|
            \item \lstinline|-Wextra|
            \item \lstinline|-pedantic|
            \item \lstinline|-fmessage-length=0|
            \item \lstinline|-Wno-unused-label|
            \item \lstinline|-O3|
            \item \lstinline|-DNDEBUG|
            \item \lstinline|-DSORTED_INSERT|
            \item \lstinline|-DBLOCKING|
            \item \lstinline|-DUSE_DYNAMIC_ARRAY_SIZE=0|
        \end{itemize}
    \item \lstinline|LDFLAGS=|
        \begin{itemize}[noitemsep]
            \item \lstinline|-pthread|
            \item \lstinline|-shared|
            \item \lstinline|-Wl,--version-script,/usr/local/MATLAB/R2012a/extern/lib/glnxa64/mexFunction.map|
            \item \lstinline|-Wl,--no-undefined|
            \item \lstinline|-O3|
        \end{itemize}
\end{itemize}

The compiler flags that were used with \command{gcc} for the standalone
\progLang{C} application are as follows:
\begin{itemize}[noitemsep]
    \item \lstinline|-D__C__|
    \item \lstinline|-D_GNU_SOURCE|
    \item \lstinline|-fexceptions|
    \item \lstinline|-fPIC|
    \item \lstinline|-fno-omit-frame-pointer|
    \item \lstinline|-pthread|
    \item \lstinline|-std=c99|
    \item \lstinline|-W|
    \item \lstinline|-Wall|
    \item \lstinline|-Wextra|
    \item \lstinline|-pedantic|
    \item \lstinline|-fmessage-length=0|
    \item \lstinline|-Wno-unused-label|
    \item \lstinline|-O3|
    \item \lstinline|-DNDEBUG|
    \item \lstinline|-DSORTED_INSERT|
    \item \lstinline|-DBLOCKING|
    \item \lstinline|-DUSE_DYNAMIC_ARRAY_SIZE=0|
    \item \lstinline|-pg|
\end{itemize}

%%%%%%%%%%%%%%%%%%%%%%%%%%%%%%%%%%%%%%%%%%%%%%%%%%%%%%%%%%%%%%%%%%%%%%%%%%%%%%%%
% Results
%%%%%%%%%%%%%%%%%%%%%%%%%%%%%%%%%%%%%%%%%%%%%%%%%%%%%%%%%%%%%%%%%%%%%%%%%%%%%%%%
\subsection{Results}
\label{profiling:c:results}
The following plots compare the `self time' of functions comprising the
\algm{anomaly detection using commute distance} algorithm, specifically for a
\progLang{C} implementation. Please note that in the following plots, any
functions with a self time less than 1\% of the total execution time have been
aggregated and are represented in the plots as ``Other'' functions.

The legend shown in \autoref{profiling:c:legend} is common amongst all of the
the \progLang{C} profiling results plots shown in
\autoref{fig:profiling:c}.

\begin{figure}
    \centering
    \pielegend[bound,color={yellow!60,blue!60,white!60,orange!60}]{
1/\escape{top_n_outlier_pruning_block},
1/\escape{distance_squared},
1/\escape{Other},
1/\escape{add_neighbour}
}

    \caption{Common legend for C profiling plots}
    \label{profiling:c:legend}
\end{figure}
\begin{pieplots}{fig:profiling:c}{C profiling plots}
\pieplot{\escape{testCDST}}{\pie[bound,text=none,radius=1.5,color={yellow!60,blue!60,white!60,orange!60}]{
1.93/\escape{top_n_outlier_pruning_block},
90.50/\escape{distance_squared},
0.00/\escape{Other},
7.57/\escape{add_neighbour}
}
}
\pieplot{\escape{spam}}{\pie[bound,text=none,radius=1.5,color={yellow!60,blue!60,white!60,orange!60}]{
4.38/\escape{top_n_outlier_pruning_block},
88.65/\escape{distance_squared},
0.00/\escape{Other},
6.97/\escape{add_neighbour}
}
}
\pieplot{\escape{testCD}}{\pie[bound,text=none,radius=1.5,color={yellow!60,blue!60,white!60,orange!60}]{
1.39/\escape{top_n_outlier_pruning_block},
89.58/\escape{distance_squared},
0.00/\escape{Other},
9.03/\escape{add_neighbour}
}
}
\pieplot{\escape{runningex30k}}{\pie[bound,text=none,radius=1.5,color={yellow!60,blue!60,orange!60,white!60}]{
1.61/\escape{top_n_outlier_pruning_block},
91.05/\escape{distance_squared},
7.30/\escape{add_neighbour},
0.04/\escape{Other}
}
}
\pieplot{\escape{testCDST2}}{\pie[bound,text=none,radius=1.5,color={yellow!60,blue!60,white!60,orange!60}]{
1.39/\escape{top_n_outlier_pruning_block},
91.91/\escape{distance_squared},
0.00/\escape{Other},
6.70/\escape{add_neighbour}
}
}
\pieplot{\escape{segmentation}}{\pie[bound,text=none,radius=1.5,color={yellow!60,blue!60,white!60,orange!60}]{
2.52/\escape{top_n_outlier_pruning_block},
87.89/\escape{distance_squared},
0.18/\escape{Other},
9.41/\escape{add_neighbour}
}
}
\pieplot{\escape{magicgamma}}{\pie[bound,text=none,radius=1.5,color={yellow!60,blue!60,white!60,orange!60}]{
13.28/\escape{top_n_outlier_pruning_block},
81.50/\escape{distance_squared},
0.02/\escape{Other},
5.20/\escape{add_neighbour}
}
}
\pieplot{\escape{ball1}}{\pie[bound,text=none,radius=1.5,color={yellow!60,blue!60,white!60,orange!60}]{
1.96/\escape{top_n_outlier_pruning_block},
89.18/\escape{distance_squared},
0.00/\escape{Other},
8.86/\escape{add_neighbour}
}
}
\pieplot{\escape{pendigits}}{\pie[bound,text=none,radius=1.5,color={yellow!60,blue!60,orange!60,white!60}]{
8.30/\escape{top_n_outlier_pruning_block},
85.53/\escape{distance_squared},
6.17/\escape{add_neighbour},
0.00/\escape{Other}
}
}
\pieplot{\escape{runningex20k}}{\pie[bound,text=none,radius=1.5,color={yellow!60,blue!60,orange!60,white!60}]{
1.58/\escape{top_n_outlier_pruning_block},
91.02/\escape{distance_squared},
7.35/\escape{add_neighbour},
0.04/\escape{Other}
}
}
\pieplot{\escape{musk}}{\pie[bound,text=none,radius=1.5,color={yellow!60,blue!60,orange!60,white!60}]{
3.09/\escape{top_n_outlier_pruning_block},
90.07/\escape{distance_squared},
6.79/\escape{add_neighbour},
0.05/\escape{Other}
}
}
\pieplot{\escape{runningex10k}}{\pie[bound,text=none,radius=1.5,color={yellow!60,blue!60,orange!60,white!60}]{
1.54/\escape{top_n_outlier_pruning_block},
90.98/\escape{distance_squared},
7.43/\escape{add_neighbour},
0.05/\escape{Other}
}
}
\pieplot{\escape{runningex40k}}{\pie[bound,text=none,radius=1.5,color={yellow!60,blue!60,orange!60,white!60}]{
1.82/\escape{top_n_outlier_pruning_block},
90.92/\escape{distance_squared},
7.21/\escape{add_neighbour},
0.04/\escape{Other}
}
}
\pieplot{\escape{spam_train}}{\pie[bound,text=none,radius=1.5,color={yellow!60,blue!60,white!60,orange!60}]{
4.75/\escape{top_n_outlier_pruning_block},
89.87/\escape{distance_squared},
0.00/\escape{Other},
5.39/\escape{add_neighbour}
}
}
\pieplot{\escape{testoutrank}}{\pie[bound,text=none,radius=1.5,color={yellow!60,blue!60,white!60,orange!60}]{
1.76/\escape{top_n_outlier_pruning_block},
94.73/\escape{distance_squared},
0.00/\escape{Other},
3.51/\escape{add_neighbour}
}
}
\pieplot{\escape{testCDST3}}{\pie[bound,text=none,radius=1.5,color={yellow!60,blue!60,orange!60,white!60}]{
1.61/\escape{top_n_outlier_pruning_block},
89.93/\escape{distance_squared},
8.46/\escape{add_neighbour},
0.00/\escape{Other}
}
}
\pieplot{\escape{runningex1k}}{\pie[bound,text=none,radius=1.5,color={yellow!60,blue!60,white!60,orange!60}]{
1.96/\escape{top_n_outlier_pruning_block},
94.09/\escape{distance_squared},
0.00/\escape{Other},
3.94/\escape{add_neighbour}
}
}
\pieplot{\escape{mesh_network}}{\pie[bound,text=none,radius=1.5,color={yellow!60,blue!60,orange!60,white!60}]{
1.71/\escape{top_n_outlier_pruning_block},
91.14/\escape{distance_squared},
7.15/\escape{add_neighbour},
0.00/\escape{Other}
}
}
\pieplot{\escape{letter-recognition}}{\pie[bound,text=none,radius=1.5,color={yellow!60,blue!60,orange!60,white!60}]{
7.14/\escape{top_n_outlier_pruning_block},
86.85/\escape{distance_squared},
5.94/\escape{add_neighbour},
0.08/\escape{Other}
}
}
\pieplot{\escape{connect4}}{\pie[bound,text=none,radius=1.5,color={yellow!60,blue!60,orange!60,white!60}]{
15.28/\escape{top_n_outlier_pruning_block},
79.49/\escape{distance_squared},
5.18/\escape{add_neighbour},
0.04/\escape{Other}
}
}
\end{pieplots}


%%%%%%%%%%%%%%%%%%%%%%%%%%%%%%%%%%%%%%%%%%%%%%%%%%%%%%%%%%%%%%%%%%%%%%%%%%%%%%%%
% Discussion
%%%%%%%%%%%%%%%%%%%%%%%%%%%%%%%%%%%%%%%%%%%%%%%%%%%%%%%%%%%%%%%%%%%%%%%%%%%%%%%%
\subsection{Discussion}
\label{profiling:c:discussion}
The profiling results from the \progLang{C} implementation support the results
obtained using the \progLang{MATLAB} implementation --- i.e., that improvements
to the performance of the \command{distance_squared} function will yield the
greatest improvement to the algorithm collectively.
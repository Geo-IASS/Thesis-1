In order to choose which step of the outlier detection algorithm would be best
suited for implementation on \gls{FPGA}-based hardware, it was first necessary
to \emph{profile} the execution of the algorithm under various test cases. This
task was performed using \software{MATLAB}, using code supplied by
\citeauthor{Khoa:2012} that was used to test and verify the conclusions of
\citetitle{Khoa:2012}. Using \software{MATLAB}'s \command{profile} command, I
was able to analyse the algorithm and make an assessment of the running time of
the algorithm. The results of the algorithm profiling appear in the following
sections.

%%%%%%%%%%%%%%%%%%%%%%%%%%%%%%%%%%%%%%%%%%%%%%%%%%%%%%%%%%%%%%%%%%%%%%%%%%%%%%%%
% Performance Bottleneck
%%%%%%%%%%%%%%%%%%%%%%%%%%%%%%%%%%%%%%%%%%%%%%%%%%%%%%%%%%%%%%%%%%%%%%%%%%%%%%%%
\subsection{Performance Bottleneck}
\label{algorithmPerformance:bottleneck}
From observations of the results of the algorithm profiling, it was observed
that the performance of the `anomaly detection using commute-distance' algorithm
is bottle-necked significantly by a function named
\command{TopN_Outlier_Pruning_Block}. The MATLAB code for this function can be
found in \autoref{sourceCode:matlab}. Analysis of this function, as well as
discussions with \citeauthor{Khoa:2012} revealed that the algorithm was
originally devised by \citeauthor{Bay:2003} and published in the paper
\citetitle{Bay:2003}. The general steps of the algorithm are described in
\autoref{algm:TopNOutlierPruningBlock}.

\begin{algorithm}
    \LinesNumbered

\SetKwInput{InputK}{k}\InputK{the number of nearest neighbours to consider}
\SetKwInput{InputN}{N}\InputN{the number of outliers to return}
\SetKwInput{InputData}{data}\InputData{a set of examples in random order}
\SetKwInOut{Output}{outliers}\Output{a set of outliers}
\SetKwInOut{OutputScores}{outlier\_scores}\OutputScores{}

\SetKwData{BlockIterator}{b}
\SetKwData{Block}{block}
\SetKwData{Cutoff}{cutoff}
\SetKwData{DataIterator}{d}
\SetKwData{Data}{data}
\SetKwData{NumK}{k}
\SetKwData{NumN}{N}
\SetKwData{Neighbours}{neighbours}
\SetKwData{OutlierIterator}{o}
\SetKwData{Outliers}{outliers}
\SetKwData{Score}{score}

\SetKwFunction{Closest}{closest}
\SetKwFunction{Distance}{distance}
\SetKwFunction{LoadBlock}{load\_next\_block}
\SetKwFunction{MaxDist}{max\_dist}
\SetKwFunction{Min}{min}
\SetKwFunction{Top}{top}

\Begin{
    \Cutoff $\leftarrow$ 0\tcp*[l]{set the cutoff for pruning to $0$}
    \Outliers $\leftarrow\emptyset$\tcp*[l]{initialize to the empty set}
    \BlankLine
    \While(\tcp*[h]{load a block of examples from \Data}){\Block $\leftarrow$ \LoadBlock{\Data}}{
        \ForEach{\BlockIterator $\in$ \Block}{
            $\Neighbours[\BlockIterator]\leftarrow\emptyset$\;
        }
        \BlankLine
        \ForEach{\DataIterator $\in$ \Data}{
            \ForEach{\BlockIterator $\in$ \Block $:$ \BlockIterator $\neq$ \DataIterator}{
                \If{$|\Neighbours[\BlockIterator]|<$ \NumK $\:\lor\:$ \Distance{\BlockIterator, \DataIterator} $<$ \MaxDist{\BlockIterator, $\Neighbours[\BlockIterator]$}}{
                    %$\Neighbours[\BlockIterator] \leftarrow$ \Closest{\BlockIterator, $\Neighbours[\BlockIterator]$} $\cup$ \DataIterator,\NumK)$\;
                    \If{$\Score[\Neighbours[\BlockIterator],\BlockIterator)<\Cutoff$}{
                        \Block $\leftarrow$ \Block $\setminus$ \BlockIterator\;
                    }
                }
            }
        }
        \BlankLine
        \Outliers $\leftarrow$ \Top{\Block $\cup$ \Outliers, \NumN}\;
        \Cutoff $\leftarrow$ $\Min_{\OutlierIterator \in \Outliers}(\Score(\OutlierIterator))$\tcp*[l]{the cutoff is the score of the weakest outlier}
    }

    \BlankLine
    \KwRet{\Outliers}\;
}

    \caption{TopN\_Outlier\_Pruning\_Block}
    \label{algm:TopNOutlierPruningBlock}
\end{algorithm}

In this algorithm, the \command{score} function can be any monotonically
decreasing function of the nearest neighbor distances such as the distance to
the $k$th nearest neighbor, or the average distance to the $k$ neighbours
\cite{Bay:2003}.

The main idea of the nested loop algorithm is that for each example in the
input set \verb+Data+, the algorithm keeps track of the closest neighbours found
so far. When an example's closest neighbours achieve a score lower than the
\verb+cutoff+, the example is discarded because it can no longer be an outlier.
As more examples are processed, the algorithm finds more extreme outliers and
the \verb+cutoff+ increases along with pruning efficiency \cite{Bay:2003}.

In the worst case, the performance of the algorithm is very poor --- due to the
nested loops, it could require $O(N^{2})$ distance computations and
$O(\frac{N}{blocksize} \times N)$ data accesses. However, \citeauthor{Bay:2003}
proved (through application of the algorithm to both real and synthetic data
sets) that the algorithm performs considerably better than the expected
$O(N^{2})$ running time in the average case. The performance improvements over
similar algorithms were attributed to the application of randomization and
pruning techniques. The outlier pruning problem can be considered similar to the
problem of conducting a set of independent Bernoulli trials in which examples
are analysed until $k$ examples within distance $d$ are found, or until the data
set is exhausted. \citeauthor{Bay:2003} proved that the number of trials
expected to achieve $k$ examples within distance $d$ is given by:

\begin{equation}
\label{eqn:outlierPruneComplexity}
E[Y] \leq \frac{k}{\pi(\textbf{x})} + \Bigg(1 - \sum_{y=k}^{N} P(Y=y)\Bigg) \times N
\end{equation}

Where $\pi(x)$ is the probability that a random drawn example lies within
distance $d$ of point $x$ and $P(Y=y)$ is the probability of obtaining the $k$th
success on trial $y$.

The first term in \autoref{eqn:outlierPruneComplexity} represents the number of
distance computations to eliminate non-outliers, and does not depend on $N$. The
second term represents the expected cost of outliers, and does depend on $N$,
yielding an overall quadratic dependency to process $N$ examples in total.
However, note that we typically set the program parameters to return a small an
possibly fixed number of outliers. Thus the first term dominates and we obtain
near linear performance \cite{Bay:2003}. More specifically, it was determined
that the primary factor determining the scaling is how the cutoff changes as $N$
increases.

There are, however, some limitations of the aforementioned algorithm.
Specifically:
\begin{enumerate}
    \item The algorithm assumes that the data is in random order. If the data is
        not in random order and is sorted then the performance can be poor.
    \item The algorithm depends on the independence of examples. If examples are
        dependent in such a way that they have similar values (and will likely
        be in the set of $k$ nearest neighbors) this can cause performance to be
        poor as the algorithm may have to scan the entire data set to find the
        dependent examples.
    \item The algorithm can perform poorly occurs when the data does not contain
        outliers.
\end{enumerate}

% TODO
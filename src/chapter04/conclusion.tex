%%%%%%%%%%%%%%%%%%%%%%%%%%%%%%%%%%%%%%%%%%%%%%%%%%%%%%%%%%%%%%%%%%%%%%%%%%%%%%%%
% The Function
%%%%%%%%%%%%%%%%%%%%%%%%%%%%%%%%%%%%%%%%%%%%%%%%%%%%%%%%%%%%%%%%%%%%%%%%%%%%%%%%
\section{The Function}
\label{software:algorithmProfiling:theFunction}
The function that was identified in the aforementioned tests as being most
significant to the execution time of the \algm{anomaly detection using commute
time} algorithm, and thus an ideal candidate for hardware acceleration, was the
\command{Top_N_Outlier_Pruning_Block} function. This function was originally
proposed by \citeauthor{Bay:2003} in \citetitle{Bay:2003}.

Essentially, this function simply performs a nested loops comparison of the
distances between pairwise vectors. Consequently, in the worst case, the
performance of this algorithm is very poor --- and it could require $O(N^2)$
distance computations and $O(N/blocksize \times N)$ data accesses
\cite{Bay:2003}. However, the use of randomization as well as pruning improves
the average-case complexity of the algorithm to be somewhat near-linear. In
particular, \citeauthor{Bay:2003}

The main idea of the nested loop algorithm is that for each vector in the data
set, we keep track of the closest neighbours found so far. When an vector's
closest neighbours achieve a score lower than the cutoff, the data set is pruned
from the current block because it can no longer be an outlier. As the algorithm
proceeds to process more vectors, it is able to find more extreme outliers and
the cutoff increases along with pruning efficiency \cite{Bay:2003}. The score
function can be any monotonically decreasing function of the nearest neighbour
distances such as the distance to the $k$th nearest neighbour, or the average
distance to the k neighbours \cite{Bay:2003}.

In \citetitle{Bay:2003}, \citeauthor{Bay:2003} proves that the performance of
this method is dependent

Surprisingly, the first term which represents the number of distance
computations to eliminate non-outliers does not depend on $N$ . The second term,
which represents the expected cost of outliers (i.e, we must compare with
everything in the database and then conclude that nothing is close) does depend
on $N$, yielding an overall quadratic dependency to process $N$ examples in
total. However, note that we typically set the program parameters to return a
small and possibly fixed number of outliers. Thus the first term dominates and
we obtain near linear performance.

The main goal of our experimental study was to show that our algorithm could
scale to very large data sets. We showed that on large, real, high-dimensional
data sets the algorithm had near linear scaling performance. However, the
algorithm depends on a number of assumptions, violations of which can lead to
poor performance.

Although this pr

\begin{enumerate}
    \item The algorithm assumes that the data is in random order. If the data is
        not in random order and is sorted then the performance can be poor.
    \item The outlier algorithm depends on the independence of examples. If
        examples are dependent in such a way that they have similar values (and
        will likely be in the set of $k$ nearest neighbours) this can cause
        performance to be poor as the algorithm may have to scan the entire data
        set to find the dependent examples.
    \item The third situation when our algorithm can perform poorly occurs when
        the data does not contain outliers. For example, our experiment with the
        examples drawn from a uniform distribution had very poor scaling.
        However, we believe data sets of this type are likely to be rare as most
        physical quantities one can measure have distributions with tails.
\end{enumerate}

The prediction of the stock market has become an issue of great interest in the
areas of finance, mathematics and engineering; due mainly to the great potential
financial gain. Researchers have devised various algorithms and differing
approaches to the problem of stock market analysis, with varying degrees of
success.

A major outstanding issue for stock market analysis is the effective and
efficient detection of local anomalies in the input data sets, which are
inherently highly multidimensional. Many na\"{\i}ve algorithms are highly
inefficient and others fail to adequately detect local anomalies altogether. It
had become a time-vs-correctness trade-off in which no acceptable compromise
could be reached.

However, researchers are starting to explore the relatively new concept of
applying ``random projections'' to the highly multidimensional data sets.
Research has suggested that by applying these random projections, they are able
to significantly reduce the dimensionality (and consequently the computational
complexity) of the data sets, whilst sufficiently retaining the inherent
properties of that data set --- at least so much so as anomaly detection is
concerned.

Anomaly detection is important because it allows otherwise-accurate machine
learning algorithms such as neural networks to more accurately model and predict
the stock exchange data by ignoring anomalous data, which likely doesn't effect
the state of the model to any significant degree.

\getPerson{Chawla} from \universityName{} has in recent years conducted and
supervised new and exciting research oriented around random projections. In
particular, \getPerson{Khoa}, under the supervision of \getPerson{Chawla}
evaluated the use of traditional distance metrics, such as Euclidean distance
and Mahalanobis distance, in the application of local anomaly detection.
\getPerson{Khoa} proposed the use of the `commute time' metric, derived from
random walks on graphs, in anomaly detection.

The source code that was used throughout this project can be freely downloaded
from \gitRepoHTTP{} (over HTTP) or \gitRepoSSH{} (over SSH).
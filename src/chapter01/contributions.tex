One key difficulty in anomaly detection is the efficient scaling of a general
algorithm to apply to highly multivariate data \citeNeeded{}. In this Thesis, I
explore the use of randomization techniques (such as random projections and
commute time) in anomaly detection algorithms. These techniques provide
encouraging results with regards to the run-time complexity of an algorithm.

Furthermore, in this Thesis I attempt to make observations and an analysis of
the run-time performance of anomaly detection algorithms using randomization
techniques, so as to identify steps in the algorithms that bottleneck the
algorithm's performance. Through this identification it would be possible to
improve the \emph{actual} run-time performance of the algorithm by utilising the
advantages of reconfigurable computing.
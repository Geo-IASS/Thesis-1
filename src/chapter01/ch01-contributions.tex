One key difficulty in anomaly detection is the efficient scaling of a general 
algorithm to apply to highly multivariate data. In this thesis, I explore the 
use of randomization techniques (such as random projections and commute time) 
in anomaly detection algorithms. These techniques provide encouraging results 
with regards to the run-time complexity of an algorithm.

Furthermore, in this thesis I attempt to make observations and analysis of the 
run-time of anomaly detection using randomization techniques, so as to identify
steps in the algorithms that bottleneck the algorithm's performance. Through 
this identification it would be possible to improve the \emph{actual} run-time 
performance of the algorithm by utilisation the advantages of reconfigurable 
computing.

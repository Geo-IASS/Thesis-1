The rest of this \thesis{} is organized as follows. In \autoref{background}, I
provide a background to various anomaly detection techniques, as well as
randomization techniques. In order to provide the reader with an understanding
of the background topics, a brief background is given to various topics in
linear algebra, vector calculus and graph theory. However, the provided
background material is not intended to be comprehensive, and it is expected that
the reader is familiar with these fields of mathematics. In
\autoref{reconfigurableComputing}, I provide an overview of reconfigurable
computing, including an explanation of \glspl{FPGA} and an overview of
\gls{HLS}.

In \autoref{software}, I profile the execution of an anomaly detection algorithm
and explore possible improvements to the algorithm, in particular by outsourcing
various stages of the algorithm to an FPGA device. I proceed to make iterative
refinements to the algorithm from its base implementation towards a hardware
design.

In \autoref{hardware}, I propose several hardware implementations and discuss
design considerations, as well as the design process itself, that must be
considered for these designs. I also detail the results of some initial testing
investigating the performance expectations from differing hardware designs.

I conclude in \autoref{conclusions} by reflecting upon the results obtained
through this research, and making suggestions for further research in this
topic.

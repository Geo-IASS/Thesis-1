The rest of this \thesis{} is organized as follows. In \autoref{background}, I
provide a background to various anomaly detection techniques, as well as
randomization techniques. In order to provide the reader with an understanding
of the background topics, a brief background is given to various topics in
linear algebra, vector calculus and graph theory. However, the provided
background material is not intended to be comprehensive, and it is expected that
the reader is familiar with these fields of mathematics. In
\autoref{reconfigurableComputing}, I provide an overview of reconfigurable
computing, including an explanation of \glspl{FPGA}.

In \autoref{software}, I profile the execution of an anomaly detection algorithm
and explore possible improvements to the algorithm, in particular by outsourcing
various stages of the algorithm to an FPGA device. In \autoref{hardware},
I describe the implementation of the improved algorithm and detail the process
that was followed in order to construct the hardware processing device.
\begin{comment}
In \autoref{results}, I record results obtained by benchmarking the device that
was previously designed and constructed, and comparing the expected improvements
to the algorithm's execution with the measured results.
\end{comment}

I conclude in \autoref{conclusions} by reflecting upon the results obtained
through this research, and making suggestions for further research in this
topic.
\LinesNumbered

% sample graph A using a random walk with restart
% function [Anew, sampled_vertices_all] = vertex_sampling(A, pSampling, outlying_cluster_size)

\SetKwInOut{AdjacencyMatrix}{$A$}\adjacencyMatrix{similarity matrix}
\SetKwInput{PSampling}{$p_{sampling}$}\pSampling{percentage of samples}
\SetKwInput{OutlyingClusterSize}{$ocs$}\OutlyingClusterSize{threshold b/w sizes of normal and outlying clusters}
\SetKwInOut{Sampled}{$sampled$}\sampled{sampled vertices}

\SetKwData{NumClusters}{$nClusters$}

\SetKwFunction{Components}{components}\Components{Compute the connected components of a graph}
% [ci sizes] = components(A) returns the component index vector (ci) and
% the size of each of the connected components (sizes).  The number of
% connected components is max(components(A)).  The algorithm used computes
% the strongly connected components of A, which are the connected
% components of A if A is undirected (i.e. symmetric).

\SetKwFunction{RandomPermutation}{randperm}\RandomPermutation{returns a vector containing a random permutation of the integers $1\rightarrow N$}
%   P = RANDPERM(N) returns a vector containing a random permutation of the
%   integers 1:N.  For example, RANDPERM(6) might be [2 4 5 6 1 3].

\SetKwFunction{Rows}{rows}

\Begin{
    % connected components
    [ci, sizes] = \Components{\AdjacencyMatrix}\;

    % sampling each big component
    \Sampled$\leftarrow\emptyset$\;

    \For{$i\leftarrow 1$ \KwTo \Rows{$sizes$}}{
        cluster_index = find(ci' == i);
        cluster_size = size(cluster_index,2);

        if cluster_size > outlying_cluster_size % sampling
            index = randperm(cluster_size);
            nSamples = round(pSampling*cluster_size);
            sampled_vertices = cluster_index(index(1:nSamples));

            sampled_vertices_all = [sampled_vertices_all sampled_vertices];
        else
            sampled_vertices_all = [sampled_vertices_all cluster_index];
        end
    }

    Anew = A(sampled_vertices_all, sampled_vertices_all);

    \KwRet{\varOutliers}\;
}

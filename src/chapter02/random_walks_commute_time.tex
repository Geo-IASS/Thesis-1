Assume we are given a connected undirected and weighted graph $G=(V,E,W)$ with
edge weights $(w_{ij})_{i,j \in V}>=0$ be the graph adjacency matrix. A degree
of a node $i$ is $d_i=\sum_{j\in N(i)}w_{ij}$ where $N(i)$ is a set of
neighbours of node $i$. All nodes nonadjacent to $i$ are assumed to have a
weight of $w_{ij}=0$.

A random walk is a sequence of nodes on a graph visited by a random walker:
starting from a node, the random walker moves to one of its neighbours with some
probability. Then from that node, it proceeds to one of its own neighbours with
some probability, and so on \cite{Khoa:2012}. The random walk is a finite Markov
chain that is time-reversible, which means the reverse Markov chain has the same
transition probability matrix as the original Markov chain \cite{Lovasz:1996}.

The probability that a random walker selects a particular node from is
neighbours is determined by the edge weights of the graph. The larger the weight
$w_{ij}$ of the edge connecting nodes $i$ and $j$, the more often the random
walker travels through that edge.

%%%%%%%%%%%%%%%%%%%%%%%%%%%%%%%%%%%%%%%%%%%%%%%%%%%%%%%%%%%%%%%%%%%%%%%%%%%%%%%%
% Similarity Graphs
%%%%%%%%%%%%%%%%%%%%%%%%%%%%%%%%%%%%%%%%%%%%%%%%%%%%%%%%%%%%%%%%%%%%%%%%%%%%%%%%
\subsection{Similarity Graphs}
\label{similarityGraphs}
% TODO

%%%%%%%%%%%%%%%%%%%%%%%%%%%%%%%%%%%%%%%%%%%%%%%%%%%%%%%%%%%%%%%%%%%%%%%%%%%%%%%%
% Hitting Time
%%%%%%%%%%%%%%%%%%%%%%%%%%%%%%%%%%%%%%%%%%%%%%%%%%%%%%%%%%%%%%%%%%%%%%%%%%%%%%%%
\subsection{Hitting Time}
\label{hittingTime}
% TODO

%%%%%%%%%%%%%%%%%%%%%%%%%%%%%%%%%%%%%%%%%%%%%%%%%%%%%%%%%%%%%%%%%%%%%%%%%%%%%%%%
% Commute Time
%%%%%%%%%%%%%%%%%%%%%%%%%%%%%%%%%%%%%%%%%%%%%%%%%%%%%%%%%%%%%%%%%%%%%%%%%%%%%%%%
\subsection{Commute Time}
\label{commuteTime}

% Introduction
\subsubsection{Introduction}
\label{commuteTime:introduction}
Commute time is a robust distance metric derived from a random walk on graphs
\cite{Khoa:2012}. In \citetitle{Khoa:2012}, \citeauthor{Khoa:2012} demonstrated
how commute time can be used as a distance measure for data mining tasks such as
anomaly detection and clustering. A prohibitive limitation of this technique is
that the calculation of commute time involves the eigen decomposition of the
graph Laplacian, making it impractical for large graphs.

A major advantage of using commute time as a distance metric for outlier
detection is that it effectively captures not only the distances between data
points but also the density of the data \citeNeeded{}. This property results in
a distance metric that can be effectively used to capture global, local and
group anomalies.

The commute time between two nodes $i$ and $j$ in a graph is the number of steps
that a random walk, starting from $i$ will take to visit $j$ and then come back
to $i$ for the first time. The fact that the commute time is averaged over all
paths (and not just the shortest path) makes it more robust to data
perturbations and it can also capture graph density \cite{Khoa:2012}. Since it
is a measure which can capture the geometrical structure of the data and is
robust to noise, commute time can be applied in methods where Euclidean or other
distances are used and thus the limitations of these metrics can be avoided.

% Limitations
\subsubsection{Limitations}
\label{commuteTime:limitations}
The computation of commute time requires the eigen decomposition (see
\autoref{eigenDecomposition}) of the graph Laplacian matrix (see
\autoref{laplacianMatrices}), a computation which takes $O(n^3)$ time and thus
is not practical for large graphs \citeNeeded{}. Methods to approximate the
commute time to reduce the computational time are required in order to
efficiently use commute time in large data sets.
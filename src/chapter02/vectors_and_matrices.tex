%%%%%%%%%%%%%%%%%%%%%%%%%%%%%%%%%%%%%%%%%%%%%%%%%
% Eigenvectors and Eigenvalues
%%%%%%%%%%%%%%%%%%%%%%%%%%%%%%%%%%%%%%%%%%%%%%%%%
\subsection{Eigenvectors and Eigenvalues}
\label{eigenvectors}
\label{eigenvalues}
This section will briefly recall some basic definitions of eigenvectors and
eigenvalues, as well as some of their basic properties.

A vector $\mathbf{v}$ is an eigenvector of a matrix $M$ of eigenvalue $\lambda$
if:
\begin{equation}
M\mathbf{v} = \lambda\textbf{v}
\end{equation}

If $\mathbf{v_1}$ is an eigenvector of $M$ of eigenvalue $\lambda_1$,
$\mathbf{v_2}$ is an eigenvector of $M$ of eigenvalue $\lambda_2\neq\lambda_1$,
and $M$ is symmetric, then $\mathbf{v_1}$ is orthogonal to $\mathbf{v_2}$.

For a symmetric matrix $M$, the multiplicity of an eigenvalue $\lambda$ is the
dimension of the space of eigenvectors of eigenvalue $\lambda$. Also recall that
every $n\times n$ symmetric matrix has $n$ eigenvalues, counted with
multiplicity. Thus, it has an orthonormal basis of eigenvectors,
$\begin{Bmatrix} \mathbf{v_1} & \ldots & \mathbf{v_n} \end{Bmatrix}$ with
eigenvalues $\lambda_1\leq\lambda_2\leq\ldots\leq\lambda_3$ so that:
\begin{equation}
M\mathbf{v_i} = \lambda_i \mathbf{v_i} \quad \forall i
\end{equation}

If we let $V$ be the matrix whose $i$th column is $v_i$ and $\Lambda$ be the
diagonal matrix whose $i$th diagonal is $\lambda_i$, we can write this more
compactly as:
\begin{equation}
MV = V\Lambda
\end{equation}

Multiplying by $V^T$ on the right, we obtain the eigen-decomposition of $M$:
\begin{equation}
M = MVV^T = V{\Lambda}V^T = \sum_i \lambda_i\mathbf{v_i}\mathbf{v_i^T}
\end{equation}

%%%%%%%%%%%%%%%%%%%%%%%%%%%%%%%%%%%%%%%%%%%%%%%%%
% Eigen Decomposition
%%%%%%%%%%%%%%%%%%%%%%%%%%%%%%%%%%%%%%%%%%%%%%%%%
\subsection{Eigen Decomposition}
\label{eigenDecomposition}
% TODO

%%%%%%%%%%%%%%%%%%%%%%%%%%%%%%%%%%%%%%%%%%%%%%%%%
% Laplacian Matrices
%%%%%%%%%%%%%%%%%%%%%%%%%%%%%%%%%%%%%%%%%%%%%%%%%
\subsection{Laplacian Matrices}
\label{laplacianMatrices}
\nocite{Berkeley:1999,Pati:2011,Spielman:2006}
Recall that a weighted, undirected graph $G = (V,E,w)$ is essentially an
undirected graph $G = (V,E)$ along with a function $w : E \rightarrow \Re^+$,
where $\Re^+$ denotes the set of positive real numbers.

The adjacency matrix of a weighted graph $G$ is be denoted $A_G$, and is given
by:
\begin{equation}
A_{G}(i,j) :=
    \left\{
        \begin{array}{ll}
            \mathit{w}(i,j) &   \quad \text{if $(i,j) \in E$}\\
            0 &                 \quad \text{otherwise}
        \end{array}
    \right.
\end{equation}

The degree matrix of a weighted graph $G$, denoted $D_G$, is the diagonal matrix
such that:
\begin{equation}
D_G(i,i) = \sum_j A_G(i,j)
\end{equation}

A Laplacian Matrix is a matrix representation of a graph, defined by the
equation:
\begin{equation}
L_G = D_G - A_G
\end{equation}

For a vector $\textbf{x} \in \Re^V$, the Laplacian quadratic form of $G$ is:
\begin{equation}
\label{laplacianQuadraticForm}
\textbf{x}^T L \textbf{x} = \sum_{(u,v) \in E} w_{u,v}(\textbf{x}(u) - \textbf{x}(v))^2
\end{equation}

From \autoref{laplacianQuadraticForm}, it can be seen that $L$ provides a
measure of the smoothness of $\textbf{x}$ over the edges in $G$. The more
$\textbf{x}$ jumps over an edge, the larger the quadratic form becomes.

It is often more convenient to consider the normalized Laplacian of a graph
instead of the Laplacian \cite{Spielman:2010}. The normalized Laplacian is given
by $D^{-1/2}LD^{-1/2}$ and is more closely related to the behaviour of random
walks.

Now, let $G_{1,2}$ be a graph on two vertices with a single edge of weight $1$.
\begin{equation}
L_{G_{1,2}} :=
    \begin{bmatrix}
        1 & -1\\
        -1 & 1
    \end{bmatrix}
\end{equation}

For the graph with $n$ vertices and just one edge between vertices $u$ and $v$,
we can define the Laplacian Matrix similarly. For concreteness, I'll call this
graph $G_{u,v}$. It's Laplacian Matrix is the $n \times n$ matrix whose only
non-zero entries are in the intersections of rows and columns $u$ and $v$. The
$2 \times 2$ matrix at the intersections of these rows and columns is, of
course:
\begin{equation}
    \begin{bmatrix}
        1 & -1\\
        -1 & 1
    \end{bmatrix}
\end{equation}

For a weighted graph $G = (V,E,w)$, we define:
\begin{equation}
L_G := \sum_{(u,v) \in E} w(u,v)L_{G_{u,v}}
\end{equation}

% Properties
\subsubsection{Properties}
\label{laplacianMatrices:properties}
Laplacian matrices of graphs are symmetric, have zero row-sums, and have
nonpositive off-diagonal entries. We call any matrix that satisfies these
properties a Laplacian matrix, as there always exists some graph for which it is
the Laplacian \cite{Spielman:2010}.

For a graph $G$ and its Laplacian Matrix $L$ with eigenvalues $\lambda_0<
\lambda_1<\ldots<\lambda_{n-1}$:

\begin{itemize}
\item $L$ is a symmetric matrix. This means the eigenvalues of $L$ are real, and
its eigenvectors are real and orthogonal.
\item $L$ is always positive-semidefinite ($\forall i,\lambda_i\geq 0;
\lambda_0=0$).
\item Let $G=(V,E)$ be a graph, and let $0=\lambda_1\leq\lambda_2\leq\ldots\leq
\lambda_n$ be the eigenvalues of its Laplacian Matrix. Then, $\lambda_2>0$ if
and only if $G$ is connected.
\item The number of times $0$ appears as an eigenvalue in the Laplacian Matrix
is the number of connected components in the graph.
\item $\lambda_0$ is always $0$ because every Laplacian Matrix has an
eigenvector of $\begin{bmatrix} 1 & 1 & \ldots & 1 \end{bmatrix}$ that, for each
row, adds the corresponding node's degree to a ``-1'' for each neighbour,
thereby producing zero by definition.
\item The smallest non-zero eigenvalue of $L$ is called the spectral gap.
\item If we arbitrarily assign an orientation to the edges in $G$ and label each
edge, then we can define the vertex edge incidence matrix $Q$ by:
\begin{equation}
Q_{ij} := 
    \left\{
        \begin{array}{ll}
            1 &     \quad \text{if $e_j$ starts from $i$}\\
            -1 &    \quad \text{if $e_j$ ends at $i$}\\
            0 &     \quad \text{otherwise}
        \end{array}
    \right.
\end{equation}
Then the Laplacian Matrix $L$ satisfies $L = Q^TQ$, regardless of the
orientation of the edges.
\item The second smallest eigenvalue of $L$ ($\lambda_2$) is the algebraic
connectivity of $G$. $\lambda_2>0$ if and only if $G$ is connected.
\end{itemize}

% Applications
\subsubsection{Applications}
\label{laplacianMatrices:applications}
An interesting application of Laplacian matrices is that of modelling electrical
flow in a network resistors. In this model, the vertices of the graph correspond
to points at which current can be added to or removed from the circuit. Edges in
the graph correspond to resistors, with the edge weight equal to the conductance
of the electrical resistor.

If $\textbf{p} \in \Re^V$ denotes the vector of potentials and $\textbf{i}_{ext}
\in \Re^V$ the vectors of currents entering and leaving vertices, then these
satisfy the relation:
\begin{equation}
L\textbf{p} = \textbf{i}_{ext}
\end{equation}

This equation can be exploited to compute the effective resistance between
pairs of vertices \cite{Spielman:2010}. The effective resistance between
vertices $u$ and $v$ is the difference in potential one must impose between $u$
and $v$ to flow one unit of current from $u$ to $v$. To measure this, we compute
the vector $\textbf{p}$ for which $L\textbf{p} = \textbf{i}_{ext}$, where:
\begin{equation}
\textbf{i}_{ext}(x) =
    \left\{
        \begin{array}{ll}
            1 &     \quad \text{for $x=u$}\\
            -1 &    \quad \text{for $x=v$}\\
            0 &     \quad \text{otherwise}
        \end{array}
    \right.
\end{equation}

We then measure the difference between $\textbf{p}(u)$ and $\textbf{p}$(v).

A `Markov chain' is a chance process in which the outcome of a given experiment
can affect the outcome of the next experiment \cite{Grinstead:1997}. For a
Markov chain, we have a set of states $S = \left\{s_1,s_2,\ldots,s_r\right\}$
with a process starting in one of the states and moving from state $s_i$ to
$s_j$ with a probability $p_{ij}$ not dependent upon which states the chain was
in before the current state. The probabilities $p_{ij}$ are called
\emph{transition probabilities}, and the complete matrix $\mathbf{P}$ of
probabilities is known as the \emph{transition matrix}.

The probability that, given the chain is in state $i$ now, it will be in state
$j$ in two steps is denoted by $p_{ij}^{(2)}$. In general, if a Markov chain has
$r$ states, then:

\begin{equation}
p_{ij}^{(2)} = \sum_{k=1}^r p_{ik} p_{kj}
\end{equation}
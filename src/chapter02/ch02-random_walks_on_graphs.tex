Assume we are given a connected undirected and weighted graph $G = (V,E,W)$ with
edge weights $(w_{ij})_{i,j \in V} >= 0$ be the graph adjacency matrix. A degree
of a node $i$ is $d_{i} = \sum_{j \in N(i)} w_{ij}$ where $N(i)$ is a set of 
neighbours of node $i$. All nodes nonadjacent to $i$ are assumed to have a 
weight of $w_{ij} = 0$.

A random walk is a sequence of nodes on a graph visited by a random walker: 
starting from a node, the random walker moves to one of its neighbours with some
probability. Then from that node, it proceeds to one of its own neighbours with 
some probability, and so on \cite{Khoa:2012}. The random walk is a finite Markov
chain that is time-reversible, which means the reverse Markov chain has the same
transition probability matrix as the original Markov chain \cite{Lovasz:1996}.

The probability that a random walker selects a particular node from is 
neighbours is determined by the edge weights of the graph. The larger the weight
${w_ij}$ of the edge connecting nodes $i$ and $j$, the more often the random 
walker travels through that edge.

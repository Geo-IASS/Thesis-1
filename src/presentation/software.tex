%%%%%%%%%%%%%%%%%%%%%%%%%%%%%%%%%%%%%%%%%%%%%%%%%%%%%%%%%%%%%%%%%%%%%%%%%%%%%%%%
% Method
%%%%%%%%%%%%%%%%%%%%%%%%%%%%%%%%%%%%%%%%%%%%%%%%%%%%%%%%%%%%%%%%%%%%%%%%%%%%%%%%
\subsection{Method}
\begin{frame}[label=method]{Method}
    \begin{enumerate}[<+->]
        \item MATLAB source code
        \bigskip
        \item MATLAB source code + MEX file
        \bigskip
        \item C++ source code
        \bigskip
        \item AutoESL high level synthesis
        \bigskip
        \item Hardware design
    \end{enumerate}
\end{frame}

%%%%%%%%%%%%%%%%%%%%%%%%%%%%%%%%%%%%%%%%%%%%%%%%%%%%%%%%%%%%%%%%%%%%%%%%%%%%%%%%
% Algorithm Profiling
%%%%%%%%%%%%%%%%%%%%%%%%%%%%%%%%%%%%%%%%%%%%%%%%%%%%%%%%%%%%%%%%%%%%%%%%%%%%%%%%
\subsection{Algorithm Profiling}
\begin{frame}[label=profiling]{Algorithm Profiling}
        {\escape{TopN_Outlier_Pruning_Block}}
    \profilingPlots{testoutrank}{runningex40k}{connect4}

    \note{These graphs show the relative execution times of functions comprising
        the anomaly detection algorithm, executed on 3 different data sets. The
        graphs for the other 18 data sets are very similar.

        These graphs shows that the vast majority of the algorithm's execution
        time is attributed to the \escape{distance_squared} function --- a
        simple function which returns the square of the Euclidean distance
        between two given vectors.}
\end{frame}

%%%%%%%%%%%%%%%%%%%%%%%%%%%%%%%%%%%%%%%%%%%%%%%%%%%%%%%%%%%%%%%%%%%%%%%%%%%%%%%%
% Block Size Analysis
%%%%%%%%%%%%%%%%%%%%%%%%%%%%%%%%%%%%%%%%%%%%%%%%%%%%%%%%%%%%%%%%%%%%%%%%%%%%%%%%
\subsection{Block Size Analysis}
\begin{frame}[label=functionExecutionTime]{Function Execution Time}
        {\escape{TopN_Outlier_Pruning_Block}}
    \blockSizeProfilingPlot{function_execution_time}

    \note{This graph shows the total execution time of the
        \escape{TopN_Outlier_Pruning_Block} function, sampled using 21 data
        sets with a size ranging from 441 2-dimensional vectors to 67557
        43-dimensional vectors.

        Each dataset was profiled with 13 different block sizes ranging from
        $10^0$ to $10^6$. In addition, a modified algorithm which operated
        without blocking was profiled for comparison.

        Each coloured plot in this graph represents a single data set. The solid
        lines show how the total execution time varies with block size. The
        dashed plots of the same colour illustrate a baseline comparison to the
        modified (no blocking) algorithm.

        An interesting observation from this graph shows that there is no
        significant difference in the algorithm's performance for block sizes
        from $10^0$ to $10^3$.}
\end{frame}

\begin{frame}{Distance Calls and Vectors Pruned}
        {\escape{TopN_Outlier_Pruning_Block}}
    \begin{columns}[c]
        \column{0.5\textwidth}
        \begin{figure}[H]
            \blockSizeProfilingPlot{distance_calls}
            \caption{Distance calls}
        \end{figure}

        \column{0.5\textwidth}
        \begin{figure}[H]
            \blockSizeProfilingPlot{vectors_pruned}
            \caption{Vectors pruned}
        \end{figure}
    \end{columns}
\end{frame}

\begin{frame}[label=runTimeComplexity]{Function Run Time Complexity}
        {\escape{TopN_Outlier_Pruning_Block}}
    \begin{columns}[c]
        \column{0.5\textwidth}
        \begin{figure}[H]
            \blockSizeProfilingPlot{function_run_time_complexity.lin}
            \caption{Linear plot}
        \end{figure}

        \column{0.5\textwidth}
        \begin{figure}[H]
            \blockSizeProfilingPlot{function_run_time_complexity.log}
            \caption{Logarithmic plot}
        \end{figure}
    \end{columns}
\end{frame}

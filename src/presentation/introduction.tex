%%%%%%%%%%%%%%%%%%%%%%%%%%%%%%%%%%%%%%%%%%%%%%%%%%%%%%%%%%%%%%%%%%%%%%%%%%%%%%%%
% Overview
%%%%%%%%%%%%%%%%%%%%%%%%%%%%%%%%%%%%%%%%%%%%%%%%%%%%%%%%%%%%%%%%%%%%%%%%%%%%%%%%
% Briefly tell the audience what you are going to cover.
\subsection{Overview}
\begin{frame}{Overview}
    \note<1>{Good afternoon...

        My name is \studentName{}. In this presentation, I will discuss the
        works and results of my \thesis{} project, which was conducted during
        the year of 2012 under the supervision of \supervisorName{}, and with
        assistance from PhD student \getPerson{Frechtling}.}

    \begin{itemize}
        \item<2-> Analysis and benchmarking of an anomaly detection algorithm.
        \begin{itemize}
            \item<2-> Uses commute time as a distance metric.
            \item<2-> Uses eigenspace embedding.
        \end{itemize}
        \note<2>{I will provide an overview of my analysis and
            benchmarking of an existing anomaly detection algorithm. This
            algorithm:
            \begin{itemize}
                \item Uses commute time as a distance metric. Commute time
                    effectively captures both the distance between nodes, as
                    well as the data density around these nodes.
                \item The algorithm also uses eigenspace embedding in order to
                    alleviate the $O(n^3)$ computational complexity of the
                    commute time calculations.
                %\item The graph components are sampled in order to reduce the
                %   graph size, and then eigenspace approximation is applied to
                %   approximate the commute time on the sampled graph.
            \end{itemize}
        }

        \bigskip
        \item<3-> Exploration of the use of reconfigurable computing to solve
            algorithmic problems.
        \note<3>{I will also briefly discuss the use of reconfigurable computing
            to solve algorithmic problems.}

        \bigskip
        \item<4-> Discuss an architecture design for using hardware to
            accelerate algorithms that involve repeated pairwise computations.
        \note<4>{And finally, I will also discuss an architecture which I have
            designed for the acceleration of a class of algorithms using
            reconfigurable hardware.}
    \end{itemize}
\end{frame}

%%%%%%%%%%%%%%%%%%%%%%%%%%%%%%%%%%%%%%%%%%%%%%%%%%%%%%%%%%%%%%%%%%%%%%%%%%%%%%%%
% Motivation
%%%%%%%%%%%%%%%%%%%%%%%%%%%%%%%%%%%%%%%%%%%%%%%%%%%%%%%%%%%%%%%%%%%%%%%%%%%%%%%%
% Briefly tell the audience why you are doing your research. Sell your audience
% on why your topic is important and of interest to them.
\subsection{Motivation}
\begin{frame}{Motivation}
    \note<1>{There are several reasons that anomaly detection is an important
        problem that deserves the attention of software and hardware engineers.}

    \begin{itemize}
        \item<2-> Anomaly detection is a fundamental data mining task.
        \note<2>{Anomaly detection is a fundamental data mining task with the
            aim to identify data points, events and transactions which deviate
            from the norm.}

        \medskip
        \item<3-> The amount of data being collected and stored is continually
            increasing.
        \note<3>{The amount of data being collected and stored is continually
            increasing.}

        \medskip
        \item<4-> Scalability problems exist with current algorithms.
        \note<4>{When this data is analysed, the time taken, cost of analysis,
            and energy required are all critical factors to consider, and
            algorithms with scalability problems in these areas cannot be
            feasibly used.}

        \medskip
        \item<5-> Real time appliations.
        \note<5>{There is an increased demand for real-time anomaly detection,
            particularly in the financial sectors.}

        \medskip
        \item<6-> Anomaly detection is an important technique which can be
            applied to a wide range of applications.
        \note<6>{Anomaly detection is an important technique which can be
            applied to a wide range of applications. These applications
            include...}
    \end{itemize}
\end{frame}

\begin{frame}{Applications}
    \begin{itemize}
        \item Stock market analysis
        \note[item]{Stock market analysis}

        \smallskip
        \item Network intrusion detection
        \note[item]{Network intrusion detection}

        \smallskip
        \item Image comparison
        \note[item]{Image comparison}

        \smallskip
        \item Fraud detection
        \note[item]{Fraud detection}

        \smallskip
        \item Fault detection
        \note[item]{Fault detection}

        \smallskip
        \item Event detection in sensor networks
        \note[item]{Event detection in sensor networks}

        \smallskip
        \item Detecting ecosystem disturbances
        \note[item]{Detecting ecosystem disturbances}

        \smallskip
        \item And many other applications...
        \note[item]{And many other applications...}
    \end{itemize}
\end{frame}

%%%%%%%%%%%%%%%%%%%%%%%%%%%%%%%%%%%%%%%%%%%%%%%%%%%%%%%%%%%%%%%%%%%%%%%%%%%%%%%%
% Aims
%%%%%%%%%%%%%%%%%%%%%%%%%%%%%%%%%%%%%%%%%%%%%%%%%%%%%%%%%%%%%%%%%%%%%%%%%%%%%%%%
% This section should be short and memorable.
\subsection{Aims}
\begin{frame}{Aims}\relax
    {\huge To \emph{investigate} the \emph{performance benefits} that can be
        obtained by \emph{applying} \emph{reconfigurable computing design
        principles} to an \textbf{anomaly detection algorithm} using commute
        time and eigenspace sampling.}

    \note{The aim of this \thesis{} was to \emph{investigate} the
        \textbf{performance benefits} that can be obtained by \emph{applying}
        \textbf{reconfigurable computing design principles} to an
        \textbf{anomaly detection algorithm} using commute time and eigenspace
        sampling.}
\end{frame}

%%%%%%%%%%%%%%%%%%%%%%%%%%%%%%%%%%%%%%%%%%%%%%%%%%%%%%%%%%%%%%%%%%%%%%%%%%%%%%%%
% Contributions
%%%%%%%%%%%%%%%%%%%%%%%%%%%%%%%%%%%%%%%%%%%%%%%%%%%%%%%%%%%%%%%%%%%%%%%%%%%%%%%%
% What I contributed towards this research area.
\subsection{Contributions}
\begin{frame}{Contributions}
    \note<1>{My contributions towards this research area include:}

    \begin{enumerate}
        \item<2-> A quantitative analysis and benchmarking of the properties of
            the chosen anomaly detection algorithm.
        \note<2>{A quantitative analysis and benchmarking of the properties of
            the chosen anomaly detection algorithm.}

        \bigskip
        \item<3-> An architecture for accelerating a wide class of algorithms,
            involving repeated pairwise computations.
        \note<3>{An architecture for accelerating a wide class of algorithms,
            involving repeated pairwise computations.}
        \begin{itemize}
            \item<3-> Anomaly detection
            \note[item]<3>{Anomaly detection}

            \item<3-> Cluster analysis
            \note[item]<3>{Cluster analysis}

            \item<3-> Statistical classification
            \note[item]<3>{Statistical classification}

            \item<3-> Regression analysis
            \note[item]<3>{Regression analysis}
        \end{itemize}

        \bigskip
        \item<4-> Initial implementation and performance estimates for a
            software-hardware co-design for anomaly detection.
        \note<4>{An initial implementation and performance estimates for a
            software-hardware co-design for anomaly detection.}
    \end{enumerate}
\end{frame}

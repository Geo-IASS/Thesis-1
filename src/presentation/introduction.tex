%%%%%%%%%%%%%%%%%%%%%%%%%%%%%%%%%%%%%%%%%%%%%%%%%%%%%%%%%%%%%%%%%%%%%%%%%%%%%%%%
% Overview
%%%%%%%%%%%%%%%%%%%%%%%%%%%%%%%%%%%%%%%%%%%%%%%%%%%%%%%%%%%%%%%%%%%%%%%%%%%%%%%%
% Briefly tell the audience what you are going to cover.
\subsection{Overview}
\begin{frame}[label=overview]{Overview}
    \note{In this presentation, I will discuss the works and results of my
        \thesis{} project, which was conducted during the year of 2012 under the
        supervision of \supervisorName{}, and with assistance from PhD student
        \getPerson{Frechtling}.}

    \begin{itemize}
        \item<1-> Analysis and benchmarking of an anomaly detection algorithm.
        \begin{itemize}
            \item<1-> Uses commute time as a distance metric.
            \item<1-> Uses eigenspace embedding.
        \end{itemize}
        \note[item]{I will provide an overview of my analysis and
            benchmarking of an existing anomaly detection algorithm. This
            algorithm:
            \begin{itemize}
                \item Uses commute time as a distance metric. Commute time
                effectively captures both the distance between nodes, as well as
                the data density around these nodes.
                \item The use of eigenspace embedding alleviates the $O(n^3)$
                computational complexity of the commute time calculations.
                \item The graph components are sampled in order to reduce the
                graph size, and then eigenspace approximation is applied to
                approximate the commute time on the sampled graph.
            \end{itemize}
        }

        \bigskip
        \item<2-> Exploration of the use of reconfigurable computing to solve
            computing problems.
        \note[item]{I will also discuss an architecture which I have designed
            for the acceleration of a class of algorithms using reconfigurable
            hardware.}

        \bigskip
        \item<3->  Discuss an architecture design for using hardware to
            accelerate algorithms that involve repeated pairwise computations.
    \end{itemize}
\end{frame}

%%%%%%%%%%%%%%%%%%%%%%%%%%%%%%%%%%%%%%%%%%%%%%%%%%%%%%%%%%%%%%%%%%%%%%%%%%%%%%%%
% Motivation
%%%%%%%%%%%%%%%%%%%%%%%%%%%%%%%%%%%%%%%%%%%%%%%%%%%%%%%%%%%%%%%%%%%%%%%%%%%%%%%%
% Briefly tell the audience why you are doing your research. Sell your audience
% on why your topic is important and of interest to them.
\subsection{Motivation}
\begin{frame}[label=motivation]{Motivation}
    \note{There are several reasons that anomaly detection is an important
        problem that deserves the attention of software and hardware engineers.}

    \begin{itemize}[<+->]
        \item Anomaly detection is a fundamental data mining task with the aim
            to identify data points, events, transactions which deviate from the
            norm.

        \medskip
        \item The amount of data being collected and stored is continually
            increasing.
        \note[item]{The amount of data being collected and stored is
            continually increasing.}

        \medskip
        \item When this data is analysed, the time taken, cost of analysis, and
            energy required are all critical factors to consider, and algorithms
            with scalability problems in these areas cannot be feasibly used.

        \medskip
        \item Anomaly detection is an important technique which can be
            applied to a wide range of applications.
        \note[item]{Anomaly detection is an important technique which can be
            applied to a wide range of applications. These applications
            include...}
    \end{itemize}
\end{frame}

\begin{frame}[label=applications]{Applications}
    \note{Anomaly detection is an interesting problem with a wide range of
        applications, including:}

    \begin{itemize}
        \item Stock market analysis
        \note[item]{Stock market analysis}

        \smallskip
        \item Network intrusion detection
        \note[item]{Network intrusion detection}

        \smallskip
        \item Image comparison
        \note[item]{Image comparison}

        \smallskip
        \item Fraud detection
        \note[item]{Fraud detection}

        \smallskip
        \item Fault detection
        \note[item]{Fault detection}

        \smallskip
        \item Event detection in sensor networks
        \note[item]{Event detection in sensor networks}

        \smallskip
        \item Detecting ecosystem disturbances
        \note[item]{Detecting ecosystem disturbances}

        \smallskip
        \item And many other applications...
        \note[item]{And many other applications...}
    \end{itemize}
\end{frame}

%%%%%%%%%%%%%%%%%%%%%%%%%%%%%%%%%%%%%%%%%%%%%%%%%%%%%%%%%%%%%%%%%%%%%%%%%%%%%%%%
% Aims
%%%%%%%%%%%%%%%%%%%%%%%%%%%%%%%%%%%%%%%%%%%%%%%%%%%%%%%%%%%%%%%%%%%%%%%%%%%%%%%%
% This section should be short and memorable.
\subsection{Aims}
\begin{frame}[label=aims]{Aims}\relax
    {\huge To \emph{investigate} the \emph{performance benefits} that can be
        obtained by \emph{applying} \emph{reconfigurable computing design
        principles} to an \textbf{anomaly detection algorithm} using commute
        time and eigenspace sampling.}

    \note{The aim of this \thesis{} was to \emph{investigate} the
        \textbf{performance benefits} that can be obtained by \emph{applying}
        \textbf{reconfigurable computing design principles} to an
        \textbf{anomaly detection algorithm} using commute time and eigenspace
        sampling.}
\end{frame}

%%%%%%%%%%%%%%%%%%%%%%%%%%%%%%%%%%%%%%%%%%%%%%%%%%%%%%%%%%%%%%%%%%%%%%%%%%%%%%%%
% Contributions
%%%%%%%%%%%%%%%%%%%%%%%%%%%%%%%%%%%%%%%%%%%%%%%%%%%%%%%%%%%%%%%%%%%%%%%%%%%%%%%%
% What I contributed towards this research area.
\subsection{Contributions}
\begin{frame}[label=contributions]{Contributions}
    \note{My contributions towards this research area include:}

    \begin{enumerate}
        \item<1-> A quantitative analysis and benchmarking of the properties of
            the chosen anomaly detection algorithm.

        \bigskip
        \item<2-> An architecture for accelerating a wide class of algorithms,
            involving repeated pairwise computations.
        \begin{itemize}
            \item<2-> Anomaly detection
            \item<2-> Cluster analysis
            \item<2-> Statistical classification
            \item<2-> Regression analysis
        \end{itemize}

        \bigskip
        \item<3-> Implementation and performance estimates for a
            software-hardware co-design for anomaly detection.
    \end{enumerate}
\end{frame}

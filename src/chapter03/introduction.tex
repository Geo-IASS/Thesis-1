In the computer and electronics world, computations are performed either in
hardware or in software (or in some combination of both!). Computer hardware
provides highly optimized resources for quickly performing critical tasks, but
is permanently configured to a single task or application. Computer software
offers a flexible approach, but is orders of magnitude worse than a hardware
implementation in terms of performance, silicon area efficiency and power
consumption \cite{Hauck:2007}.

\glspl{FPGA} are devices that combine the advantages of hardware implementations
with the flexibility of software implementations. The computations are
programmed into the silicon chip such that an \gls{FPGA} system can be
programmed and reprogrammed many times. The utility of FPGAs does, however, come
at a price. Whilst, compared to a microprocessor, \glspl{FPGA} are typically
several orders of magnitude faster and more power efficient, the task of
creating efficient programs for these devices is difficult.

It is the goal of an \gls{FPGA} to provide more performance than pure software
implementations running on general-purpose processors, as well as providing
flexibility over a fixed-solution \gls{ASIC} implementation. In order to realise
this goal, it is essential that the \gls{FPGA} is a reconfigurable device
\cite{Hauck:2007}. A defining characteristic that allows this purpose is that
an \gls{FPGA} is used as a `blank hardware device for the end user.

Typically, \glspl{FPGA} are useful only for operations that process large
streams of data, such as signal processing, networking, and the like. Compared
to integrated circuit, they may be 5 to 25 times worse in terms of area, delay,
and performance \cite{Hauck:2007}. However, while an integrated circuit design
may take months to years to develop and have a multimillion-dollar price tag, an
\gls{FPGA} design might only take days to create and cost tens to hundreds of
dollars.

Typically, computer software is designed as sequential code, exploiting a
microprocessor's ability to rapidly step through a series of instructions.
Hardware, however, is best suited to a design that consider spatial parallelism
--- that is, simultaneously using multiple resources spread across a chip to
yield a huge amount of computations \cite{Hauck:2007}.

The combination of the high perform of \glspl{ASIC} and the flexibility of
microprocessors has made possible entirely new types of applications
\cite{Hauck:2007}. In order to most effectively utilise the benefits from a
hardware-software co-design, designers need to be aware of both hardware and
software limitations and issues. Essentially, a designer must be aware of the
hardware architecture that best supports a design, as well as the software flow
that supports the design process.